\chapter{Preliminaries}

\section{Generalised motion groups}

\begin{definition}[Semi-direct product]
\label{definition:semi-direct_products}
\index{semi-direct product}
\index{motion group|see {generalised motion group}}
    Let $N$ be a group and $H$ be a subgroup of $\Aut(N)$.
    The \emph{semi-direct product} $N \rtimes H$ is the group whose elements are that of $N \times H$ with the group law
    \begin{align}
        (n_1, h_1) (n_2, h_2) \defeq (n_1 h_1(n_2), h_1 h_2), \quad (n_1, h_1), (n_2, h_2) \in N \times H.
    \end{align}

    Given $(n, h)$ in $N \rtimes H$, its \emph{inverse} is given by $(h_1^{-1}(n_1^{-1}), h_1^{-1})$,
    while $(e_N, e_H)$ is the \emph{identity element} of the group.
\end{definition}

\begin{definition}[Generalised motion group]
\label{definition:generalised_motion_group}
\index{generalised motion group}
    Let $\Group$ be a group.
    We shall say that $\Group$ is a \emph{generalised motion group}
    if there exists a (finite dimensional) real Euclidean space $\VectorSpace$
    and a compact group $\CompactGroup \subset \SpecialOrthogonalGroup{\VectorSpace}$
    such that $\Group = \VectorSpace \rtimes \CompactGroup$
    and $\CompactGroup$ acts transitively on $\VectorSpace$.
\end{definition}

\begin{remark}
    Let $x \in \VectorSpace$ and $k \in \CompactGroup$.
    We will never identify $x$ with $(x, \Id{\VectorSpace}) \in \VectorSpace \rtimes \CompactGroup$,
    or $k$ with $(0, k) \in \VectorSpace \rtimes \CompactGroup$.

    Therefore, when we write $k x$, it will \emph{always} mean the vector obtained by rotating $x$ by $k$, i.e.\ $k(x)$.
    If we see them as elements of $\VectorSpace \rtimes \CompactGroup$,
    we shall explicitely write
    \begin{align*}
        (0, k) (x, \Id{\VectorSpace}) = (k x, k).
    \end{align*}
\end{remark}

\begin{example}[Euclidean Motion Groups]
\label{example:Euclidean_motion_groups}
\index{Euclidean motion group}
    For each $n \in \N$, let
    \begin{align*}
        \MotionGroup{n} \defeq \{g \in \AffineTransformations{\R^n} : \det g = 1\}.
    \end{align*}
    The elements of $\MotionGroup{n}$ are called \emph{rigid motions},
    while $\MotionGroup{n}$ is called the \emph{Euclidean motion group}.

    It is easily shown that associating $(x, k) \in \R^n \rtimes \SpecialOrthogonalGroup{n}$ to the motion
    \begin{align*}
        g_{(x, k)} : \R^n \to \R^n : y \mapsto x + ky
    \end{align*}
    defines a group isomorphism between $\R^n \rtimes \SpecialOrthogonalGroup{n}$ and $\MotionGroup{n}$.
    We shall therefore identify $\MotionGroup{n}$ with $\R^n \rtimes \SpecialOrthogonalGroup{n}$ from now on.
\end{example}

\begin{example}[$2$-dimensional Euclidean motion group]
\label{example:Euclidean_motion_groups:dimension_2}
\index{Euclidean motion group!dimenion 2}
    The $2$-dimensional case is worth mentioning because
    in that case, $\SpecialOrthogonalGroup{2}$ is \emph{abelian}.

    Since $\SpecialOrthogonalGroup{2}$ is isomorphic to $\T$,
    we shall often identify $\MotionGroup{2}$ with $\R^2 \times \T$ and the group law
    \begin{align*}
        (x, t) (y, s) \defeq (x + \e^{\i \turn t} y, t + s), \quad (x, t), (y, s) \in \R^2 \times \T.
    \end{align*}
\end{example}

\begin{example}
    \label{example:complex_motion_groups}
    Let $n \in \N$.
    Consider the group
    \begin{align*}
        \{g \in \AffineTransformations{\C^n} : \det_{\C^n} g = 1\}
    \end{align*}
    where the law is the composition of functions.

    Arguing like in Example~\ref{example:Euclidean_motion_groups},
    the above group can be identified with $\C^n \rtimes \SpecialUnitaryGroup{n}$.
\end{example}

\begin{remark}
    Since in our examples (e.g. Example \ref{example:complex_motion_groups}) our vector space might be $\C^n$,
    we choose to use $\VectorSpace$ to denote the vector space instead of simply $\R^n$ to avoid any confusion.
\end{remark}

From now on, unless stated otherwise,
$\Group$ will denote a generalised motion group,
with $\VectorSpace$ its underlying Eucliean space and $\CompactGroup$ its associated compact group.

The requirement that $\CompactGroup$ should be a subgroup of $\OrthogonalGroup{\VectorSpace}$ is motivated by the following result

\begin{lemma}[Haar measure]
\label{lemma:Haar_measure}
    If $\dd x$ is a Lebesgue measure on $\VectorSpace$ and $\dd k$ is the normalised Haar measure on $\CompactGroup$,
    then the the product measure $\dd x \dd k$ is a Haar measure on $\Group = \VectorSpace \rtimes \CompactGroup$,
    which is both left and right-invariant.
\end{lemma}
\begin{proof}
    Let $(x, k) \in \Group$.
    \begin{align*}
        \int_\Group f((x, k) (y, l)) \dd (y, l)
        = \int_\VectorSpace \int_\CompactGroup f(x + ky, k l) \dd l \dd y
    \end{align*}

    Now, let us substitute $y$ for $k^{-1}(y - x)$ and $l$ for $k^{-1} l$ in the above.
    As the Lebesgue measure is invariant under $\OrthogonalGroup{\VectorSpace}$ and under translations,
    and because the Haar measure $\dd l$ is left-invariant,
    we obtain
    \begin{align*}
        \int_\Group f((x, k) (y, l)) \dd (y, l)
        &= \int_\VectorSpace \int_\CompactGroup f(y, l) \dd l \dd y\\
        &= \int_\Group f(y, l) \dd (y, l),
    \end{align*}
    showing that $\dd y \dd l$ is indeed a Haar measure on $\Group$.

    Since by Proposition~\ref{proposition:sufficient_conditions_to_be_unimodular} $\dd l$ is also right-invariant,
    arguing similarly shows that $\dd y \dd l$ is also right-invariant.
\end{proof}

\begin{proposition}
    Let $\VectorSpace$ be a finite dimensional vector space
    and $\CompactGroup$ be a subgroup of $\SpecialOrthogonalGroup\VectorSpace$.
    The following properties hold:
    \begin{enumerate}
        \item The Lebesgue measure on $\VectorSpace$ is invariant under $\CompactGroup$,
            i.e.\ for every $k \in \CompactGroup$ and each Borel set $A \subset \VectorSpace$, we have
            \begin{align*}
                \int_A 1 \dd x = \int_{kA} 1 \dd x;
            \end{align*}
        \item The Laplacian on $\VectorSpace$ is invariant under $\CompactGroup$,
            i.e.\ for every $k \in \CompactGroup$ and every $\phi \in \SmoothFunctions\VectorSpace$, we have
            \begin{align*}
                \Laplacian[\VectorSpace] (\phi \circ k)(x) = \Laplacian[\VectorSpace] \phi(k x);
            \end{align*}
        \item The action of $\CompactGroup$ on $\VectorSpace$ commutes with the dilation structure of $\VectorSpace$.
    \end{enumerate}
\end{proposition}
\begin{proof}
    \begin{enumerate}
        \item This follows easily from the change of variables formula,
            \begin{align*}
                \int_{k A} 1 \dd x
                = \int_A (1 \circ k) \det k \dd x
                = \int_A 1 \dd x,
            \end{align*}
            where we used the fact that $\det k = 1$ since $k \in \SpecialOrthogonalGroup\VectorSpace$.
        \item First, observe that
            \begin{align*}
                \dd (\phi \circ k)(x) = \dd \phi(k x) k
            \end{align*}
            which implies that $\grad (\phi \circ k)(x) = k^{-1} \grad \phi(k x)$.

            From there, using the fact that $\Hessian = \dd \grad$,
            \begin{align*}
                \Hessian (\phi \circ k)(x) = (\dd \grad(\phi \circ k))(x) = k^{-1} (\Hessian \phi(k x)) k.
            \end{align*}

            Therefore, we conclude by observing that
            \begin{align*}
                \Laplacian[\VectorSpace] (\phi \circ k)(x)
                = \tr (\Hessian (\phi \circ k)(x))
                = \tr (\Hessian \phi (k x))
                = \Laplacian[\VectorSpace] \phi(k x).
            \end{align*}
    \end{enumerate}
\end{proof}

\section{Lie groups}

\begin{definition}[Topological group]
\label{definition:topological_group}
    Let $\Group$ be a group and a Haussdorf topological space.
    We shall say that $\Group$ is a \emph{topological group} if the map
    \begin{align*}
        \Group \times \Group \to \Group :
        (g_1, g_2) \mapsto g_1^{-1} g_2
    \end{align*}
    is continuous.
\end{definition}

\begin{definition}[Lie group]
\label{definition:Lie_group}
\index{Lie group}
    Let $\Group$ be a group.
    We say that $\Group$ is a \emph{Lie group}
    if $\Group$ is a smooth manifold and the map
    \begin{align*}
        \Group \times \Group \to \Group :
        (g_1, g_2) \mapsto g_1^{-1} g_2
    \end{align*}
    is smooth.

    If moreover $\Group$ is (locally) compact as a manifold,
    then we shall say that $\Group$ is a \emph{(locally) compact Lie group}.
\end{definition}

\begin{definition}[Haar measure]
\index{Haar measure}
    Let $\Group$ be a locally compact topological group.
    A positive Radon measure on $\Group$ is called a \emph{Haar measure}
    if it is in addition \emph{left-invariant},
    i.e.\ for each $g \in \Group$ and each Borel set $A \subset G$, we have
    \begin{align*}
        \mu(g A) = \mu(A).
    \end{align*}
\end{definition}

\begin{proposition}[Haar measure]
    If $\Group$ is a locally compact topological group,
    then there exists a Haar measure $\mu$ on $\Group$.

    Moreover, if $\nu$ is another left-invariant Radon measure on $\Group$,
    then we can find $c \geq 0$ such that $\nu = c \mu$.
\end{proposition}

\begin{definition}[Unimodular group]
\label{definition:unimodular_group}
    Let $\Group$ be a locally compact group.
    If a Haar measure on $\Group$ is also right-invariant,
    we shall say that $\Group$ is \emph{unimodular}.
\end{definition}

\begin{proposition}
\label{proposition:sufficient_conditions_to_be_unimodular}
    Let $\Group$ be a topological group.
    \begin{enumerate}
        \item If $\Group$ is compact, then $\Group$ is \emph{unimodular}.
        \item If $\Group$ is abelian and locally compact, then $\Group$ is unimodular.
    \end{enumerate}
\end{proposition}

\section{Representation Theory}

\begin{definition}[Unitary representations]
\label{definition:unitary_representation}
\index{representations!unitary representations}
    Let $\Group$ be a group and $\Hil$ be a Hilbert space.
    A map
    \begin{align*}
        \xi : \Group \mapsto \Hom(\Hil)
    \end{align*}
    is called a \emph{unitary representation (on $\Hil$)} if
    \begin{enumerate}
        \item for each $g \in \Group$, the map $\xi(g)$ is unitary:
            \begin{align*}
                {\xi(g)}^{-1} = \adj{\xi(g)};
            \end{align*}
        \item if $g, h \in \Group$, then we have $\xi(g h) = \xi(g) \xi(h)$.
    \end{enumerate}

    The \emph{dimension} of $\xi$ is that of $\Hil$.
    If $\Hil$ is finite-dimensional,
    we let $\dimRep{\xi} \defeq \dim{\Hil}$ denote the dimension of $\xi$.
\end{definition}

\begin{example}[Left-regular representation]
    Let $\Group$ be a unimodular topological group.
    The \emph{left-regular representation} is the representation
    \begin{align*}
        \pi_L : \Group \to \Hom(\Lebesgue{2}{\Group})
    \end{align*}
    defined via
    \begin{align*}
        \pi_L(h) f(g) = f(h^{-1} g)
    \end{align*}
    for every $g, h \in \Group$.
\end{example}

\begin{definition}[Invariant subspaces]
\label{definition:invariant_subspaces}
    Let $\Group$ be a group and $\xi$ be a unitary representation of $\Group$ on a Hilbert space $\Hil$.
    We shall say that a vector subspace $W \subset \Hil$ is \emph{invariant} under $\xi$
    if for each $g \in \Group$, we have $\xi(g) W \subset W$.
\end{definition}

\begin{definition}[Irreducibility]
\label{definition:irreducible_representations}
    Let $\Group$ be a group and $\xi$ be a unitary representation of $\Group$ on a Hilbert space $\Hil$.
    \begin{enumerate}
        \item If the only invariant subspaces of $\xi$ are $\{0\}$ and $\Hil$,
            then $\xi$ is said to be \emph{irreducible}.
        \item Otherwise, if there exists a non-trivial invariant subspace,
            then $\xi$ is \emph{reducible}.
    \end{enumerate}
\end{definition}

\begin{definition}[Equivalent representations]
\label{definition:equivalent_representations}
    Let $\Group$ be a group and $\Hil_1, \Hil_2$ be Hilbert spaces.
    Suppose $\xi_1$, $\xi_2$ are representations of $\Group$ on $\Hil_1$ and $\Hil_2$ respectively.
    We shall say that $\xi_1$ and $\xi_2$ are \emph{equivalent}
    if there exists an invertible linear map
    \begin{align*}
        A : \Hil_1 \to \Hil_2
    \end{align*}
    such that for each $g \in \Group$, we have
    \begin{align*}
        \xi_2(g) = A \circ \xi_1(g) \circ A^{-1}.
    \end{align*}
    In that case, $A$ is called an \emph{intertwining operator}.
\end{definition}

\begin{definition}[Strongly continuous representations]
\label{definition:strongly_continuous_representation}
\index{strongly continuous}
    Let $\Group$ be a group and $\Hil$ be a Hilbert space.
    Suppose further that $\xi$ is a representation of $\Group$ on $\Hil$.
    We shall say that $\xi$ is \emph{strongly continuous}
    if for each $x \in \Hil$,
    the map
    \begin{align*}
        \Group \to \Hil : g \mapsto \xi(g) v
    \end{align*}
    is continuous.
\end{definition}

\begin{definition}[Unitary dual]
\label{definition:unitary_dual}
    Let $\Group$ be a locally compact topological group.
    The \emph{unitary dual} of $\Group$, denoted by $\dualGroup\Group$,
    is the set of all equivalence classes of
    \emph{strongly continuous, irreducible, unitary representations} of $\Group$.
\end{definition}

\begin{remark}
    Let $\Group$ be a locally compact topological group.
    We shall often abuse the notation and use $\dualGroup\Group$ to denote a set consisting of
    exactly one representation in each equivalence class of the actual unitary dual.
\end{remark}

\begin{example}[$\dualGroup\VectorSpace$]
    For each $\lambda \in \VectorSpace$,
    define
    \begin{align}
        \xi_\lambda : \VectorSpace \to \UnitaryGroup{1} : x \mapsto \e^{\i \turn \ip{\lambda}{x}}.
        \label{eq:elements_of_dual_of_vector_space}
    \end{align}

    It can be shown that
    \begin{align*}
        \dualGroup\VectorSpace = \{ \xi_\lambda : \lambda \in \VectorSpace \}.
    \end{align*}

    Therefore, the map
    \begin{align}
        \lambda \mapsto \xi_\lambda
        \label{eq:isomorphism_between_vector_space_and_its_dual_group}
    \end{align}
    is a group isomorphism which allows us to give $\dualGroup\VectorSpace$ a vector space structure.
\end{example}

\begin{theorem}[Peter-Weyl theorem]
\label{theorem:Peter-Weyl_theorem}
    Let $\Group$ be a compact topological group.
    The set
    \begin{align*}
        \left\{
            \sqrt{\dimRep\xi} \xi_{ij} : \xi \in \dualGroup\Group,\ i, j = 1, \dots, \dimRep\xi
        \right\}
    \end{align*}
    is an orthonormal basis of $\Lebesgue{2}{\Group}$ with respect to the \emph{normalised} Haar measure.

    Writing $\Hilbert{\xi}{j} \defeq \Span\{\xi_{ij} : i = 1, \dots, \dimRep\xi\}$
    for each $\xi \in \dualGroup\Group$,
    we obtain the following decomposition:
    \begin{align*}
        \Lebesgue{2}{\Group} \defeq
        \bigoplus_{\xi \in \dualGroup\Group} \bigoplus_{j = 1}^{\dimRep \xi} \Hilbert{\xi}{j}.
    \end{align*}
    Therefore,
    since $\Hilbert{\xi}{j}$ is an invariant subspace of $\pi_L$
    and since $\eval{\pi_L}{\Hilbert{\xi}{j}} \sim \xi$,
    we have
    \begin{align*}
        \pi_L \sim
        \bigoplus_{\xi \in \dualGroup\Group} \bigoplus_{j = 1}^{\dimRep \xi} \xi.
    \end{align*}
    In particular, the unitary dual can be generated by the left-regular representation.
\end{theorem}

\section{Fourier Transform}

\begin{definition}[Fourier coefficient]
    Let $\xi \in \dualGroup\Group$.
    If $f \in \Lebesgue{1}{\Group}$,
    we define the \emph{Fourier coefficient of $f$ at $\xi$}, $\Fourier f(\xi)$, via
    \begin{align*}
        \Fourier f(\xi) \defeq \int_\Group f(g) \adj{\xi(g)} \dd g.
    \end{align*}

    The map
    \begin{align*}
        \Fourier f : \xi \mapsto \Fourier f(\xi)
    \end{align*}
    is called the \emph{Fourier transform of $f$}.
\end{definition}

\begin{proposition}
    Let $f, f_1, f_2 \in \Lebesgue{1}{\Group}$ and $\xi \in \dualGroup\Group$.
    The Fourier Transform satisfies the following properties:
    \begin{enumerate}
        \item For each $g \in \Group$, we have
            \begin{align*}
                \Fourier \{f(\dummy g)\} (\xi)
                = \xi(g) \Fourier f(\xi), \quad
                \Fourier \{f(g \dummy)\} (\xi)
                = \Fourier f(\xi) \xi(g).
            \end{align*}
        \item We have
            \begin{align*}
                \Fourier \{\conv{f_1}{f_2}\}(\xi)
                = \Fourier f_2(\xi) \Fourier f_1(\xi).
            \end{align*}
        \item If $f \in \SmoothFunctions\Group$ and $X \in \LieAlgebra$,
            \begin{align*}
                \Fourier \{\LeftDifferentialOperatorFirstOrder{X} f\}(\xi)
                = \xi(X) \Fourier f(\xi), \quad
                \Fourier \{\RightDifferentialOperatorFirstOrder{X} f\}(\xi)
                = \Fourier f(\xi) \xi(X).
            \end{align*}
    \end{enumerate}
\end{proposition}

\begin{proposition}
    There exists a measure $\Plancherel\Group$ on $\dualGroup\Group$ such that the following property holds:
    if $f \in \Schwartz\Group$, we have
    \begin{align*}
        \int_\Group \abs{f}^2 \dd g
        = \int_{\dualGroup\Group} \tr \left( \Fourier f(\xi) \adj{\Fourier f(\xi)} \right) \dd \Plancherel\Group(\xi).
    \end{align*}
\end{proposition}

\begin{example}[Plancherel measure on $\dualGroup\VectorSpace$]
    The \emph{Plancherel measure} on $\VectorSpace$ is the pushforward measure of the Lebesgue measure on $\VectorSpace$ via the isomorphism~\eqref{eq:isomorphism_between_vector_space_and_its_dual_group}.
    In other words, if $f \in \Schwartz\VectorSpace$, then
    \begin{align*}
        \int_\VectorSpace \abs{f}^2 \dd x = \int_\VectorSpace \abs{\Fourier[\VectorSpace] f(\xi_\lambda)}^2 \dd \lambda,
    \end{align*}
    where $\xi_\lambda$ was defined in~\eqref{eq:elements_of_dual_of_vector_space}.
\end{example}

\begin{example}[Plancherel Measure on $\dualGroup\CompactGroup$]
    If $f \in \SmoothFunctions{\CompactGroup}$,
    then the Peter-Weyl theorem implies:
    \begin{align*}
        \int_\CompactGroup \abs{f}^2 \dd g
        = \sum_{\tau \in \dualGroup\CompactGroup}
            \dimRep\tau
            \tr \left(
                \Fourier[\CompactGroup] f(\tau)
                \adj{\Fourier[\CompactGroup] f(\tau)}
            \right)
    \end{align*}
\end{example}

\begin{proposition}[Inverse Fourier Transform]
    Let $g \in \Group$.
    If $f \in \Schwartz\Group$,
    then we have
    \begin{align*}
        f(g) =
        \int_\dualGroup\Group
            \tr\left(
                \xi(g)
                \Fourier f(\xi)
            \right)
        \dd \Plancherel\Group(\xi).
    \end{align*}
\end{proposition}

\section{Difference operators}

\begin{proposition}
    Let $\Group$ be a compact topological group.
    There exists a finite subfamily $\{q_1, \dots, q_M\}$ of
    \begin{align*}
        \{ \tau_{ij} - \Kronecker{i}{j} : \tau \in \dualGroup\Group,\ i, j = 1, \dots, \dimRep\tau \}
    \end{align*}
    which is \emph{strongly admissible}.
\end{proposition}

\section{Pseudo-differential calculus}
