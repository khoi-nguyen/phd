\chapter{Preliminaries}

\section{Lie groups}

To develop a \emph{pseudo-differential calculus} on groups,
a reasonable prerequisite is that the group be equipped with a differential structure.

\begin{definition}[Lie group]
\label{definition:Lie_group}
\index{Lie group}
    Let $\Group$ be a group.
    We say that $\Group$ is a \emph{Lie group}
    if $\Group$ is a smooth manifold and the map
    \begin{align*}
        \Group \times \Group \to \Group :
        (g_1, g_2) \mapsto g_1^{-1} g_2
    \end{align*}
    is smooth.

    If moreover $\Group$ is (locally) compact as a manifold,
    then we shall say that $\Group$ is a \emph{(locally) compact Lie group}.
\end{definition}

\subsection{Lie algebra}

\begin{definition}[Lie algebra]
    A (real) Lie algebra is a (real) vector space $\LieAlgebra$
    equipped with a bilinear map
    \begin{align*}
        \LieBracket \dummy \dummy : \VectorSpace \times \VectorSpace \to \VectorSpace,
    \end{align*}
    called the \emph{Lie bracket} or \emph{commutator},
    such that
    \begin{enumerate}
        \item $\LieBracket X X = 0$ for every $X \in \LieAlgebra$;
        \item for every $X, Y, Z \in \LieAlgebra$, we have the following \emph{Jacobi identity}
            \begin{align*}
                \LieBracket X {\LieBracket Y Z} +
                \LieBracket Y {\LieBracket Z X} +
                \LieBracket Z {\LieBracket X Y}
                = 0.
            \end{align*}
    \end{enumerate}
\end{definition}

\begin{example}[Left-invariant vector fields]
    Let $\Group$ be a Lie group.
    If we define $\LieAlgebra$ to be the set of all left-invariant vector fields,
    i.e.\ the set of all vector fields $X$ such that
    \begin{align*}
        X f(x) = X_{y = e} f(x y)
    \end{align*}
    for every $f \in \SmoothFunctions \Group$.

    Given two left-invariant vector fields $X, Y$,
    we define
    \begin{align*}
        \LieBracket X Y f(x) = X Y f(x) - Y X f(x),
    \end{align*}
    and we can show that $\LieBracket X Y$ is also a left-invariant vector field.

    We can check that $\LieBracket \dummy \dummy$ gives $\LieAlgebra$ a \emph{Lie algebra} structure.
    Moreover, since left-invariant vector fields are completely determined by the tangent vector at the identity,
    $\LieBracket \dummy \dummy$ gives a \emph{Lie algebra} structure to the tangent plane of $\Group$ at the identity.
\end{example}

\begin{definition}[Lie algebra of Lie group]
\label{definition:Lie_algebra_of_Lie_group}
    Let $\Group$ be a Lie group.
    The tangent plane of $\Group$ at the identity is called
    the \emph{Lie algebra} of $\Group$.
\end{definition}

\subsection{Haar measure}

\begin{remark}
    Every Lie group $\Group$ is also topological space.
    As a result, all the topological definitions apply to Lie groups.
\end{remark}

\begin{definition}[Haar measure]
\index{Haar measure}
    Let $\Group$ be a Lie group.
    A positive Radon measure on $\Group$ is called a \emph{Haar measure}
    if it is in addition \emph{left-invariant},
    i.e.\ for each $g \in \Group$ and each Borel set $A \subset G$, we have
    \begin{align*}
        \mu(g A) = \mu(A).
    \end{align*}
\end{definition}

\begin{proposition}[Haar measure]
    If $\Group$ is a locally compact Lie group,
    then there exists a Haar measure $\mu$ on $\Group$.

    Moreover, if $\nu$ is another left-invariant Radon measure on $\Group$,
    then we can find $c \geq 0$ such that $\nu = c \mu$.
\end{proposition}

\begin{definition}[Unimodular group]
\label{definition:unimodular_group}
    Let $\Group$ be a locally compact Lie group.
    If a non-zero Haar measure on $\Group$ is also right-invariant,
    or equivalently if all Haar measure are right-invariant,
    we shall say that $\Group$ is \emph{unimodular}.
\end{definition}

\begin{proposition}
\label{proposition:sufficient_conditions_to_be_unimodular}
    Let $\Group$ be a Lie group.
    \begin{enumerate}
        \item If $\Group$ is compact, then $\Group$ is \emph{unimodular}.
        \item If $\Group$ is abelian and locally compact, then $\Group$ is unimodular.
    \end{enumerate}
\end{proposition}

\section{Representation Theory}

\begin{definition}[Unitary representations]
\label{definition:unitary_representation}
\index{representations!unitary representations}
    Let $\Group$ be a group and $\Hil$ be a Hilbert space.
    A map
    \begin{align*}
        \xi : \Group \mapsto \Hom(\Hil)
    \end{align*}
    is called a \emph{unitary representation (on $\Hil$)} if
    \begin{enumerate}
        \item for each $g \in \Group$, the map $\xi(g)$ is unitary:
            \begin{align*}
                {\xi(g)}^{-1} = \adj{\xi(g)};
            \end{align*}
        \item if $g, h \in \Group$, then we have $\xi(g h) = \xi(g) \xi(h)$.
    \end{enumerate}

    The \emph{dimension} of $\xi$ is that of $\Hil$.
    If $\Hil$ is finite-dimensional,
    we let $\dimRep{\xi} \defeq \dim{\Hil}$ denote the dimension of $\xi$.
\end{definition}

\begin{example}[Right-regular representation]
    Let $\Group$ be a unimodular topological group.
    The \emph{right-regular representation} is the representation
    \begin{align*}
        \RightRegularRepresentation : \Group \to \Hom(\Lebesgue{2}{\Group})
    \end{align*}
    defined via
    \begin{align*}
        \RightRegularRepresentation(h) f(g) = f(g h)
    \end{align*}
    for every $g, h \in \Group$.
\end{example}

\begin{definition}[Invariant subspaces]
\label{definition:invariant_subspaces}
    Let $\Group$ be a group and $\xi$ be a unitary representation of $\Group$ on a Hilbert space $\Hil$.
    We shall say that a vector subspace $W \subset \Hil$ is \emph{invariant} under $\xi$
    if for each $g \in \Group$, we have $\xi(g) W \subset W$.
\end{definition}

\begin{definition}[Irreducibility]
\label{definition:irreducible_representations}
    Let $\Group$ be a group and $\xi$ be a unitary representation of $\Group$ on a Hilbert space $\Hil$.
    \begin{enumerate}
        \item If the only invariant subspaces of $\xi$ are $\{0\}$ and $\Hil$,
            then $\xi$ is said to be \emph{irreducible}.
        \item Otherwise, if there exists a non-trivial invariant subspace,
            then $\xi$ is \emph{reducible}.
    \end{enumerate}
\end{definition}

\begin{definition}[Equivalent representations]
\label{definition:equivalent_representations}
    Let $\Group$ be a group and $\Hil_1, \Hil_2$ be Hilbert spaces.
    Suppose $\xi_1$, $\xi_2$ are representations of $\Group$ on $\Hil_1$ and $\Hil_2$ respectively.
    We shall say that $\xi_1$ and $\xi_2$ are \emph{equivalent}
    if there exists an invertible linear map
    \begin{align*}
        A : \Hil_1 \to \Hil_2
    \end{align*}
    such that for each $g \in \Group$, we have
    \begin{align*}
        \xi_2(g) = A \circ \xi_1(g) \circ A^{-1}.
    \end{align*}
    In that case, $A$ is called an \emph{intertwining operator}.
\end{definition}

\begin{definition}[Strongly continuous representations]
\label{definition:strongly_continuous_representation}
\index{strongly continuous}
    Let $\Group$ be a group and $\Hil$ be a Hilbert space.
    Suppose further that $\xi$ is a representation of $\Group$ on $\Hil$.
    We shall say that $\xi$ is \emph{strongly continuous}
    if for each $x \in \Hil$,
    the map
    \begin{align*}
        \Group \to \Hil : g \mapsto \xi(g) v
    \end{align*}
    is continuous.
\end{definition}

\begin{definition}[Unitary dual]
\label{definition:unitary_dual}
    Let $\Group$ be a locally compact topological group.
    The \emph{unitary dual} of $\Group$, denoted by $\dualGroup\Group$,
    is the set of all equivalence classes of
    \emph{strongly continuous, irreducible, unitary representations} of $\Group$.
\end{definition}

\begin{remark}
    Let $\Group$ be a locally compact topological group.
    We shall often abuse the notation and use $\dualGroup\Group$ to denote a set consisting of
    exactly one representation in each equivalence class of the actual unitary dual.
\end{remark}

\begin{example}[$\dualGroup\VectorSpace$]
    For each $\lambda \in \VectorSpace$,
    define
    \begin{align}
        \xi_\lambda : \VectorSpace \to \UnitaryGroup{1} : x \mapsto \e^{\i \turn \ip{\lambda}{x}}.
        \label{eq:elements_of_dual_of_vector_space}
    \end{align}

    It can be shown that
    \begin{align*}
        \dualGroup\VectorSpace = \{ \xi_\lambda : \lambda \in \VectorSpace \}.
    \end{align*}

    Therefore, the map
    \begin{align}
        \lambda \mapsto \xi_\lambda
        \label{eq:isomorphism_between_vector_space_and_its_dual_group}
    \end{align}
    is a group isomorphism which allows us to give $\dualGroup\VectorSpace$ a vector space structure.
\end{example}

\section{Distributions}

\begin{theorem}[Schwartz Kernel Theorem]
\label{theorem:Schwartz_Kernel_Theorem}
\index{Schwartz Kernel Theorem}
    If $T : \Schwartz \GroupDirect \to \TemperedDistributions \GroupDirect$ is a continuous linear operator,
    then there exists a unique distribution
    $\kappa \in \TemperedDistributions {\GroupDirect \times \GroupDirect}$ such that
    \begin{align*}
        T \phi(x, k) = \int_{\GroupDirect} \kappa(x, k; y, l) \phi(y, l) \dd (y, l).
    \end{align*}
\end{theorem}

\section{Fourier Transform}

\begin{definition}[Fourier coefficient]
    Let $\xi \in \dualGroup\Group$.
    If $f \in \Lebesgue{1}{\Group}$,
    we define the \emph{Fourier coefficient of $f$ at $\xi$}, $\Fourier f(\xi)$, via
    \begin{align*}
        \Fourier f(\xi) \defeq \int_\Group f(g) \adj{\xi(g)} \dd g.
    \end{align*}

    The map
    \begin{align*}
        \Fourier f : \xi \mapsto \Fourier f(\xi)
    \end{align*}
    is called the \emph{Fourier transform of $f$}.
\end{definition}

\begin{definition}
    Let $\xi \in \VectorSpace$.
    If $f \in \Lebesgue{1}\VectorSpace$,
    we define the \emph{Fourier coefficient of $f$ at $\lambda$}, $\Fourier[\VectorSpace] f(\xi)$, via
    \begin{align*}
        \Fourier[\VectorSpace] f(\xi) \defeq \int_\VectorSpace f(x) \e^{-\i \turn \ip{x}\xi} \dd x.
    \end{align*}
\end{definition}

\begin{proposition}
\label{proposition:elementary_properties_of_the_Fourier_transform}
    Let $f, f_1, f_2 \in \Lebesgue{1}{\Group}$ and $\xi \in \dualGroup\Group$.
    The Fourier Transform satisfies the following properties:
    \begin{enumerate}
        \item For each $g \in \Group$, we have
            \begin{align*}
                \Fourier \{f(\dummy g)\} (\xi)
                = \xi(g) \Fourier f(\xi), \quad
                \Fourier \{f(g \dummy)\} (\xi)
                = \Fourier f(\xi) \xi(g).
            \end{align*}
        \item We have
            \begin{align*}
                \Fourier \{\conv{f_1}{f_2}\}(\xi)
                = \Fourier f_2(\xi) \Fourier f_1(\xi).
            \end{align*}
        \item If $f \in \SmoothFunctions\Group$ and $X \in \LieAlgebra$,
            \begin{align*}
                \Fourier \{\LeftDifferentialOperatorFirstOrder{X} f\}(\xi)
                = \xi(X) \Fourier f(\xi), \quad
                \Fourier \{\RightDifferentialOperatorFirstOrder{X} f\}(\xi)
                = \Fourier f(\xi) \xi(X).
            \end{align*}
    \end{enumerate}
\end{proposition}

\begin{proposition}
    There exists a measure $\Plancherel\Group$ on $\dualGroup\Group$ such that the following property holds:
    if $f \in \Schwartz\Group$, we have
    \begin{align*}
        \int_\Group \abs{f}^2 \dd g
        = \int_{\dualGroup\Group} \tr \left( \Fourier f(\xi) \adj{\Fourier f(\xi)} \right) \dd \Plancherel\Group(\xi).
    \end{align*}
\end{proposition}

\begin{example}[Plancherel Measure on $\dualGroup\CompactGroup$]
    If $f \in \SmoothFunctions{\CompactGroup}$,
    then the Peter-Weyl theorem implies:
    \begin{align*}
        \int_\CompactGroup \abs{f}^2 \dd g
        = \sum_{\tau \in \dualGroup\CompactGroup}
            \dimRep\tau
            \tr \left(
                \Fourier[\CompactGroup] f(\tau)
                \adj{\Fourier[\CompactGroup] f(\tau)}
            \right)
    \end{align*}
\end{example}

\begin{proposition}[Inverse Fourier Transform]
    Let $g \in \Group$.
    If $f \in \Schwartz\Group$,
    then we have
    \begin{align*}
        f(g) =
        \int_\dualGroup\Group
            \tr\left(
                \xi(g)
                \Fourier f(\xi)
            \right)
        \dd \Plancherel\Group(\xi).
    \end{align*}
\end{proposition}

\section{Compact Lie groups}

\begin{definition}
    Let $\tau \in \dualGroup\Group$.
    We define the sets
    \begin{align*}
        \HilbertCompactGroupColumn{\tau}{i} \defeq \span \{ \tau_{i, j} : j = 1, \dots, \dimRep\tau \}
        \subset \Lebesgue{2}\CompactGroup, \quad
        \HilbertCompactGroup{\tau} = \bigoplus_{i = 1}^\dimRep\tau \HilbertCompactGroupColumn{\tau}{i}.
    \end{align*}
\end{definition}

\begin{theorem}[Peter-Weyl theorem]
\label{theorem:Peter-Weyl_theorem}
    Let $\Group$ be a compact topological group.
    The set
    \begin{align*}
        \left\{
            \sqrt{\dimRep\tau} \tau_{ij} : \tau \in \dualGroup\Group,\ i, j = 1, \dots, \dimRep\tau
        \right\}
    \end{align*}
    is an orthonormal basis of $\Lebesgue{2}{\Group}$ with respect to the \emph{normalised} Haar measure.

    Therefore, we obtain the following decomposition:
    \begin{align*}
        \Lebesgue{2}{\Group} \defeq
        \bigoplus_{\tau \in \dualGroup\Group} \bigoplus_{j = 1}^{\dimRep \tau} \Hilbert{\tau}{j}.
    \end{align*}

    Since $\Hilbert{\tau}{j}$ is an invariant subspace of $\RightRegularRepresentation$
    and since $\eval{\RightRegularRepresentation}{\Hilbert{\tau}{j}} \sim \tau$,
    we have
    \begin{align*}
        \RightRegularRepresentation \sim
        \bigoplus_{\tau \in \dualGroup\Group} \dimRep\tau \tau.
    \end{align*}
    In particular, the unitary dual can be generated by the right-regular representation.
\end{theorem}

As our representations on the motion group will act on $\Lebesgue{2}\CompactGroup$,
we will define our Fourier Transform on $\CompactGroup$ so that it acts on $\Lebesgue{2}\CompactGroup$ as well.
This will allow useful comparisons later on.

\begin{definition}[Fourier Transform]
    Let $f \in \Lebesgue{1}\CompactGroup$.
    We define the \emph{Fourier transform of $f$}, denoted via $\Fourier[\CompactGroup] f$, via
    \begin{align*}
        \Fourier[\CompactGroup] f \defeq \int_\CompactGroup f(k) \adj{\RightRegularRepresentation(k)} \dd k.
    \end{align*}
\end{definition}

\begin{proposition}[Inverse formula]
    Let $f \in \SmoothFunctions\CompactGroup$.
    If $k \in \CompactGroup$,
    then $f$ can be recovered via the formula
    \begin{align*}
        f(k) = \tr\left(\RightRegularRepresentation(k) \Fourier[\CompactGroup] f \right)
    \end{align*}
\end{proposition}

\begin{lemma}
    For each $\tau \in \dualGroup\CompactGroup$,
    the space $\HilbertCompactGroup{\tau}$ is an eigenspace of the operator $\Laplacian[\CompactGroup]$.
    Denoting by $\JapaneseBracket{\CompactGroup}{\tau}$ the eigenvalue associated with the operator $\BesselPotential[\CompactGroup]{1}$ on the eigenspace $\HilbertCompactGroup\tau$, we obtain
    \begin{align*}
        \eval{\RightRegularRepresentation\left(\BesselPotential[\CompactGroup]{1}\right)}{\HilbertCompactGroup\tau}
        = \JapaneseBracket{\CompactGroup}{\tau} \Id{\HilbertCompactGroup\tau}
    \end{align*}
\end{lemma}

%\section{Difference operators}
%
%\begin{proposition}
%    Let $\Group$ be a compact topological group.
%    There exists a finite subfamily $\{q_1, \dots, q_M\}$ of
%    \begin{align*}
%        \{ \tau_{ij} - \Kronecker{i}{j} : \tau \in \dualGroup\Group,\ i, j = 1, \dots, \dimRep\tau \}
%    \end{align*}
%    which is \emph{strongly admissible}.
%\end{proposition}
%
%\section{Pseudo-differential calculus}
