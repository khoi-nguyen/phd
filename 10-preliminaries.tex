\chapter{Preliminaries}

\section{Lie groups}

To develop a \emph{pseudo-differential calculus} on groups,
a reasonable prerequisite is that the group be equipped with a \emph{differential} structure.
This naturally leads us to the notion of \emph{Lie group}.

\begin{definition}[Lie group]
\label{definition:Lie_group}
\index{Lie group}
    Let $\Group$ be a group.
    We say that $\Group$ is a \emph{Lie group}
    if $\Group$ is a smooth manifold and the map
    \begin{align*}
        \Group \times \Group \to \Group :
        (g_1, g_2) \mapsto g_1^{-1} g_2
    \end{align*}
    is smooth.

    If moreover $\Group$ is (locally) compact as a manifold,
    then we shall say that $\Group$ is a \emph{(locally) compact Lie group}.
\end{definition}

In the sequel,
$\Group$ and $\CompactGroup$ will be used to denote Lie groups,
with the additional assumption that $\CompactGroup$ is compact.
Moreover, we shall denote the \emph{identity element} of $\Group$ by $e$,
or $e_\Group$ should an ambiguity arise.

Closed Lie subgroups consisting of invertible matrices form an important class of Lie groups called \emph{linear Lie groups},
a non-exhaustive list of which is now given.

\begin{example}[Linear Lie groups]
    Let $\VectorSpace$ be a Euclidean space.
    \begin{enumerate}
        \item The set $\GeneralLinear \VectorSpace$ of invertible linear maps,
            equipped with the composition of functions,
            naturally forms a Lie Group called the \emph{general linear group} of $\VectorSpace$.
        \item The set
            \begin{align*}
                \SpecialOrthogonalGroup \VectorSpace
                \defeq
                \{ A \in \Lin{\VectorSpace} : \det A = 1 \}
            \end{align*}
            is a compact Lie subgroup of $\GeneralLinear \VectorSpace$
            called the \emph{special orthogonal group} of $\VectorSpace$.
    \end{enumerate}
\end{example}

\subsection{Lie algebra}

\begin{definition}[Lie algebra]
    A (real) Lie algebra is a (real) vector space $\LieAlgebra$
    equipped with a bilinear map
    \begin{align*}
        \LieBracket \dummy \dummy : \VectorSpace \times \VectorSpace \to \VectorSpace,
    \end{align*}
    called the \emph{Lie bracket} or \emph{commutator},
    such that
    \begin{enumerate}
        \item $\LieBracket X X = 0$ for every $X \in \LieAlgebra$;
        \item for every $X, Y, Z \in \LieAlgebra$, we have the following \emph{Jacobi identity}
            \begin{align*}
                \LieBracket X {\LieBracket Y Z} +
                \LieBracket Y {\LieBracket Z X} +
                \LieBracket Z {\LieBracket X Y}
                = 0.
            \end{align*}
    \end{enumerate}
\end{definition}

Given a Lie group $\Group$,
let us denote by $\VectorFields$ the set of all \emph{smooth vector fields} on $\Group$.

\begin{definition}[Left-invariant vector fields]
    Let $\Group$ be a Lie group.
    We shall say that $X \in \VectorFields$ is \emph{left-invariant}
    if for every $g \in \Group$,
    \begin{align*}
        X \circ L_g = \dd L_g \circ X,
    \end{align*}
    where
    \begin{align}
        L_g : \Group \to \Group : h \mapsto g h
        \label{eq:left_translation_on_Lie_group}
    \end{align}
    is the left translation on $\Group$ by $g$.
    We shall denote the set of left-invariant vector fields by $\LeftInvariantVectorFields$.
\end{definition}

\begin{example}[Left-invariant vector fields]
\label{example:Lie_algebra_of_left-invariant_vector_fields}
    Let $\Group$ be a Lie group.
    Given two left-invariant vector fields $X, Y \in \LeftInvariantVectorFields$,
    we define
    \begin{align*}
        \LieBracket [\LeftInvariantVectorFields] X Y f(x) \defeq X Y f(x) - Y X f(x),
        \quad f \in \SmoothFunctions \Group
    \end{align*}
    and we can show that $\LieBracket [\LeftInvariantVectorFields] X Y$ is also a left-invariant vector field.
    Therefore, $\LeftInvariantVectorFields$ is a Lie algebra.
\end{example}

Let $X, Y \in \TangentSpace \Group e$.
Defining for each $g \in \Group$
\begin{align*}
    \tilde X(g) \defeq \dd L_g(X), \quad \tilde Y(g) = \dd L_g(Y),
\end{align*}
we check that $\tilde X, \tilde Y \in \LeftInvariantVectorFields$.

The Lie algebra structure of $\LeftInvariantVectorFields$ allows us then to define
\begin{align}
    \LieBracket [\TangentSpace \Group e] X Y \defeq \LieBracket [\LeftInvariantVectorFields] {\tilde X} {\tilde Y}(e).
    \label{eq:Lie_bracket_on_the_tangent_space}
\end{align}

\begin{definition}[Lie algebra of Lie group]
\label{definition:Lie_algebra_of_Lie_group}
    Let $\Group$ be a Lie group.
    The \emph{Lie algebra} of $\Group$, denoted by $\Lie(\Group)$, is the tangent space at the identity $\TangentSpace \Group e$
    with the Lie bracket $\LieBracket [\TangentSpace \Group e] \dummy \dummy$
    defined in \eqref{eq:Lie_bracket_on_the_tangent_space}.
\end{definition}

\begin{example}[Lie algebra of $\GeneralLinear n$]
\label{example:Lie_algebra_of_GL_n}
    Suppose $\Group \defeq \GeneralLinear n$.
    Its Lie algebra is
    \begin{align*}
        \LieAlgebra \defeq \SquareMatrices n
    \end{align*}
    equipped with the Lie bracket
    \begin{align*}
        \LieBracket [\LieAlgebra] \dummy \dummy : \LieAlgebra \times \LieAlgebra \to \LieAlgebra : (A, B) \mapsto AB - BA.
    \end{align*}
\end{example}

\begin{proposition}[Lie algebra of Lie subgroups]
    Let $\Group$ be a Lie group with Lie algebra $\LieAlgebra$.
    If $F \subset \Group$ is a \emph{closed} Lie subgroup of $\Group$,
    then its Lie algebra $\Lie(F)$ is a subset of $\LieAlgebra$,
    and the Lie bracket on $\Lie(F)$ is the restriction of that of $\LieAlgebra$.
\end{proposition}

\begin{example}[Lie algebra of $\SpecialOrthogonalGroup n$]
\label{example:Lie_algebra_of_SO_n}
    Suppose $\CompactGroup \defeq \SpecialOrthogonalGroup n$.
    Its Lie algebra is
    \begin{align*}
        \LieAlgebraCompactGroup \defeq \SkewSymmetric n
    \end{align*}
    equipped with the Lie bracket
    \begin{align*}
        \LieBracket [\LieAlgebraCompactGroup] \dummy \dummy : \LieAlgebraCompactGroup \times \LieAlgebraCompactGroup \to \LieAlgebraCompactGroup : (A, B) \mapsto AB - BA.
    \end{align*}
\end{example}

\begin{lemma}[Orthonormal basis on $\LieAlgebraCompactGroup$]
    Let $\CompactGroup$ be a compact Lie group.
    There exists an inner product on $\LieAlgebraCompactGroup = \Lie(\CompactGroup)$ such that
    any orthonormal basis $X_1, \cdots, X_{\dim \CompactGroup} \in \LieAlgebraCompactGroup$ satisfies
    \begin{align*}
        \Laplacian [\CompactGroup] = \sum_{j = 1}^{\dim \CompactGroup} X_j^2,
    \end{align*}
    where $\Laplacian [\CompactGroup]$ is
    the \emph{Laplace-Beltrami operator} associated with the normalized bi-invariant metric.
\end{lemma}

Suppose that $\Group$ is a Lie group
and that $\LieAlgebra$ is its Lie algebra.
If $X_1, \cdots, X_{\dim \Group} \in \LieAlgebra$ is a given basis,
and $\alpha \in \N^{\dim \Group}$
we let
\begin{align*}
    X^\alpha \defeq X_1^{\alpha_1} \cdots X_{\dim \Group}^{\alpha_{\dim \Group}}.
\end{align*}

\subsubsection{Exponential map}

\begin{definition}[Exponential map]
    Let $\Group$ be a Lie group,
    and fix $X \in \LieAlgebra$.
    We define the \emph{exponential} of $X$ via
    \begin{align*}
        \exp_\Group(X) \defeq \gamma_X(1),
    \end{align*}
    where $\gamma_X : I \subset \R \to \Group$ is the unique solution to the following ordinary differential equation
    \begin{align*}
        \begin{cases}
            \D* 1 [\gamma_X] t(t) &= \eval{\dd L_{\gamma_X(t)}}{e}(X)\\
            \gamma_X(0) &= e.
        \end{cases}
        \quad
    \end{align*}
    In the above,
    $L_g$ designates the left-translation by $g$ on $\Group$
    as defined in~\eqref{eq:left_translation_on_Lie_group}.
\end{definition}

\begin{example}[Exponential map on $\GeneralLinear n$]
    Suppose that $\Group \defeq \GeneralLinear n$,
    so that $\LieAlgebra = \SquareMatrices n$ by Example~\eqref{example:Lie_algebra_of_GL_n}.
    The \emph{exponential map} on $\Group$ is
    \begin{align}
        \exp_{\GeneralLinear n} : \SquareMatrices n \to \GeneralLinear n :
        A \to \sum_{k = 1}^{+\infty} \frac 1 {k!} A^k.
        \label{eq:definition_of_exponential_of_matrices}
    \end{align}
\end{example}

The convergence of the series~\eqref{eq:definition_of_exponential_of_matrices} can be shown
by using a \emph{submultiplicative norm} on $\SquareMatrices n$
and the convergence of the exponential power series for real numbers.

\begin{example}[Exponential map on $\SpecialOrthogonalGroup n$]
    Suppose that $\CompactGroup \defeq \SpecialOrthogonalGroup n$.
    By Example~\ref{example:Lie_algebra_of_SO_n},
    its Lie algebra is $\LieAlgebraCompactGroup \defeq \SkewSymmetric n$.
    Given $A \in \CompactGroup$,
    its exponential is given by
    \begin{align*}
        \exp_\CompactGroup(A) = \sum_{k = 0}^{+\infty} \frac 1 {k!} A^k.
    \end{align*}
\end{example}

\begin{proposition}
    Let $\Group$ be a connected simply connected Lie group.
    The exponential map is a diffeomorphism from $\LieAlgebra \defeq \Lie(\Group)$ onto $\Group$.
\end{proposition}

\subsection{Haar measure}

\begin{remark}
    Every Lie group $\Group$ is also topological space.
    As a result, all the topological definitions apply to Lie groups.
\end{remark}

\begin{definition}[Haar measure]
\index{Haar measure}
    Let $\Group$ be a Lie group.
    A positive Radon measure on $\Group$ is called a \emph{Haar measure}
    if it is in addition \emph{left-invariant},
    i.e.\ for each $g \in \Group$ and each Borel set $A \subset G$, we have
    \begin{align*}
        \mu(g A) = \mu(A).
    \end{align*}
\end{definition}

\begin{proposition}[Haar measure]
    If $\Group$ is a locally compact Lie group,
    then there exists a Haar measure $\mu$ on $\Group$.

    Moreover, if $\nu$ is another left-invariant Radon measure on $\Group$,
    then we can find $c \geq 0$ such that $\nu = c \mu$.
\end{proposition}

\begin{definition}[Unimodular group]
\label{definition:unimodular_group}
    Let $\Group$ be a locally compact Lie group.
    If a non-zero Haar measure on $\Group$ is also right-invariant,
    or equivalently if all Haar measure are right-invariant,
    we shall say that $\Group$ is \emph{unimodular}.
\end{definition}

\begin{proposition}
\label{proposition:sufficient_conditions_to_be_unimodular}
    Let $\Group$ be a Lie group.
    \begin{enumerate}
        \item If $\Group$ is compact, then $\Group$ is \emph{unimodular}.
        \item If $\Group$ is abelian and locally compact, then $\Group$ is unimodular.
    \end{enumerate}
\end{proposition}

\begin{definition}[Lebesgue measure]
\label{definition:Lebesgue_measure}
    Let $(\VectorSpace, \ip \dummy \dummy)$ be a Euclidean space.
    If $e_1, \cdots, e_{\dim \VectorSpace} \in \VectorSpace$,
    we call the \emph{Lebesgue measure} the unique Haar measure such that the set
    \begin{align*}
        \{ x \in \VectorSpace : 0 \leq \ip x {e_i} \leq 1 \text{ for each } i = 1, \cdots, \dim \VectorSpace \}
    \end{align*}
    has measure $1$.
\end{definition}

From now on,
everytime we integrate on a Euclidean space,
we do so with respect to the Lebesgue measure.

\section{Representation Theory}

To generalise the Fourier Transform to other Lie groups,
we need an adequate substitute for the complex exponentials.
Many of the properties of the Fourier Transform rely on the fact that the maps
\begin{align*}
    x \in \R^n \mapsto \e^{\i \turn \ip \xi x},
    \quad \xi \in \R^n
\end{align*}
are group homomorphisms $\R^n \to \UnitaryGroup 1$ for each fixed $\xi \in \R^n$.

\begin{definition}[Unitary representations]
\label{definition:unitary_representation}
\index{representations!unitary representations}
    Let $\Group$ be a group and $\Hil$ be a Hilbert space.
    A map
    \begin{align*}
        \xi : \Group \mapsto \Hom(\Hil)
    \end{align*}
    is called a \emph{unitary representation (on $\Hil$)} if
    \begin{enumerate}
        \item for each $g \in \Group$, the map $\xi(g)$ is unitary:
            \begin{align*}
                {\xi(g)}^{-1} = \adj{\xi(g)};
            \end{align*}
        \item if $g, h \in \Group$, then we have $\xi(g h) = \xi(g) \xi(h)$.
    \end{enumerate}

    The \emph{dimension} of $\xi$ is that of $\Hil$.
    If $\Hil$ is finite-dimensional,
    we let $\dimRep{\xi} \defeq \dim{\Hil}$ denote the dimension of $\xi$.
\end{definition}

\begin{example}[Right-regular representation]
    Let $\Group$ be a unimodular topological group.
    The \emph{right-regular representation} is the representation
    \begin{align*}
        \RightRegularRepresentation : \Group \to \Hom(\Lebesgue{2}{\Group})
    \end{align*}
    defined via
    \begin{align*}
        \RightRegularRepresentation(h) f(g) = f(g h)
    \end{align*}
    for every $g, h \in \Group$.
\end{example}

\begin{definition}[Invariant subspaces]
\label{definition:invariant_subspaces}
    Let $\Group$ be a group and $\xi$ be a unitary representation of $\Group$ on a Hilbert space $\Hil$.
    We shall say that a vector subspace $W \subset \Hil$ is \emph{invariant} under $\xi$
    if for each $g \in \Group$, we have $\xi(g) W \subset W$.
\end{definition}

\begin{definition}[Irreducibility]
\label{definition:irreducible_representations}
    Let $\Group$ be a group and $\xi$ be a unitary representation of $\Group$ on a Hilbert space $\Hil$.
    \begin{enumerate}
        \item If the only invariant subspaces of $\xi$ are $\{0\}$ and $\Hil$,
            then $\xi$ is said to be \emph{irreducible}.
        \item Otherwise, if there exists a non-trivial invariant subspace,
            then $\xi$ is \emph{reducible}.
    \end{enumerate}
\end{definition}

\begin{definition}[Equivalent representations]
\label{definition:equivalent_representations}
    Let $\Group$ be a group and $\Hil_1, \Hil_2$ be Hilbert spaces.
    Suppose $\xi_1$, $\xi_2$ are representations of $\Group$ on $\Hil_1$ and $\Hil_2$ respectively.
    We shall say that $\xi_1$ and $\xi_2$ are \emph{equivalent}
    if there exists an invertible linear map
    \begin{align*}
        A : \Hil_1 \to \Hil_2
    \end{align*}
    such that for each $g \in \Group$, we have
    \begin{align*}
        \xi_2(g) = A \circ \xi_1(g) \circ A^{-1}.
    \end{align*}
    In that case, $A$ is called an \emph{intertwining operator}.
\end{definition}

\begin{definition}[Strongly continuous representations]
\label{definition:strongly_continuous_representation}
\index{strongly continuous}
    Suppose that $\Group$ is a group and $\Hil$ be a Hilbert space.
    Suppose further that $\xi$ is a representation of $\Group$ on $\Hil$.
    We shall say that $\xi$ is \emph{strongly continuous}
    if for each $x \in \Hil$,
    the map
    \begin{align*}
        \Group \to \Hil : g \mapsto \xi(g) v
    \end{align*}
    is continuous.
\end{definition}

\begin{definition}[Unitary dual]
\label{definition:unitary_dual}
    Let $\Group$ be a locally compact topological group.
    The \emph{unitary dual} of $\Group$, denoted by $\dualGroup\Group$,
    is the set of all equivalence classes of
    \emph{strongly continuous, irreducible, unitary representations} of $\Group$.
\end{definition}

\begin{remark}
    Let $\Group$ be a locally compact topological group.
    We shall often abuse the notation and use $\dualGroup\Group$ to denote a set consisting of
    exactly one representation in each equivalence class of the actual unitary dual.
\end{remark}

\begin{example}[$\dualGroup\VectorSpace$]
    For each $\lambda \in \VectorSpace$,
    define
    \begin{align}
        \xi_\lambda : \VectorSpace \to \UnitaryGroup{1} : x \mapsto \e^{\i \turn \ip{\lambda}{x}}.
        \label{eq:elements_of_dual_of_vector_space}
    \end{align}

    It can be shown that
    \begin{align*}
        \dualGroup\VectorSpace = \{ \xi_\lambda : \lambda \in \VectorSpace \}.
    \end{align*}

    Therefore, the map
    \begin{align}
        \lambda \mapsto \xi_\lambda
        \label{eq:isomorphism_between_vector_space_and_its_dual_group}
    \end{align}
    is a group isomorphism which allows us to give $\dualGroup\VectorSpace$ a vector space structure.
\end{example}

\section{Distributions}

For this section,
we assume that $\VectorSpace$ is a vector space
and $\CompactGroup$ is a compact Lie group.
Also, when we integrate on $\VectorSpace$, $\CompactGroup$ or $\GroupDirect$,
we always mean with respect to a \emph{Haar measure}.

We shall denote by $\LieAlgebraCompactGroup$ the Lie algebra of $\CompactGroup$.
We fix a basis $Y_1$, \cdots, $Y_{\dim \CompactGroup}$ of $\LieAlgebraCompactGroup$ and write
\begin{align*}
    Y^\beta \defeq Y_1^{\beta_1} \cdots Y_{\dim \CompactGroup}^{\beta_{\dim \CompactGroup}},
    \quad \beta \in \N^{\dim \CompactGroup}.
\end{align*}

\begin{definition}[Schwartz space]
    We shall say that $f \in \SmoothFunctions \GroupDirect$ is \emph{rapidly decaying}
    if for every $N \in \N$,
    \begin{align*}
        \seminorm [\Schwartz \GroupDirect] N {f}
        \defeq
        \sup_{\abs \alpha, \abs \beta \leq N}
        \abs{%
            {(1 + \abs x)}^N \D{\abs \alpha}{x^\alpha} Y^{\beta}_k f(x, k)
        } < \infty.
    \end{align*}
    The set of all rapidly decaying functions will denoted by $\Schwartz \GroupDirect$.
    The family $\{\seminorm [\Schwartz \GroupDirect] N \dummy : N \in \N\}$ gives $\Schwartz \GroupDirect$
    the structure of a Fr\'echet space.
\end{definition}

\begin{definition}[Tempered distributions]
    We shall say that
    \begin{align*}
        \kappa : \Schwartz \GroupDirect \to \C
    \end{align*}
    is a \emph{tempered distribution} if it is linear and continuous.
    The set of all tempered distributions will be denoted by $\TemperedDistributions \GroupDirect$.
\end{definition}

As usual, if $\kappa \in \TemperedDistributions \GroupDirect$ and $\phi \in \Schwartz \GroupDirect$
\begin{align*}
    \dualBracket [\GroupDirect] \kappa \phi \defeq \kappa(\phi)
\end{align*}
We shall also write
\begin{align*}
    \int_{\GroupDirect} \kappa(x, k) \phi(x, k) d(x, k) \defeq \dualBracket [\GroupDirect] \kappa \phi,
\end{align*}
motivated by the inclusion $\Lebesgue 1 \GroupDirect \subset \TemperedDistributions \GroupDirect$,
and shall say that the above integral is interpreted \emph{in the sense of distributions}.

\begin{theorem}[Schwartz Kernel Theorem]
\label{theorem:Schwartz_Kernel_Theorem}
\index{Schwartz Kernel Theorem}
    If the map
    \begin{align*}
        T : \Schwartz \GroupDirect \to \TemperedDistributions \GroupDirect
    \end{align*}
    is a continuous linear operator,
    then there exists a unique distribution
    $\kappa \in \TemperedDistributions {\GroupDirect \times \GroupDirect}$ such that
    \begin{align*}
        T \phi(x, k) = \int_{\GroupDirect} \kappa(x, k; y, l) \phi(y, l) \dd (y, l).
    \end{align*}
\end{theorem}

\section{Convolutions}

As the notion of convolution depends on the group structure,
we introduce it in the abstract, general case of a locally compact Lie group $\Group$.
Naturally, every integration on $\Group$ is performed with respect to a Haar measure.

\begin{definition}[Convolution]
    Let $f_1, f_2 \in \Lebesgue 1 \Group$.
    We define the convolution of $f_1$ and $f_2$,
    denoted by $\conv {f_1} {f_2}$, via
    \begin{align}
        \conv {f_1} {f_2} (g)
        &\defeq \int_\Group f_1(h) f_2(h^{-1} g) \dd h
        \label{eq:definition_of_convolution}\\
        &= \int_\Group f_1(g h^{-1}) f_2(h) \dd h.
        \label{eq:alternative_definition_of_convolution}
    \end{align}
\end{definition}

\begin{proposition}
\label{proposition:integrability_of_the_convolution_of_two_L1_functions}
    Let $f_1, f_2 \in \Lebesgue 1 \Group$.
    The convolution of $f_1$ and $f_2$, $\conv {f_1} {f_2}$,
    belongs to $\Lebesgue 1 \Group$.
    Moreover, we have
    \begin{align*}
        \norm [\Lebesgue 1 \Group] {\conv {f_1} {f_2}}
        \leq
        \norm [\Lebesgue 1 \Group] {f_1}
        \norm [\Lebesgue 1 \Group] {f_2}.
    \end{align*}
\end{proposition}
\begin{proof}
    By the positivity of the integral,
    we get
    \begin{align*}
        \int_\Group \abs {\int_\Group f_1(h) f_2(h^{-1} g) \dd h} \dd g
        \leq \int_\Group \int_\Group \abs{f_1(h)} \abs{f_2(h^{-1} g)} \dd h \dd g
    \end{align*}

    By the Fubini-Tonelli Theorem,
    we can change the order of integration above
    and therefore substitute $g$ for $h g$,
    which yields
    \begin{align*}
        \int_\Group \abs {\int_\Group f_1(h) f_2(h^{-1} g) \dd h} \dd g
        &\leq \int_\Group \int_\Group \abs{f_1(h)} \abs{f_2(g)} \dd g \dd h\\
        &\leq \norm [\Lebesgue 1 \Group] {f_1} \norm [\Lebesgue 1 \Group] {f_2},
    \end{align*}
    concluding the prood.
\end{proof}

\begin{lemma}
    Let $f_1, f_2 \in \Lebesgue 2 \Group$.
    The map
    \begin{align*}
        f \defeq \conv {f_1} {f_2}
    \end{align*}
    is continuous.
\end{lemma}
\begin{proof}
    Let $g \in \Group$.
    For each $h \in \Group$,
    we have
    \begin{align*}
        \abs {f(h) - f(g)} &\leq \abs {\int_\Group f_1(h l^{-1}) f_2(l) \dd l}\\
        &\leq \norm [\Lebesgue 2 \Group] {f_1} \norm [\Lebesgue 2 \Group] {f_2(h \dummy) - f_2(g \dummy)}.
    \end{align*}

    By %TODO Add reference,
    \begin{align*}
        \lim_{h \to g} \norm [\Lebesgue 2 \Group] {f_2(h \dummy) - f_2(g \dummy)} = 0,
    \end{align*}
    showing that $f$ is continuous at $g$.
    We conclude the proof by observing that $g$ is arbitrary.
\end{proof}

\subsection{Convolutions of distributions}

Suppose that $\kappa, \phi \in \Schwartz \Group$.
It follows that \eqref{eq:alternative_definition_of_convolution} can be rewritten
\begin{align*}
    \conv \phi \kappa &= \int_\Group \phi(g h^{-1}) \kappa(h) \dd h\\
    &= \int_\Group \kappa(h) \RightRegularRepresentation(g^{-1}) \InverseFunctionArgument \phi(h) \dd g,
\end{align*}
where $\InverseFunctionArgument \phi(h) \defeq \phi(h^{-1})$ for each $h \in \Group$.
Similarly, we show that \eqref{eq:definition_of_convolution} becomes
\begin{align*}
    \conv \kappa \phi &= \int_\Group \kappa(h) \phi({(g^{-1} h)}^{-1}) \dd h\\
    &= \int_\Group \kappa(h) \LeftRegularRepresentation(g^{-1}) \InverseFunctionArgument \phi(h) \dd g.
\end{align*}

Both of the above can be generalised for $\kappa \in \TemperedDistributions \Group$
provided that we interpret the above in the sense of distributions,
leading to the following definition.

% TODO: define Schwartz space on Lie group
\begin{definition}[Convolution]
    Let $\kappa \in \TemperedDistributions \Group$ and $\phi \in \Schwartz \Group$.
    For every $g \in \Group$,
    we let
    \begin{align*}
        \conv \phi \kappa(g) &\defeq \dualBracket [\Group] \kappa {\RightRegularRepresentation(g^{-1}) \InverseFunctionArgument \phi}\\
        \conv \kappa \phi(g) &\defeq \dualBracket [\Group] \kappa {\LeftRegularRepresentation(g) \InverseFunctionArgument \phi},
    \end{align*}
    where $\InverseFunctionArgument \phi(h) \defeq \phi(h^{-1})$ for each $h \in \Group$.
\end{definition}

\section{Fourier Transform}

\begin{definition}[Fourier coefficient]
\label{definition:Fourier_coefficient}
    Let $\xi \in \dualGroup\Group$.
    If $f \in \Lebesgue{1}{\Group}$,
    we define the \emph{Fourier coefficient of $f$ at $\xi$}, $\Fourier f(\xi)$, via
    \begin{align}
        \Fourier f(\xi) \defeq \int_\Group f(g) \adj{\xi(g)} \dd g.
        \label{eq:definition_of_Fourier_coefficient}
    \end{align}

    The map
    \begin{align*}
        \Fourier f : \xi \mapsto \Fourier f(\xi)
    \end{align*}
    is called the \emph{Fourier transform of $f$}.
\end{definition}

\begin{definition}
    Let $\xi \in \VectorSpace$.
    If $f \in \Lebesgue{1}\VectorSpace$,
    we define the \emph{Fourier coefficient of $f$ at $\lambda$}, $\Fourier[\VectorSpace] f(\xi)$, via
    \begin{align*}
        \Fourier[\VectorSpace] f(\xi) \defeq \int_\VectorSpace f(x) \e^{-\i \turn \ip{x}\xi} \dd x.
    \end{align*}
\end{definition}

\begin{proposition}
\label{proposition:elementary_properties_of_the_Fourier_transform}
    Let $f, f_1, f_2 \in \Lebesgue{1}{\Group}$ and $\xi \in \dualGroup\Group$.
    The Fourier Transform satisfies the following properties:
    \begin{enumerate}
        \item For each $g \in \Group$, we have
            \begin{align*}
                \Fourier \{f(\dummy g)\} (\xi)
                = \xi(g) \Fourier f(\xi), \quad
                \Fourier \{f(g \dummy)\} (\xi)
                = \Fourier f(\xi) \xi(g).
            \end{align*}
        \item We have
            \begin{align*}
                \Fourier \{\conv{f_1}{f_2}\}(\xi)
                = \Fourier f_2(\xi) \Fourier f_1(\xi).
            \end{align*}
        \item If $f \in \Schwartz\Group \cap \Lebesgue 1 \Group$ and $X \in \LieAlgebra$,
            \begin{align*}
                \Fourier \{\LeftDifferentialOperatorFirstOrder{X} f\}(\xi)
                = \xi(X) \Fourier f(\xi), \quad
                \Fourier \{\RightDifferentialOperatorFirstOrder{X} f\}(\xi)
                = \Fourier f(\xi) \xi(X).
            \end{align*}
    \end{enumerate}
\end{proposition}
\begin{proof}
    Let us prove each of those claims separately.

    \begin{enumerate}
        \item
            Let $g \in \Group$.
            To obtain the first identity,
            let us observe that
            \begin{align*}
                \Fourier \{f(\dummy g)\}(\xi)
                &= \int_\Group f(h g) \adj{\xi(h)} \dd h\\
                &= \int_\Group f(h) \adj{\xi(h g^{-1})} \dd h,
            \end{align*}
            where we used the invariance of the Haar measure under the group action.
            Continuing the above calculation,
            we obtain
            \begin{align*}
                \Fourier \{f(\dummy g)\}(\xi)
                = \int_\Group f(h) \xi(g) \adj{\xi(h)} \dd h
                = \xi(g) \Fourier f(\xi),
            \end{align*}
            which is what we wanted to show.

            Let us proceed similarly for the left-translation.
            Let us observe that this time
            \begin{align*}
                \Fourier \{f(g \dummy)\}(\xi)
                &= \int_\Group f(g h) \adj{\xi(h)} \dd h\\
                &= \int_\Group f(h) \adj{\xi(g^{-1} h)} \dd h,
            \end{align*}
            where we again used the invariance of the Haar measure under the group action.
            Continuing the above calculation,
            we now obtain
            \begin{align*}
                \Fourier \{f(g \dummy)\}(\xi)
                = \int_\Group f(h) \adj{\xi(h)} \xi(g) \dd h
                = \Fourier f(\xi) \xi(g),
            \end{align*}
            which is what we wanted to show.
        \item
            By Proposition~\ref{proposition:integrability_of_the_convolution_of_two_L1_functions},
            $\conv {f_1} {f_2} \in \Lebesgue 1 \Group$.
            Therefore,
            we can compute its Fourier Transform,
            yielding
            \begin{align*}
                \Fourier \{\conv {f_1} {f_2}\}(\xi)
                &= \int_\Group \int_\Group f_1(h) f_2(h^{-1} g) \adj{\xi(g)} \dd g
            \end{align*}

            Using $\adj{\xi(g)} = \adj{(\xi(h) \xi(h^{-1} g))} = \adj{\xi(h^{-1} g)} \adj{\xi(h)}$ in the above,
            we obtain
            \begin{align*}
                \Fourier \{\conv {f_1} {f_2}\}(\xi)
                &= \int_\Group \int_\Group f_2(h^{-1} g) \adj{\xi(h^{-1} g)} \dd g f_1(h) \adj{\xi(h)} \dd h\\
                &= \Fourier f_2(\xi) \Fourier f_1(\xi).
            \end{align*}
        \item
            Let $X \in \LieAlgebra$.
            Using integration by parts,
            we get
            \begin{align*}
                \Fourier \{\LeftDifferentialOperatorFirstOrder X f\}(\xi)
                = \int_\Group \LeftDifferentialOperatorFirstOrder X f(g) \adj{\xi(g)} \dd g
                = -\int_\Group f(g) \LeftDifferentialOperatorFirstOrder X \adj{\xi(g)} \dd g.
            \end{align*}

            Now,
            we know that
            \begin{align*}
                \LeftDifferentialOperatorFirstOrder X \adj{\xi(g)}
                &= \eval {\D* 1 t} {t = 0} \adj{\xi(g \exp_\Group(t X))}
                = \eval {\D* 1 t} {t = 0} \adj{\xi(\exp_\Group(t X))} \adj{\xi(g)}\\
                &= - \eval {\D* 1 t} {t = 0} \xi(\exp_\Group(t X)) \adj{\xi(g)},
            \end{align*}
            which implies that
            \begin{align*}
                \LeftDifferentialOperatorFirstOrder X \adj{\xi(g)}
                &= - \xi(X) \adj{\xi(g)}.
            \end{align*}

            Substituing the above into the Fourier transform above,
            we obtain
            \begin{align*}
                \Fourier \{\LeftDifferentialOperatorFirstOrder X f\}(\xi)
                = \int_\Group f(g) \xi(X) \adj{\xi(g)} \dd g
                = \xi(X) \Fourier f(\xi).
            \end{align*}.

            Let us now turn our attention to the case of the right-invariant vector fields.
            In this case,
            \begin{align*}
                \RightDifferentialOperatorFirstOrder X \adj{\xi(g)}
                &= \eval {\D* 1 t} {t = 0} \adj{\xi(\exp_\Group(t X) g)}
                = \eval {\D* 1 t} {t = 0} \adj{\xi(g)} \adj{\xi(\exp_\Group(t X))}\\
                &= - \adj{\xi(g)} \xi(X)
            \end{align*}
            which allows us to conclude that
            \begin{align*}
                \Fourier \{\RightDifferentialOperatorFirstOrder X f\}(\xi)
                &= \int_\Group \RightDifferentialOperatorFirstOrder X f(g) \adj{\xi(g)} \dd g
                = -\int_\Group f(g) \RightDifferentialOperatorFirstOrder X \adj{\xi(g)} \dd g\\
                &= \int_\Group f(g) \adj{\xi(g)} \xi(X) \dd g
                = \Fourier f(\xi) \xi(X).
            \end{align*}
    \end{enumerate}
\end{proof}

\begin{proposition}
    There exists a measure $\Plancherel\Group$ on $\dualGroup\Group$ such that the following property holds:
    if $f \in \Schwartz\Group$, we have
    \begin{align*}
        \int_\Group \abs{f}^2 \dd g
        = \int_{\dualGroup\Group} \tr \left( \Fourier f(\xi) \adj{\Fourier f(\xi)} \right) \dd \Plancherel\Group(\xi).
    \end{align*}
\end{proposition}

\begin{example}[Plancherel Measure on $\dualGroup\CompactGroup$]
    If $f \in \SmoothFunctions{\CompactGroup}$,
    then the Peter-Weyl theorem implies:
    \begin{align*}
        \int_\CompactGroup \abs{f}^2 \dd g
        = \sum_{\tau \in \dualGroup\CompactGroup}
            \dimRep\tau
            \tr \left(
                \Fourier[\CompactGroup] f(\tau)
                \adj{\Fourier[\CompactGroup] f(\tau)}
            \right)
    \end{align*}
\end{example}

\begin{proposition}[Inverse Fourier Transform]
    Let $g \in \Group$.
    If $f \in \Schwartz\Group$,
    then we have
    \begin{align*}
        f(g) =
        \int_\dualGroup\Group
            \tr\left(
                \xi(g)
                \Fourier f(\xi)
            \right)
        \dd \Plancherel\Group(\xi).
    \end{align*}
\end{proposition}

\section{The direct product case}

As our representations on the motion group will act on $\Lebesgue{2}\CompactGroup$,
we will define our Fourier Transform on $\CompactGroup$ so that it acts on $\Lebesgue{2}\CompactGroup$ as well.
This will allow useful comparisons later on.

\begin{definition}[Representations on $\GroupDirect$]
    Let $\lambda \in \VectorSpace$.
    We define the representation associated to $\lambda$,
    \begin{align*}
        \Rep [\GroupDirect] \lambda : \Group \to \Lin{\Lebesgue 2 \CompactGroup},
    \end{align*}
    via
    \begin{align*}
        \Rep [\GroupDirect] \lambda (x, k) \defeq \e^{\i \turn \ip \lambda x} \RightRegularRepresentation (k),
    \end{align*}
    where $(x, k) \in \GroupDirect$.
\end{definition}

\begin{definition}[Fourier Transform]
    Let $f \in \Lebesgue 1 \GroupDirect$ and $\lambda \in \VectorSpace$.
    We define the \emph{Fourier coefficient of $f$} at $\lambda$,
    denoted via $\Fourier[\GroupDirect] f(\lambda)$, via
    \begin{align*}
        \Fourier [\GroupDirect] f(\lambda) \defeq \int_\CompactGroup f(x, k) \e^{\i \turn \ip \lambda x} \adj{\RightRegularRepresentation(k)} \dd (x, k).
    \end{align*}
    The map
    \begin{align*}
        \lambda \in \VectorSpace \mapsto \Fourier [\GroupDirect] f(\lambda)
    \end{align*}
    is called the \emph{Fourier Transform} of $f$ on $\GroupDirect$.
\end{definition}

\begin{proposition}[Plancherel formula]
    Let $f_1, f_2 \in \Lebesgue 1 \GroupDirect$.
    If in addition $f_1, f_2 \in \Lebesgue 2 \GroupDirect$,
    then the following identity holds
    \begin{align*}
        \ip [\Lebesgue 2 \GroupDirect] {f_1} {f_2}
        = \int_\VectorSpace \tr\left(\Fourier [\GroupDirect] f_1(\lambda) \adj{\Fourier [\GroupDirect] f_2(\lambda)}\right) \dd \lambda.
    \end{align*}
\end{proposition}

\begin{proposition}[Inverse formula]
    Let $f \in \SmoothFunctions \GroupDirect$.
    If $(x, k) \in \GroupDirect$,
    then $f$ can be recovered via the formula
    \begin{align*}
        f(x, k) = \int_\VectorSpace \tr\left(\e^{\i \turn \ip x \lambda} \RightRegularRepresentation(k) \Fourier[\GroupDirect] f(\lambda) \right) \dd \lambda.
    \end{align*}
\end{proposition}

\begin{lemma}
    For each $\tau \in \dualGroup\CompactGroup$,
    the space $\HilbertCompactGroup{\tau}$ is an eigenspace of the operator $\Laplacian[\CompactGroup]$.
    Denoting by $\JapaneseBracket{\CompactGroup}{\tau}$ the eigenvalue associated with the operator $\BesselPotential[\CompactGroup]{1}$ on the eigenspace $\HilbertCompactGroup\tau$, we obtain
    \begin{align*}
        \eval{\RightRegularRepresentation\left(\BesselPotential[\CompactGroup]{1}\right)}{\HilbertCompactGroup\tau}
        = \JapaneseBracket{\CompactGroup}{\tau} \Id{\HilbertCompactGroup\tau}
    \end{align*}
\end{lemma}

\begin{lemma}[Infinitesimal representations of the Laplacian]
    Let $\lambda \in \VectorSpace$.
    The infinitesimal representation of the Laplacian $\Laplacian [\GroupDirect] = \Laplacian [\VectorSpace] + \Laplacian [\CompactGroup]$
    is given by
    \begin{align*}
        \Rep [\GroupDirect] \lambda (\Laplacian [\GroupDirect])
        = - (\turn)^2 \norm \lambda^2 \Id{\Lebesgue 2 \CompactGroup} + \Laplacian [\CompactGroup]
    \end{align*}
\end{lemma}

\begin{proposition}[Sobolev inequality]
    Let $s \in \R$.
    We have
    \begin{align*}
        \norm [\Lebesgue 2 \GroupDirect] {\BesselPotentialKernel [\GroupDirect] {s}}^2
        \defeq \int_\VectorSpace \tr(\Rep [\GroupDirect] \lambda (\BesselPotential [\GroupDirect] {2s}) \dd \lambda < \infty
    \end{align*}
    if and only if $s > (\dim \VectorSpace + \dim \CompactGroup) / 2$.
\end{proposition}

%\section{Difference operators}
%
%\begin{proposition}
%    Let $\Group$ be a compact topological group.
%    There exists a finite subfamily $\{q_1, \cdots, q_M\}$ of
%    \begin{align*}
%        \{ \tau_{ij} - \Kronecker{i}{j} : \tau \in \dualGroup\Group,\ i, j = 1, \cdots, \dimRep\tau \}
%    \end{align*}
%    which is \emph{strongly admissible}.
%\end{proposition}
%
%\section{Pseudo-differential calculus}
