\chapter{Preliminaries}

\section{Lie groups}

\begin{definition}[Lie group]
\label{definition:Lie_group}
\index{Lie group}
    Let $\Group$ be a group.
    We say that $\Group$ is a \emph{Lie group}
    if $\Group$ is a smooth manifold and the map
    \begin{align*}
        \Group \times \Group \to \Group :
        (g_1, g_2) \mapsto g_1^{-1} g_2
    \end{align*}
    is a smooth.

    If moreover $\Group$ is (locally) compact as a manifold,
    then we shall say that $\Group$ is a \emph{(locally) compact Lie group}.
\end{definition}

\begin{definition}[Haar measure]
    Let $\Group$ be a locally compact topological group.
    A positive Radon measure on $\Group$ is called a \emph{Haar measure}
    if it is in addition \emph{left-invariant},
    i.e.\ for each $g \in \Group$ and each Borel set $A \subset G$, we have
    \begin{align*}
        \mu(g A) = \mu(A).
    \end{align*}
\end{definition}

\begin{proposition}[Haar measure]
    If $\Group$ is a locally compact topological group,
    then there exists a Haar measure $\mu$ on $\Group$.

    Moreover, if $\nu$ is another left-invariant Radon measure on $\Group$,
    then we can find $c \geq 0$ such that $\nu = c \mu$.
\end{proposition}

\begin{definition}[Unimodular group]
\label{definition:unimodular_group}
    Let $\Group$ be a locally compact group.
    If a Haar measure on $\Group$ is also right-invariant,
    we shall say that $\Group$ is \emph{unimodular}.
\end{definition}

\begin{proposition}
\label{proposition:sufficient_conditions_to_be_unimodular}
    Let $\Group$ be a topological group.
    \begin{enumerate}
        \item If $\Group$ is compact, then $\Group$ is \emph{unimodular}.
        \item If $\Group$ is abelian and locally compact, then $\Group$ is unimodular.
    \end{enumerate}
\end{proposition}

\section{Representation Theory}

\begin{definition}
\label{definition:unitary_representation}
\index{representations!unitary representations}
    Let $\Group$ be a group and $\Hil$ be a Hilbert space.
    A map
    \begin{align*}
        \xi : \Group \mapsto \Hom(\Hil)
    \end{align*}
    is called a \emph{unitary representation (on $\Hil$)} if
    \begin{enumerate}
        \item for each $g \in \Group$, the map $\xi(g)$ is unitary:
            \begin{align*}
                {\xi(g)}^{-1} = \adj{\xi(g)};
            \end{align*}
        \item if $g, h \in \Group$, then we have $\xi(g h) = \xi(g) \xi(h)$.
    \end{enumerate}

    The \emph{dimension} of $\xi$ is that of $\Hil$.
    If $\Hil$ is finite-dimensional,
    we let $\dimRep{\xi} \defeq \dim{\Hil}$ denote the dimension of $\xi$.
\end{definition}

\begin{definition}
    Let $\Group$ be a group and $\xi$ be a unitary representation of $\Group$ on a Hilbert space $\Hil$.
    We shall say that a vector subspace $W \subset \Hil$ is \emph{invariant} under $\xi$
    if for each $g \in \Group$, we have $\xi(g) W \subset W$.
\end{definition}

\begin{definition}
\label{definition:irreducible_representations}
    Let $\Group$ be a group and $\xi$ be a unitary representation of $\Group$ on a Hilbert space $\Hil$.
    \begin{enumerate}
        \item If the only invariant subspaces of $\xi$ are $\{0\}$ and $\Hil$,
            then $\xi$ is said to be \emph{irreducible}.
        \item Otherwise, if there exists a non-trivial invariant subspace,
            then $\xi$ is \emph{reducible}.
    \end{enumerate}
\end{definition}

\section{Fourier Transform}

\section{Difference operators}

\section{Pseudo-differential calculus}
