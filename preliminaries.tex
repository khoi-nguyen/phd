\chapter{Preliminaries}

\section{Lie groups}

\begin{definition}[Lie group]
\label{definition:Lie_group}
\index{Lie group}
    Let $\Group$ be a group.
    We say that $\Group$ is a \emph{Lie group}
    if $\Group$ is a smooth manifold and the map
    \begin{align*}
        \Group \times \Group \to \Group :
        (g_1, g_2) \mapsto g_1^{-1} g_2
    \end{align*}
    is a smooth.

    If moreover $\Group$ is (locally) compact as a manifold,
    then we shall say that $\Group$ is a \emph{(locally) compact Lie group}.
\end{definition}

\begin{definition}[Haar measure]
    Let $\Group$ be a locally compact topological group.
    A positive Radon measure on $\Group$ is called a \emph{Haar measure}
    if it is in addition \emph{left-invariant},
    i.e.\ for each $g \in \Group$ and each Borel set $A \subset G$, we have
    \begin{align*}
        \mu(g A) = \mu(A).
    \end{align*}
\end{definition}

\begin{proposition}[Haar measure]
    If $\Group$ is a locally compact topological group,
    then there exists a Haar measure $\mu$ on $\Group$.

    Moreover, if $\nu$ is another left-invariant Radon measure on $\Group$,
    then we can find $c \geq 0$ such that $\nu = c \mu$.
\end{proposition}

\begin{definition}[Unimodular group]
\label{definition:unimodular_group}
    Let $\Group$ be a locally compact group.
    If a Haar measure on $\Group$ is also right-invariant,
    we shall say that $\Group$ is \emph{unimodular}.
\end{definition}

\begin{proposition}
\label{proposition:sufficient_conditions_to_be_unimodular}
    Let $\Group$ be a topological group.
    \begin{enumerate}
        \item If $\Group$ is compact, then $\Group$ is \emph{unimodular}.
        \item If $\Group$ is abelian and locally compact, then $\Group$ is unimodular.
    \end{enumerate}
\end{proposition}

\section{Representation Theory}

\section{Fourier Transform}

\section{Difference operators}

\section{Pseudo-differential calculus}
