\RequirePackage{silence}
\WarningFilter{scrbook}{Usage of package `titlesec'}
\WarningFilter{scrbook}{Activating an ugly workaround}
\WarningFilter{titlesec}{Non standard sectioning command detected}
\WarningFilter{biblatex}{Data encoding is 'utf8'}
% Warnings from arsclassica
\WarningFilter{hyperref}{Option `hyperfootnotes' has already been used}
\WarningFilter{hyperref}{Option `pdfpagelabels' has already been used}
\WarningFilter{hyperref}{Option `pdfpagelabels' has already been used}
\WarningFilter{scrlayer-scrpage}{Very small head height detected!}
\WarningFilter{scrlayer-scrpage}{\headheight to low}
\documentclass[dottedtoc, headinclude, footinclude=true]{scrbook}
\usepackage[pdfspacing]{classicthesis}
\usepackage{arsclassica}
\usepackage[backend=biber,style=alphabetic]{biblatex}

\usepackage{setspace}
\setstretch{1.25}

% Index
%\usepackage{makeidx}

\usepackage[utf8]{inputenc}

\usepackage{amsmath, amsthm, amssymb, amsfonts}
\usepackage{mathtools}

\usepackage{xparse}

% Quotes
\def\signed #1{{\leavevmode\unskip\nobreak\hfil\penalty50\hskip2em
  \hbox{}\nobreak\hfil(#1)%
  \parfillskip=0pt \finalhyphendemerits=0 \endgraf}}

\newsavebox\mybox
\newenvironment{aquote}[1]
  {\savebox\mybox{#1}\begin{quote}}
  {\signed{\usebox\mybox}\end{quote}}

% Environments
\newtheorem{theorem}{Theorem}[section]
\newtheorem{application}[theorem]{Application}
\newtheorem{corollary}[theorem]{Corollary}
\newtheorem{definition}[theorem]{Definition}
\newtheorem{example}[theorem]{Example}
\newtheorem{lemma}[theorem]{Lemma}
\newtheorem{proposition}[theorem]{Proposition}
\newtheorem{remark}[theorem]{Remark}

\newcounter{claimcounter}
\numberwithin{claimcounter}{theorem}
\newenvironment{claim}{\stepcounter{claimcounter}{\emph{Claim \theclaimcounter:}}}{}

% Notation
\DeclarePairedDelimiter{\Ceiling}{\lceil}{\rceil}
\DeclarePairedDelimiter{\Floor}{\lfloor}{\rfloor}
\newcommand{\abs}[1]{\left|{#1}\right|}
\newcommand{\adj}[1]{{#1}^\star}
\newcommand{\conj}[1]{\overline{#1}}
\newcommand{\conv}[2]{#1 \star #2}
\newcommand{\dualGroup}[1]{{\widehat{#1}}}
\newcommand{\defeq}{\mathrel{\overset{\makebox[0pt]{\mbox{\normalfont\tiny\sffamily def}}}{=}}}
\newcommand{\defsim}{\mathrel{\overset{\makebox[0pt]{\mbox{\normalfont\tiny\sffamily def}}}{\sim}}}
\newcommand{\grad}[1][\VectorSpace]{\nabla_{#1}}
\newcommand{\directionalDerivative}[1]{\partial_{#1}}
\newcommand{\dd}{\,\mathrm{d}}
\newcommand{\dimRep}[1]{{d_{#1}}}
\newcommand{\dimDifferenceOperators}[1][\Group]{n_{\Delta, #1}}
\newcommand{\dist}[3][\Group]{d_{#1}(#2, #3)}
\newcommand{\dualBracket}[3][\Group]{{\langle #2, #3 \rangle}_{#1}}
\newcommand{\dummy}{\cdot}
\newcommand{\eval}[2]{\left. #1 \right|_{#2}}
\newcommand{\g}{\mathfrak{g}}
\newcommand{\lcsum}[1]{\sum_{#1}^{--}}
\newcommand{\norm}[2][\VectorSpace]{{\left\| #2 \right\|}_{#1}}
\newcommand{\seminorm}[3][\VectorSpace]{\norm[#1, #2]{#3}}
\newcommand{\ip}[3][\VectorSpace]{{\left(#2, #3\right)_{#1}}}
\newcommand{\transpose}[1]{{#1}^T}
\newcommand{\AbelianGroup}{A}
\newcommand{\AffineTransformations}[1]{{\mathrm{Affine} (#1)}}
\newcommand{\Ball}[3][\VectorSpace]{{B_{#1} (#2, #3)}}
\newcommand{\BesselPotential}[2][\Group]{(I - \Laplacian[#1])^\frac{#2}{2}}
\newcommand{\BesselPotentialSquared}[2][\Group]{(I - \Laplacian[#1])^{#2}}
\newcommand{\BesselPotentialKernel}[2][\Group]{\mathfrak B^{#1}_{#2}}
\newcommand{\BigO}{\mathcal{O}}
\newcommand{\Character}[1]{{\chi_{#1}}}
\newcommand{\Class}[2][\Group]{C^{#2}(#1)}
\newcommand{\ContinuousFunctions}[1]{{C(#1)}}
\newcommand{\DifferenceOperator}[2][\Group]{{\Delta^{#1}_{#2}}}
\newcommand{\DifferenceOperatorOrder}[2][\Group]{{\Delta_{#1}^{#2}}}
\newcommand{\DiracDelta}[1]{\delta_{#1}}
\newcommand{\Distributions}[1]{{\mathcal{D}' (#1)}}
\newcommand{\EquivalenceClass}[2]{{{[#2]}_{#1}}}
\newcommand{\Fourier}[1][\Group]{\mathcal{F}_{#1}}
\newcommand{\Group}{G}
\newcommand{\GroupDirect}{\VectorSpace \times \CompactGroup}
\newcommand{\GeneralLinear}[1]{\mathrm{GL}(#1)}
\newcommand{\Hil}{\mathcal{H}}
\newcommand{\Hilbert}[2]{\mathfrak{H}_{#1, #2}}
\newcommand{\HilbertRep}[1]{{\mathfrak{H}_{#1}}}
\newcommand{\HilbertCompactGroupColumn}[2]{\mathfrak{H}_{#1, #2}}
\newcommand{\HilbertCompactGroup}[1]{\mathfrak{H}_{#1}}
\newcommand{\InverseFourier}[1][\Group]{\mathcal{F}^{-1}_{#1}}
\newcommand{\CompactGroup}{K}
\newcommand{\HilbertSchmidt}[1]{{\mathcal{HS} \left(#1\right)}}
\newcommand{\Id}[1]{{I_{#1}}}
\newcommand{\InverseFunctionArgument}[1][\Group]{\iota_{#1}}
\newcommand{\IsotropySubgroup}[2]{{{#1}_{#2}}}
\newcommand{\JapaneseBracket}[2]{{\langle #2 \rangle}_{\dualGroup{#1}}}
\newcommand{\Kronecker}[2]{\delta_{#1,#2}}
\newcommand{\Kernels}[1][\Group]{\mathcal{K}(#1)}
\newcommand{\KernelsSobolev}[3][\Group]{\mathcal{K}_{#2, #3}(#1)}
\newcommand{\Lebesgue}[2]{{L^{#1} (#2)}}
\newcommand{\LebesgueDual}[3][]{{L^{#2}_{#1} (\dualGroup{#3})}}
\newcommand{\LeftDifferentialOperatorFirstOrder}[1]{{#1}}
\newcommand{\LeftDifferentialOperator}[2][]{X^{#2}\if #1\empty \else_{#1 }\fi}
\newcommand{\LeftDifferentialOperatorOnCompactGroup}[2][]{Y^{#2}_{#1}}
\newcommand{\LeftRegularRepresentation}[1][\CompactGroup]{\pi^L_{#1}}
\newcommand{\RightRegularRepresentation}[1][\CompactGroup]{\pi^R_{#1}}
\newcommand{\Lie}{\mathfrak{Lie}}
\newcommand{\LieAlgebra}{\mathfrak{g}}
\newcommand{\LieAlgebraCompactGroup}{\mathfrak{k}}
\newcommand{\LieAlgebraVectorSpace}{\mathfrak{v}}
\newcommand{\LieBracket}[3][\LieAlgebra]{{[#2, #3]}_{#1}}
\newcommand{\Lin}[1]{{\mathcal{L} (#1)}}
\newcommand{\Laplacian}[1][\Group]{{\mathcal{L}_{#1}}}
\newcommand \SquareMatrices [2][\R] {{#1}^{#2 \times #2}}
\newcommand{\MotionGroup}[1]{{\mathrm{SE} (#1)}}
\newcommand{\OrthogonalGroup}[1]{{\mathrm{O} (#1)}}
\newcommand{\Op}[1][\Group]{\mathrm{Op}_{#1}}
\newcommand{\Plancherel}[1]{\mu_{\dualGroup{#1}}}
\newcommand{\Polynomials}[1]{{\mathrm{Pol}_{#1}}}
\newcommand{\Projection}[1]{\mathrm{Proj}_{#1}}
\newcommand{\LeftQuotient}[2]{{{#1} \backslash{} {#2}}}
\newcommand{\RightQuotient}[2]{{{#1} \slash{} {#2}}}
\newcommand{\Rep}[2][\Group]{\xi^{#2}_{#1}}
\newcommand{\RightDifferentialOperatorFirstOrder}[1]{\tilde{#1}}
\newcommand{\RightDifferentialOperator}[2][]{\tilde X^{#2}\if #1\empty \else_{#1 }\fi}
\newcommand{\RightLaplacian}[1][\Group]{{\tilde{\mathcal{L}}_{#1}}}
\newcommand{\Rotation}[1]{\tilde R \left(#1\right)}
\newcommand{\InverseRotation}[1]{R\left(#1\right)}
\newcommand{\SmoothFunctions}[1]{{C^\infty(#1)}}
\newcommand{\SmoothVectors}[1]{#1^\infty}
\newcommand{\ScalarImageSchwartz}[1]{\tilde{\mathcal{S}}(#1)}
\newcommand{\SchattenClasses}[2]{S_{#1}(#2)}
\newcommand{\Schwartz}[1]{{\mathcal{S} (#1)}}
\newcommand{\SkewSymmetric}[1]{\mathrm{Skew}(#1)}
\newcommand{\Sobolev}[2][\Group]{L^2_{#2}(#1)}
\newcommand{\SobolevOrder}[3][\Group]{L^{#2}_{#3}(#1)}
\newcommand{\SpecialOrthogonalGroup}[1]{{\mathrm{SO} (#1)}}
\newcommand{\SpecialUnitaryGroup}[1]{{\mathrm{SU} (#1)}}
\newcommand{\TangentSpace}[2]{T_{#2} #1}
\newcommand{\TaylorLeftDifferentialOperator}[1]{X^{(#1)}}
\newcommand{\TaylorPolynomial}[3]{P^{#1}_{#2, #3}}
\newcommand{\TaylorRemainder}[3]{R^{#1}_{#2, #3}}
\newcommand{\TemperedDistributions}[1]{{\mathcal{S}' (#1)}}
\newcommand{\UnitaryGroup}[1]{{\mathrm{U} (#1)}}
\newcommand{\LeftInvariantVectorFields}[1][\Group]{\mathfrak{X}_L(#1)}
\newcommand{\VectorFields}[1][\Group]{\mathfrak{X}(#1)}
\newcommand{\VectorSpace}{V}
\newcommand{\Volume}[1]{\mathrm{Vol}(#1)}

\DeclareMathOperator{\Aut}{Aut}
\DeclareMathOperator{\End}{End}
\DeclareMathOperator{\Hessian}{Hess}
\DeclareMathOperator{\Hom}{Hom}
\DeclareMathOperator{\Image}{Im}
\DeclareMathOperator{\Span}{span}
\DeclareMathOperator{\tr}{tr}
\DeclareMathOperator{\order}{order}
\DeclareMathOperator{\rank}{rank}
\DeclareMathOperator{\supp}{supp}
\DeclareMathOperator*{\esssup}{ess\,sup}

\makeatletter
\DeclareDocumentCommand \D{s m O{} m}{%
    % Choice of right d
    \IfBooleanTF{#1}{\def\@der{\dd}}{\def\@der{\partial}}
    % Write the derivative
    \mathchoice{%
        \frac{
            \@der\ifnum\pdfstrcmp{#2}{1}=0\else^{#2}\fi {#3}
        }{%
            \@for\@token:={#4}\do{\@der \@token}
        }
    } {%
        %\@for\@token:={#4}\do{\@der_\@token} #3
        \iD{#4} #3
    } {} {}
}
\DeclareDocumentCommand \iD {m}{%
    \@for\@token:={#1}\do{\partial_\@token}
}
\makeatother

% Sets
\newcommand{\C}{\mathbb{C}}
\newcommand{\N}{\mathbb{N}}
\newcommand{\R}{\mathbb{R}}
\newcommand{\T}{\mathbb{T}}
\newcommand{\Z}{\mathbb{Z}}

% Constants
\newcommand{\e}{e}
\newcommand{\turn}{2 \pi}
\renewcommand{\i}{i}
\renewcommand{\epsilon}{\varepsilon}

% Pseudo-differential calculus
\newcommand{\SmoothingSymbols}[1][\Group]{{S^{-\infty} (#1)}}
\newcommand{\SmoothingOperators}[1][\Group]{{\Psi^{-\infty} (#1)}}
\newcommand{\SymbolClass}[3][\Group]{S^{#2}_{#3}(#1)}
\newcommand{\Symbols}[1][\Group]{S(#1)}
\newcommand{\OperatorClass}[3][\Group]{\Psi^{#2}_{#3}(#1)}
\newcommand{\SymbolSemiNorm}[5][\Group]{\norm[S^{#2}_{#3}(#1), #4]{#5}}

\title{Pseudo-Differential Calculus on Generalized Motion Groups}
\author{Binh-Khoi Nguyen}

%\makeindex

\addbibresource{Bibliography.bib}

\begin{document}

\maketitle

\chapter*{Copyright}

This thesis is distributed under the terms of the
\href{http://creativecommons.org/licenses/by-sa/4.0/}{Creative Commons Attribution-ShareAlike 4.0 International License}.
Anyone is free to use, share and adapt the following document
provided they give appropriate credit to the author,
and share their modifications under similar terms.
The last requirement is not part of the university policy,
but it is in my opinion crucial that \emph{anyone} be able to access, use and build upon our scientific heritage.

Prospective research students might wonder what it is like to write a thesis.
For this reason, I am releasing the \emph{complete history} of the \LaTeX\ sources of this document
under the \href{https://www.gnu.org/licenses/gpl.html}{GNU General Public License v3+}.
All the versions of this document since 13 December 2015,
when this document contained only one definition,
can be viewed at
\url{https://git.nguyen.me.uk/khoi/public/phd.git}.
For more experienced researchers, I hope this will encourage adoption of \emph{version control systems}.

\chapter*{Acknowledgements}

Many teachers and professors contributed a great deal to my education,
and I will aways be grateful to them.
Some had more of an impact than others,
and I would like to seize the opportunity that those pages provide me to thank them.
The two people who in my opinion have influenced me the most mathematically are Jean Van Schaftingen and Augusto Ponce.

\begin{itemize}
    \item Michael Ruzhansky
    \item Veronique Fischer
    \item Augusto Ponce, Laurent Moonens, Jean Van Schaftingen, Pierre Bieliavsky
    \item Chloe
    \item Family
\end{itemize}


\tableofcontents

\chapter{Introduction}

In the Euclidean setting,
it is a known fact that constant-coefficient differential operators act like their characteristic polynomials on the Fourier Transform side.
Moreover, composing, taking the adjoint, or inverting such operators can be done by multiplication, adjunction or inversion of their associated polynomials.

Although the above properties do not hold for general differential operators with smooth coefficients,
the composition and adjunction of such operators can still be written solely in terms of their characteristic polynomials.
As an example,
if $T_a$ and $T_b$ are two operators arising from the polynomials $a(x, \xi)$ and $b(x, \xi)$ via
\begin{align*}
    T_a \defeq \sum_{\abs \alpha \leq N} \frac 1 {\alpha!} \iD{\xi^\alpha} a(x, 0) \iD{x^\alpha}
    \quad
    T_b \defeq \sum_{\abs \alpha \leq N} \frac 1 {\alpha!} \iD{\xi^\alpha} b(x, 0) \iD{x^\alpha},
\end{align*}
then it follows from an elementary computation
that the characteristic polynomial of $T_c \defeq T_a \circ T_b$ is given by the finite sum
\begin{align}
    c(x, \xi) = \sum_{\alpha \in \N^n} \frac 1 {\alpha!} \iD{\xi^\alpha} a(x, \xi) \iD{x^\alpha} b(x, \xi)
    \label{eq:composition_formula_for_characteristic_polynomials}.
\end{align}
Importantly,
\eqref{eq:composition_formula_for_characteristic_polynomials} means that
composition is reduced to summing and multiplying smooth functions.

Laurent Schwartz's \emph{Kernel Theorem} and his extension of the \emph{Fourier transform} allows us to extend the concept of characteristic polynomial much beyond the class of ordinary differential operators.
The resulting concept, called the \emph{symbol} of an operator,
is formally defined by taking the Fourier transform of the convolution kernel,
i.e.\ if $T$ is formally given by
\begin{align*}
    T \phi(x) \defeq \conv \phi {\kappa_x}(x),
\end{align*}
then we define its symbol via
\begin{align*}
    \sigma(x, \xi) \defeq \Fourier \kappa_x(\xi).
\end{align*}

Can we define a generalised notion of differential operator
with an approprate notion of order,
stable under composition and adjunction,
and for which both these operations can be expressed in terms of their symbols?

\chapter*{Notation}

\section*{Sets and symbols}

\begin{itemize}
    \item $\R$, $\R^+$ and $\R^-$ denote the sets of all, positive (including $0$) and negative (including $0$) real numbers respectively.
    \item $\Z$ denotes the sets of all integers,
        while $\N$ contains all positive integers including $0$.
    \item $\Kronecker i j$ is the \emph{Kronecker delta};
        if is equal to $1$ if $i = j$,
        and $0$ otherwise.
\end{itemize}

\section*{Positive and negative}

In this document, we shall use the following conventions.

\begin{itemize}
    \item $0$ is both \emph{``positive''} and \emph{``negative''}.
    \item $\N, \R^+, \R^-$ contain $0$.
    \item Any number is both \emph{``greater than''} and \emph{``less than''} itself.
    \item We shall use strict inequalities only
        when the corresponding non-strict inequalities are not suitable.
\end{itemize}

We shall do this to ensure that
when a strict inequality is necessary (i.e. $< +\infty$ or $> 0$),
it gets the emphasis it deserves.
This is why we shall always insist on writing \emph{``strictly positive''} or \emph{``strictly greater than''} when we require the strict inequality.
To make it less verbose,
we give the non-strict meaning to \emph{``greater or less than''}.

Note that the above conventions originated from Laurent Schwartz's \citetitle{Schwartz1981} (\cite{Schwartz1981})
and have now become standard in Francophone literature.

\begin{aquote}{Laurent Schwartz, \cite[p.~17]{Schwartz1981}}
    Notons que nous rompons ici avec l'usage ant\'erieurement acquis
    en appelant inf\'erieur ce qu'on appelait inf\'erieur ou \'egal,
    et strictement inf\'erieur ce qu'on appelait inf\'erieur.
    La raison d'\^etre de ces changements,
    pleinement justifi\'es par la suite,
    est que la notion la plus g\'en\'eralement utilis\'ee est $\leq$,
    et qu'il est bon qu'elle ait l'appelation la plus courte.
    On devra toujours utiliser $\leq$ plut\^ot que $<$,
    toutes les fois que cela sera possible;
    quand on \'ecrira une in\'egalit\'e stricte avec $<$,
    ce sera pour avertir le lecteur qu'il y a un point d\'elicat,
    et que l'in\'egalit\'e large $\leq$ ne conviendrait pas.
    \footnote{%
        \emph{Translation:}
        Let us observe that we break from traditional usage
        by saying ``less than'' instead of ``less than or equal to'',
        and ``strictly less than'' instead of ``less than''.
        The reason for these changes,
        completely justified hereafter,
        is that $\leq$ is the more widely used notion
        and should therefore have the shortest name.
        We shall always use $\leq$ rather than $<$
        every time we can;
        when we use the strict inequality with $<$,
        it will be to warn the reader that there is a subtlety
        and that the non-strict inequality is not suitable.
    }
\end{aquote}

Naturally, the same terminology applies to operators and matrices.
The operator $0$ is \emph{``positive definite''} but not \emph{``strictly positive definite''}.

\chapter{Preliminaries}

\section{Lie groups}

To develop a \emph{pseudo-differential calculus} on groups,
a reasonable prerequisite is that the group be equipped with a differential structure.

\begin{definition}[Lie group]
\label{definition:Lie_group}
\index{Lie group}
    Let $\Group$ be a group.
    We say that $\Group$ is a \emph{Lie group}
    if $\Group$ is a smooth manifold and the map
    \begin{align*}
        \Group \times \Group \to \Group :
        (g_1, g_2) \mapsto g_1^{-1} g_2
    \end{align*}
    is smooth.

    If moreover $\Group$ is (locally) compact as a manifold,
    then we shall say that $\Group$ is a \emph{(locally) compact Lie group}.
\end{definition}

\begin{example}[Special orthogonal group]
    The set
    \begin{align*}
        \SpecialOrthogonalGroup \VectorSpace
        \defeq
        \{ A \in \Lin{\VectorSpace} : \det A = 1 \}
    \end{align*}
    is a compact Lie group.
\end{example}

\subsection{Lie algebra}

\begin{definition}[Lie algebra]
    A (real) Lie algebra is a (real) vector space $\LieAlgebra$
    equipped with a bilinear map
    \begin{align*}
        \LieBracket \dummy \dummy : \VectorSpace \times \VectorSpace \to \VectorSpace,
    \end{align*}
    called the \emph{Lie bracket} or \emph{commutator},
    such that
    \begin{enumerate}
        \item $\LieBracket X X = 0$ for every $X \in \LieAlgebra$;
        \item for every $X, Y, Z \in \LieAlgebra$, we have the following \emph{Jacobi identity}
            \begin{align*}
                \LieBracket X {\LieBracket Y Z} +
                \LieBracket Y {\LieBracket Z X} +
                \LieBracket Z {\LieBracket X Y}
                = 0.
            \end{align*}
    \end{enumerate}
\end{definition}

Given a Lie group $\Group$,
let us denote by $\VectorFields$ the set of all \emph{smooth vector fields} on $\Group$.

\begin{definition}[Left-invariant vector fields]
    Let $\Group$ be a Lie group.
    We shall say that $X \in \VectorFields$ is \emph{left-invariant}
    if for every $g \in \Group$,
    \begin{align*}
        X \circ L_g = \dd L_g \circ X,
    \end{align*}
    where
    \begin{align*}
        L_g : \Group \to \Group : h \mapsto g h
    \end{align*}
    is the left translation on $\Group$ by $g$.
    We shall denote the set of left-invariant vector fields by $\LeftInvariantVectorFields$.
\end{definition}

\begin{example}[Left-invariant vector fields]
\label{example:Lie_algebra_of_left-invariant_vector_fields}
    Let $\Group$ be a Lie group.
    Given two left-invariant vector fields $X, Y \in \LeftInvariantVectorFields$,
    we define
    \begin{align*}
        \LieBracket [\LeftInvariantVectorFields] X Y f(x) \defeq X Y f(x) - Y X f(x),
        \quad f \in \SmoothFunctions \Group
    \end{align*}
    and we can show that $\LieBracket [\LeftInvariantVectorFields] X Y$ is also a left-invariant vector field.
    Therefore, $\LeftInvariantVectorFields$ is a Lie algebra.
\end{example}

Let $X, Y \in \TangentSpace \Group e$.
Defining for each $g \in \Group$
\begin{align*}
    \tilde X(g) \defeq \dd L_g(X), \quad \tilde Y(g) = \dd L_g(Y),
\end{align*}
we check that $\tilde X, \tilde Y \in \LeftInvariantVectorFields$.

The Lie algebra structure of $\LeftInvariantVectorFields$ allows us then to define
\begin{align}
    \LieBracket [\TangentSpace \Group e] X Y \defeq \LieBracket [\LeftInvariantVectorFields] {\tilde X} {\tilde Y}(e).
    \label{eq:Lie_bracket_on_the_tangent_space}
\end{align}

\begin{definition}[Lie algebra of Lie group]
\label{definition:Lie_algebra_of_Lie_group}
    Let $\Group$ be a Lie group.
    The \emph{Lie algebra} of $\Group$ is the tangent space at the identity $\TangentSpace \Group e$
    with the Lie bracket $\LieBracket [\TangentSpace \Group e] \dummy \dummy$
    defined in \eqref{eq:Lie_bracket_on_the_tangent_space}.
\end{definition}

\subsection{Haar measure}

\begin{remark}
    Every Lie group $\Group$ is also topological space.
    As a result, all the topological definitions apply to Lie groups.
\end{remark}

\begin{definition}[Haar measure]
\index{Haar measure}
    Let $\Group$ be a Lie group.
    A positive Radon measure on $\Group$ is called a \emph{Haar measure}
    if it is in addition \emph{left-invariant},
    i.e.\ for each $g \in \Group$ and each Borel set $A \subset G$, we have
    \begin{align*}
        \mu(g A) = \mu(A).
    \end{align*}
\end{definition}

\begin{proposition}[Haar measure]
    If $\Group$ is a locally compact Lie group,
    then there exists a Haar measure $\mu$ on $\Group$.

    Moreover, if $\nu$ is another left-invariant Radon measure on $\Group$,
    then we can find $c \geq 0$ such that $\nu = c \mu$.
\end{proposition}

\begin{definition}[Unimodular group]
\label{definition:unimodular_group}
    Let $\Group$ be a locally compact Lie group.
    If a non-zero Haar measure on $\Group$ is also right-invariant,
    or equivalently if all Haar measure are right-invariant,
    we shall say that $\Group$ is \emph{unimodular}.
\end{definition}

\begin{proposition}
\label{proposition:sufficient_conditions_to_be_unimodular}
    Let $\Group$ be a Lie group.
    \begin{enumerate}
        \item If $\Group$ is compact, then $\Group$ is \emph{unimodular}.
        \item If $\Group$ is abelian and locally compact, then $\Group$ is unimodular.
    \end{enumerate}
\end{proposition}

\section{Representation Theory}

\begin{definition}[Unitary representations]
\label{definition:unitary_representation}
\index{representations!unitary representations}
    Let $\Group$ be a group and $\Hil$ be a Hilbert space.
    A map
    \begin{align*}
        \xi : \Group \mapsto \Hom(\Hil)
    \end{align*}
    is called a \emph{unitary representation (on $\Hil$)} if
    \begin{enumerate}
        \item for each $g \in \Group$, the map $\xi(g)$ is unitary:
            \begin{align*}
                {\xi(g)}^{-1} = \adj{\xi(g)};
            \end{align*}
        \item if $g, h \in \Group$, then we have $\xi(g h) = \xi(g) \xi(h)$.
    \end{enumerate}

    The \emph{dimension} of $\xi$ is that of $\Hil$.
    If $\Hil$ is finite-dimensional,
    we let $\dimRep{\xi} \defeq \dim{\Hil}$ denote the dimension of $\xi$.
\end{definition}

\begin{example}[Right-regular representation]
    Let $\Group$ be a unimodular topological group.
    The \emph{right-regular representation} is the representation
    \begin{align*}
        \RightRegularRepresentation : \Group \to \Hom(\Lebesgue{2}{\Group})
    \end{align*}
    defined via
    \begin{align*}
        \RightRegularRepresentation(h) f(g) = f(g h)
    \end{align*}
    for every $g, h \in \Group$.
\end{example}

\begin{definition}[Invariant subspaces]
\label{definition:invariant_subspaces}
    Let $\Group$ be a group and $\xi$ be a unitary representation of $\Group$ on a Hilbert space $\Hil$.
    We shall say that a vector subspace $W \subset \Hil$ is \emph{invariant} under $\xi$
    if for each $g \in \Group$, we have $\xi(g) W \subset W$.
\end{definition}

\begin{definition}[Irreducibility]
\label{definition:irreducible_representations}
    Let $\Group$ be a group and $\xi$ be a unitary representation of $\Group$ on a Hilbert space $\Hil$.
    \begin{enumerate}
        \item If the only invariant subspaces of $\xi$ are $\{0\}$ and $\Hil$,
            then $\xi$ is said to be \emph{irreducible}.
        \item Otherwise, if there exists a non-trivial invariant subspace,
            then $\xi$ is \emph{reducible}.
    \end{enumerate}
\end{definition}

\begin{definition}[Equivalent representations]
\label{definition:equivalent_representations}
    Let $\Group$ be a group and $\Hil_1, \Hil_2$ be Hilbert spaces.
    Suppose $\xi_1$, $\xi_2$ are representations of $\Group$ on $\Hil_1$ and $\Hil_2$ respectively.
    We shall say that $\xi_1$ and $\xi_2$ are \emph{equivalent}
    if there exists an invertible linear map
    \begin{align*}
        A : \Hil_1 \to \Hil_2
    \end{align*}
    such that for each $g \in \Group$, we have
    \begin{align*}
        \xi_2(g) = A \circ \xi_1(g) \circ A^{-1}.
    \end{align*}
    In that case, $A$ is called an \emph{intertwining operator}.
\end{definition}

\begin{definition}[Strongly continuous representations]
\label{definition:strongly_continuous_representation}
\index{strongly continuous}
    Let $\Group$ be a group and $\Hil$ be a Hilbert space.
    Suppose further that $\xi$ is a representation of $\Group$ on $\Hil$.
    We shall say that $\xi$ is \emph{strongly continuous}
    if for each $x \in \Hil$,
    the map
    \begin{align*}
        \Group \to \Hil : g \mapsto \xi(g) v
    \end{align*}
    is continuous.
\end{definition}

\begin{definition}[Unitary dual]
\label{definition:unitary_dual}
    Let $\Group$ be a locally compact topological group.
    The \emph{unitary dual} of $\Group$, denoted by $\dualGroup\Group$,
    is the set of all equivalence classes of
    \emph{strongly continuous, irreducible, unitary representations} of $\Group$.
\end{definition}

\begin{remark}
    Let $\Group$ be a locally compact topological group.
    We shall often abuse the notation and use $\dualGroup\Group$ to denote a set consisting of
    exactly one representation in each equivalence class of the actual unitary dual.
\end{remark}

\begin{example}[$\dualGroup\VectorSpace$]
    For each $\lambda \in \VectorSpace$,
    define
    \begin{align}
        \xi_\lambda : \VectorSpace \to \UnitaryGroup{1} : x \mapsto \e^{\i \turn \ip{\lambda}{x}}.
        \label{eq:elements_of_dual_of_vector_space}
    \end{align}

    It can be shown that
    \begin{align*}
        \dualGroup\VectorSpace = \{ \xi_\lambda : \lambda \in \VectorSpace \}.
    \end{align*}

    Therefore, the map
    \begin{align}
        \lambda \mapsto \xi_\lambda
        \label{eq:isomorphism_between_vector_space_and_its_dual_group}
    \end{align}
    is a group isomorphism which allows us to give $\dualGroup\VectorSpace$ a vector space structure.
\end{example}

\begin{definition}
    Let $\tau \in \dualGroup\Group$.
    We define the sets
    \begin{align*}
        \HilbertCompactGroupColumn{\tau}{i} \defeq \span \{ \tau_{i, j} : j = 1, \dots, \dimRep\tau \}
        \subset \Lebesgue{2}\CompactGroup, \quad
        \HilbertCompactGroup{\tau} = \bigoplus_{i = 1}^\dimRep\tau \HilbertCompactGroupColumn{\tau}{i}.
    \end{align*}
\end{definition}

\begin{theorem}[Peter-Weyl theorem]
\label{theorem:Peter-Weyl_theorem}
    Let $\Group$ be a compact topological group.
    The set
    \begin{align*}
        \left\{
            \sqrt{\dimRep\tau} \tau_{ij} : \tau \in \dualGroup\Group,\ i, j = 1, \dots, \dimRep\tau
        \right\}
    \end{align*}
    is an orthonormal basis of $\Lebesgue{2}{\Group}$ with respect to the \emph{normalised} Haar measure.

    Therefore, we obtain the following decomposition:
    \begin{align*}
        \Lebesgue{2}{\Group} \defeq
        \bigoplus_{\tau \in \dualGroup\Group} \bigoplus_{j = 1}^{\dimRep \tau} \Hilbert{\tau}{j}.
    \end{align*}

    Since $\Hilbert{\tau}{j}$ is an invariant subspace of $\RightRegularRepresentation$
    and since $\eval{\RightRegularRepresentation}{\Hilbert{\tau}{j}} \sim \tau$,
    we have
    \begin{align*}
        \RightRegularRepresentation \sim
        \bigoplus_{\tau \in \dualGroup\Group} \dimRep\tau \tau.
    \end{align*}
    In particular, the unitary dual can be generated by the right-regular representation.
\end{theorem}

\section{Distributions}

For this section,
we assume that $\VectorSpace$ is a vector space
and $\CompactGroup$ is a compact Lie group.
Also, when we integrate on $\VectorSpace$, $\CompactGroup$ or $\GroupDirect$,
we always mean with respect to a \emph{Haar measure}.

We shall denote by $\LieAlgebraCompactGroup$ the Lie algebra of $\CompactGroup$.
We fix a basis $Y_1, \dots, Y_{\dim \CompactGroup}$ of $\LieAlgebraCompactGroup$ and write
\begin{align*}
    Y^\beta \defeq Y_1^{\beta_1} \dots Y_{\dim \CompactGroup}^{\beta_{\dim \CompactGroup}},
    \quad \beta \in \N^{\dim \CompactGroup}.
\end{align*}

\begin{definition}[Schwartz space]
    We shall say that $f \in \SmoothFunctions \GroupDirect$ is \emph{rapidly decaying}
    if for every $N \in \N$,
    \begin{align*}
        \seminorm [\Schwartz \GroupDirect] N {f}
        \defeq
        \sup_{\abs \alpha, \abs \beta \leq N}
        \abs{%
            {(1 + \abs x)}^N \D{\abs \alpha}{x^\alpha} Y^{\beta}_k f(x, k)
        } < \infty.
    \end{align*}
    The set of all rapidly decaying functions will denoted by $\Schwartz \GroupDirect$.
    The family $\{\seminorm [\Schwartz \GroupDirect] N \dummy : N \in \N\}$ gives $\Schwartz \GroupDirect$
    the structure of a Fr\'echet space.
\end{definition}

\begin{definition}[Tempered distributions]
    We shall say that
    \begin{align*}
        \kappa : \Schwartz \GroupDirect \to \C
    \end{align*}
    is a \emph{tempered distribution} if it is linear and continuous.
    The set of all tempered distributions will be denoted by $\TemperedDistributions \GroupDirect$.
\end{definition}

As usual, if $\kappa \in \TemperedDistributions \GroupDirect$ and $\phi \in \Schwartz \GroupDirect$
\begin{align*}
    \dualBracket [\GroupDirect] \kappa \phi \defeq \kappa(\phi)
\end{align*}
We shall also write
\begin{align*}
    \int_{\GroupDirect} \kappa(x, k) \phi(x, k) d(x, k) \defeq \dualBracket [\GroupDirect] \kappa \phi,
\end{align*}
motivated by the inclusion $\Lebesgue 1 \GroupDirect \subset \TemperedDistributions \GroupDirect$,
and shall say that the above integral is interpreted \emph{in the sense of distributions}.

\begin{theorem}[Schwartz Kernel Theorem]
\label{theorem:Schwartz_Kernel_Theorem}
\index{Schwartz Kernel Theorem}
    If $T : \Schwartz \GroupDirect \to \TemperedDistributions \GroupDirect$ is a continuous linear operator,
    then there exists a unique distribution
    $\kappa \in \TemperedDistributions {\GroupDirect \times \GroupDirect}$ such that
    \begin{align*}
        T \phi(x, k) = \int_{\GroupDirect} \kappa(x, k; y, l) \phi(y, l) \dd (y, l).
    \end{align*}
\end{theorem}

\section{Fourier Transform}

\begin{definition}[Fourier coefficient]
    Let $\xi \in \dualGroup\Group$.
    If $f \in \Lebesgue{1}{\Group}$,
    we define the \emph{Fourier coefficient of $f$ at $\xi$}, $\Fourier f(\xi)$, via
    \begin{align*}
        \Fourier f(\xi) \defeq \int_\Group f(g) \adj{\xi(g)} \dd g.
    \end{align*}

    The map
    \begin{align*}
        \Fourier f : \xi \mapsto \Fourier f(\xi)
    \end{align*}
    is called the \emph{Fourier transform of $f$}.
\end{definition}

\begin{definition}
    Let $\xi \in \VectorSpace$.
    If $f \in \Lebesgue{1}\VectorSpace$,
    we define the \emph{Fourier coefficient of $f$ at $\lambda$}, $\Fourier[\VectorSpace] f(\xi)$, via
    \begin{align*}
        \Fourier[\VectorSpace] f(\xi) \defeq \int_\VectorSpace f(x) \e^{-\i \turn \ip{x}\xi} \dd x.
    \end{align*}
\end{definition}

\begin{proposition}
\label{proposition:elementary_properties_of_the_Fourier_transform}
    Let $f, f_1, f_2 \in \Lebesgue{1}{\Group}$ and $\xi \in \dualGroup\Group$.
    The Fourier Transform satisfies the following properties:
    \begin{enumerate}
        \item For each $g \in \Group$, we have
            \begin{align*}
                \Fourier \{f(\dummy g)\} (\xi)
                = \xi(g) \Fourier f(\xi), \quad
                \Fourier \{f(g \dummy)\} (\xi)
                = \Fourier f(\xi) \xi(g).
            \end{align*}
        \item We have
            \begin{align*}
                \Fourier \{\conv{f_1}{f_2}\}(\xi)
                = \Fourier f_2(\xi) \Fourier f_1(\xi).
            \end{align*}
        \item If $f \in \SmoothFunctions\Group$ and $X \in \LieAlgebra$,
            \begin{align*}
                \Fourier \{\LeftDifferentialOperatorFirstOrder{X} f\}(\xi)
                = \xi(X) \Fourier f(\xi), \quad
                \Fourier \{\RightDifferentialOperatorFirstOrder{X} f\}(\xi)
                = \Fourier f(\xi) \xi(X).
            \end{align*}
    \end{enumerate}
\end{proposition}

\begin{proposition}
    There exists a measure $\Plancherel\Group$ on $\dualGroup\Group$ such that the following property holds:
    if $f \in \Schwartz\Group$, we have
    \begin{align*}
        \int_\Group \abs{f}^2 \dd g
        = \int_{\dualGroup\Group} \tr \left( \Fourier f(\xi) \adj{\Fourier f(\xi)} \right) \dd \Plancherel\Group(\xi).
    \end{align*}
\end{proposition}

\begin{example}[Plancherel Measure on $\dualGroup\CompactGroup$]
    If $f \in \SmoothFunctions{\CompactGroup}$,
    then the Peter-Weyl theorem implies:
    \begin{align*}
        \int_\CompactGroup \abs{f}^2 \dd g
        = \sum_{\tau \in \dualGroup\CompactGroup}
            \dimRep\tau
            \tr \left(
                \Fourier[\CompactGroup] f(\tau)
                \adj{\Fourier[\CompactGroup] f(\tau)}
            \right)
    \end{align*}
\end{example}

\begin{proposition}[Inverse Fourier Transform]
    Let $g \in \Group$.
    If $f \in \Schwartz\Group$,
    then we have
    \begin{align*}
        f(g) =
        \int_\dualGroup\Group
            \tr\left(
                \xi(g)
                \Fourier f(\xi)
            \right)
        \dd \Plancherel\Group(\xi).
    \end{align*}
\end{proposition}

\section{The $\GroupDirect$ case }

As our representations on the motion group will act on $\Lebesgue{2}\CompactGroup$,
we will define our Fourier Transform on $\CompactGroup$ so that it acts on $\Lebesgue{2}\CompactGroup$ as well.
This will allow useful comparisons later on.

\begin{definition}[Fourier Transform]
    Let $f \in \Lebesgue 1 \GroupDirect$ and $\lambda \in \VectorSpace$.
    We define the \emph{Fourier coefficient of $f$} at $\lambda$,
    denoted via $\Fourier[\GroupDirect] f(\lambda)$, via
    \begin{align*}
        \Fourier [\GroupDirect] f(\lambda) \defeq \int_\CompactGroup f(x, k) \e^{\i \turn \ip \lambda x} \adj{\RightRegularRepresentation(k)} \dd (x, k).
    \end{align*}
    The map
    \begin{align*}
        \lambda \in \VectorSpace \mapsto \Fourier [\GroupDirect] f(\lambda)
    \end{align*}
    is called the \emph{Fourier Transform} of $f$ on $\GroupDirect$.
\end{definition}

\begin{proposition}[Inverse formula]
    Let $f \in \SmoothFunctions \GroupDirect$.
    If $(x, k) \in \GroupDirect$,
    then $f$ can be recovered via the formula
    \begin{align*}
        f(x, k) = \int_\VectorSpace \tr\left(\e^{\i \turn \ip x \lambda} \RightRegularRepresentation(k) \Fourier[\GroupDirect] f(\lambda) \right) \dd \lambda.
    \end{align*}
\end{proposition}

\begin{lemma}
    For each $\tau \in \dualGroup\CompactGroup$,
    the space $\HilbertCompactGroup{\tau}$ is an eigenspace of the operator $\Laplacian[\CompactGroup]$.
    Denoting by $\JapaneseBracket{\CompactGroup}{\tau}$ the eigenvalue associated with the operator $\BesselPotential[\CompactGroup]{1}$ on the eigenspace $\HilbertCompactGroup\tau$, we obtain
    \begin{align*}
        \eval{\RightRegularRepresentation\left(\BesselPotential[\CompactGroup]{1}\right)}{\HilbertCompactGroup\tau}
        = \JapaneseBracket{\CompactGroup}{\tau} \Id{\HilbertCompactGroup\tau}
    \end{align*}
\end{lemma}

%\section{Difference operators}
%
%\begin{proposition}
%    Let $\Group$ be a compact topological group.
%    There exists a finite subfamily $\{q_1, \dots, q_M\}$ of
%    \begin{align*}
%        \{ \tau_{ij} - \Kronecker{i}{j} : \tau \in \dualGroup\Group,\ i, j = 1, \dots, \dimRep\tau \}
%    \end{align*}
%    which is \emph{strongly admissible}.
%\end{proposition}
%
%\section{Pseudo-differential calculus}

%\chapter{Survey}

\section{The compact case}

In this section,
$\CompactGroup$ denotes a \emph{connected, compact Lie group}.
We equip $\CompactGroup$ with the unique normalized bi-invariant Riemannian metric,
and denote the corresponding Laplace-Beltrami operator by $\Laplacian [\CompactGroup]$.

The description of the unitary dual is given by the following \emph{Peter-Weyl theorem}.

\begin{theorem}[Peter-Weyl theorem]
\label{theorem:Peter-Weyl_theorem}
    Let $\CompactGroup$ be a compact topological group.
    The unitary dual $\dualGroup \CompactGroup$ of $\Group$ is discrete and
    the set
    \begin{align*}
        \left\{
            \sqrt{\dimRep\tau} \tau_{ij} : \tau \in \dualGroup\CompactGroup,\ i, j = 1, \dots, \dimRep\tau
        \right\}
    \end{align*}
    is an orthonormal basis of $\Lebesgue{2}{\CompactGroup}$ with respect to the \emph{normalised} Haar measure.
\end{theorem}

In particuar,
the Peter-Weyl theorems implies that $\Lebesgue 2 \CompactGroup$ has the following orthogonal decomposition
\begin{align}
    \Lebesgue 2 \CompactGroup
    = \bigoplus_{\tau \in \dualGroup \CompactGroup} \dimRep \tau H_{\tau, j},
    \label{eq:decomposition_of_L2_for_compact_groups}
\end{align}
where $H_{\tau, j} \defeq \Span\{\tau_{j l} \in \Lebesgue 2 \CompactGroup : l = 1, \dots, \dim \CompactGroup\}$,
while
\begin{align*}
    e^{\tau, j}_l \defeq \sqrt {\dimRep \tau} \tau_{j l}, \quad l = 1, \dots, \dim \CompactGroup
\end{align*}
forms an orthonormal basis of $H_{\tau, j}$.

Observe that
\begin{align*}
    \ip [\Lebesgue 2 \CompactGroup] {\RightRegularRepresentation(k) e^{\tau, j}_n} {e^{\tau, j}_m}
    &= \dimRep \tau \int_\CompactGroup \tau_{j n}(h k) \conj{\tau_{j m}}(h) \dd h\\
    &= \sum_{p = 1}^{\dimRep \tau} \left( \dimRep \tau \int_\CompactGroup \tau_{j p}(h) \conj{\tau_{j m}}(h) \dd h \right) \tau_{p n}(k)\\
    &= \sum_{p = 1}^{\dimRep \tau} \Kronecker p m \tau_{p n}(k) = \tau_{m n}(k),
\end{align*}
which means that
\begin{align*}
    (\RightRegularRepresentation(k), H_{\tau, j}) \simeq \tau.
\end{align*}

Combining the above with~\eqref{eq:decomposition_of_L2_for_compact_groups},
we obtain
\begin{align*}
    \RightRegularRepresentation \simeq \bigoplus_{\tau \in \dualGroup \CompactGroup} \dimRep \tau \tau,
\end{align*}
i.e.\ the right-regular representation of $\CompactGroup$ generates the unitary dual $\dualGroup \CompactGroup$.

\subsection{Fourier analysis on Lie groups}

\section{The graded case}

\chapter{Motion Groups}

From now on,
$\VectorSpace$ will denote a finite-dimensional vector space over $\R$,
while $\CompactGroup$ is a compact Lie subgroup of $\GeneralLinear\VectorSpace$.
We then define the \emph{motion group} as the \emph{semi-direct product} between $\VectorSpace$ and $\CompactGroup$.

In this chapter,
we start by observing that the compactness of $\CompactGroup$ is a very rigid condition,
as it implies that $\CompactGroup \subset \SpecialOrthogonalGroup \VectorSpace$.
Using the fact that both $\Laplacian [\VectorSpace]$ and the Lebesgue measure are invariant under $\SpecialOrthogonalGroup \VectorSpace$,
we show that the motion group has the \emph{same Haar measure} and \emph{same Laplacian} as $\GroupDirect$.
Moreover,
provided that we group the representations on either side,
the representations of the motion group seem formally very close to those of $\GroupDirect$.

All these observations lead to the conclusion
that the Fourier theory on our group might seem identical to the one of $\GroupDirect$.
However,
the resemblance is very deceptive,
as we shall see in Chapter \ref{chapter:symbols}.

\section{Motion groups}

\begin{definition}[Motion group]
\label{definition:motion_group}
\index{motion group}
    Let $\Group$ be a group.
    We shall say that $\Group$ is a \emph{motion group}
    if there exists a finite dimensional real vector space $\VectorSpace$
    and a compact connected Lie group $\CompactGroup \subset \GeneralLinear\VectorSpace$
    such that $\Group$ is the \emph{semi-direct product} $\VectorSpace \ltimes \CompactGroup$.
    More precisely, $\Group = \VectorSpace \times \CompactGroup$ as a set, and
    \begin{align*}
        (x, k) (y, l) \defeq (x + ky, k l)
    \end{align*}
    for every $(x, k), (y, l) \in \Group$.
\end{definition}

\begin{remark}
\label{remark:notation_kx}
    Let $x \in \VectorSpace$ and $k \in \CompactGroup$.
    We will never identify $x$ with $(x, \Id{\VectorSpace}) \in \VectorSpace \ltimes \CompactGroup$,
    or $k$ with $(0, k) \in \VectorSpace \ltimes \CompactGroup$.
    Therefore, when we write $k x$, it will \emph{always} mean the vector obtained by rotating $x$ by $k$, i.e.\ $k(x)$.
\end{remark}

\begin{example}[Vector spaces]
\label{example:trivial_case_of_motion_groups}
    Suppose that $\CompactGroup = \{\Id\VectorSpace\}$.
    It follows that $\Group \defeq \VectorSpace \ltimes \CompactGroup$ is isomorphic to $\VectorSpace$.
\end{example}

\begin{example}[Euclidean Motion Groups]
\label{example:Euclidean_motion_groups}
\index{Euclidean motion group}
    For each $n \in \N$, let
    \begin{align*}
        \MotionGroup{n} \defeq \{g \in \AffineTransformations{\R^n} : \det g = 1\}.
    \end{align*}
    The elements of $\MotionGroup{n}$ are called \emph{rigid motions},
    while $\MotionGroup{n}$ is called the \emph{Euclidean motion group}.

    It is easily shown that associating $(x, k) \in \R^n \ltimes \SpecialOrthogonalGroup{n}$ to the motion
    \begin{align*}
        g_{(x, k)} : \R^n \to \R^n : y \mapsto x + ky
    \end{align*}
    defines a group isomorphism between $\R^n \ltimes \SpecialOrthogonalGroup{n}$ and $\MotionGroup{n}$.
    We shall therefore identify $\MotionGroup{n}$ with $\R^n \ltimes \SpecialOrthogonalGroup{n}$ from now on.
\end{example}

\begin{example}
    \label{example:complex_motion_groups}
    Let $n \in \N$.
    Consider the group
    \begin{align*}
        \{g \in \AffineTransformations{\C^n} : \det_{\C^n} g = 1\}
    \end{align*}
    where the law is the composition of functions.

    Arguing like in Example~\ref{example:Euclidean_motion_groups},
    the above group can be identified with $\C^n \ltimes \SpecialUnitaryGroup{n}$.
\end{example}

\begin{remark}
    Since in our examples (e.g. Example \ref{example:complex_motion_groups}) our vector space might be $\C^n$,
    we choose to use $\VectorSpace$ to denote the vector space instead of simply $\R^n$ to avoid any confusion.
\end{remark}

\begin{lemma}[$\CompactGroup$-invariant inner product]
\label{lemma:existence_of_K-invariant_inner_product}
    Let $\VectorSpace$ be a vector space,
    and $\CompactGroup$ be a compact Lie group acting acting on $\VectorSpace$.
    There exists an inner product $\ip{\dummy}{\dummy} : \VectorSpace \times \VectorSpace \to \R$ which is $\CompactGroup$-invariant,
    i.e.\ for each $k \in \CompactGroup$ and every $x, y \in \VectorSpace$, we have
    \begin{align*}
        \ip{x}{y} = \ip{k x}{k y}.
    \end{align*}
\end{lemma}
\begin{proof}
    Let $Q : \VectorSpace \times \VectorSpace \to \R$ be an arbitrary inner product.
    Given $x, y \in \VectorSpace$, we let
    \begin{align*}
        \ip{x}{y} \defeq \int_\CompactGroup Q(k x, k y) \dd k,
    \end{align*}
    where $\dd k$ is the Haar measure on $\CompactGroup$.

    It follows that if $k \in \CompactGroup$,
    then using the right-invariance of the Haar measure on compact groups,
    (for this, see for example \cite[Theorem 7.4.21]{RuzhanskyTurunen10})
    we obtain
    \begin{align*}
        \ip{k x}{k y}
        = \int_\CompactGroup Q(h k x, h k y) \dd h
        = \int_\CompactGroup Q(h x, h y) \dd h
        = \ip{x}{y},
    \end{align*}
    i.e.\ $\ip{\dummy}{\dummy}$ is $\CompactGroup$-invariant.

    The fact that $\ip{\dummy}{\dummy}$ is bilinear and nonnegative definite follows immediately from the fact that $Q$ has those properties.
    Now, if $x, y \in \VectorSpace$ are such that $\ip{x}{x} = 0$.
    It follows that $Q(k x, k x) = 0$ for almost every $k \in \CompactGroup$, hence for at least one such $k$.
    However, that means that $k x = 0$, hence $x = 0$ as $k$ is invertible.
\end{proof}

From now on, $\VectorSpace$ will be given the structure of \emph{Euclidean space} with a $\CompactGroup$-invariant inner product
whose existence is given by Lemma~\ref{lemma:existence_of_K-invariant_inner_product}.

\begin{definition}[Lebesgue measure]
    We call the \emph{Lebesgue measure} on $\VectorSpace$
    the unique Haar measure $\mu$ on $\VectorSpace$ such that
    \begin{align*}
        \mu\left(\{x \in \VectorSpace : \ip{x}{x} \leq 1\}\right) = \frac{\pi^{\dim \VectorSpace/2}}{\Gamma(\frac{\dim \VectorSpace}{2} + 1)}.
    \end{align*}
\end{definition}

From now on, integration on $\VectorSpace$ will always be performed with respect to the above Lebesgue measure.

\begin{lemma}
\label{lemma:K_subset_of_SO_and_invariance_of_the_induced_Lebesgue_measure}
    With the inner product defined in Lemma~\ref{lemma:existence_of_K-invariant_inner_product},
    $\CompactGroup$ is a Lie subgroup of $\SpecialOrthogonalGroup\VectorSpace$.

    Therefore, any Haar measure on $\VectorSpace$ is also invariant under the action of $\CompactGroup$.
\end{lemma}

\begin{lemma}[Haar measure]
\label{lemma:Haar_measure}
    If $\dd x$ is the Lebesgue measure on $\VectorSpace$ and $\dd k$ is the normalised Haar measure on $\CompactGroup$,
    then the the product measure $\dd x \dd k$ is a Haar measure on $\Group = \VectorSpace \ltimes \CompactGroup$,
    which is both left and right-invariant.
\end{lemma}
\begin{proof}
    Let $(x, k) \in \Group$.
    \begin{align*}
        \int_\Group f((x, k) (y, l)) \dd (y, l)
        = \int_\VectorSpace \int_\CompactGroup f(x + ky, k l) \dd l \dd y
    \end{align*}

    Now, let us substitute $y$ for $k^{-1}(y - x)$ and $l$ for $k^{-1} l$ in the above.
    As the Lebesgue measure is invariant under $\OrthogonalGroup{\VectorSpace}$ and under translations,
    and because the Haar measure $\dd l$ is left-invariant,
    we obtain
    \begin{align*}
        \int_\Group f((x, k) (y, l)) \dd (y, l)
        &= \int_\VectorSpace \int_\CompactGroup f(y, l) \dd l \dd y\\
        &= \int_\Group f(y, l) \dd (y, l),
    \end{align*}
    showing that $\dd y \dd l$ is indeed a Haar measure on $\Group$.

    Since by Proposition~\ref{proposition:sufficient_conditions_to_be_unimodular} $\dd l$ is also right-invariant,
    arguing similarly shows that $\dd y \dd l$ is also right-invariant.
\end{proof}

\begin{proposition}
    Let $\VectorSpace$ be a finite dimensional vector space
    and $\CompactGroup$ be a subgroup of $\SpecialOrthogonalGroup\VectorSpace$.
    The following properties hold:
    \begin{enumerate}
        \item The Lebesgue measure on $\VectorSpace$ is invariant under $\CompactGroup$,
            i.e.\ for every $k \in \CompactGroup$ and each Borel set $A \subset \VectorSpace$, we have
            \begin{align*}
                \int_A 1 \dd x = \int_{kA} 1 \dd x;
            \end{align*}
        \item The Laplacian on $\VectorSpace$ is invariant under $\CompactGroup$,
            i.e.\ for every $k \in \CompactGroup$ and every $\phi \in \SmoothFunctions\VectorSpace$, we have
            \begin{align*}
                \Laplacian[\VectorSpace] (\phi \circ k)(x) = \Laplacian[\VectorSpace] \phi(k x);
            \end{align*}
        \item The action of $\CompactGroup$ on $\VectorSpace$ commutes with the dilation structure of $\VectorSpace$.
    \end{enumerate}
\end{proposition}
\begin{proof}
    \begin{enumerate}
        \item This follows easily from the change of variables formula,
            \begin{align*}
                \int_{k A} 1 \dd x
                = \int_A (1 \circ k) \det k \dd x
                = \int_A 1 \dd x,
            \end{align*}
            where we used the fact that $\det k = 1$ since $k \in \SpecialOrthogonalGroup\VectorSpace$.
        \item First, observe that
            \begin{align*}
                \dd (\phi \circ k)(x) = \dd \phi(k x) k
            \end{align*}
            which implies that $\grad (\phi \circ k)(x) = k^{-1} \grad \phi(k x)$.

            From there, using the fact that $\Hessian = \dd \grad$,
            \begin{align*}
                \Hessian (\phi \circ k)(x) = (\dd \grad(\phi \circ k))(x) = k^{-1} (\Hessian \phi(k x)) k.
            \end{align*}

            Therefore, we conclude by observing that
            \begin{align*}
                \Laplacian[\VectorSpace] (\phi \circ k)(x)
                = \tr (\Hessian (\phi \circ k)(x))
                = \tr (\Hessian \phi (k x))
                = \Laplacian[\VectorSpace] \phi(k x).
            \end{align*}
    \end{enumerate}
\end{proof}

\section{Lie algebra structure}

We shall denote by $\LieAlgebraCompactGroup$ the Lie algebra of $\CompactGroup$.

Suppose, to simplify notation, that $V = \R^n$.
The map
\begin{align*}
    (x, k) \mapsto
        \begin{pmatrix}
            k & x\\
            0 & 1
        \end{pmatrix}
\end{align*}
defines a \emph{faithful representation}
and a diffeomorphism to a subgroup of $\GeneralLinear n$.

\begin{definition}[Lie algebra of $\Group$]
    We shall call the set
    \begin{align*}
        \LieAlgebra \defeq
        \VectorSpace \oplus \LieAlgebraCompactGroup
    \end{align*}
    the \emph{Lie algebra} of $\Group$.  
    Given $(X_1, Y_1), (X_2, Y_2) \in \LieAlgebra$,
    we define its \emph{Lie bracket} via
    \begin{align*}
        \LieBracket{(X_1, Y_1)}{(X_2, Y_2)}
        \defeq \left(Y_1 X_2 - Y_2 X_1, \LieBracket[\LieAlgebraCompactGroup]{Y_1}{Y_2}\right).
    \end{align*}
\end{definition}

\begin{lemma}[Commutation relations]
    Suppose that $X_1$, $X_2$, $X \in \VectorSpace$ and $Y_1$, $Y_2$, $Y \in \LieAlgebraCompactGroup$.
    We have the following \emph{commutation relations}
    \begin{align*}
        \LieBracket{X_1}{X_2} = 0,\quad
        \LieBracket{Y}{X} = (Y X, 0),\quad
        \LieBracket{Y_1}{Y_2} = \LieBracket[\LieAlgebraCompactGroup]{Y_1}{Y_2}.
    \end{align*}
\end{lemma}

\begin{definition}[Exponential map]
\label{definition:exponential_map}
\index{motion group!exponential map}
    The \emph{exponential map} on $\Group$ is the map
    \begin{align*}
        \exp_\Group : \LieAlgebra \to \Group : (X, Y) \mapsto \left(\sum_{k = 0}^\infty \frac{Y^k}{{(k + 1)}!} X, \exp_\CompactGroup Y\right)
    \end{align*}
\end{definition}

\begin{proposition}
    Fix an orthonormal basis $e_1, \dots, e_n$ of $\VectorSpace$,
    and consider the map
    \begin{align*}
        \Phi : \Group \to \GeneralLinear{n + 1} : (x, k) \mapsto
        \begin{pmatrix}
            \tilde k & \tilde x\\
            0 & 1,
        \end{pmatrix}
    \end{align*}
    where $\tilde k \in \SpecialOrthogonalGroup n$ and $\tilde x \in \R^n$ satisfy
    \begin{align*}
        {\tilde k}_{ij} = \ip {k e_j} {e_i} \text{ and } {\tilde x}_i = \ip x {e_i}
    \end{align*}
    for all $i, j \in \{1, \dots, n\}$.

    The following properties hold.
    \begin{enumerate}
        \item $\Phi$ is a faithful representation of $\Group$.
        \item The map
            \begin{align*}
                \eval {\dd \Phi} {(0, \Id \VectorSpace)} : \LieAlgebra \to \Lie(\Phi(\Group))
            \end{align*}
            is an isomorphism of Lie algebras.
        \item The exponential map $\exp_\LieAlgebra$ satisfies
            \begin{align*}
                (\Phi \circ \exp_\LieAlgebra) (X) = \sum_{j = 0}^{+\infty} \frac {{(\eval {\dd \Phi} {(0, \Id \VectorSpace)}(X))}^j} {j!}
            \end{align*}
            for every $X \in \LieAlgebra$.
    \end{enumerate}
\end{proposition}
\begin{proof}
    Assume without loss of generality that $\VectorSpace = \R^n$
    and that the chosen basis is the canonical one so that $k = \tilde k$ and $x = \tilde x$.

    \begin{enumerate}
        \item
            Suppose $(x, k), (y, l) \in \Group$.
            It follows that
            \begin{align*}
                \Phi(x, k) \Phi(y, l)
                &=
                \begin{pmatrix}
                    k & x\\
                    0 & 1
                \end{pmatrix}
                \begin{pmatrix}
                    l & y\\
                    0 & 1
                \end{pmatrix}\\
                &=
                \begin{pmatrix}
                    kl & ky + x\\
                    0 & 1
                \end{pmatrix}
                = \Phi((x, k) (y, l)),
            \end{align*}
            so $\Phi$ is indeed a group representation.
            The fact that $\Phi$ is faithful is trivial.
        \item
            Suppose $X \oplus Y \in \LieAlgebra$.
            We can check that
            \begin{align*}
                \eval {\dd \Phi} {(0, \Id \VectorSpace)} (X \oplus Y)
                &=
                [(\LeftDifferentialOperatorFirstOrder X + \LeftDifferentialOperatorFirstOrder Y) \Phi]
                (0_\VectorSpace, \Id \VectorSpace)\\
                &=
                \begin{pmatrix}
                    0 & X\\
                    0 & 0
                \end{pmatrix}
                +
                \begin{pmatrix}
                    Y & 0\\
                    0 & 0
                \end{pmatrix}\\
                &=
                \begin{pmatrix}
                    Y & X\\
                    0 & 0
                \end{pmatrix}.
            \end{align*}

            Then, we can check that
            \begin{align*}
                &\eval {\dd \Phi} {(0, \Id \VectorSpace)} (\LieBracket {(X_1, Y_1)} {(X_2, Y_2)})\\
                &\quad= \eval {\dd \Phi} {((0, \Id \VectorSpace)} (Y_1 X_2 - Y_2 X_1, Y_1 Y_2 - Y_2 Y_1)\\
                &\quad=
                \begin{pmatrix}
                    Y_1 Y_2 & Y_1 X_2\\
                    0 & 0
                \end{pmatrix}
                -
                \begin{pmatrix}
                    Y_2 Y_1 & Y_2 X_1\\
                    0 & 0
                \end{pmatrix}\\
                &\quad=
                \begin{pmatrix}
                    Y_1 & X_1\\
                    0 & 0
                \end{pmatrix}
                \begin{pmatrix}
                    Y_2 & X_2\\
                    0 & 0
                \end{pmatrix}
                -
                \begin{pmatrix}
                    Y_2 & X_2\\
                    0 & 0
                \end{pmatrix}
                \begin{pmatrix}
                    Y_1 & X_1\\
                    0 & 0
                \end{pmatrix},
            \end{align*}
            which means that we have
            \begin{align*}
                &\eval {\dd \Phi} {(0, \Id \VectorSpace)} (\LieBracket {(X_1, Y_1)} {(X_2, Y_2)})\\
                &\quad =
                \LieBracket [\eval {\dd \Phi} {(0, \Id \VectorSpace)} (\LieAlgebra)]
                {\eval {\dd \Phi} {(0, \Id \VectorSpace)} (X_1, Y_1)}
                {\eval {\dd \Phi} {(0, \Id \VectorSpace)} (X_2, Y_2)}.
            \end{align*}
        \item
            Suppose again that $X \oplus Y \in \LieAlgebra$.
            We can easily check by induction that
            \begin{align*}
                (\eval {\dd \Phi} {(0, \Id \VectorSpace)} (X, Y))^k
                =
                \begin{pmatrix}
                    Y^k & Y^{k - 1} X\\
                    0 & 0
                \end{pmatrix}
            \end{align*}
            for $k \geq 1$.

            Therefore, it follows that
            \begin{align*}
                \sum_{k = 0}^{+\infty} \frac 1 {k!}
                (\eval {\dd \Phi} {(0, \Id \VectorSpace)} (X, Y))^k
                &=
                \begin{pmatrix}
                    \Id \VectorSpace & 0\\
                    0 & 1
                \end{pmatrix}
                +
                \sum_{k = 1}^{+\infty} \frac 1 {k!}
                \begin{pmatrix}
                    Y^k & Y^{k - 1} X\\
                    0 & 0
                \end{pmatrix}\\
                &=
                \begin{pmatrix}
                    \exp_\LieAlgebraCompactGroup Y & \sum_{k = 0}^{+\infty} \frac{1}{(k + 1)!} Y^{k} X\\
                    0 & 1
                \end{pmatrix}\\
                &= (\Phi \circ \exp_\LieAlgebra) (X, Y),
            \end{align*}
            which is what we wanted to show.
    \end{enumerate}
\end{proof}

\begin{definition}[Left and right-invariant vector fields]
\label{definition:invariant_differential_operators}
    Let $X \in \LieAlgebra$.
    We define $\LeftDifferentialOperatorFirstOrder{X}$,
    the \emph{left-invariant differential operator associated with $X$}, via
    \begin{align*}
        \LeftDifferentialOperatorFirstOrder{X} f(g)
            \defeq \eval{\D*{1}{t}}{t = 0} f(g \exp_\Group(t X)),
    \end{align*}
    for each $f \in \SmoothFunctions{\Group}$.

    Similarly,
    we define $\RightDifferentialOperatorFirstOrder{X}$,
    the \emph{right-invariant differential operator associated with $X$}, via
    \begin{align*}
        \RightDifferentialOperatorFirstOrder{X} f(g)
            \defeq \eval{\D*{1}{t}}{t = 0} f(\exp_\Group(t X) g),
    \end{align*}
    where $f \in \SmoothFunctions{\Group}$.
\end{definition}

\begin{proposition}
    Let $X \in \LieAlgebra$.
    The differential operator $\LeftDifferentialOperatorFirstOrder{X}$ is the only differential operator satisfying the following properties:
    \begin{enumerate}
        \item $\LeftDifferentialOperatorFirstOrder{X}$ is \emph{left-invariant},
            i.e. for every $h \in \Group$, we have
            \begin{align*}
                (X f(h \dummy))(g) = (X f)(h g);
            \end{align*}
        \item The vector in $T_e \Group$ corresponding to the differentiation by $\LeftDifferentialOperatorFirstOrder{X}$ at $e$ is precisely $X$.
        \item Given $X, Y \in \LieAlgebra$, we have
            \begin{align*}
                \eval{%
                    (\LeftDifferentialOperatorFirstOrder{X} \LeftDifferentialOperatorFirstOrder{Y} - \LeftDifferentialOperatorFirstOrder{Y} \LeftDifferentialOperatorFirstOrder{X})
                }{e}
                = \LeftDifferentialOperatorFirstOrder{\LieBracket{X}{Y}}.
            \end{align*}
    \end{enumerate}
\end{proposition}

\begin{example}[$2$-dimensional Euclidean motion group]
\label{example:Lie_Algebra_of_2-dimensional_Euclidean_motion_group}
    Assume $\Group = \R^2 \ltimes \T$.
    The Lie Algebra is the vector space
    \begin{align*}
        \LieAlgebra \defeq \R^2 \oplus \SkewSymmetric{\R^2}.
    \end{align*}
    
    The vectors
    \begin{align*}
        X_1 = (1, 0) &\oplus
            \begin{pmatrix}
                0 & 0\\
                0 & 0
            \end{pmatrix},\quad
        X_2 = (0, 1) \oplus
            \begin{pmatrix}
                0 & 0\\
                0 & 0
            \end{pmatrix},\\
        &X_3 = (0, 0) \oplus
            \begin{pmatrix}
                0 & -1\\
                1 &  0
            \end{pmatrix},
    \end{align*}
    form a basis of $\LieAlgebra$
    which satisfies the commutation relations
    \begin{align*}
        \LieBracket{X_1}{X_2} = 0,\quad
        \LieBracket{X_2}{X_3} = X_1,\quad
        \LieBracket{X_3}{X_1} = X_2.
    \end{align*}

    Moreover, if $f \in \SmoothFunctions{\Group}$,
    then the associated left-invariant operators act via
    \begin{align*}
        \LeftDifferentialOperatorFirstOrder{X_1} f(x, t)
            &= \cos(\turn t) \D{1}[f]{{x_1} }(x, t) + \sin(\turn t) \D{1}[f]{{x_2} }(x, t)\\
        \LeftDifferentialOperatorFirstOrder{X_2} f(x, t)
            &= -\sin(\turn t) \D{1}[f]{{x_1} }(x, t) + \cos(\turn t) \D{1}[f]{{x_2} }(x, t)\\
        \LeftDifferentialOperatorFirstOrder{X_3} f(x, t)
            &= \D{1}[f]{t}(x, t),
    \end{align*}
    where $(x, t) \in \R^2 \ltimes \T$.
\end{example}

\begin{example}[Euclidean motion groups]
    Assume $G = \R^n \ltimes \SpecialOrthogonalGroup n$.
    Its Lie algebra is the vector space
    \begin{align*}
        \LieAlgebra \defeq \R^n \oplus \SkewSymmetric {\R^n}.
    \end{align*}

    For each pair of distinct $i, j \in \{1, \dots, n\}$,
    we let $E_{ij}$ be the only matrix satisfying
    \begin{align*}
        E_{ij} e_k = \Kronecker i j e_j - \Kronecker j k e_i
    \end{align*}
    for each $k \in \{1, \dots, n\}$.

    A basis of $\LieAlgebra$ can now be given by
    \begin{align*}
        X_i &\defeq e_i \oplus 0_{\SquareMatrices n}, &\quad &i = 1, \dots, n\\
        X_{ij} &\defeq 0_{\R^n} \oplus E_{ij}, &\quad &i,j \in \{1, \dots, n\} \text{ with } i < j,
    \end{align*}
    and these vectors satisfy the following commutation relations
    \begin{align*}
        \LieBracket {X_i} {X_j} &= 0\\
        \LieBracket {X_{ij}} {X_{k}} &= \Kronecker i k X_j - \Kronecker j k X_i\\
        \LieBracket {X_{ij}} {X_{kl}} &= \Kronecker j l X_{i k} - \Kronecker j k X_{i l}
    \end{align*}
    where in the above $i < j$, $k < l$ and
    for the last commutation relation we additionally assume $i < k$.
\end{example}

\begin{lemma}
    Let $X \in \VectorSpace \subset \LieAlgebra$.
    If $\phi \in \SmoothFunctions\Group$,
    \begin{align*}
        \LeftDifferentialOperatorFirstOrder{X} \phi(x, k)
        &= \ip{k^{-1} \grad \phi(x, k)}{X}.
    \end{align*}
\end{lemma}
\begin{proof}
    By a simple calculation,
    \begin{align*}
        \LeftDifferentialOperatorFirstOrder{X} \phi(x, k)
        =& \eval{\D*{1}{t}}{t = 0} \phi((x, k) (t X, \Id\VectorSpace))\\
        =& \eval{\D*{1}{t}}{t = 0} \phi(x + t k X, k)\\
        =& \ip{\grad \phi(x, k)}{k X}.
    \end{align*}

    From there, it follows that:
    \begin{align*}
        \LeftDifferentialOperatorFirstOrder{X} \phi(x, k)
        = \ip{k^{-1} \grad \phi(x, k)}{X}.
    \end{align*}
\end{proof}

Since $\CompactGroup$ is compact,
$\LieAlgebraCompactGroup$ is \emph{reductive}.
Therefore, it follows that
\begin{align*}
    \LieAlgebra = \VectorSpace \oplus \mathfrak a \oplus \mathfrak s,
\end{align*}
where $\mathfrak a$ is \emph{abelian} and $\mathfrak s$ is \emph{semisimple}.
In particular, we can naturally define an inner product $\ip[\LieAlgebra]\dummy\dummy$ on $\LieAlgebra$ via the Killing form of $\mathfrak s$.

From now on,
$X_1, \dots, X_{\dim \Group}$ will denote a basis such that
\begin{enumerate}
    \item the collection is orthonormal with respect to $\ip[\LieAlgebra]\dummy\dummy$;
    \item If $1 \leq j \leq \dim \VectorSpace$, then $X_j \in \VectorSpace$,
        otherwise $X_j \in \LieAlgebraCompactGroup$.
\end{enumerate}

\begin{definition}
    Let $\alpha \in \N^{\dim \Group}$.
    We define the left-invariant differential operator $\LeftDifferentialOperator{\alpha}$ via
    \begin{align*}
        \LeftDifferentialOperator{\alpha} =
        \LeftDifferentialOperatorFirstOrder{X_1}^{\alpha_1} \dots
        \LeftDifferentialOperatorFirstOrder{X_{\dim \Group}}^{\alpha_{\dim \Group}}
    \end{align*}
\end{definition}

\begin{definition}[Left-invariant Laplacian]
\label{definition:left-invariant_Laplacian}
\index{Laplacian}
    The \emph{left-invariant Laplacian} $\Laplacian$ is the left-invariant differential operator
    \begin{align*}
        \Laplacian \defeq \sum_{j = 1}^{\dim \Group} \LeftDifferentialOperatorFirstOrder{X_j}^2
    \end{align*}
\end{definition}

\begin{proposition}
    Let $f \in \SmoothFunctions\Group$.
    If $(x, k) \in \Group$, we check that
    \begin{align*}
        \Laplacian f(x, k) = \Laplacian[\VectorSpace] f(x, k) + \Laplacian[\CompactGroup] f(x, k)
    \end{align*}
\end{proposition}
\begin{proof}
    Let $(x, k) \in \Group$.
    By ,
    we check that for each $j \in \{1, \dots, \dim \VectorSpace\}$,
    \begin{align*}
        \LeftDifferentialOperatorFirstOrder{X_j} \LeftDifferentialOperatorFirstOrder{X_j} f(x, k)
        &= \LeftDifferentialOperatorFirstOrder{X_j} \ip{k^{-1} \grad f(x, k)}{e_j}\\
        &= \ip{k^{-1} \dd \grad f(x, k)[k e_j]}{e_j}\\
        &= \ip{k^{-1} \Hessian f(x, k) k e_j}{e_j}.
    \end{align*}

    Summing with respect to $j$, we get that
    \begin{align*}
        \sum_{j = 0}^{\dim \VectorSpace} \LeftDifferentialOperatorFirstOrder{X_j} \LeftDifferentialOperatorFirstOrder{X_j} f(x, k)
        = \tr \Hessian f(x, k) = \Laplacian[\VectorSpace] f(x, k).
    \end{align*}
\end{proof}

\section{Unitary representations}

\begin{definition}
\label{definition:reducible_representation}
    Let $\lambda \in \VectorSpace$.
    We define a unitary representation $\Rep{\lambda} \in \Hom(\Group, \End(\Lebesgue{2}{\CompactGroup}))$ of $\Group$ via
    \begin{align}
        \Rep{\lambda} (x, k) F(u) \defeq \e^{\i \turn \ip \lambda {u x}} F(u k),
        \label{eq:reducible_representations_on_the_motion_groups}
    \end{align}
    where $(x, k) \in \CompactGroup$, $F \in \Lebesgue{2}{\CompactGroup}$ and $u \in \CompactGroup$.
\end{definition}

Unfortunately, the above representation is often reducible.
However, as we shall see later, the Fourier Transform on $\Group$ can be written exclusively with those representations.

\begin{example}[$2$-dimensional Euclidean motion group]
    Let $\lambda \in \R^2$.
    Let $(x, t) \in \Group = \R^2 \ltimes \SpecialOrthogonalGroup{2}$.
    If $\lambda \neq 0$, then $\Rep{\lambda}$ is irreducible.

    Using the isomorphism $\SpecialOrthogonalGroup{2} \sim \T$,
    \eqref{eq:reducible_representations_on_the_motion_groups} takes the form
    \begin{align*}
        \Rep\lambda(x, t) F(u)
        = \e^{\i \turn \ip \lambda {\e^{\i \turn u} x}} F(u + t),
    \end{align*}
    where $F \in \Lebesgue{2}{\T}$, $(x, t) \in \Group$, and $u \in \T$.

    Defining ${\Rep\lambda(x, t)}_{m n} \defeq \ip[\Lebesgue{2}\T]{\Rep\lambda(x, t) \e^{\i \turn n \dummy}}{\e^{\i \turn m \dummy}}$ for every $m, n \in \Z$,
    it follows that
    \begin{align*}
        {\Rep\lambda(x, t)}_{m n}
        &= \int_\T \e^{\i \turn \ip \lambda {\e^{\i \turn u} x}} \e^{\i \turn [n (u + t) - m u]} \dd u\\
        &= \e^{\i \turn n t} \int_\T \e^{\i \turn \ip \lambda {\e^{\i \turn u} x}} \e^{\i \turn (n - m)u} \dd u.
    \end{align*}
\end{example}

\begin{lemma}[Invariance property]
    Let $\lambda \in \VectorSpace$.
    For each $k \in \CompactGroup$, we have
    \begin{align*}
        \Rep{k \lambda}(y, l) = \LeftRegularRepresentation(k) \Rep{\lambda}(y, l) \LeftRegularRepresentation(k^{-1})
    \end{align*}
\end{lemma}
\begin{proof}
    Let $F \in \Lebesgue{2}\CompactGroup$ and $u \in \CompactGroup$.
    It follows that
    \begin{align*}
        \LeftRegularRepresentation(k) \Rep\lambda(y, l) \LeftRegularRepresentation(k^{-1}) F(u)
        &= \LeftRegularRepresentation(k) \e^{\i \turn \ip \lambda {u x}} F(k u l)\\
        &= \e^{\i \turn \ip \lambda {k^{-1} u x}} F(u l)
        = \Rep{k \lambda}(y, l) F(u).
    \end{align*}
\end{proof}

\begin{lemma}
    Suppose $\Group = \R^2 \ltimes \SpecialOrthogonalGroup{2}$.
    If $(x, t) \in \Group$, then for each $m, n \in \Z$, we have
    \begin{align*}
        (m - n) {\Rep\lambda(x, t)}_{m n}
        = - \conj{x} \D{1}{{\conj{x}} } {\Rep\lambda(x, t)}_{mn}
    \end{align*}
\end{lemma}
\begin{proof}
    Using integration by parts, we obtain:
    \begin{align}
        (m - n) {\Rep\lambda(x, t)}_{m n}
    = -\frac{\e^{-\i \turn n t}}{\i \turn} \int_\T \lambda(-\e^{\i \turn u}x) \D*{1}{u} \e^{\i \turn (n - m)u} \dd u \notag\\
        = \frac{\e^{-\i \turn n t}}{\i \turn} \int_\T \D*{1}{u} \lambda(-\e^{\i \turn u}x) \e^{\i \turn (n - m)u} \dd u.
        \label{eq:lemma:off_diagonal_decay_in_dimension_2}
    \end{align}

    By simple calculations, we can show that
    \begin{align*}
        \D*{1}{u} \lambda(-\e^{\i \turn u} x) = {(-\i \turn)} \conj{x} \D{1}{{\conj{x}} } \lambda(-\e^{\i \turn u} x).
    \end{align*}
    From there, it follows that~\eqref{eq:lemma:off_diagonal_decay_in_dimension_2} becomes
    \begin{align*}
        (m - n) {\Rep\lambda(x, t)}_{m n}
        = - \conj{x} \D{1}{{\conj{x}} } {\Rep\lambda(x, t)}_{mn}
    \end{align*}
\end{proof}

%\subsection{Unitary dual}
%
%Our description of the unitary dual comes from \cite{Kumahara73},
%which itself is a minor adaptation of the one in \cite{Ito52}.
%Although both articles only specifically mention the case of the Euclidean motion group,
%the author of \cite{Ito52} mentions that the arguments generalise verbatim to motion groups.
%
%Throughout this section, fix $\lambda \in \dualGroup{\VectorSpace}$
%and denote by $\IsotropySubgroup{\CompactGroup}{\lambda}$ its isotropy subgroup.
%
%Let $\tau \in \dualGroup{\IsotropySubgroup{\CompactGroup}{\lambda}}$ and denote by $\dimRep{\tau}$ its dimension.
%For $q = 1, \dots, \dimRep{\tau}$, let
%\begin{align}
%    P^\tau_q F(u) \defeq \dimRep{\tau} \int_\IsotropySubgroup{\CompactGroup}{\lambda} \conj{\tau_{qq}(m)} F(u m) \dd m,
%    \quad F \in \Lebesgue{2}{\CompactGroup}, u \in \CompactGroup.
%    \label{eq:projection_on_L2_of_the_compact_group}
%\end{align}
%
%By the Inverse Fourier Transform at $e \in \IsotropySubgroup{\CompactGroup}{\lambda}$,
%if $F \in \Lebesgue{2}{\CompactGroup}$ and $u \in \CompactGroup$, then
%\begin{align*}
%    F(u)
%    &= F(u e) = \sum_{\tau \in \dualGroup{\IsotropySubgroup{\CompactGroup}{\lambda}}} \dimRep{\tau} \tr\left(\Rep[\IsotropySubgroup{\CompactGroup}{\lambda}]{\tau}(e) \Fourier[\IsotropySubgroup{\CompactGroup}{\lambda}]{F(u \dummy)}\right)\\
%    &= \sum_{\tau \in \dualGroup{\IsotropySubgroup{\CompactGroup}{\lambda}}} \dimRep{\tau} \sum_{q = 1}^{\dimRep{\tau}} {\Fourier[\IsotropySubgroup{\CompactGroup}{\lambda}]{F(u \dummy)}}_{qq}
%    = \sum_{\tau \in \dualGroup{\IsotropySubgroup{\CompactGroup}{\lambda}}} \sum_{q = 1}^\dimRep\tau P^\tau_q F(u).
%\end{align*}
%Now, writing $\Hilbert{\tau}{q} \defeq P^\tau_q \Lebesgue{2}{\CompactGroup}$,
%it then follows that
%\begin{align*}
%    \Lebesgue{2}{\CompactGroup}
%    = \sum_{\tau \in \dualGroup{\IsotropySubgroup{\CompactGroup}{\lambda}}} \sum_{q = 1}^\dimRep\tau \Hilbert{\tau}{q}.
%\end{align*}
%
%In fact, we shall see that the $P^\tau_q$ are \emph{orthogonal projections}.
%
%\begin{lemma}
%    Let $\mu, \tau \in \dualGroup{\IsotropySubgroup{\CompactGroup}{\lambda}}$, $q \in \{1, \dots, \dimRep{\tau}\}$ and $m, n \in \{1, \dots, \dimRep{\tau} \}$.
%    If $f \in \Lebesgue{2}{\RightQuotient{\CompactGroup}{\IsotropySubgroup{\CompactGroup}{\lambda}}}$, then
%    \begin{align*}
%        P^\mu_q (f \otimes \tau_{m n}) = \Kronecker{\mu}{\tau} \Kronecker{n}{q} (f \otimes \tau_{m n}).
%    \end{align*}
%
%    In particular, the following properties hold:
%    \begin{enumerate}
%        \item $P^\mu_q$ is an orthogonal projection onto
%            \begin{align*}
%                \Hilbert{\mu}{q} =
%                    \Lebesgue{2}{\RightQuotient{\CompactGroup}{\IsotropySubgroup{\CompactGroup}{\lambda}}}
%                    \otimes
%                    \Span \{\mu_{p q} : p = 1, \dots, \dimRep{\mu}\};
%            \end{align*}
%        \item The Hilbert spaces
%            \begin{align*}
%                \{\Hilbert{\tau}{q} : \tau \in \dualGroup{\IsotropySubgroup{\CompactGroup}{\lambda}}, q = 1, \dots, \dimRep{\tau} \}
%            \end{align*}
%            are mutually orthogonal;
%        \item We have the decomposition
%            \begin{align*}
%                \Lebesgue{2}{\CompactGroup} = \bigoplus_{\tau \in \dualGroup{\IsotropySubgroup{\CompactGroup}{\lambda}}} \bigoplus_{q = 1}^{\dimRep{\tau}} \Hilbert{\tau}{q}.
%            \end{align*}
%    \end{enumerate}
%\end{lemma}
%\begin{proof}
%    Fix $u \in \CompactGroup$ and write $u = u' u''$,
%    where $u' \in \RightQuotient{\CompactGroup}{\IsotropySubgroup{\CompactGroup}{\lambda}}$
%    and $u'' \in \IsotropySubgroup{\CompactGroup}{\lambda}$.
%    It follows that
%    \begin{align*}
%        P^\mu_q (f \otimes \tau_{m n}) (u)
%        &= \dimRep{\mu}
%            \int_\IsotropySubgroup{\CompactGroup}{\lambda}
%                \conj{\mu_{q q}(m)}
%                f(u')
%                \tau_{m n}(u'' m)
%            \dd m\\
%        &= \sum_{p = 1}^\dimRep{\tau}
%                f(u')
%                \tau_{m p}(u'')
%                \dimRep{\mu}
%                \int_\IsotropySubgroup{\CompactGroup}{\lambda}
%                    \conj{\mu_{q q}(m)}
%                    \tau_{p n}(m)
%                \dd m.
%    \end{align*}
%
%    Using the Peter-Weyl Theorem, we conclude that in fact
%    \begin{align*}
%        P^\mu_q (f \otimes \tau_{m n}) (u)
%        &= \sum_{p = 1}^\dimRep{\tau}
%            \Kronecker{\mu}{\tau}
%            \Kronecker{q}{p}
%            \Kronecker{q}{n}
%            f(u')
%            \tau_{m p}(u'')\\
%        &= \Kronecker{\mu}{\tau}
%            \Kronecker{q}{n}
%            f(u')
%            \tau_{m q}(u'')\\
%        &= \Kronecker{\mu}{\tau}
%            \Kronecker{q}{n}
%            (f \otimes \tau_{m n})(u),
%    \end{align*}
%    which is what we wanted to show.
%\end{proof}
%
%The following result and its proof can be found in \cite[Theorem 1.1, 1.2, 1.3]{Ito52}.
%Although the paper specifically treats the case of the Euclidean motion groups,
%the author remarks (\cite[Remark p. 84]{Ito52}) that the argument works for motion groups.
%
%\begin{proposition}[Unitary dual]
%\label{proposition:unitary_dual}
%    Let $\lambda, \lambda' \in \dualGroup{\VectorSpace} \setminus \{0\}$
%    and $\tau \in \dualGroup{\IsotropySubgroup{\CompactGroup}{\lambda}}$,
%    $\tau' \in \dualGroup{\IsotropySubgroup{\CompactGroup}{\lambda'}}$.
%    The following properties hold
%    \begin{enumerate}
%        \item $\Rep{\lambda}$ restricts to an infinite-dimensional irreducible unitary representation on each $\Hilbert{\tau}{q}$;
%        \item $(\Hilbert{\tau}{q}, \Rep{\lambda})$ and $(\Hilbert{\tau'}{q'}, \Rep{\lambda'})$ are equivalent if and only if
%            \begin{align*}
%                \lambda' = k \lambda \quad \text{and} \quad \EquivalenceClass{\dualGroup{\IsotropySubgroup{\CompactGroup}{\lambda}}}{\tau} = \EquivalenceClass{\dualGroup{\IsotropySubgroup{\CompactGroup}{\lambda}}}{\tau'(k \dummy k^{-1})}
%            \end{align*}
%            for some $k \in \CompactGroup$.
%            In particular, if $q_1, q_2 \in \{1, \dots, \dimRep\tau\}$,
%            then $(\Hilbert{\tau}{q_1}, \Rep{\lambda})$ and $(\Hilbert{\tau}{q_2}, \Rep{\lambda})$ are equivalent.
%    \end{enumerate}
%\end{proposition}
%
%\begin{definition}[Unitary dual]
%\label{definition:unitary_dual_of_motion_group}
%\index{motion group!unitary dual}
%    Fix $\lambda_0 \in \dualGroup{\VectorSpace} \setminus \{0\}$.
%    We define the \emph{unitary dual} of $\Group$, denoted by $\dualGroup{\Group}$, via
%    \begin{align*}
%        \dualGroup{\Group} \defeq \{ (\Hilbert{\tau}{1}, \Rep{\lambda \lambda_0}) : \lambda \in \R^+, \tau \in \dualGroup{\IsotropySubgroup{\CompactGroup}{\lambda}} \}.
%    \end{align*}
%\end{definition}
%
%\begin{definition}
%    A \emph{measurable field of operators on $\dualGroup\Group$} is a map
%    \begin{align*}
%        \sigma : \dualGroup\VectorSpace \to \End(\SmoothFunctions{\CompactGroup})
%    \end{align*}
%    satisfying the following properties.
%    \begin{enumerate}
%        \item For each $\lambda \in \dualGroup\VectorSpace$,
%            each $\tau \in \dualGroup{\IsotropySubgroup{\CompactGroup}{\lambda}}$, and each $q \in \{1, \dots, \dimRep \tau\}$, we have
%            \begin{align*}
%                \sigma(\lambda) \SmoothVectors{\Hilbert{\tau}{q}} \subset \SmoothVectors{\Hilbert{\tau}{q}}
%            \end{align*}
%        \item If $\lambda_1, \lambda_2 \in \dualGroup\VectorSpace$ and if $H_1, H_2$ are two Hilbert subspaces of $\Lebesgue{2}{\CompactGroup}$ such that
%            $(H_1, \Rep{\lambda_1})$ and $(H_2, \Rep{\lambda_2})$ are equivalent,
%            then if $T : H_1 \to H_2$ is the intertwining operator,
%            we have
%            \begin{align*}
%                \eval{\sigma(\lambda_1)}{H_1}  = T \sigma(\lambda_2) T^{-1}
%            \end{align*}
%    \end{enumerate}
%\end{definition}

\subsection{Infinitesimal representations}

\begin{definition}[Infinitesimal Representation]
\label{definition:infinitesimal_representation}
\index{motion group!infinitesimal representation}
    Let $X \in \g$.
    We define the infinitesimal representation of $X$ as the operator
    \begin{align*}
        \Rep{\lambda}(X) : \SmoothFunctions{\CompactGroup} \to \SmoothFunctions{\CompactGroup}
    \end{align*}
    defined via
    \begin{align*}
        \Rep{\lambda}(X) F(u) \defeq \eval{\D*{1}{t}}{t=0} \Rep{\lambda}(\exp(t X)) F(u),
    \end{align*}
    where $F \in \SmoothFunctions{\CompactGroup}$.
\end{definition}

\begin{proposition}[Infinitesimal representations]
\label{proposition:infinitesimal_representations_of_differential_operators}
    Let $\lambda \in \VectorSpace$ and let $j \in \{1, \dots, \dim \Group\}$.
    Fix also $F \in \SmoothFunctions \CompactGroup$ and $u \in \CompactGroup$.
    The infinitesimal representation of $X_j$ has the following expression:
    \begin{enumerate}
        \item if $j \leq \dim \VectorSpace$, then
            \begin{align*}
                \Rep{\lambda}(X_j) F(u) = \i \turn \ip \lambda {u e_j} F(u)
            \end{align*}
        \item if $j > \dim \VectorSpace$, then
            \begin{align*}
                \Rep{\lambda}(X_j) F(u) = \LeftDifferentialOperatorFirstOrder{X_j} F(u),
            \end{align*}
            where in the right-hand side $\LeftDifferentialOperatorFirstOrder{X_j}$ is the right-invariant differential operator on $\CompactGroup$ associated with $X_j \in \LieAlgebraCompactGroup$.
    \end{enumerate}
\end{proposition}
\begin{proof}
    Let $F \in \SmoothFunctions \CompactGroup$ and $u \in \CompactGroup$.
    \begin{enumerate}
        \item Fix $j \in \{1, \dots, \dim \VectorSpace\}$.
            Since $\exp_\Group(t X_j) = (t e_j, \Id{\VectorSpace})$.
            It follows that
            \begin{align*}
                \Rep{\lambda}(X_j) F(u) = \eval{\D*{1}{t}}{t = 0} \e^{\i \turn \ip \lambda {t u e_j}} F(u)
                = \i \turn \ip \lambda {u e_j} F(u)
            \end{align*}
            which is what we wanted to show.
        \item If $j > \dim \VectorSpace$, $X_j \in \LieAlgebraCompactGroup$ so that
            \begin{align*}
                \exp_\Group(t X_j) = (0, \exp_\CompactGroup(t X_j)).
            \end{align*}

            From there, it immediately follows that
            \begin{align*}
                \Rep{\lambda}(X_j) F(u)
                = \eval{\D*{1}{t}}{t = 0} F(u \exp_\CompactGroup(t X_j)),
            \end{align*}
            which by definition is $\LeftDifferentialOperatorFirstOrder{X_j} F(u)$.
    \end{enumerate}
\end{proof}

\begin{corollary}[Infinitesimal representation of $\Laplacian$]
\label{corollary:infinitesimal_representation_of_the_Laplacian}
    Let $\lambda \in \VectorSpace$.
    The infinitesimal representation of $\Laplacian$ is given by
    \begin{align*}
        \Rep{\lambda}(\Laplacian) = - {(\turn)}^2 \norm{\lambda}^2 \Id{\Lebesgue{2}{\CompactGroup}} + \Laplacian[\CompactGroup].
    \end{align*}
\end{corollary}

\section{Fourier Transform}

\subsection{Definition and elementary properties}

\begin{definition}[Schwartz space]
    We let $\Schwartz\Group$ be the space of all $f \in \SmoothFunctions\Group$
    which are \emph{rapidly decaying},
    i.e.\ such that for each $N \in \N$,
    \begin{align*}
        \seminorm[\Schwartz\Group]{N}{f} \defeq
        \sup_{\abs\alpha \leq N}
        \abs{%
            {(1 + \abs{x})}^N
            \LeftDifferentialOperator{\alpha} f(x, k)
        }
        < \infty.
    \end{align*}

    The maps ${(\seminorm[\Schwartz\Group]{N}{\dummy})}_{N \in \N}$ define a collection of semi-norms on $\Schwartz\Group$
    and give it a structure of \emph{Fr\'echet space}.
\end{definition}

\begin{definition}[Fourier transform]
\label{definition:Fourier_Transform}
\index{motion group!Fourier transform}
    Let $f \in \Lebesgue{1}{\Group}$ and $\lambda \in \dualGroup{\VectorSpace}$.
    We define its \emph{Fourier coefficient} at $\lambda$ via
    \begin{align*}
        \Fourier{f}(\lambda) \defeq \int_\Group f(g) \adj{\Rep{\lambda}(g)} \dd g.
    \end{align*}

    Moreover, the map
    \begin{align*}
        \Fourier{f} : \dualGroup{\VectorSpace} \to \End(\Lebesgue{2}{\CompactGroup}) :
        \lambda \mapsto \Fourier{f}(\lambda)
    \end{align*}
    is called the \emph{Fourier Transform} of $f$.
\end{definition}

\begin{lemma}
\label{lemma:kernels_of_Fourier_coefficients}
    Let $\lambda \in \VectorSpace$, and $f \in \Lebesgue{1}\Group$.
    The \emph{integral kernel} of the operator $\Fourier f(\lambda)$ is given by
    \begin{align}
        K_{f}(\lambda; u, k) \defeq \Fourier[\VectorSpace] f(k^{-1} \lambda, k^{-1} u),
        \label{integral_kernel_of_Fourier_coefficient}
    \end{align}
    i.e. for every $F \in \Lebesgue{2}\CompactGroup$ and every $u \in \CompactGroup$, we have
    \begin{align*}
        \Fourier f(\lambda) F(u) = \int_\CompactGroup K_f(\lambda; u, k) F(k) \dd k.
    \end{align*}

    In particular, the following properties hold:
    \begin{enumerate}
        \item if $f \in \Schwartz\Group$, then $K_f$ is smooth,
            $\Fourier f(\lambda)$ is trace class and
            \begin{align*}
                \tr(\Fourier f(\lambda)) = \int_\CompactGroup \Fourier[\VectorSpace] f(k \lambda, e) \dd k;
            \end{align*}
        \item if $f \in \Lebesgue{1}\Group \cap \Lebesgue{2}\Group$, then for almost every $\lambda \in \VectorSpace$,
            $\Fourier f(\lambda)$ is Hilbert-Schmidt and
            \begin{align*}
                \norm[\SchattenClasses{2}{\Lebesgue{2}\CompactGroup}]{\Fourier f(\lambda)}^2
                = \int_\CompactGroup \int_\CompactGroup \abs{\Fourier[\VectorSpace] f(k \lambda, u)}^2 \dd u \dd k
            \end{align*}
    \end{enumerate}
\end{lemma}
\begin{proof}
    Let $F \in \Lebesgue{2}\CompactGroup$ and $u \in \CompactGroup$.
    By definition of the Fourier Transform,
    \begin{align*}
        \Fourier f(\lambda) F(u) =
        \int_\VectorSpace
            \int_\CompactGroup
                f(x, k) \e^{-\i \turn \ip{k u^{-1} \lambda}{x}} F(u k^{-1})
            \dd k
        \dd x
    \end{align*}

    Recognising the Fourier Transform on $\VectorSpace$ in the above, we obtain
    \begin{align*}
        \Fourier f(\lambda) F(u)
        &=
        \int_\CompactGroup
            \Fourier[\VectorSpace] f(k u^{-1} \lambda, k) F(u k^{-1})
        \dd k\\
        &=
        \int_\CompactGroup
            \Fourier[\VectorSpace] f(k^{-1} \lambda, k^{-1} u) F(k)
        \dd k
    \end{align*}
    where we substituted $k$ for $k^{-1} u$ to obtain the last line.

    From there, it follows that the kernel is indeed given by~\eqref{integral_kernel_of_Fourier_coefficient}.
    Let us now prove the two remaining claims.

    \begin{enumerate}
        \item If $f \in \Schwartz\Group$, it follows that the integral kernel is smooth.
            Using~\cite[Corollary 4.1]{DelgadoRuzhansky14}, it follows that $\Fourier f(\lambda)$ is trace-class, and
            \begin{align*}
            \tr(\Fourier f(\lambda))
            = \int_\CompactGroup K_f(\lambda; k, k) \dd k
            = \int_\CompactGroup \Fourier[\VectorSpace] f(k \lambda, e) \dd k.
        \end{align*}
    \item Now, if $f \in \Lebesgue{1}\Group \cap \Lebesgue{2}\Group$,
        then $K_{f} \in \Lebesgue{2}{\CompactGroup \times \CompactGroup}$ for almost every $\lambda \in \VectorSpace$.
        For such $\lambda$, it follows by~\cite[Theorem VI.23]{Reed72} that $\Fourier f(\lambda)$ is Hilbert-Schmidt and
        \begin{align*}
            \norm[\SchattenClasses{2}{\Lebesgue{2}\CompactGroup}]{\Fourier f(\lambda)}^2
            &= \int_\CompactGroup \int_\CompactGroup \abs{K_f(\lambda; u, k)}^2 \dd k \dd u\\
            &= \int_\CompactGroup \int_\CompactGroup \abs{\Fourier[\VectorSpace] f(k^{-1} \lambda, k^{-1} u)}^2 \dd k \dd u.
        \end{align*}
        Substituing $k$ for $k^{-1}$ and then $k$ for $k^{-1}$ and then $k$ for $k^{-1}$ and then $k$ for $k^{-1}$ and then $u$ for $k^{-1} u$ in the above, we obtain
        \begin{align*}
            \norm[\SchattenClasses{2}{\Lebesgue{2}\CompactGroup}]{\Fourier f(\lambda)}^2
            &= \int_\CompactGroup \int_\CompactGroup \abs{\Fourier[\VectorSpace] f(k \lambda, u)}^2 \dd k \dd u,
        \end{align*}
        as required.
    \end{enumerate}
\end{proof}

\begin{corollary}
    Let $\lambda \in \VectorSpace$ and $f \in \Lebesgue 1 \Group$.
    If for every $l \in \CompactGroup$,
    \begin{align}
        f(l x, k) = f(x, k), \quad (x, k) \in \Group,
        \label{eq:function_on_motion_group_invariant_under_K}
    \end{align}
    then we have
    \begin{align*}
        \Fourier f(\lambda) = \Fourier [\GroupDirect] f(\lambda).
    \end{align*}
\end{corollary}
\begin{proof}
    By Lemma~\ref{lemma:kernels_of_Fourier_coefficients} and \eqref{eq:function_on_motion_group_invariant_under_K},
    we know that
    \begin{align*}
        \Fourier f(\lambda) F(u)
        &=
        \int_\CompactGroup \Fourier [\VectorSpace] f(\lambda, k^{-1} u) F(k) \dd k\\
        &=
        \int_\CompactGroup \Fourier [\VectorSpace] f(\lambda, k) F(u k^{-1}) \dd k,
    \end{align*}
    where the last line was obtained by substituing $k$ for $u k^{-1}$.

    Recognising a Fourier transform on $\CompactGroup$ is the right-hand side of the above,
    we obtain
    \begin{align*}
        \Fourier f(\lambda) F(u)
        &=
        \Fourier [\GroupDirect] f(\lambda) F(u),
    \end{align*}
    which shows that both Fourier transforms are indeed equal for $f$.
\end{proof}

%\begin{definition}
%    We shall say that $L \in \SmoothFunctions{\VectorSpace \times \CompactGroup \times \CompactGroup}$ belongs to $\ScalarImageSchwartz\Group$ if and only if
%    \begin{enumerate}
%        \item $L$ is rapidly decaying in $x$, i.e.\ for each $N \in \N$,
%            \begin{align*}
%                \seminorm[\ScalarImageSchwartz\Group]{N}{L}
%                \defeq
%                \sup_{\abs\alpha, \abs\beta, \abs{\beta'} \leq N}
%                \abs{%
%                    {(1 + \abs{x})}^N
%                    \LeftDifferentialOperatorOnCompactGroup[k_1]{\beta}
%                    \LeftDifferentialOperatorOnCompactGroup[k_2]{\beta'}
%                    \D[L]{x^\alpha}(x, k_1, k_2)
%                }
%                < \infty.
%            \end{align*}
%        \item for each $k \in \CompactGroup$, we have
%            \begin{align*}
%                L(k \lambda; k_1, k_2) = L(\lambda, k_1 k, k_2 k).
%            \end{align*}
%    \end{enumerate}
%\end{definition}
%
%The proof of the following result can be found in~\cite{Kumahara76}.
%\begin{proposition}
%    The map
%    \begin{align*}
%        f \in \Schwartz\Group \to K_f \in \ScalarImageSchwartz\Group
%    \end{align*}
%    is a topological isomorphism from $\Schwartz\Group$ onto $\ScalarImageSchwartz\Group$.
%\end{proposition}

\subsection{Plancherel formula}

\begin{proposition}[Plancherel formula]
\label{proposition:Plancherel_formula}
\index{motion group!Fourier transform!Plancherel formula}
    Let $f \in \Lebesgue{1}{\Group} \cap \Lebesgue{2}{\Group}$.
    The following formula holds
    \begin{align}
        \int_G \abs{f}^2 \dd g = \int_\dualGroup{\VectorSpace} \norm[\HilbertSchmidt{\Lebesgue{2}{\CompactGroup}}]{\Fourier{f}(\lambda)}^2 \dd \Plancherel{\VectorSpace}(\lambda).
        \label{proposition:Plancherel_formula:formula}
    \end{align}
\end{proposition}
\begin{proof}
    It follows from Lemma~\ref{lemma:kernels_of_Fourier_coefficients} that for almost every $\lambda \in \VectorSpace$,
    $\Fourier f(\lambda)$ is trace class and
    \begin{align*}
        \norm[\SchattenClasses{2}{\Lebesgue{2}\CompactGroup}]{\Fourier f(\lambda)}^2
        = \int_\CompactGroup \int_\CompactGroup \abs{\Fourier[\VectorSpace] f(k \lambda, u)}^2 \dd u \dd k.
    \end{align*}

    Now, integrating with respect to $\lambda$,
    we obtain
    \begin{align*}
        \int_\dualGroup{\VectorSpace} \norm[\HilbertSchmidt{\Lebesgue{2}{\CompactGroup}}]{\Fourier{f}(\lambda)}^2 \dd \Plancherel{\VectorSpace}(\lambda)
        &= \int_\dualGroup{\VectorSpace} \int_\CompactGroup \abs{\Fourier[\VectorSpace]{f}(\lambda, k)}^2 \dd k \dd \Plancherel{\VectorSpace}(\lambda)\\
        &= \int_\VectorSpace \int_\CompactGroup \abs{f(x, k)}^2 \dd u \dd k,
    \end{align*}
    where the last line was obtained by applying the Plancherel formula on $\VectorSpace$.
\end{proof}

%\begin{lemma}
%    Let $\phi \in \Schwartz\Group$.
%    For each $\lambda \in \VectorSpace$,
%    the operator $\Fourier \phi(\lambda)$ is trace class.
%    Moreover, for each $N \in \N$, there exists $C \geq 0$ such that
%    \begin{align*}
%        \norm[\SchattenClasses{1}{\Lebesgue{2}{\CompactGroup}}]{\Fourier \phi(\lambda)}
%        &\leq C {(1 + \abs\lambda)}^{-N}.
%    \end{align*}
%    In particular, the map
%    \begin{align*}
%        \lambda \in \VectorSpace \mapsto \norm[\SchattenClasses{1}{\Lebesgue{2}{\CompactGroup}}]{\Fourier \phi(\lambda)}
%    \end{align*}
%    is integrable.
%\end{lemma}
%\begin{proof}
%    Let $\alpha > \dim \CompactGroup$.
%    It follows by \cite[Proposition 3.3]{DelgadoRuzhansky14} that
%    \begin{align}
%        \BesselPotential[\CompactGroup]{-\alpha} \in \SchattenClasses{1}{\Lebesgue{2}\CompactGroup}
%        \label{lemma:preparation_for_inverse_formula:Bessel_potential_in_trace_class}
%    \end{align}
%
%    Let $\lambda \in \VectorSpace$.
%    We check that for each $F \in \Lebesgue{2}\CompactGroup$,
%    \begin{align*}
%        \BesselPotential[\CompactGroup]{\alpha} \Fourier \phi(\lambda) F(u)
%        = \int_\CompactGroup \BesselPotential[\CompactGroup]{\alpha}_u K_{\phi, \lambda}(u, k) F(k) \dd k,
%    \end{align*}
%    where $K_{\phi, \lambda}(u, k)$ represents the kernel of $\Fourier \phi(\lambda)$.
%
%    Since $K_{\phi, \lambda}$ is smooth with respect to $(u, k)$ and is rapidly decaying in $\lambda$,
%    it follows that for each $N \in \N$, there exists $C \geq 0$ such that
%    \begin{align*}
%        \norm[\Lin{\Lebesgue{2}\CompactGroup}]{\BesselPotential[\CompactGroup]{\alpha} \Fourier \phi(\lambda)}
%        \leq C {(1 + \abs\lambda)}^{-N}.
%    \end{align*}
%
%    Combining the above with~\eqref{lemma:preparation_for_inverse_formula:Bessel_potential_in_trace_class},
%    we obtain
%    \begin{align*}
%        \norm[\SchattenClasses{1}{\Lebesgue{2}{\CompactGroup}}]{\Fourier \phi(\lambda)}
%        &\leq
%        \norm[\SchattenClasses{1}{\Lebesgue{2}\CompactGroup}]{\BesselPotential[\CompactGroup]{-\alpha}}
%            \norm[\Lin{\Lebesgue{2}\CompactGroup}]{\BesselPotential[\CompactGroup]{\alpha} \Fourier \phi(\lambda)}\\
%        &\leq C {(1 + \abs\lambda)}^{-N},
%    \end{align*}
%    which is the desired estimate.
%\end{proof}

\begin{proposition}[Inverse Fourier Transform]
\label{proposition:inverse_Fourier_Transform}
\index{motion group!Fourier transform!inverse formula}
    Let $\phi \in \Schwartz{\Group}$.
    For each $g \in \Group$,
    we have
    \begin{align*}
        \phi(g)
        = \int_\dualGroup{\VectorSpace}
        \tr \left( \Rep{\lambda}(g) \Fourier \phi(\lambda) \right) \dd \Plancherel{\VectorSpace}(\lambda).
    \end{align*}
\end{proposition}
\begin{proof}
    Let us assume that $g = e$.
    By Lemma~\ref{lemma:kernels_of_Fourier_coefficients}, we know that $\Fourier \phi(\lambda)$ is trace class and
    \begin{align*}
        \tr(\Fourier \phi(\lambda))
        = \int_\CompactGroup \Fourier[\VectorSpace] \phi(k \lambda, e) \dd k.
    \end{align*}

    Integrating with respect to $\lambda$, we obtain
    \begin{align*}
        \int_\VectorSpace \tr(\Fourier \phi(\lambda)) \dd \lambda
        &= \int_\VectorSpace \int_\CompactGroup \Fourier[\VectorSpace] \phi(k \lambda, e) \dd k \dd \lambda\\
        &= \int_\VectorSpace \Fourier[\VectorSpace] \phi(\lambda, e) \dd \lambda,
    \end{align*}
    where the last line was obtained by a change of variables after permuting the integrals.

    Recognising an inverse Fourier Transform in the right-hand side of the above, we obtain
    \begin{align*}
        \int_\VectorSpace \tr(\Fourier \phi(\lambda)) \dd \lambda
        = \phi(0, e),
    \end{align*}
    concluding the case $g = e$.

    The general case follows immediately, since
    \begin{align*}
        \phi(g) = \phi(e g) = \int_\VectorSpace \tr(\Fourier \{\phi(\dummy g)\}(\lambda)) \dd \lambda
        = \int_\VectorSpace \tr(\Rep\lambda(g) \Fourier \phi(\lambda)) \dd \lambda,
    \end{align*}
    where the last equality was obained by Proposition~\ref{proposition:elementary_properties_of_the_Fourier_transform}.
\end{proof}

\section{Applications to engineering}

\subsection{Phase-noise Fokker-Planck equation}

Suppose that
\begin{align*}
    s : \R^+ \times \Omega \to \R
\end{align*}
is a Brownian motion on a probability space $(\Omega, \mathcal F, \mathbb P)$
that represents the (normalised) phase of a signal
\begin{align*}
    x : \R^+ \times \Omega \to \C : (t, \omega) \mapsto x(t) \defeq \e^{\i \turn s(t, \omega)}.
\end{align*}

Assume further that after going through a bandpass filter,
the signal $x$ becomes
\begin{align*}
    z : \R^+ \times \Omega \to \C : (t, \omega) \mapsto z(t, \omega) \defeq \int_0^t h(t') \e^{\i \turn s(t - t', \omega)} \dd t',
\end{align*}
where $h \in \Distributions {\R}$.

[Citation] modelises the probability density
\begin{align*}
    f : \R^+ \to \TemperedDistributions{\R^2 \times \T} : t \mapsto f(\dummy, t)
\end{align*}
of the random variable $(z, s)$ as the solution of
\begin{align}
    \begin{cases}
        \D{1}[f]{t} &= -h(t) \cos (\turn s) \D{1}[f]{{y_1} } - h(t) \sin(\turn s) \D{1}[f]{{y_2} } + C \D{2}[f]{s^2}\\
        f(\dummy; 0) &= \DiracDelta{(0_{\R^2}, 0)}.
    \end{cases}
    \label{eq:Fokker-Planck_phase_noise}
\end{align}

By Example~\ref{example:Lie_Algebra_of_2-dimensional_Euclidean_motion_group},
\eqref{eq:Fokker-Planck_phase_noise} can be rewritten
\begin{align}
    \begin{cases}
        \D{1}[f]{t}(y, s; t) &= -h(t) \LeftDifferentialOperatorFirstOrder {X_1}_{(y, s)} f(y, s; t) + C \LeftDifferentialOperatorFirstOrder {X_3}^2_{(y, s)} f(y, s; t)\\
        f(\dummy; 0) &= \DiracDelta{(0_{\R^2}, 0)}.
    \end{cases}
    \label{eq:Fokker-Planck_phase_noise:left-invariant_operators}
\end{align}

Applying the Fourier Transform on both sides of the above with respect to $(y, s)$,
we obtain

\begin{align}
    \begin{cases}
        \D{1}{t} \hat f(\lambda; t) &= A(\lambda, t) \hat f(\lambda; t)\\
        \hat f(\lambda; 0) &= \Id{\Lebesgue 2 \CompactGroup},
    \end{cases}
    \quad
    \label{eq:Fokker-Planck_phase_noise:Fourier_side}
\end{align}
where in the above $\hat f(\lambda, t) \defeq \Fourier f(\lambda, t)$, and
\begin{align*}
    A(\lambda, t) \defeq \left( -h(t) \Rep \lambda (X_1) + C \Rep \lambda ({X_3})^2 \right).
\end{align*}

\section{Fourier Transform of distributions}

\begin{definition}[Image of the Schwartz space]
    We shall denote by $\Schwartz{\dualGroup\Group}$
    the set of all functions $F \in \SmoothFunctions{\VectorSpace, \Lin{\Lebesgue{2}\CompactGroup}}$ such that:
    \begin{enumerate}
        \item For each $\alpha \in \N^{\dim \VectorSpace}$,
            $\D{\abs \alpha}[F]{\lambda^\alpha}$ leaves $\SmoothFunctions\CompactGroup$ stable;
        \item For each $N \in \N$, the quantity
            \begin{align*}
                \seminorm[\Schwartz{\dualGroup\Group}]{N}{F} \defeq
                \sup_{\abs\alpha, \abs\beta, \abs{\beta'} \leq N}
                &\sup_{\lambda \in \VectorSpace}
                {(1 + \abs\lambda)}^N\\
                &\norm[\Lin{\Lebesgue{2}\CompactGroup}]{%
                    \Rep\lambda(\LeftDifferentialOperator\beta)
                    \D{\abs \alpha}[F]{\lambda^\alpha}(\lambda)
                    \Rep\lambda(\LeftDifferentialOperator{\beta'})
                }
            \end{align*}
            is finite;
        \item For every $k \in \VectorSpace$ and each $\lambda \in \VectorSpace$,
            \begin{align*}
                F(k \lambda) = \LeftRegularRepresentation(k) F(k) \LeftRegularRepresentation(k^{-1}).
            \end{align*}
    \end{enumerate}
\end{definition}

\begin{lemma}
    The space $\Schwartz{\dualGroup\Group}$ is a Fr\'echet space
    whose topology is given by the seminorms $\seminorm[\Schwartz{\dualGroup\Group}]{N}{\dummy}$, $N \in \N$.
\end{lemma}

\begin{proposition}[Fourier transform and duality]
    Let $f \in \Schwartz\Group$, and $H \in \Schwartz{\dualGroup\Group}$.
    We have the identities
    \begin{align*}
        \ip[\dualGroup\Group]{\Fourier f}{H}
        = \ip[\Group]{f}{\iota \circ \InverseFourier H},\quad
        \ip[\Group]{\InverseFourier H}{f}
        = \ip[\dualGroup\Group]{H}{\Fourier (\iota \circ f)},
    \end{align*}
    where $(\iota \circ f)(g) = f(g^{-1})$.
\end{proposition}

\begin{definition}[Fourier Transform on distributions]
    Let $f \in \TemperedDistributions\Group$.
    We define the \emph{Fourier Transform} of $f$, $\Fourier f$,
    an element of $\TemperedDistributions{\dualGroup\Group}$, via
    \begin{align*}
        \ip[\dualGroup\Group]{\Fourier f}{H}
        \defeq \ip[\Group]{f}{\iota \circ \InverseFourier H},\quad
    \end{align*}
    for any $H \in \Schwartz{\dualGroup\Group}$.

    Similarly, let $H \in \TemperedDistributions{\dualGroup\Group}$.
    We define the tempered distribution $\InverseFourier H \in \TemperedDistributions\Group$,
    called the \emph{inverse Fourier Transform} of $H$, via
    \begin{align*}
        \ip[\Group]{\InverseFourier H}{f}
        = \ip[\dualGroup\Group]{H}{\Fourier (\iota \circ f)},
    \end{align*}
    for any $f \in \TemperedDistributions\Group$.
\end{definition}

\begin{proposition}
    The Fourier Transform is a topological linear isomorphism
    of $\TemperedDistributions\Group$ onto $\TemperedDistributions{\dualGroup\Group}$.
\end{proposition}

%\begin{definition}[$\LebesgueDual{2}{\Group}$]
%    We shall say that a map
%    \begin{align*}
%        \sigma : \dualGroup{\VectorSpace} \to \HilbertSchmidt{\Lebesgue{2}{\CompactGroup}}
%    \end{align*}
%    belongs to $\LebesgueDual{2}{\Group}$ if and only if the following conditions are met:
%    \begin{enumerate}
%        \item $\sigma$ is measurable;
%        \item for each $k \in \CompactGroup$, we have
%            \begin{align*}
%                \sigma(k \lambda) = R_k \sigma(\lambda) R_k^{-1}
%            \end{align*}
%        \item the quantity
%            \begin{align*}
%                \norm[\LebesgueDual{2}{\Group}]{\sigma} \defeq
%                    \left(
%                        \int_\dualGroup{\VectorSpace}
%                            \norm[\HilbertSchmidt{\Lebesgue{2}{\CompactGroup}}]{\sigma(\lambda)}^2
%                        \dd \Plancherel{\VectorSpace}(\lambda)
%                    \right)^{\frac{1}{2}}
%            \end{align*}
%            is finite.
%    \end{enumerate}
%
%    If $\sigma_1, \sigma_2 \in \LebesgueDual{2}{\Group}$, then we let
%    \begin{align*}
%        \ip[\LebesgueDual{2}{\Group}]{\sigma_1}{\sigma_2} \defeq
%        \int_\dualGroup{\VectorSpace}
%            \tr\left(
%                \sigma_1(\lambda) \adj{\sigma_2(\lambda)}
%            \right)
%        \dd \Plancherel{\VectorSpace}(\lambda).
%    \end{align*}
%    If we quotient $\LebesgueDual{2}{\Group}$ by $\Plancherel{\VectorSpace}$-almost everywhere equality,
%    which we shall do from now onwards,
%    then the above gives $\LebesgueDual{2}{\Group}$ the structure of a Hilbert space.
%\end{definition}
%
%\begin{definition}[$\Kernels{\Group}$]
%    We shall say that a tempered distribution $\kappa \in \TemperedDistributions{\Group}$ belongs to $\Kernels{\Group}$
%    if and only if the map
%    \begin{align}
%        T_\kappa : \Schwartz{\Group} \to \TemperedDistributions{\Group} : f \mapsto \conv{f}{\kappa}
%    \end{align}
%    extends to a continuous map from $\Lebesgue{2}{\Group}$ into itself.
%    In this case, we let
%    \begin{align*}
%        \norm[\Kernels{\Group}]{\kappa} \defeq \norm[\Lin{\Lebesgue{2}{\Group}}]{T_\kappa}.
%    \end{align*}
%\end{definition}
%
%\begin{definition}
%    We shall say that a map
%    \begin{align*}
%        \sigma : \dualGroup{\VectorSpace} \to \Lin{\Lebesgue{2}{\CompactGroup}}
%    \end{align*}
%    belongs to $\LebesgueDual{\infty}{\Group}$ if and only if it is invariant under $\CompactGroup$ and the quantity
%    \begin{align*}
%        \norm[\LebesgueDual{\infty}{\Group}]{\sigma} \defeq
%            \esssup_{\lambda \in \dualGroup{\VectorSpace}}
%                \norm[\Lin{\Lebesgue{2}{\CompactGroup}}]{\sigma(\lambda)}
%    \end{align*}
%    is finite.
%    The essential supremum is taken with respect to the Plancherel measure.
%\end{definition}
%
%\begin{theorem}[Abstract Plancherel formula]
%    The Fourier Transform can be extended to a \emph{surjective} isometry
%    \begin{align*}
%        \Fourier : \Lebesgue{2}{\Group} \to \LebesgueDual{2}{\Group}.
%    \end{align*}
%
%    Moreover, for every left-invariant operator $T \in \Lin{\Lebesgue{2}{\Group}}$,
%    there exists a unique element $\sigma \in \LebesgueDual{\infty}{\Group}$ such that
%    \begin{align*}
%        \Fourier\{T f\}(\lambda) = \sigma(\lambda) \Fourier f(\lambda)
%    \end{align*}
%    holds for $\Plancherel{\Group}$-almost every $\lambda \in \dualGroup{\VectorSpace}$.
%\end{theorem}

\subsection{Sobolev spaces}

\begin{definition}[Sobolev norm]
    Let $s \in \R$.
    If $\phi \in \Schwartz\Group$, we let
    \begin{align*}
        \norm[\Sobolev{s}]{\phi} \defeq
        \left(
            \int_\dualGroup\VectorSpace
                \norm[\HilbertSchmidt{\Lebesgue{2}{\CompactGroup}}]{%
                    \Rep\lambda \BesselPotential{s}
                    \Fourier \phi(\lambda)
                    }^2
            \dd \Plancherel\VectorSpace(\lambda)
        \right)^{1 / 2}.
    \end{align*}
\end{definition}

\begin{definition}[Sobolev spaces]
\label{definition:Sobolev_spaces}
    Let $s \in \R$.
    We define the \emph{Sobolev space of order $s$} to be the completion of $\Schwartz\Group$ with the norm $\norm[\Sobolev{s}]{\dummy}$.
\end{definition}

\subsubsection{Sobolev embeddings}

\begin{proposition}[Sobolev embedding]
\label{proposition:Sobolev_embedding}
    If $s > \dim \Group / 2$, then we have the following continuous inclusion
    \begin{align*}
        \Sobolev{s} \subset \ContinuousFunctions\Group \cap \Lebesgue{\infty}{\Group}.
    \end{align*}
    More precisely, there exists $C \geq 0$ such that the following property holds:
    for every $f \in \Sobolev{s}$,
    there exists a continuous function $\tilde{f} \in \ContinuousFunctions\Group$ such that $f = \tilde{f}$ almost everywhere and
    \begin{align*}
        \norm[\ContinuousFunctions\Group]{\tilde{f}} \leq C \norm[\Sobolev{s}]{f}.
    \end{align*}
\end{proposition}

\subsection{Fourier Transform of distributions}

For the sequel, we need to be able to take the Fourier Transform of certain distributions.
To this end, we follow the ideas of \cite{FischerRuzhansky15}.

\begin{definition}
    Let $a, b \in \R$.
    We shall say that $f \in \TemperedDistributions\Group$ belongs to $\KernelsSobolev{a}{b}$ if and only if the map
    \begin{align*}
        \Schwartz\Group \to \Schwartz\Group : \phi \to \conv{\phi}{f}
    \end{align*}
    is a continuous map in $\Lin{\Sobolev{a}, \Sobolev{b}}$.
\end{definition}

\begin{definition}
    Let $a, b \in \R$.
    We shall say that a map
    \begin{align*}
        \sigma : \dualGroup\VectorSpace \to \End(\SmoothFunctions{\CompactGroup})
    \end{align*}
    belongs to $\LebesgueDual[a, b]{\infty}{\Group}$ if and only if
    \begin{enumerate}
        \item TODO: Invariance condition
        \item The map
            \begin{align*}
                \lambda \in \dualGroup\VectorSpace \mapsto
                \Rep\lambda \BesselPotential{b} \sigma(\lambda) \Rep\lambda \BesselPotential{-a}
            \end{align*}
            belongs to $\LebesgueDual{\infty}{\Group}$.
    \end{enumerate}
\end{definition}

\begin{proposition}[Extension of the Fourier Transform]
    Define the sets
    \begin{align*}
        K \defeq \bigcup_{a, b \in \R} \KernelsSobolev{a}{b}, \quad
        L \defeq \bigcup_{a, b \in \R} \LebesgueDual[a, b]{\infty}{\Group}.
    \end{align*}

    The Fourier Transform can be extended as a bijective map
    \begin{align*}
        \Fourier : K \to L,
    \end{align*}
    and preserve the following properties.
    \begin{enumerate}
        \item If $f \in \Lebesgue{1}{\Group}$, then it coincides with Definition~\ref{definition:Fourier_Transform}.
        \item If $f_1, f_2 \in K$ are such that $\conv{f_1}{f_2} \in K$, then
            \begin{align*}
                \Fourier\{\conv{f_1}{f_2}\} = \Fourier f_2 \Fourier f_1.
            \end{align*}
        \item If $f \in \SmoothFunctions{\Group} \cap K$, $X \in \LieAlgebra$ and $\LeftDifferentialOperatorFirstOrder{X} f \in K$, then
            \begin{align*}
                \Fourier\{\LeftDifferentialOperatorFirstOrder{X} f\}(\lambda) = \Rep\lambda(X) \Fourier f(\lambda).
            \end{align*}
    \end{enumerate}
\end{proposition}

\section{Taylor formula}

\begin{proposition}[Taylor remainder]
    There exists an admissible collection of smooth functions $q_1, \dots, q_M \in \SmoothFunctions\Group$
    and a collection $\{\TaylorLeftDifferentialOperator{\alpha}\}_{\alpha \in \N^M}$ of left-invariant differential operators satisfying the following properties:
    \begin{enumerate}
        \item for each $\alpha \in \N^M$, $\TaylorLeftDifferentialOperator\alpha$'s order is less than $\abs\alpha$;
        \item if $f \in \SmoothFunctions\Group$ and $(x, k) \in \Group$,
            we have the following Taylor development
            \begin{align*}
                f(x, k) &= \sum_{\abs\alpha \leq N} \frac{1}{\alpha!} q^\alpha({(x, k)}^{-1}) \TaylorLeftDifferentialOperator\alpha f(0, e) + \BigO(h(x, k)^N),
            \end{align*}
            where $h : \Group \to \R^+$ denotes the geodesic distance to the identity.
    \end{enumerate}
\end{proposition}
\begin{proof}
    Let $f \in \SmoothFunctions\Group$.
    Choose an embedding
    \begin{align*}
        \iota : \CompactGroup \to \R^D.
    \end{align*}

    By REFERENCE, there exists an open neighbourhood $O \subset \R^D$ containing $\iota(\CompactGroup)$
    such that an orthogonal projection $p : O \to \iota(\CompactGroup)$ is defined and smooth.
    We can therefore extend $f$ onto $O$ via:
    \begin{align*}
        F : \VectorSpace \times O : (x, y) \mapsto f(x, (\iota^{-1} \circ p)(y)).
    \end{align*}

    By the Taylor Theorem on $\VectorSpace \times O$ at $(0, \iota(e))$, we get
    \begin{align*}
        F(x, y) =
        \sum_{\substack{\alpha = (\alpha_1, \alpha_2)\\ \abs\alpha \leq N}}
            \frac{1}{\alpha!}
            &x^{\alpha_1} {(y - \iota(e))}^{\alpha_2}
            \D{\abs \alpha}[F]{x^{\alpha_1}, y^{\alpha_2}}(0, \iota(e))\\
            &+ \BigO\left((\norm{x} + \norm[\R^N]{y - \iota(e)})^N\right).
    \end{align*}
    In particular, denoting by $h(g)$ the geodesic distance between $g \in \Group$ and $e \in \Group$,
    we observe that
    \begin{align}
        f(x, k) =
        \sum_{\substack{\alpha = \alpha_1 + \alpha_2\\ \abs\alpha \leq N}}
            \frac{1}{\alpha!}
            x^{\alpha_1} {(\iota(k) - \iota(e))}^{\alpha_2}
            \D{\abs \alpha}[F]{x^{\alpha_1}, y^{\alpha_2}}(0, \iota(e))
            + \BigO\left({h(x, k)}^N\right).
        \label{proposition:Taylor_remainder_theorem:Taylor_development}
    \end{align}

    Now, let
    \begin{align*}
        q_j(x, k) &= -\ip{x}{k e_i}, \quad &1 \leq &j \leq \dim \VectorSpace\\
        q_j(x, k) &= \ip[\R^N]{\iota(k^{-1}) - \iota(e)}{e_{j}}, \quad &\dim \VectorSpace < &j \leq \dim \VectorSpace + D
    \end{align*}
    where $e_1$, \dots, $e_{\dim V}$ is an orthonormal basis of $\VectorSpace$
    and $e_{\dim V + 1}$, \dots, $e_{\dim \VectorSpace + D}$ is an orthonormal basis of $\R^D$.
    Moreover, we know that for each $\alpha = (\alpha_1, \alpha_2)$,
    there exists a left-invariant differential operator $\TaylorLeftDifferentialOperator{\alpha}$ of order at most $\abs\alpha$ such that
    \begin{align*}
        \TaylorLeftDifferentialOperator{\alpha} f(e) = \D{\abs \alpha}[F]{x^{\alpha_1}, y^{\alpha_2}}(0, \iota(e)),
    \end{align*}
    since $F$ doesn't locally vary in the directions perpendicular to the tangent plane of $\iota(\CompactGroup)$.

    Now, let us check that if $\alpha = (\alpha_1, \alpha_2)$, then
    \begin{align*}
        q^\alpha({(x, k)}^{-1})
        = q^\alpha(-k^{-1} x, k^{-1})
        = x^{\alpha_1} {(\iota(k) - \iota(e))}^{\alpha_2}.
    \end{align*}

    It follows that~\eqref{proposition:Taylor_remainder_theorem:Taylor_development} becomes
    \begin{align*}
        f(x, k) &= \sum_{\abs\alpha \leq N} \frac{1}{\alpha!} q^\alpha({(x, k)}^{-1}) \TaylorLeftDifferentialOperator\alpha f(0, e) + \BigO(h(x, k)^N),
    \end{align*}
    which concludes our proof.
\end{proof}

\chapter{Symbols}

\section{Difference operators}

\section{Symbols and Kohn-Nirenberg quantization}

\section{Symbol classes}

\section{Link with the Hormander classes}

\section{Littlewood-Paley decomposition}

\begin{lemma}
\label{lemma:derivatives_of_radial_functions}
    Let $\alpha \in \N^n$,
    and fix a radial function $\chi \in \SmoothFunctions{\R^n}$.
    If $\alpha \in \N^n$, then
    \begin{align}
        \D[\chi]{x^\alpha}(x)
        = \sum_{r = 1}^{C_\alpha} f_r(\norm[\R^n]{x}) P_r(x),
    \end{align}
    where $P_r$ is a polynomial depending only on $\alpha$.

    Moreover, if $\supp \chi$ is compact
    and if there exists $\delta > 0$ such that $\D[\chi]{\lambda} = 0$ on $\Ball[\R^n]{0}{\delta}$,
    then we have
    \begin{align*}
        \sup_r \sup_{\lambda \in \R^+} \abs{f_r} < \infty
    \end{align*}
\end{lemma}
\begin{proof}
    Using the chain rule, we know that for a purely radial function $f$
    \begin{align}
        \D[f]{x_i} = \D[\lambda]{x_i} \D[f]{\lambda} = \frac{\D[f]{\lambda}}{\norm[\R^n]{x}} x_i.
    \end{align}

    We know proceed to show the claim by induction on $\alpha$.
    The result is clearly true when $\abs{\alpha} = 0$.
    If we assume it is true for some $\alpha \in \N^n$, then by the above,
    \begin{align}
        \D[\chi]{x_i,x^\alpha}(x)
        &= \D{x_i} \sum_{r = 1}^{C_\alpha} f_r(\norm[\R^n]{x}) P_r(x)\\
        &= \sum_{r = 1}^{C_\alpha} \frac{\D[f_r]{\lambda}}{\norm[\R^n]{x}}(\norm[\R^n]{x}) x_i P_r(x)
        + \sum_{r = 1}^{C_\alpha} f_r(\norm[\R^n]{x}) \D[P_r]{x_i}(x),
    \end{align}
    which concludes the proof.
\end{proof}

\begin{lemma}
\label{lemma:left_regular_representation_of_polynomials}
    Let $P \in \SmoothFunctions{\dualGroup{\VectorSpace}}$ be a polynomial.
    We can find functions $q_i \in \Polynomials{\CompactGroup}$, $f_i \in \Lebesgue{2}{\dualGroup{\VectorSpace}}$, $i = 1, \dots, N$ such that
    \begin{align*}
        P(k \lambda) = \sum_{i = 1}^N q_i(k) f_i(\lambda)
    \end{align*}
    for each $k \in \CompactGroup$ and each $\lambda \in \dualGroup{\VectorSpace}$.

    Moreover, the $q_i$ satisfy the bound
    \begin{align*}
        \sup_i \sup_\CompactGroup \abs{q_i} < \infty.
    \end{align*}
\end{lemma}
\begin{proof}
    Let $\mathcal{H}$ be the set of polynomials of order at most $m$ on $\dualGroup{\VectorSpace}$,
    with the norm
    \begin{align}
        {\left( \int_\dualGroup{\VectorSpace} \abs{P}^2 \right)}^\frac{1}{2}, \quad P \in \mathcal{H}.
    \end{align}

    Consider $L_m$ the left-regular representaion of $\CompactGroup$ on the above Hilbert space.
    Decomposing $L_m$ into irreducible representations
    \begin{align}
        \mathcal{H} = \bigoplus_{i = 1}^N \mathcal{H}_i,
        \quad L_m = \bigoplus_{i = 1}^N \left. L_m \right|_{\mathcal{H}_i},
    \end{align}
    then $P$ can be written as
    \begin{align*}
        P = \sum_{i = 1}^N \sum_{q = 1}^{d_i} c_{i, q} e_{i, q},
    \end{align*}
    and if $\tau_{i, pq}(k) = \int_\dualGroup{\VectorSpace} L_m(k) e_{i, q} \conj{e_{i, p}}$, then
    \begin{align*}
        P(k \lambda) &= L_m(k^{-1}) P(\lambda)
        = \sum_{i = 1}^N \sum_{q = 1}^{d_i} c_{i, q} L_m(k^{-1}) e_{i, q}(\lambda)\\
        &= \sum_{i = 1}^N \sum_{q = 1}^{d_i} c_{i, q} \tau_{i, pq}(k^{-1}) e_{i, p}(\lambda),
    \end{align*}
    which concludes the proof.
\end{proof}

\begin{theorem}[Littlewood-Paley decomposition]
\label{theorem:Littlewood-Paley_decomposition}
\index{Littlewood-Paley decomposition}
    Let $\AbelianGroup$ be a \emph{locally compact abelian Lie group},
    and suppose that $\CompactGroup \subset \Aut(\AbelianGroup)$ is a \emph{compact} group
    such that the \emph{Haar measure of $\AbelianGroup$ is invariant under the action of $\CompactGroup$}.

    We now consider the group $\Group = \AbelianGroup \rtimes \CompactGroup$.

    If there exists a sequence ${(\chi_j)}_{j \in \N} \subset \SmoothFunctions{\dualGroup\AbelianGroup}$ such that
    \begin{enumerate}
        \item they sum to 1, i.e.\ we have $\sum_{j = 0}^\infty \chi_j = 1$.
        \item the functions $\chi_j$, $j \in \N$, are invariant under $\CompactGroup$:
            for each $\lambda \in \dualGroup\AbelianGroup$ and each $k \in \CompactGroup$, we have $\chi_j(k \lambda) = \chi_j(\lambda)$.
        \item for each $q \in \Polynomials\AbelianGroup$,
            there exists a finite family $q_1, \dots, q_{C_q} \in \Polynomials\CompactGroup$ such that
            \begin{align*}
                q(k a) = \sum_{r = 1}^{C_q} f_r(a) q_j(k)
            \end{align*}
            for some bounded functions $f_r : A \to \C$, $r \in \{1, \dots, C_q\}$.
        \item There exists $C > 0$ such that for every $j \in \N$, we have
            \begin{align*}
                \chi_j(\lambda) = 0 \quad \text{if} \quad \norm[\dualGroup\AbelianGroup]{\lambda} \geq C 2^j.
            \end{align*}
    \end{enumerate}

    There exists a sequence $\eta_l \in \SmoothSymbols$, $l \in \N$ of smoothing symbols satisfying the following properties
    \begin{enumerate}
        \item the semi-norms decay in the following way
            \begin{align}
                \SymbolSemiNorm{m}{\rho, \delta}{\eta_l} \leq C 2^{-lm}
            \end{align}
        \item the associated kernels $\kappa_l$ satisfy
            \begin{align*}
                \sum_{l = 0}^\infty \kappa_l = \DiracDelta{e_\Group}
            \end{align*}
            in the sense of distributions.
    \end{enumerate}
\end{theorem}
\begin{proof}
    \begin{description}
        \item[Step 1] Constructing the dyadic decomposition.

            First, let us find a smooth function $\chi_0 \in \SmoothFunctions{\dualGroup{\VectorSpace}}$ invariant under $\CompactGroup$ such that
            \begin{align*}
                \chi_0(\lambda) = 1 \  \text{if}\  \norm[\dualGroup{\VectorSpace}]{\lambda} \leq 1, \quad \text{and} \quad
                \chi_0(\lambda) \equiv 0 \ \text{if}\  \norm[\dualGroup{\VectorSpace}]{\lambda} \geq 2.
            \end{align*}

            Then, for each $l \in \N$ satisfying $l \leq 1$, let
            \begin{align*}
                \chi_l = \chi_0(2^{-l} \dummy) - \chi_0(2^{-l + 1} \dummy).
            \end{align*}
            so that $\supp \chi_l \subset \Ball{0}{2^{l + 1}}$.

            In particular, it should be clear that
            \begin{align*}
                \sum_{l = 0}^N \chi_l = \chi_0(2^{-N} \dummy)
            \end{align*}
            so that in fact
            \begin{align}
                \sum_{l = 0}^\infty \chi_l = 1.
                \label{eq:theorem:Littlewood-Paley_decomposition:partition_of_unity}
            \end{align}

            Fix $l \in \N$.
            We define our symbol $\eta_l$ as follows.
            For each $\tau \in \dualGroup{\CompactGroup}$, we let
            \begin{align*}
                \eta_l(\lambda)
                = \sum_{\JapaneseBracket{\CompactGroup}{\tau} \leq 2^l}
                \chi_{l - \Ceiling{\log_2 \JapaneseBracket{\CompactGroup}{\tau}}}(\lambda) \Id{V_\tau},
            \end{align*}
            where $V_\tau = \Span \{ \tau_{ij} : i, j = 1, \dots, \dimRep{\tau} \}$.

            Note that since $\supp \chi_l \subset \Ball{0}{2^{l + 1}}$,
            we get that
            \begin{align}
                \eta_l(\lambda) = 0 \quad \text{if } \norm[\dualGroup{\CompactGroup}]{v} \geq 2^{l + 1}
                \label{eq:theorem:Littlewood-Paley_decomposition:cancellation_condition}
            \end{align}

            We check that
            \begin{align*}
                \sum_{l = 0}^\infty \eta_l
                &= \sum_{l = 0}^\infty
                    \sum_{\JapaneseBracket{\CompactGroup}{\tau} \leq 2^l}
                        \chi_{l - \Ceiling{\log_2 \JapaneseBracket{\CompactGroup}{\tau}}} \Id{V_\tau}\\
                &= \sum_{\tau \in \dualGroup{\CompactGroup}}
                    \sum_{l = \Ceiling{\log_2 \JapaneseBracket{\CompactGroup}{\tau}}}^\infty
                        \chi_{l - \Ceiling{\log_2 \JapaneseBracket{\CompactGroup}{\tau}}} \Id{V_\tau},
            \end{align*}
            where the last line was obtained by commuting the two sums.

            Substituing $l$ for $l + \Ceiling{\log_2 \JapaneseBracket{\CompactGroup}{\tau}}$ in the inner sum,
            the above becomes
            \begin{align*}
                \sum_{l = 0}^\infty \eta_l
                = \sum_{\tau \in \dualGroup{\CompactGroup}}
                    \sum_{l = 0}^\infty
                        \chi_l \Id{V_\tau}
                = \sum_{\tau \in \dualGroup{\CompactGroup}}
                    \Id{V_\tau}
                = \Id{\Lebesgue{2}{\CompactGroup}},
            \end{align*}
            where the second to last inequality was obtained from~\eqref{eq:theorem:Littlewood-Paley_decomposition:partition_of_unity}.

        \item[Step 2] Computing the associated kernels $\kappa_l$.

            By applying the inverse Fourier Transform (Proposition~\ref{proposition:inverse_Fourier_Transform})
            we obtain that the kernel is given by
            \begin{align}
                \kappa_l(x, k)
                = \sum_{\JapaneseBracket{\CompactGroup}{\tau} \leq 2^l}
                    \int_\dualGroup{\VectorSpace}
                        \chi_{l - \Ceiling{\log_2 \JapaneseBracket{\CompactGroup}{\tau}}}(\lambda) \tr( \left. \Rep{\lambda}(x, k) \right|_{V_\tau} )
                    \dd \Plancherel{\VectorSpace}(\lambda)
                \label{eq:theorem:Littlewood-Paley_decomposition:computing_kernel}
            \end{align}

            By the Peter-Weyl Theorem,
            $\{ \sqrt{\dimRep{\tau}} \tau_{pq} : p, q = 1, \dots, \dimRep{\tau} \}$
            is an orthonormal basis of $V_\tau$,
            allowing us to compute the trace as
            \begin{align*}
                \tr( \left. \Rep{\lambda}(x, k) \right|_{V_\tau})
                &= \sum_{p = 1}^\dimRep{\tau}
                    \dimRep{\tau}
                    \int_\Lebesgue{2}{\CompactGroup}
                    (u \lambda)(x) \tau_{pp}(k^{-1} u) \conj{\tau_{pp}(u)}
                    \dd u\\
                &= \sum_{p,q = 1}^\dimRep{\tau}
                    \dimRep{\tau}
                    \int_\Lebesgue{2}{\CompactGroup}
                        (u \lambda)(x) \tau_{pq}(k^{-1}) \tau_{q p}(u) \conj{\tau_{pp}(u)}
                    \dd u.
            \end{align*}

            Using the above in~\eqref{eq:theorem:Littlewood-Paley_decomposition:computing_kernel},
            and substituing $\lambda$ for $u^{-1} \lambda$,
            we obtain
            \begin{align}
                \kappa_l (x, k)
                = &\sum_{\JapaneseBracket{\CompactGroup}{\tau} \leq 2^l}
                        \sum_{p,q = 1}^\dimRep{\tau}
                        \dimRep{\tau}
                        \int_\dualGroup{\VectorSpace}
                                \int_\Lebesgue{2}{\CompactGroup} \notag\\
                                    &\chi_{l - \Ceiling{\log_2 \JapaneseBracket{\CompactGroup}{\tau}}}(u^{-1}\lambda) \lambda(x) \tau_{pq}(k^{-1}) \tau_{qp}(u) \conj{\tau_{pp}(u)}
                                \dd u
                            \dd \Plancherel{\VectorSpace}(\lambda)
                    \label{eq:theorem:Littlewood-Paley_decomposition:computing_kernel:2}
            \end{align}

            Using the invariance of $\chi_{k}$ under $\CompactGroup$ and
            \begin{align*}
                \dimRep{\tau} \int_\Lebesgue{2}{\CompactGroup} \tau_{qp}(u) \conj{\tau_{pp}(u)} \dd u = \Kronecker{p}{q},
            \end{align*}
            then~\eqref{eq:theorem:Littlewood-Paley_decomposition:computing_kernel:2} becomes
            \begin{align*}
                \kappa_l (x, k)
                = &\sum_{\JapaneseBracket{\CompactGroup}{\tau} \leq 2^l}
                    \int_\dualGroup{\VectorSpace}
                        \chi_{l - \Ceiling{\log_2 \JapaneseBracket{\CompactGroup}{\tau}}}(\lambda) \lambda(x)
                    \dd \Plancherel{\VectorSpace}(\lambda)
                    \conj{\Character{\tau}(k)}
            \end{align*}
            which, after recognising the inverse Fourier Transform on $\dualGroup{\VectorSpace}$,
            yields the following expression for the kernel
            \begin{align}
                \kappa_l (x, k)
                = &\sum_{\JapaneseBracket{\CompactGroup}{\tau} \leq 2^l}
                    \InverseFourier[\VectorSpace]{\chi_{l - \Ceiling{\log_2 \JapaneseBracket{\CompactGroup}{\tau}}}}(x) \conj{\Character{\tau}(k)}.
                \label{eq:theorem:Littlewood-Paley_decomposition:kernel}
            \end{align}

        \item[Step 3] We show that for every $q \in \Polynomials{\Group}$
            and every $\lambda \in \dualGroup{\VectorSpace}$,
            \begin{align}
                \sup_{l \in \N} \norm[\Lin{\Lebesgue{2}{\CompactGroup}}]{\DifferenceOperator{q} \eta_l(\lambda)} < \infty.
            \end{align}

            Fix $q_1 \in \Polynomials{\VectorSpace}$, $q_2 \in \Polynomials{\CompactGroup}$
            and write $q(x, k) = q_1(x) q_2(k)$.
            We also choose an arbitrary function $F \in \Lebesgue{2}{\CompactGroup}$,
            and an element $u \in \CompactGroup$.

            Informally, the idea behind the proof of this step is the following
            if we can write
            \begin{align*}
                \DifferenceOperator{q} \eta_l(\lambda) F(u)
                = \sum_{\tau \in \dualGroup{\CompactGroup}}
                    \dimRep{\tau}
                    \tr\left( \tau(u) \sigma_{l, \lambda, q, \lambda}(u, \tau) \Fourier[\CompactGroup] F(\tau) \right),
            \end{align*}
            then a bound on the operator norm of $\DifferenceOperator{q} \eta_l (\lambda)$ can be obtained by finding an appropriate bound on $\sigma_{l, q, \lambda}$.
            Looking at~\eqref{eq:theorem:Littlewood-Paley_decomposition:kernel},
            we can see the latter is the right approach
            as we have a sum on $\dualGroup{\CompactGroup}$ already.

            Multiplying $\kappa_l$ par $q$ and taking the Fourier Transform, we get
            \begin{align}
                \DifferenceOperator{q} \eta_l (\lambda) F(u)
                = &\sum_{\JapaneseBracket{\CompactGroup}{\tau} \leq 2^l}
                    \int_\VectorSpace
                        \int_\CompactGroup
                            q_1(x) \InverseFourier[\VectorSpace]{\chi_{l - \Ceiling{\log_2 \JapaneseBracket{\CompactGroup}{\tau}}}}(x) (k u \lambda)(-x)
                            q_2(k) \conj{\Character{\tau}(k)} F(k u)
                        \dd k
                    \dd x\notag\\
                = &\sum_{\JapaneseBracket{\CompactGroup}{\tau} \leq 2^l}
                    \int_\CompactGroup
                        \DifferenceOperator[\VectorSpace]{q_1} \chi_{l - \Ceiling{\log_2 \JapaneseBracket{\CompactGroup}{\tau}}}(k u \lambda)
                        q_2(k) \conj{\Character{\tau}(k)} F(k u)
                    \dd k \label{eq:theorem:Littlewood-Paley_decomposition:rho_condition},
            \end{align}
            where the second line was obtained by integrating with respect to $x$.

            Substituing $k$ for $k u^{-1}$ in the above,
            and using the Leibniz rule for polynomials on $q_2$, we obtain
            \begin{align*}
                \DifferenceOperator{q} \eta_l (\lambda) F(u)
                = &\sum_{\JapaneseBracket{\CompactGroup}{\tau} \leq 2^l}
                    \sum_{p = 1}^{C_q}
                        \int_\CompactGroup
                            \DifferenceOperator[\VectorSpace]{q_1} \chi_{l - \Ceiling{\log_2 \JapaneseBracket{\CompactGroup}{\tau}}}(k \lambda)
                            q_{2, p}(k) {q'}_{2, p}(u^{-1}) \conj{\Character{\tau}(k u^{-1})} F(k)
                        \dd k,
            \end{align*}

            \begin{claim}
                We have the decomposition
                \begin{align*}
                    \DifferenceOperator[\VectorSpace]{q_1} \chi_{l - \Ceiling{\log_2 \JapaneseBracket{\CompactGroup}{\tau}}}(k \lambda) = \sum_{r = 1}^{C_q} f_{l, r}(\tau, \lambda) q_r(k),
                \end{align*}
                where $q_r$ and $f_{l, r}$ satisfies the following bound
                \begin{align}
                    \sup_{l \in \N} \sup_{\tau \in \dualGroup{\CompactGroup}} \sup_{\lambda \in \dualGroup{\VectorSpace}} \abs{f_{l, r}(\tau, \lambda)} < \infty,\quad
                    \sup_{r} \sup_{k \in \CompactGroup} \abs{q(k)} < \infty
                    \label{eq:theorem:Littlewood-Paley_decomposition:claim_bound}
                \end{align}
            \end{claim}
            \begin{proof}[Proof of the claim]
                Assume first that $l - \Ceiling{\log_2 \JapaneseBracket{\CompactGroup}{\tau}} \neq 0$.
                Since
                \begin{align*}
                    \chi_{l - \Ceiling{\log_2 \JapaneseBracket{\CompactGroup}{\tau}}}(\lambda)
                    = \chi_1(2^{-l + \Ceiling{\log_2 \JapaneseBracket{\CompactGroup}{\tau} + 1}} \lambda),
                \end{align*}
                it follows that
                \begin{align*}
                    \DifferenceOperator[\VectorSpace]{q_1} \chi_{l - \Ceiling{\log_2 \JapaneseBracket{\CompactGroup}{\tau}}}(k \lambda)
                    =
                    2^{(-l + \Ceiling{\log_2 \JapaneseBracket{\CompactGroup}{\tau} + 1}) \order{q_1}}
                    \DifferenceOperator[\VectorSpace]{q_1} \chi_1(2^{-l + \Ceiling{\log_2 \JapaneseBracket{\CompactGroup}{\tau} + 1}} k \lambda).
                \end{align*}

                Using Lemma~\ref{lemma:derivatives_of_radial_functions} and Lemma~\ref{lemma:left_regular_representation_of_polynomials}
                we obtain
                \begin{align*}
                    \DifferenceOperator[\VectorSpace]{q_1} &\chi_{l - \Ceiling{\log_2 \JapaneseBracket{\CompactGroup}{\tau}}}(k \lambda) =
                    2^{(-l + \Ceiling{\log_2 \JapaneseBracket{\CompactGroup}{\tau} + 1}) \order{q_1}}\\
                    &\sum_{r = 1}^{C_q} f_r(2^{-l + \Ceiling{\log_2 \JapaneseBracket{\CompactGroup}{\tau} + 1}} \norm[\dualGroup{\VectorSpace}]{\lambda}) q_r(k) P_r(2^{-l + \Ceiling{\log_2 \JapaneseBracket{\CompactGroup}{\tau} + 1}} \lambda),
                \end{align*}
                where each $f_r$, $q_r$ and each $P_r$ is independent of $l$ and $\tau$.
                Now, writing
                \begin{align*}
                    f_{l, r}(\tau, \lambda) &=
                    2^{(-l + \Ceiling{\log_2 \JapaneseBracket{\CompactGroup}{\tau} + 1}) \order{q_1}}
                    f_r(2^{-l + \Ceiling{\log_2 \JapaneseBracket{\CompactGroup}{\tau} + 1}} \norm[\dualGroup{\VectorSpace}]{\lambda}) P_r(2^{-l + \Ceiling{\log_2 \JapaneseBracket{\CompactGroup}{\tau} + 1}} \lambda),
                \end{align*}
                we obtain the desired formula.
                The bound comes from the bounds in Lemma~\ref{lemma:derivatives_of_radial_functions} and~\ref{lemma:left_regular_representation_of_polynomials}.

                The case $l - \Ceiling{\log_2 \JapaneseBracket{\CompactGroup}{\tau}} = 0$ can be treated similarly.
            \end{proof}

            Using the above claim,
            and the identity $\conj{\Character{\tau}(k u^{-1})} = \sum_{i, j = 1}^\dimRep{\tau} \tau_{ij}(u) \conj{{\tau_{ij}(k)}}$,
            we observe that
            \begin{align}
                \DifferenceOperator{q} \eta_l (\lambda) F(u)
                = &\sum_{p, r = 1}^{C_q}
                    \sum_{\JapaneseBracket{\CompactGroup}{\tau} \leq 2^l}
                        \sum_{i, j = 1}^\dimRep{\tau}\notag\\
                            &\tau_{ij}(u) {q'}_{2, p}(u^{-1})
                            f_{l, r}(\tau, \lambda)
                            \int_\CompactGroup
                                q_{r}(k) q_{2, p}(k) F(k) \conj{\tau_{ij}(k)}
                            \dd k
                            \label{eq:theorem:Littlewood-Paley_decomposition:exact_expression_for_operator}\\
                = &\sum_{p, r = 1}^{C_q}
                    \sum_{\JapaneseBracket{\CompactGroup}{\tau} \leq 2^l}
                        \sum_{i, j = 1}^\dimRep{\tau}
                            \tau_{ij}(u) {q'}_{2, p}(u^{-1})
                            f_{l, r}(\tau, \lambda)
                            \Fourier[\CompactGroup]{} {\left\{ q_{r} q_{2, p} F\right\}}_{j i}(\tau).\notag
            \end{align}

            For $p, r = 1, \dots, \dimRep{\tau}$, defining the symbols
            \begin{align}
                \sigma_{l, \lambda, p, r}(u, \tau) =
                \begin{cases}
                    \frac{1}{\dimRep{\tau}} {q'}_{2, p}(u^{-1}) f_{l, r}(\tau, \lambda) \Id{\dimRep{\tau}} & \text{if } \JapaneseBracket{\CompactGroup}{\tau} \leq 2^l\\
                    0 & \text{otherwise}
                \end{cases}
            \end{align}
            and denoting by $T_{l, \lambda, p, r}$ the corresponding operators,
            we see that in fact,
            \begin{align*}
                \DifferenceOperator{q} \eta_l (\lambda) F(u)
                = &\sum_{p, r = 1}^{C_q}
                    \sum_{\tau \in \dualGroup{\CompactGroup}}
                        \dimRep{\tau}
                        \tr \left(
                            \tau(u)
                            \sigma_{l, \lambda, p, r}(u, \tau)
                            \Fourier[\CompactGroup]{} \left\{ q_r q_{2, p} F\right\}(\tau)
                        \right)\\
                = &\sum_{p, r = 1}^{C_q}
                        T_{l, \lambda, p, r} (q_r q_{2, p} F)(u).
            \end{align*}

            By~\eqref{eq:theorem:Littlewood-Paley_decomposition:claim_bound},
            we obtain
            \begin{align*}
                \norm[\Lin{\Lebesgue{2}{\CompactGroup}}]{T_{l, \lambda, p, r}}
                &\leq C \sup_{\tau \in \dualGroup{\CompactGroup}} \sup_{u \in \CompactGroup} \norm[\Lin{\HilbertRep{\tau}}]{\sigma_{l, \lambda, p, r}(u, \tau)}\\
                &\leq C_q < \infty
            \end{align*}
            From there, it follows that
            \begin{align*}
                \norm[\Lebesgue{2}{\CompactGroup}]{\DifferenceOperator{q} \eta_l (\lambda) F}
                &\leq C_q \sum_{p, r = 1}^{C_q} \norm[\Lebesgue{2}{\CompactGroup}]{q_r q_{2, p} F}
            \end{align*}
            where $C_q$ does not depend on $l$.

            Now, using the fact that
            \begin{align*}
                \sup_\CompactGroup \abs{q_r q_{2, p}} \leq C_q < \infty
            \end{align*}
            in the above, this concludes the step.

        \item[Step 4] $\ip[\Lebesgue{2}{\CompactGroup}]{\DifferenceOperator{q} \eta_l(\lambda) \mu_{mn}}{\nu_{kl}}$ is non-zero
            only if $\norm[\dualGroup{\VectorSpace}]{\lambda}, \JapaneseBracket{\CompactGroup}{\mu}, \JapaneseBracket{\CompactGroup}{\nu} \leq C_q 2^l$.

            Choose $C_q \geq 2$ so that
            \begin{align*}
                q_r q_{2, p}, q'_{2, p}(\dummy^{-1})
            \end{align*}
            can be generated by the representations $\{ \tau \in \dualGroup{\CompactGroup} : \JapaneseBracket{\CompactGroup}{\tau} \leq \frac{C_q}{2} \}$.

            Suppose now that $\max\{\JapaneseBracket{\CompactGroup}{\mu}, \JapaneseBracket{\CompactGroup}{\nu}\} > C_q 2^l$.
            It follows from our choice of $C_q$ that if $\JapaneseBracket{\CompactGroup}{\tau} \leq 2^l$,
            either of the following equations hold
            \begin{align*}
                \int_\CompactGroup q_r(k) q_{2, p}(k) \mu_{mn}(k) \conj{\tau_{ij}(k)} \dd k &= 0\\
                \int_\CompactGroup \tau_{ij}(u) q'_{2, p}(u) \conj{\nu_{mn}(k)} \dd k &= 0.
            \end{align*}

            From~\eqref{eq:theorem:Littlewood-Paley_decomposition:exact_expression_for_operator} with $F = \mu_{mn}$,
            we can see that the above implies
            \begin{align*}
                \ip[\Lebesgue{2}{\CompactGroup}]{\DifferenceOperator{q} \eta_l(\lambda) \mu_{mn}}{\nu_{kl}} = 0.
            \end{align*}

            The condition on $\lambda$ is obvious by~\eqref{eq:theorem:Littlewood-Paley_decomposition:rho_condition}
            keeping in mind that $\supp \chi_k \subset \Ball[\dualGroup{\VectorSpace}]{0}{2^{k + 1}}$.

        \item[Step 5] Conclusion.

            Let
            \begin{align*}
                L_l(\lambda) =
                \begin{cases}
                    {\left. \Rep{\lambda} \BesselPotential{1} \right|}_{\oplus_{\JapaneseBracket{\CompactGroup}{\tau} \leq C_q 2^l} V_\tau}
                    & \text{if } \norm[\dualGroup{\VectorSpace}]{\lambda} \leq C_q 2^l\\
                    0 & \text{otherwise},
                \end{cases}
            \end{align*}
            where $C_q$ is given by Step 4.

            Observe that $L_l(\lambda)$ is a bounded operator in $\Lebesgue{2}{\CompactGroup}$,
            and the operator norm is bounded by $C_q 2^l$ uniformly in $\lambda$.

            By Step 4,
            \begin{align*}
                \Rep{\lambda} \BesselPotential{- m - \order(q) + \gamma}&
                \DifferenceOperator{q} \eta_l(\lambda)
                \Rep{\lambda} \BesselPotential{-\gamma}\\
                &= {L_l(\lambda)}^{-m - \order(q) + \gamma}
                \DifferenceOperator{q} \eta_l(\lambda)
                {L_l(\lambda)}^{-\gamma},
            \end{align*}
            from which it follows that
            \begin{align*}
                \norm[\Lin{\Lebesgue{2}{\CompactGroup}}]{\Rep{\lambda} \BesselPotential{- m - \order(q) + \gamma} \DifferenceOperator{q} \eta_l(\lambda) \Rep{\lambda} \BesselPotential{-\gamma}}
                \leq C_q 2^{-l m},
            \end{align*}
            which is what we wanted to show.
    \end{description}
\end{proof}

\section{Kernel estimates}

\section{Adjoint and composition formulas}

\section{$L^2$ boundedness}

%\chapter{Towards groups with polynomial growth}

In this chapter,
$\Group$ will denote a connected locally compact Lie group,
for which we fix a non-zero left-invariant Haar measure $\mu = \dd g$.
Moreover,
we shall assume that $\Group$ has \emph{polynomial growth},
in the sense of the following definition.

\begin{definition}[Group of polynomial growth]
    We shall say that $\Group$ has \emph{polynomial growth}
    if for every compact neighbourhood $U \subset \Group$
    containing the identity $e_\Group$,
    there exists $C \geq 0$ such that
    \begin{align*}
        \mu(U^n) \leq C n^C
    \end{align*}
    for every $n \in \N$.
\end{definition}

This implies that $\Group$ is unimodular.


%\printindex

\emergencystretch=2em
\printbibliography

\nocite{*}

\end{document}
