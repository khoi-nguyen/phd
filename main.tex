\RequirePackage{silence}
\WarningFilter{scrbook}{Usage of package `titlesec'}
\WarningFilter{scrbook}{Activating an ugly workaround}
\WarningFilter{titlesec}{Non standard sectioning command detected}
\WarningFilter{biblatex}{Data encoding is 'utf8'}
% Warnings from arsclassica
\WarningFilter{hyperref}{Option `hyperfootnotes' has already been used}
\WarningFilter{hyperref}{Option `pdfpagelabels' has already been used}
\WarningFilter{hyperref}{Option `pdfpagelabels' has already been used}
\WarningFilter{scrlayer-scrpage}{Very small head height detected!}
\WarningFilter{scrlayer-scrpage}{\headheight to low}
\documentclass[dottedtoc, headinclude, footinclude=true]{scrbook}
\usepackage[pdfspacing]{classicthesis}
\usepackage{arsclassica}
\usepackage[backend=biber,style=alphabetic]{biblatex}

\usepackage{setspace}
\setstretch{1.25}

% Index
\usepackage{makeidx}
\usepackage[utf8]{inputenc}

\usepackage{amsmath, amsthm, amssymb, amsfonts}
\usepackage{mathtools}

\usepackage{xparse}

% Quotes
\def\signed #1{{\leavevmode\unskip\nobreak\hfil\penalty50\hskip2em
  \hbox{}\nobreak\hfil(#1)%
  \parfillskip=0pt \finalhyphendemerits=0 \endgraf}}

\newsavebox\mybox
\newenvironment{aquote}[1]
  {\savebox\mybox{#1}\begin{quote}}
  {\signed{\usebox\mybox}\end{quote}}

% Environments
\newtheorem{theorem}{Theorem}[section]
\newtheorem{application}[theorem]{Application}
\newtheorem{corollary}[theorem]{Corollary}
\newtheorem{definition}[theorem]{Definition}
\newtheorem{example}[theorem]{Example}
\newtheorem{lemma}[theorem]{Lemma}
\newtheorem{proposition}[theorem]{Proposition}
\newtheorem{remark}[theorem]{Remark}

\newcounter{claimcounter}
\numberwithin{claimcounter}{theorem}
\newenvironment{claim}{\stepcounter{claimcounter}{\emph{Claim \theclaimcounter:}}}{}

% Notation
\DeclarePairedDelimiter{\Ceiling}{\lceil}{\rceil}
\DeclarePairedDelimiter{\Floor}{\lfloor}{\rfloor}
\newcommand{\abs}[1]{\left|{#1}\right|}
\newcommand{\adj}[1]{{#1}^\star}
\newcommand{\conj}[1]{\overline{#1}}
\newcommand{\conv}[2]{#1 \star #2}
\newcommand{\dualGroup}[1]{{\widehat{#1}}}
\newcommand{\defeq}{\mathrel{\overset{\makebox[0pt]{\mbox{\normalfont\tiny\sffamily def}}}{=}}}
\newcommand{\grad}[1][\VectorSpace]{\nabla_{#1}}
\newcommand{\directionalDerivative}[1]{\partial_{#1}}
\newcommand{\dd}{\,\mathrm{d}}
\newcommand{\dimRep}[1]{{d_{#1}}}
\newcommand{\dimDifferenceOperators}[1][\Group]{n_{\Delta, #1}}
\newcommand{\dualBracket}[3][\Group]{{\langle #2, #3 \rangle}_{#1}}
\newcommand{\dummy}{\cdot}
\newcommand{\eval}[2]{\left. #1 \right|_{#2}}
\newcommand{\g}{\mathfrak{g}}
\newcommand{\lcsum}[1]{\sum_{#1}^{--}}
\newcommand{\norm}[2][\VectorSpace]{{\left\| #2 \right\|}_{#1}}
\newcommand{\seminorm}[3][\VectorSpace]{\norm[#1, #2]{#3}}
\newcommand{\ip}[3][\VectorSpace]{{\left(#2, #3\right)_{#1}}}
\newcommand{\transpose}[1]{{#1}^T}
\newcommand{\AbelianGroup}{A}
\newcommand{\AffineTransformations}[1]{{\mathrm{Affine} (#1)}}
\newcommand{\Ball}[3][\VectorSpace]{{B_{#1} (#2, #3)}}
\newcommand{\BesselPotential}[2][\Group]{(I - \Laplacian[#1])^\frac{#2}{2}}
\newcommand{\BesselPotentialSquared}[2][\Group]{(I - \Laplacian[#1])^{#2}}
\newcommand{\BesselPotentialKernel}[2][\Group]{\mathfrak B^{#1}_{#2}}
\newcommand{\BigO}{\mathcal{O}}
\newcommand{\Character}[1]{{\chi_{#1}}}
\newcommand{\Class}[2][\Group]{C^{#2}(#1)}
\newcommand{\ContinuousFunctions}[1]{{C(#1)}}
\newcommand{\DifferenceOperator}[2][\Group]{{\Delta^{#1}_{#2}}}
\newcommand{\DifferenceOperatorOrder}[2][\Group]{{\Delta_{#1}^{#2}}}
\newcommand{\DiracDelta}[1]{\delta_{#1}}
\newcommand{\Distributions}[1]{{\mathcal{D}' (#1)}}
\newcommand{\EquivalenceClass}[2]{{{[#2]}_{#1}}}
\newcommand{\Fourier}[1][\Group]{\mathcal{F}_{#1}}
\newcommand{\Group}{G}
\newcommand{\GroupDirect}{\VectorSpace \times \CompactGroup}
\newcommand{\GeneralLinear}[1]{\mathrm{GL}(#1)}
\newcommand{\Hil}{\mathcal{H}}
\newcommand{\Hilbert}[2]{\mathfrak{H}_{#1, #2}}
\newcommand{\HilbertRep}[1]{{\mathfrak{H}_{#1}}}
\newcommand{\HilbertCompactGroupColumn}[2]{\mathfrak{H}_{#1, #2}}
\newcommand{\HilbertCompactGroup}[1]{\mathfrak{H}_{#1}}
\newcommand{\InverseFourier}[1][\Group]{\mathcal{F}^{-1}_{#1}}
\newcommand{\CompactGroup}{K}
\newcommand{\HilbertSchmidt}[1]{{\mathcal{HS} \left(#1\right)}}
\newcommand{\Id}[1]{{I_{#1}}}
\newcommand{\InverseFunctionArgument}[1][\Group]{\iota_{#1}}
\newcommand{\IsotropySubgroup}[2]{{{#1}_{#2}}}
\newcommand{\JapaneseBracket}[2]{{\langle #2 \rangle}_{\dualGroup{#1}}}
\newcommand{\Kronecker}[2]{\delta_{#1,#2}}
\newcommand{\Kernels}[1][\Group]{\mathcal{K}(#1)}
\newcommand{\KernelsSobolev}[3][\Group]{\mathcal{K}_{#2, #3}(#1)}
\newcommand{\Lebesgue}[2]{{L^{#1} (#2)}}
\newcommand{\LebesgueDual}[3][]{{L^{#2}_{#1} (\dualGroup{#3})}}
\newcommand{\LeftDifferentialOperatorFirstOrder}[1]{{#1}}
\newcommand{\LeftDifferentialOperator}[2][]{X^{#2}\if #1\empty \else_{#1 }\fi}
\newcommand{\LeftDifferentialOperatorOnCompactGroup}[2][]{Y^{#2}_{#1}}
\newcommand{\LeftRegularRepresentation}[1][\CompactGroup]{\pi^L_{#1}}
\newcommand{\RightRegularRepresentation}[1][\CompactGroup]{\pi^R_{#1}}
\newcommand{\Lie}{\mathfrak{Lie}}
\newcommand{\LieAlgebra}{\mathfrak{g}}
\newcommand{\LieAlgebraCompactGroup}{\mathfrak{k}}
\newcommand{\LieAlgebraVectorSpace}{\mathfrak{v}}
\newcommand{\LieBracket}[3][\LieAlgebra]{{[#2, #3]}_{#1}}
\newcommand{\Lin}[1]{{\mathcal{L} (#1)}}
\newcommand{\Laplacian}[1][\Group]{{\mathcal{L}_{#1}}}
\newcommand \SquareMatrices [2][\R] {{#1}^{#2 \times #2}}
\newcommand{\MotionGroup}[1]{{\mathrm{SE} (#1)}}
\newcommand{\OrthogonalGroup}[1]{{\mathrm{O} (#1)}}
\newcommand{\Op}[1][\Group]{\mathrm{Op}_{#1}}
\newcommand{\Plancherel}[1]{\mu_{\dualGroup{#1}}}
\newcommand{\Polynomials}[1]{{\mathrm{Pol}_{#1}}}
\newcommand{\Projection}[1]{\mathrm{Proj}_{#1}}
\newcommand{\LeftQuotient}[2]{{{#1} \backslash{} {#2}}}
\newcommand{\RightQuotient}[2]{{{#1} \slash{} {#2}}}
\newcommand{\Rep}[2][\Group]{\xi^{#2}_{#1}}
\newcommand{\RightDifferentialOperatorFirstOrder}[1]{\tilde{#1}}
\newcommand{\RightDifferentialOperator}[2][]{\tilde X^{#2}\if #1\empty \else_{#1 }\fi}
\newcommand{\RightLaplacian}[1][\Group]{{\tilde{\mathcal{L}}_{#1}}}
\newcommand{\Rotation}[1]{\tilde R \left(#1\right)}
\newcommand{\InverseRotation}[1]{R\left(#1\right)}
\newcommand{\SmoothFunctions}[1]{{C^\infty(#1)}}
\newcommand{\SmoothVectors}[1]{#1^\infty}
\newcommand{\ScalarImageSchwartz}[1]{\tilde{\mathcal{S}}(#1)}
\newcommand{\SchattenClasses}[2]{S_{#1}(#2)}
\newcommand{\Schwartz}[1]{{\mathcal{S} (#1)}}
\newcommand{\SkewSymmetric}[1]{\mathrm{Skew}(#1)}
\newcommand{\Sobolev}[2][\Group]{L^2_{#2}(#1)}
\newcommand{\SpecialOrthogonalGroup}[1]{{\mathrm{SO} (#1)}}
\newcommand{\SpecialUnitaryGroup}[1]{{\mathrm{SU} (#1)}}
\newcommand{\TangentSpace}[2]{T_{#2} #1}
\newcommand{\TaylorLeftDifferentialOperator}[1]{X^{(#1)}_L}
\newcommand{\TaylorRemainder}[3]{R^{#1}_{#2, #3}}
\newcommand{\TemperedDistributions}[1]{{\mathcal{S}' (#1)}}
\newcommand{\UnitaryGroup}[1]{{\mathrm{U} (#1)}}
\newcommand{\LeftInvariantVectorFields}[1][\Group]{\mathfrak{X}_L(#1)}
\newcommand{\VectorFields}[1][\Group]{\mathfrak{X}(#1)}
\newcommand{\VectorSpace}{V}
\newcommand{\Volume}[1]{\mathrm{Vol}(#1)}

\DeclareMathOperator{\Aut}{Aut}
\DeclareMathOperator{\End}{End}
\DeclareMathOperator{\Hessian}{Hess}
\DeclareMathOperator{\Hom}{Hom}
\DeclareMathOperator{\Image}{Im}
\DeclareMathOperator{\Span}{span}
\DeclareMathOperator{\tr}{tr}
\DeclareMathOperator{\order}{order}
\DeclareMathOperator{\rank}{rank}
\DeclareMathOperator{\supp}{supp}
\DeclareMathOperator*{\esssup}{ess\,sup}

\makeatletter
\DeclareDocumentCommand \D{s m O{} m}{%
    % Choice of right d
    \IfBooleanTF{#1}{\def\@der{\dd}}{\def\@der{\partial}}
    % Write the derivative
    \mathchoice{%
        \frac{
            \@der\ifnum\pdfstrcmp{#2}{1}=0\else^{#2}\fi {#3}
        }{%
            \@for\@token:={#4}\do{\@der \@token}
        }
    } {%
        %\@for\@token:={#4}\do{\@der_\@token} #3
        \iD{#4} #3
    } {} {}
}
\DeclareDocumentCommand \iD {m}{%
    \@for\@token:={#1}\do{\partial_\@token}
}
\makeatother

% Sets
\newcommand{\C}{\mathbb{C}}
\newcommand{\N}{\mathbb{N}}
\newcommand{\R}{\mathbb{R}}
\newcommand{\T}{\mathbb{T}}
\newcommand{\Z}{\mathbb{Z}}

% Constants
\newcommand{\e}{e}
\newcommand{\turn}{2 \pi}
\renewcommand{\i}{i}
\renewcommand{\epsilon}{\varepsilon}

% Pseudo-differential calculus
\newcommand{\SmoothingSymbols}[1][\Group]{{S^{-\infty} (#1)}}
\newcommand{\SmoothingOperators}[1][\Group]{{\Psi^{-\infty} (#1)}}
\newcommand{\SymbolClass}[3][\Group]{S^{#2}_{#3}(#1)}
\newcommand{\Symbols}[1][\Group]{S(#1)}
\newcommand{\OperatorClass}[3][\Group]{\Psi^{#2}_{#3}(#1)}
\newcommand{\SymbolSemiNorm}[5][\Group]{\norm[S^{#2}_{#3}(#1), #4]{#5}}

\title{Pseudo-Differential Calculus on Generalised Motion Groups}
\author{Binh-Khoi Nguyen}

\makeindex

\addbibresource{Bibliography.bib}

\begin{document}

\maketitle

\chapter*{Copyright}

This thesis is distributed under the terms of the
\href{http://creativecommons.org/licenses/by-sa/4.0/}{Creative Commons Attribution-ShareAlike 4.0 International License}.
Anyone is free to use, share and adapt the following document
provided they give appropriate credit to the author,
and share their modifications under similar terms.
The last requirement is not part of the university policy,
but it is in my opinion crucial that \emph{anyone} be able to access, use and build upon our scientific heritage.

Prospective research students might wonder what it is like to write a thesis.
For this reason, I am releasing the \emph{complete history} of the \LaTeX\ sources of this document
under the \href{https://www.gnu.org/licenses/gpl.html}{GNU General Public License v3+}.
All the versions of this document since 13 December 2015,
when this document contained only one definition,
can be viewed at
\url{https://git.nguyen.me.uk/khoi/public/phd.git}.
For more experienced researchers, I hope this will encourage adoption of \emph{version control systems}.

\chapter*{Acknowledgements}

Last but not least,
I would like to express my sincere gratitude to my supervisor,
Professor Michael Ruzhansky,
for giving me the opportunity to work with him and for his continuous support, patience, motivation and insight.
I could not have completed this PhD without his belief in my abilities and words of encouragement.

\begin{itemize}
    \item Family and Chloe
    \item Augusto Ponce, Laurent Moonens, Jean Van Schaftingen: made me discover analysis.
    \item Pierre Bieliavsky + 3 professors above: wrote my recommendation letters
    \item Veronique Fischer
    \item Michael Ruzhansky
\end{itemize}


\tableofcontents

\chapter{Introduction}

Since their inception in the 1960s,
pseudo-differential operators have become standard and natural tools in the study of partial differential equations,
particularly when elliptic or hypoelliptic operators are concerned.

On $\R^n$,
these operators are defined \emph{globally} as arising from smooth maps called \emph{symbols} via the Euclidean Fourier Transform,
the latter generalising characteristic polynomials.
Their usefulness arises from the fact that they enjoy similar properties as their differential counterparts,
such as a notion of \emph{order} that behaves well under composition and adjunction,
as well as expected boundedness properties between the adequate Sobolev spaces.
Crucially,
the \emph{adjunction} and \emph{composition} formulae claim that
both these operations applied to pseudo-differential operators can be approximately expressed on the symbolic side via a formula that is formally virtually identical to the corresponding rule for characteristic polynomials.
In particular,
this means that elliptic pseudo-differential operators admit parametrices which are contained in the pseudo-differential calculus.
Overviews of the theory can be found in e.g.~\cite{Shubin01,Hormander07}.

While pseudo-differential operators can be defined locally via local charts on any connected manifold,
two issues arise.
Firstly,
operators are usually defined provided that a condition on the \emph{type} such as
\begin{align*}
    1 - \rho \leq \delta < \rho
\end{align*}
holds (see e.g.~\cite[Section 4]{Shubin01}),
while their Euclidean counterparts make sense without restrictions on $\rho$ and $\delta$.
Secondly,
a global notion of \emph{symbol} cannot be invariantly defined,
making results which rely on a full symbol (such as Gårding's inequality) difficult to establish.

The first successful attempt at defining a global pseudo-differential calculus
on a non-Euclidean manifold is due to \citeauthor{RuzhanskyTurunen10},
and their treatment of compact Lie groups.
They defined the global symbol of an operator via the group Fourier transform of the right-convolution kernel,
which in particular is  \emph{matrix}-valued
due to the non-commutativity of the group law.
Particularly instrumental to their success was their definition of \emph{difference operators},
which generalizes derivatives in Fourier variables from the Euclidean case.
It is worth remembering that,
while good properties of left-invariant pseudo-differential operators
are ensured by the Plancherel theory,
it is the estimates on the derivatives of the symbol and thus also in the Fourier variables,
that extend the desired properties onto more general pseudo-differential operators.
More specifically,
their concepts of difference operators allowed them to firstly ensure pseudo-differential operators
have Calder\'on-Zygmund kernels,
and secondly control the order of the terms that appear in asymptotic sums
arising from the composition and adjunction of operators.

Since then,
many results and applications have been obtained on compact Lie groups
by several authors including Ruzhansky, Wirth, Turunen, Delgado.
To cite a few,
these applications include
criterias for ellipticity and global hypoellipticity in terms of the global symbols~\cite{RuzhanskyTurunen10},
a proof of a sharp Gårding inequality~\cite{RuzhanskyTurunen11},
a global functional calculus \cite{RuzhanskyWirth14},
a study of $\Lebesgue p \Group$ Fourier multipliers~\cite{RuzhanskyWirth15},
and many others.

It is natural to wonder whether a global pseudo-differential calculus can be defined on other,
more complicated, settings.
To this end,
we observe that the Ruzhansky-Turunen theory relies on a Fourier-Plancherel theory,
while good properties of the operators rely on an understanding of the theory of singular integrals.
Therefore,
the natural settings to investigate are Lie groups with polynomial growth of the volume.

In 2013,
\citeauthor{FischerRuzhansky12} treated the case of graded nilpotent Lie groups.
In particular,
they dealt with the technical issues which arise from working with infinite-dimensional representations,
and more crucially,
non-central sub-Laplacians or positive Rockland operators.
They also present a strategy to obtain the composition and adjunction formulae,
which consists of proving symbolic estimates on the functional calculus for the sub-Laplacian or the Rockland operator,
and subsequently proving kernel estimates for general symbols.

The aim of this thesis is to study the case of the Euclidean motion group,
which is the smallest group of affine transformations on a Euclidean space containing both translations and rotations.
Our analysis will encounter difficulties due to the absence of properties such as compactness and commutativity of the group law,
and the lack of a dilation structure.
Furthermore, the infinite dimensional nature of the representations,
and the fact that the Laplacian is not central also complicates matters.
Nevertheless, our interest in this group is justified for two reasons:
firstly, it is one of the more elementary non-compact groups not yet studied in the context of global pseudo-differential calculus;
secondly, by its very definition,
it naturally describes rigid motions such as the position and orientation of a robot arm or polymer chains like DNA.
As a result,
many partial differential equations from engineering, biology and physics
are naturally expressed on the motion group.
More detail on the different applications can be found in \cite{ChirikjianWang04,ChirikjianKyatkin00,Chirikjian13}.

\section{Survey of the compact and graded case}

In this section,
we informally discuss an overview of the theory on compact and graded Lie groups.
For more detail,
the reader is invited to consult~\cite{RuzhanskyTurunen10} for the compact case
(with the understanding that the composition and adjunction formulae are only fully proved in~\cite{Fischer2015}),
and~\cite{FischerRuzhansky16} for the graded case.
To avoid repetition,
we shall discuss both cases simultaneously,
and contrast them to the case of the motion group when appropriate.

Essentially,
\cite{Fischer2015,FischerRuzhansky16} and the present document
all follow the following steps
to develop a global pseudo-differential calculus.

\begin{description}
    \item[Step 1.] Study the (sub-)Laplacian or positive Rockland operator,
        and the corresponding Sobolev spaces.
    \item[Step 2.] Study and generalise the group Fourier Transform.
    \item[Step 3.] Find a family of smooth functions which have a Leibniz-like rule and can be used for a Taylor development.
    \item[Step 4.] Define the symbol classes.
    \item[Step 5.] Study the functional calculus of the (sub-)Laplacian or positive Rockland operator.
    \item[Step 6.] Obtain kernel estimates.
    \item[Step 7.] Prove the composition and adjunction formulae.
\end{description}

We now present the above steps in more detail,
and $\Group$ will denote a compact or a graded Lie group until the end of this outline
unless stated otherwise.

\subsection*{Step 1 - Sobolev spaces}

We start by fixing a (sub-)Laplacian or, in the graded case, a positive Rockland operator.
This allows us to define \emph{Sobolev spaces}.
We then need to show their basic properties,
the most important of which is the \emph{Sobolev embedding theorem}.

As our kernels and symbols will be linked via the Fourier transform,
the \emph{Plancherel theory} will allow to get $\Lebesgue 2 \Group$ and more generally $\Sobolev s$ estimates on the kernel from symbolic estimates.
We shall therefore rely on the Sobolev embedding to obtain further regularity results (e.g.~continuity).
The role of the Sobolev embedding theorem is thus pivotal in the study of the regularity and the singularities of the kernel.

Moreover,
as we require our pseudo-differential operators to be bounded as a map between suitable Sobolev spaces,
we expect them to be part of our symbol classes definition.

In the graded case,
Sobolev spaces are studied in~\cite[Chapter 4]{FischerRuzhansky16},
following the proofs of the stratified case (see~\cite{Folland75}).

In the case of the compact group,
the \emph{Laplace-Beltrami} operator associated with a normalized bi-invariant metric appears the natural candidate to define Sobolev spaces,
since its eigenspaces arise naturally from the group representations.

From this point onwards,
$\Laplacian$ will denote the operator with respect to which we have defined our Sobolev spaces.

\subsection*{Step 2 - Fourier analysis}

The main idea is to define an analogue of the Euclidean Kohn-Nirenberg quantisation,
whereby the \emph{symbol} of a pseudo-differential operator is the \emph{Fourier transform of its convolution kernel}.
Our first step is thus to extend the group Fourier transform to a suitable subset of tempered distributions.

On our settings,
the Fourier transform is defined on $\Lebesgue 1 {\Group, \dd g}$ via
\begin{align*}
    \Fourier f(\xi)
    \defeq \int_\Group f(g) \adj{\xi(g)} \dd g,
\end{align*}
where $\dd g$ denotes a \emph{Haar measure},
$\xi \in \dualGroup \Group$,
and $\dualGroup \Group$ is the \emph{unitary dual} of $\Group$,
i.e.\ the set of all equivalence classes of irreducible, strongly continuous and unitary representations of $\Group$.

If $f \in \Lebesgue 1 \Group \cap \Lebesgue 2 \Group$,
we have the following \emph{Plancherel formula}
\begin{align*}
    \int_\Group \abs f^2 \dd g
    = \int_{\dualGroup \Group}
    \tr(\Fourier f(\xi) \adj{\Fourier f(\xi)})
    \dd \mu_{\dualGroup \Group}(\xi),
\end{align*}
where $\dd \mu_{\dualGroup \Group}$ denotes the so-called \emph{Plancherel measure},
while any $f \in \Schwartz \Group$ may be recovered from its Fourier Transform via
the following \emph{inverse formula}
\begin{align*}
    f(g) \defeq
    \int_{\dualGroup \Group}
    \tr(\xi(g) \Fourier f(\xi))
    \dd \mu_{\dualGroup \Group}(\xi),
    \quad g \in \Group.
\end{align*}

The problem of defining an extension of the Fourier Transform to make sense of a \emph{Kohn-Nirenberg quantization} procedure was solved in~\cite[Subsection 5.1.1]{FischerRuzhansky16} for a very large class of groups,
since their solution relies exclusively on the \emph{Plancherel} theorem and the definition of Sobolev spaces.

\subsection*{Step 3 - Difference operators}

In~\cite{RuzhanskyTurunen10},
\citeauthor{RuzhanskyTurunen10} introduce the following notion of difference operator
to generalize the derivatives in Fourier variables in the Euclidean definition of symbol classes.
If $q \in \SmoothFunctions \Group$,
they define
\begin{align*}
    \DifferenceOperator [\Group] q \Fourier f = \Fourier \{q f\}.
\end{align*}

When $\Group = \R^n$,
we recover the classical derivatives in frequency with the collection
$\Delta \defeq \{x_1, \dots, x_n \in \SmoothFunctions {\R^n}\}$,
where $x_i$ is the function that maps $x$ onto its $i$-th coordinate.

We now need to find the properties that a family of functions
\begin{align*}
    \Delta \defeq \{q_1, \dots, q_{\dimDifferenceOperators} \in \SmoothFunctions \Group\}
\end{align*}
must satisfy in order to develop a global pseudo-differential calculus.
We then let
\begin{align*}
    q^\alpha \defeq q^{\alpha_1}_1 \dots q^{\alpha_{\dimDifferenceOperators}}_{\dimDifferenceOperators},   
    \quad \DifferenceOperatorOrder \alpha \defeq \DifferenceOperator {q^\alpha(\dummy^{-1})}.
\end{align*}

Upon inspection of the different proofs in the Euclidean~\cite{Stein93},
compact~\cite{RuzhanskyTurunen10,Fischer2015} and graded~\cite{FischerRuzhansky16} cases,
the main requirements appear the following.
\begin{itemize}
    \item We can use a \emph{Taylor development} at the origin.
        This is necessary because the adjunction and the composition are difficult to express on the symbolic side without using a Taylor development.
    \item We have a reasonable \emph{Leibniz rule} for the associated difference operators.
        This ensures for example that the order of the composition of two left-invariant pseudo-differential operators is the sum of the orders.
    \item Ensure our pseudo-differentials of order $0$ are \emph{Calder\'on-Zygmund}.
        While we can reasonably expect the Plancherel theory to ensure $\Lebesgue 2 \Group$ boundedness,
        we rely on the theory of singular integrals to have the boundedness on $\Lebesgue p \Group$.
\end{itemize}

The first couple of conditions naturally require that each $q_j$ vanish at $e_\Group$ and that $\{dq_j(e) : 1 \leq j \leq \dimDifferenceOperators\}$ span the tangent space $T_{e_\Group} G$.
Such a family is called \emph{admissible} in~\cite{RuzhanskyTurunenWirth10} and subsequent literature.

Intuitively,
our definition of symbol classes will ensure that the regularity of the kernel will increase every time we multiply it by one of the $q_j \in \Delta$,
exactly like in the Euclidean case.
In particular,
this means that our kernels will be smooth outside of
\begin{align}
    \bigcap_{q \in \Delta} \{g \in \Group : q(g) = 0\},
    \label{eq:common_zeros_of_difference_operators}
\end{align}
while the above set may contain singularities.
However,
the Calder\'on-Zygmund theory requires that the singularity be confined to the identity.
Therefore,
\cite{RuzhanskyTurunenWirth10} defines $\Delta$ to be \emph{strongly admissible}
if~\eqref{eq:common_zeros_of_difference_operators} equals $\{e_\Group\}$.

\subsubsection*{The compact case}

In~\cite{RuzhanskyTurunenWirth10},
the authors show that a well-chosen finite subfamily $\Delta$ of
\begin{align*}
    \{(\tau - \tau(e_\Group))_{m n} : \tau \in \dualGroup \Group, 1 \leq m, n \leq \dimRep \tau\}
\end{align*}
can be strongly admissible.

Importantly,
this family yields the following \emph{Leibniz-like} rule
\begin{align*}
    \DifferenceOperatorOrder \alpha (\Fourier f_1 \Fourier f_2)
    = \sum_{\abs \alpha \leq \abs {\alpha_1} + \abs {\alpha_2} \leq 2 \abs \alpha}
    c^\alpha_{\alpha_1, \alpha_2}
    (\DifferenceOperatorOrder {\alpha_1} \Fourier f_1)
    (\DifferenceOperatorOrder {\alpha_2} \Fourier f_2),
\end{align*}
which is good enough to establish the calculus,
but slightly complicates some proofs related to the construction of parametrices.

The compactness of the group allows us to show that the calculus does \emph{not} depend on the choice of strongly admissible difference operators~\cite{RuzhanskyTurunenWirth10,Fischer2015}.
Such a proof relies strongly on the \emph{finite volume} of our group,
and it is important to note at this stage that the result does not hold even on $\R^n$.

\subsubsection*{The graded case}

The behaviour of difference operators in the graded case is somehow closer to the Euclidean setting
In~\cite{FischerRuzhansky16},
the authors consider the difference operators associated with \emph{homogeneous polynomials}.
The dilation structure
shows that we have a Leibniz rule if we consider a sum on homogeneous degrees
(see~\cite[Subsection 5.2.1]{FischerRuzhansky16}),
while the strong admissibility is easily obtained by choosing a suitable basis.

\subsection*{Step 4 - Symbol and operator classes}

Very informally,
we say that $\sigma(g, \xi)$ is a \emph{symbol} if there exists $\kappa \in \SmoothFunctions {\Group, \TemperedDistributions \Group}$ such that
\begin{align*}
    \sigma(g, \xi) \defeq \Fourier \{\kappa(g, \dummy)\}(\xi)
\end{align*}
is defined.
The distribution $\kappa$ is called the \emph{kernel associated} wih $\sigma$.

Moreover,
given $m \in \R$
we shall say $\sigma \in \SymbolClass m {\rho, \delta}$ if
\begin{align*}
    \norm [\mathcal H_\xi] {%
        \xi \BesselPotential {-m + \rho \abs \alpha - \delta \abs \beta + \gamma}
        \LeftDifferentialOperator \beta \DifferenceOperatorOrder \alpha \sigma(g, \xi)
        \xi \BesselPotential {-\gamma}
    }
    < \infty
\end{align*}
uniformly in $g \in \Group$
and essentially uniformly in $\xi \in \dualGroup \Group$ (with respect to the Plancherel measure).
Note that when the Laplacian is central,
we can assume $\gamma = 0$.
It should be clear that when $\Group = \R^n$,
we obtain the classical definition of symbol classes.

We also let
\begin{align*}
    \SmoothingSymbols \defeq \bigcap_{m \in \R} \SymbolClass m {\rho, \delta}
\end{align*}
denote the set of \emph{smoothing symbols}.

We define a \emph{pseudo-differential operator} to be an operator arising from a symbol via an analogue of the \emph{Kohn-Nirenberg quantisation}.
More precisely,
the operator
\begin{align*}
    \Op(\sigma) \phi(g)
    \defeq \InverseFourier \left\{ \xi \in \dualGroup \Group \mapsto \xi(g) \sigma(g, \xi) \Fourier \phi(\xi)\right\}(g)
\end{align*}
is the \emph{pseudo-differential operator} associated with the symbol $\sigma$.
We easily check that $\Op [\R^n]$ is the usual Kohn-Nirenberg quantisation.

Naturally,
the quantisation allows us to define \emph{operator classes} via
\begin{align*}
    \OperatorClass m {\rho, \delta} &\defeq \Op(\SymbolClass m {\rho, \delta}),\\
    \SmoothingOperators &\defeq \Op(\SmoothingSymbols).
\end{align*}

For more information,
see~\cite[Section 5.1]{FischerRuzhansky16} for the graded case;
and~\cite[Section 3.1]{Fischer2015} for the compact case.

\subsection*{Step 5 - Functional calculus}

On $\R^n$,
the composition and adjunction formulae can be proved by carrying out our computations on the frequency space $\dualGroup {\R^n} = \R^n$,
exploiting the fact that the Euclidean space is \emph{abelian} and we have explicit expressions of our difference operators.
Since a group Fourier transform is \emph{operator-valued} in more complicated settings,
it makes sense to attempt to do our calculations on the kernel side,
especially since this is how difference operators were defined.
Unfortunately,
kernels may have a singularity at the origin,
unlike symbols, which are smooth.
Therefore, two questions naturally arise.

\begin{itemize}
    \item What is the \emph{strength} of the singularity of the kernel at the origin?
    \item Can we \emph{approximate} a symbol $\sigma$ by a sequence of smoothing symbols $(\sigma_n)_{n \in \N}$ so that the kernel can be approximated by the associated sequence of smooth kernels?
\end{itemize}

Looking at pseudo-differential operators such as $\BesselPotential m$
suggests that \emph{functional calculus} is the natural framework to answer such questions.

A crucial step of our analysis is to construct a family of smoothing symbols approximating the identity symbol.
Using the algebra structure of our symbol classes,
this will yield an important density result,
as we will be able to approximate any symbol with smoothing symbols,
and by extension any kernel with smooth kernels.

Ideally,
we would like to create a family $\{\eta_\epsilon = \Fourier p_\epsilon\}_{\epsilon \in (0, 1]} \subset \SmoothingSymbols$
such that
\begin{align}
    \SymbolSemiNorm m {\rho, \delta} N {\eta_\epsilon}
    &\leq \epsilon^\frac m \nu,
    \quad m < 0,
    \label{eq:survey:density_estimate:1}
    \\
    \SymbolSemiNorm m {\rho, \delta} N {\Id {} - \eta_\epsilon}
    &\leq \epsilon^\frac m \nu,
    \quad m \geq 0,
    \label{eq:survey:density_estimate:2}
\end{align}
where $\nu$ is the order of the operator $\Laplacian$.
Moreover,
to prove the Calder\'on-Zygmund property,
we need very good estimates on the kernels $p_\epsilon$.
Therefore, the natural candidate is the Fourier transform of the \emph{heat kernel}.

The family of symbols is used in the following proofs.
\begin{description}
    \item[Kernel estimates]

        Given a symbol $\sigma \in \SymbolClass m {\rho, \delta}$,
        we estimate the singularity of its kernel at the origin via
        \begin{align*}
            \abs {\kappa_g(h)}
            \leq
            \abs {\conv {\kappa_g} {p_\epsilon}(h)}
            +
            \abs {\conv {\kappa_g} {(\delta_{e_\Group} - p_\epsilon)(h)}},
        \end{align*}
        where we choose $\epsilon$ so that $\epsilon^\nu$ equals the distance between $h$ and $e_\Group$.

        The second term of the right-hand side is the one containing the singularity,
        and we need~\eqref{eq:survey:density_estimate:2} to control it.

    \item[Construction of asymptotic limits]

        One of the inherent characteristics of symbolic calculus is that compositions and adjunctions can only be expressed via infinite sums from which we cannot reasonably expect convergence.
        As certain applications such as the construction of parametrices rely on making sense of these sums in a suitable sense,
        it is important that we define a notion of \emph{asymptotic convergence}.

        The idea is the same as in the Euclidean case.
        We know that cutting off ``low frequencies'' will not affect the result modulo $\SmoothingSymbols$.
        Therefore, given a sequence of symbols $\sigma_j$ with strictly decreasing order,
        we thus define
        \begin{align*}
            \sum_{j = 0}^{+\infty} \sigma_j \defsim
            \sum_{j = 0}^{+\infty} \sigma_j (\Id {} - \eta_{\epsilon_j})
            \quad \text{modulo } \SmoothingSymbols,
        \end{align*}
        for a well chosen sequence $(\epsilon_j) \subset (0, 1]$ decreasing to $0$.

    \item[Density of smoothing symbols]

        Letting
        \begin{align*}
            \sigma_\epsilon \defeq \sigma \eta_\epsilon
        \end{align*}
        allows us to approximate $\sigma$ by smoothing symbols,
        while the error term
        \begin{align*}
            \sigma - \sigma_\epsilon = (\Id {} - \eta_\epsilon) \sigma
        \end{align*}
        can be controlled via our symbolic estimates on $\Id {} - \eta_\epsilon$.

        In particular,
        this allows us to show the composition and adjunction formulae only for
        for smoothing symbols and conclude by density.
\end{description}

\subsubsection{The compact case}

Following the classical heat kernel methods (see e.g.~\cite{FurioliMelziVeneruso06,VaropoulosSaloffCosteCoulhon92}),
\citeauthor{Fischer2015} shows symbolic estimates for the functional calculus of the Laplace-Beltrami operator in~\cite{Fischer2015}.
However, she proceeds slightly differently by letting
\begin{align*}
    \eta_\epsilon
    \defeq \chi(\epsilon \xi(\Laplacian [\Group])),
\end{align*}
where $\chi : \R^+ \to \R$ equals $1$ on $[0, 1]$ and has compact support.
To estimate the remainder,
she uses a \emph{Littlewood-Paley} decomposition
\begin{align*}
    \Id {} - \eta_\epsilon = \sum_{j = 1}^{+\infty} \eta_{\epsilon, j},
\end{align*}
where $\eta_{\epsilon, j} \defeq \eta_{2^{-j} \epsilon} - \eta_{2^{-j + 1} \epsilon}$
and the convergence is understood in the strong operator topology.

\subsubsection{The graded case}

The existence of such a family $\eta_\epsilon$ is ensured directly by the \emph{Hulanicki Theorem} (see \cite[Theorem 4.5.1]{FischerRuzhansky16}).
We can define
\begin{align*}
    \eta_\epsilon(\xi) \defeq \Fourier \{\e^{-\epsilon \Laplacian} \delta_{e_\Group}\}(\xi),
    \quad \xi \in \dualGroup \Group,
\end{align*}
and the aforementioned theorem ensures that this defines a smoothing symbol satisfying the appropriate symbolic estimates.

\subsection*{Steps 6 and 7 - Kernel estimates and calculus proofs}

As soon as we have obtained a good approximation of the identity symbol,
whether it is via a family $\eta_\epsilon, \epsilon \in (0, 1]$
or a Littlewood-Paley decomposition,
the proofs of the kernel estimates and subsequently the composition and adjunction formulae are very similar (when $\rho > \delta$).

Let us sketch the proof of the \emph{adjunction formula}
and let us keep in mind that the arguments for the composition formula are very similar.

Suppose that $T \defeq \Op(\sigma) \in \OperatorClass m {1, 0}$.

\begin{enumerate}
    \item As we intend to use a \emph{density} argument,
        we assume first that $\sigma$ is smoothing.
    \item We perform an exact computation of the kernel $\tilde \kappa$ of $\adj T$ in terms of the kernel $\kappa$ of $T$.

        More precisely, we show that $\adj T$ has right-convolution kernel
        \begin{align*}
            \tilde \kappa_g(h) \defeq \conj \kappa_{g h^{-1}}(h^{-1}).
        \end{align*}
    \item
        In particular, its \emph{symbol} is given by taking the Fourier transform of the above
        \begin{align*}
            \tilde \sigma(g, \xi) \defeq \Fourier \tilde \kappa_g(\xi)
            = \int_\Group \conj \kappa_{g h^{-1}}(h^{-1}) \adj{\xi(h)} \dd h.
        \end{align*}
    \item
        In the integral above,
        we take the \emph{Taylor development} with respect to the variable $h$ appearing in the subscript of the kernel
        \begin{align*}
            \tilde \sigma(g, \xi)
            &= \int_\Group (\sum_{\abs \alpha \leq N} q^{\alpha}(h^{-1}) \LeftDifferentialOperator [g] \alpha \conj \kappa_{g}(h^{-1}) + R_N(h)) \adj{\xi(h)} \dd h\\
            &= \sum_{\abs \alpha \leq N} \frac 1 {\alpha!} \DifferenceOperatorOrder \alpha \LeftDifferentialOperator \alpha \adj \sigma(g, \xi) + \int_\Group R_N(h) \adj{\xi(h)} \dd h.
        \end{align*}
    \item
        Using the Taylor remainder formula,
        the integrand on the right can be estimated by an expression such as
        \begin{align*}
            \abs {R_N(h)}
            \leq C d(h, e_\Group)^{N + 1} \sup_{\beta \in B} \sup_{\tilde h \in \Gamma_h} \abs {\LeftDifferentialOperator \beta \kappa_g(\tilde h)}
        \end{align*}
        so that the integral can be controlled via good estimates on $\kappa_g$.
    \item
        We now conclude the proof by density.
\end{enumerate}

For the compact case,
see~\cite[Sections 6.3 and 7.3]{Fischer2015};
for the graded case,
see~\cite[Sections 5.4 and 5.5]{FischerRuzhansky16}.

\section{Overview of thesis}

Following the ideas of~\cite{Shubin01, RuzhanskyWirth14},
we define
\begin{align*}
    \eta_\epsilon(\xi) \defeq
    \frac 1 {\i \turn} \int_\Gamma \e^{-t z} \left(\xi(\Id {} - \Laplacian) - z \Id {}\right)^{-1} \dd z,
    \quad \xi \in \dualGroup \Group
\end{align*}
with an appropriate contour $\Gamma \subset \C$
and we show the symbolic estimates on $\eta_\epsilon$ and $\Id {} - \eta_\epsilon$
via the exponential decay on the contour and elementary estimates on the resolvent,
in particular \emph{without using the classical estimates on the heat kernel}.

This strategy generalizes verbatim to more general settings
(provided that we have good properties difference operators).
It also means that the results herein could be adapted
to provide a shorter proof for the kernel estimates,
and thus by extension also for the composition and the adjunction formulae.

This approach has however (at least) two drawbacks:
\begin{itemize}
    \item
        We only show symbolic estimates for the functional calculus associated with the functions $\e^{t z}$,
        $1 - \e^{t z}$, and $z^\gamma$,
        while~\cite{FischerRuzhansky16} and~\cite{Fischer2015} prove a much more general result in their respective cases.
    \item
        The kernel estimates when $\rho = 1$,
        essential to show the Calder\'on-Zygmund property,
        are trickier to obtain
        and will require better estimates on the kernel.
        As a result,
        we will have to follow~\cite{Fischer2015}
        to prove the $\Lebesgue p \Group$ boundedness.
\end{itemize}

It is nevertheless interesting to observe that this approach provides a shorter and more elementary way
to establish a global pseudo-differential calculus
if compared with~\cite{Fischer2015,FischerRuzhansky16}.

\chapter*{Notation}

\section*{Sets and symbols}

\begin{itemize}
    \item $\R$, $\R^+$ and $\R^-$ denote the sets of all, positive (including $0$) and negative (including $0$) real numbers respectively.
    \item $\Z$ denotes the sets of all integers,
        while $\N$ contains all positive integers including $0$.
    \item $\Kronecker i j$ is the \emph{Kronecker delta};
        if is equal to $1$ if $i = j$,
        and $0$ otherwise.
    \item $\grad$ denotes the \emph{gradient} on the vector space $\VectorSpace$.
    \item $(\LieAlgebra, \LieBracket \dummy \dummy)$ is the \emph{Lie algebra} of $\Group$.
    \item $\Laplacian$ denotes the \emph{Laplace operator}.
    \item $\adj T$ denotes the \emph{adjoint} of $T$
    \item $e_\Group$ is the identify element of a group $\Group$.
    \item $e_1, \dots, e_{\dim \VectorSpace}$ denotes an orthonormal basis of $\VectorSpace$.
    \item $\conv f g$ represents the convolution of $f$ and $g$.
    \item $\delta_g$ is the desta-distribution at $g \in \Group$.
    \item $\Schwartz \Group$ is the set of \emph{Schwartz functions}.
    \item $\TemperedDistributions \Group$ is the set of \emph{tempered distributions} on $\Group$.
    \item $\SmoothFunctions \Group$ denotes the set of \emph{smooth functions}.
    \item $\Lebesgue p \Group$ will denote the set of all $\mu$-measurable complex function $f$ on $\Group$ such that $\abs f^p$ is $\mu$-integrable;
        here, $\mu$ is a Haar measure on $\Group$.
    \item $\Sobolev s$ is the $2$-\emph{Sobolev space} of order $s$ on $\Group$.
    \item $\SobolevOrder p s$ is the $p$-\emph{Sobolev space} of order $s$ on $\Group$.
    \item $\Lin {\mathcal H_1, \mathcal H_2}$ denotes a bounded linear map between the Hilbert spaces $\mathcal H_1$ and $\mathcal H_2$.
    \item $\SchattenClasses p {\mathcal H}$ denotes the $p$-\emph{Schatten classes} on the Hilbert space $\mathcal H$; also, we write $\Lin {\mathcal H} \defeq \Lin {\mathcal H, \mathcal H}$.
    \item $\dualGroup \Group$ denotes the \emph{unitary dual} of $\Group$.
    \item $\Fourier$ denotes the group \emph{(unitary) Fourier transform} on $\Group$.
    \item $\InverseFourier$ denotes the \emph{inverse Fourier transform} on $\Group$.
    \item $\JapaneseBracket \Group \dummy$ denotes the \emph{Japanese Bracket}.
    \item $\SymbolClass m {\rho, \delta}$ denotes the class of symbols with order $m$ and type $(\rho, \delta)$ on $\Group$.
    \item $\OperatorClass m {\rho, \delta}$ denotes the class of operators with order $m$ and type $(\rho, \delta)$ on $\Group$.
\end{itemize}

\section*{Positive and negative}

In this document, we shall use the following conventions.

\begin{itemize}
    \item $0$ is both \emph{``positive''} and \emph{``negative''}.
    \item $\N, \R^+, \R^-$ contain $0$.
    \item Any number is both \emph{``greater than''} and \emph{``less than''} itself.
    \item We shall use strict inequalities only
        when the corresponding non-strict inequalities are not suitable.
\end{itemize}

We shall do this to ensure that
when a strict inequality is necessary (i.e. $< +\infty$ or $> 0$),
it gets the emphasis it deserves.
This is why we shall always insist on writing \emph{``strictly positive''} or \emph{``strictly greater than''} when we require the strict inequality.
To make it less verbose,
we give the non-strict meaning to \emph{``greater or less than''}.

Note that the above conventions originated from Laurent Schwartz's \citetitle{Schwartz1981} (\cite{Schwartz1981})
and have now become standard in Francophone literature.

\begin{aquote}{Laurent Schwartz, \cite[p.~17]{Schwartz1981}}
    Notons que nous rompons ici avec l'usage ant\'erieurement acquis
    en appelant inf\'erieur ce qu'on appelait inf\'erieur ou \'egal,
    et strictement inf\'erieur ce qu'on appelait inf\'erieur.
    La raison d'\^etre de ces changements,
    pleinement justifi\'es par la suite,
    est que la notion la plus g\'en\'eralement utilis\'ee est $\leq$,
    et qu'il est bon qu'elle ait l'appelation la plus courte.
    On devra toujours utiliser $\leq$ plut\^ot que $<$,
    toutes les fois que cela sera possible;
    quand on \'ecrira une in\'egalit\'e stricte avec $<$,
    ce sera pour avertir le lecteur qu'il y a un point d\'elicat,
    et que l'in\'egalit\'e large $\leq$ ne conviendrait pas.
    \footnote{%
        \emph{Translation:}
        Let us observe that we break from traditional usage
        by saying ``less than'' instead of ``less than or equal to'',
        and ``strictly less than'' instead of ``less than''.
        The reason for these changes,
        completely justified hereafter,
        is that $\leq$ is the more widely used notion
        and should therefore have the shortest name.
        We shall always use $\leq$ rather than $<$
        every time we can;
        when we use the strict inequality with $<$,
        it will be to warn the reader that there is a subtlety
        and that the non-strict inequality is not suitable.
    }
\end{aquote}

Naturally, the same terminology applies to operators and matrices.
The operator $0$ is \emph{``positive definite''} but not \emph{``strictly positive definite''}.

\chapter{Preliminaries}

\section{Lie groups}

To develop a \emph{pseudo-differential calculus} on groups,
a reasonable prerequisite is that the group be equipped with a differential structure.

\begin{definition}[Lie group]
\label{definition:Lie_group}
\index{Lie group}
    Let $\Group$ be a group.
    We say that $\Group$ is a \emph{Lie group}
    if $\Group$ is a smooth manifold and the map
    \begin{align*}
        \Group \times \Group \to \Group :
        (g_1, g_2) \mapsto g_1^{-1} g_2
    \end{align*}
    is smooth.

    If moreover $\Group$ is (locally) compact as a manifold,
    then we shall say that $\Group$ is a \emph{(locally) compact Lie group}.
\end{definition}

\begin{example}[Special orthogonal group]
    The set
    \begin{align*}
        \SpecialOrthogonalGroup \VectorSpace
        \defeq
        \{ A \in \Lin{\VectorSpace} : \det A = 1 \}
    \end{align*}
    is a compact Lie group.
\end{example}

\subsection{Lie algebra}

\begin{definition}[Lie algebra]
    A (real) Lie algebra is a (real) vector space $\LieAlgebra$
    equipped with a bilinear map
    \begin{align*}
        \LieBracket \dummy \dummy : \VectorSpace \times \VectorSpace \to \VectorSpace,
    \end{align*}
    called the \emph{Lie bracket} or \emph{commutator},
    such that
    \begin{enumerate}
        \item $\LieBracket X X = 0$ for every $X \in \LieAlgebra$;
        \item for every $X, Y, Z \in \LieAlgebra$, we have the following \emph{Jacobi identity}
            \begin{align*}
                \LieBracket X {\LieBracket Y Z} +
                \LieBracket Y {\LieBracket Z X} +
                \LieBracket Z {\LieBracket X Y}
                = 0.
            \end{align*}
    \end{enumerate}
\end{definition}

Given a Lie group $\Group$,
let us denote by $\VectorFields$ the set of all \emph{smooth vector fields} on $\Group$.

\begin{definition}[Left-invariant vector fields]
    Let $\Group$ be a Lie group.
    We shall say that $X \in \VectorFields$ is \emph{left-invariant}
    if for every $g \in \Group$,
    \begin{align*}
        X \circ L_g = \dd L_g \circ X,
    \end{align*}
    where
    \begin{align*}
        L_g : \Group \to \Group : h \mapsto g h
    \end{align*}
    is the left translation on $\Group$ by $g$.
    We shall denote the set of left-invariant vector fields by $\LeftInvariantVectorFields$.
\end{definition}

\begin{example}[Left-invariant vector fields]
\label{example:Lie_algebra_of_left-invariant_vector_fields}
    Let $\Group$ be a Lie group.
    Given two left-invariant vector fields $X, Y \in \LeftInvariantVectorFields$,
    we define
    \begin{align*}
        \LieBracket [\LeftInvariantVectorFields] X Y f(x) \defeq X Y f(x) - Y X f(x),
        \quad f \in \SmoothFunctions \Group
    \end{align*}
    and we can show that $\LieBracket [\LeftInvariantVectorFields] X Y$ is also a left-invariant vector field.
    Therefore, $\LeftInvariantVectorFields$ is a Lie algebra.
\end{example}

Let $X, Y \in \TangentSpace \Group e$.
Defining for each $g \in \Group$
\begin{align*}
    \tilde X(g) \defeq \dd L_g(X), \quad \tilde Y(g) = \dd L_g(Y),
\end{align*}
we check that $\tilde X, \tilde Y \in \LeftInvariantVectorFields$.

The Lie algebra structure of $\LeftInvariantVectorFields$ allows us then to define
\begin{align}
    \LieBracket [\TangentSpace \Group e] X Y \defeq \LieBracket [\LeftInvariantVectorFields] {\tilde X} {\tilde Y}(e).
    \label{eq:Lie_bracket_on_the_tangent_space}
\end{align}

\begin{definition}[Lie algebra of Lie group]
\label{definition:Lie_algebra_of_Lie_group}
    Let $\Group$ be a Lie group.
    The \emph{Lie algebra} of $\Group$ is the tangent space at the identity $\TangentSpace \Group e$
    with the Lie bracket $\LieBracket [\TangentSpace \Group e] \dummy \dummy$
    defined in \eqref{eq:Lie_bracket_on_the_tangent_space}.
\end{definition}

\subsection{Haar measure}

\begin{remark}
    Every Lie group $\Group$ is also topological space.
    As a result, all the topological definitions apply to Lie groups.
\end{remark}

\begin{definition}[Haar measure]
\index{Haar measure}
    Let $\Group$ be a Lie group.
    A positive Radon measure on $\Group$ is called a \emph{Haar measure}
    if it is in addition \emph{left-invariant},
    i.e.\ for each $g \in \Group$ and each Borel set $A \subset G$, we have
    \begin{align*}
        \mu(g A) = \mu(A).
    \end{align*}
\end{definition}

\begin{proposition}[Haar measure]
    If $\Group$ is a locally compact Lie group,
    then there exists a Haar measure $\mu$ on $\Group$.

    Moreover, if $\nu$ is another left-invariant Radon measure on $\Group$,
    then we can find $c \geq 0$ such that $\nu = c \mu$.
\end{proposition}

\begin{definition}[Unimodular group]
\label{definition:unimodular_group}
    Let $\Group$ be a locally compact Lie group.
    If a non-zero Haar measure on $\Group$ is also right-invariant,
    or equivalently if all Haar measure are right-invariant,
    we shall say that $\Group$ is \emph{unimodular}.
\end{definition}

\begin{proposition}
\label{proposition:sufficient_conditions_to_be_unimodular}
    Let $\Group$ be a Lie group.
    \begin{enumerate}
        \item If $\Group$ is compact, then $\Group$ is \emph{unimodular}.
        \item If $\Group$ is abelian and locally compact, then $\Group$ is unimodular.
    \end{enumerate}
\end{proposition}

\section{Representation Theory}

\begin{definition}[Unitary representations]
\label{definition:unitary_representation}
\index{representations!unitary representations}
    Let $\Group$ be a group and $\Hil$ be a Hilbert space.
    A map
    \begin{align*}
        \xi : \Group \mapsto \Hom(\Hil)
    \end{align*}
    is called a \emph{unitary representation (on $\Hil$)} if
    \begin{enumerate}
        \item for each $g \in \Group$, the map $\xi(g)$ is unitary:
            \begin{align*}
                {\xi(g)}^{-1} = \adj{\xi(g)};
            \end{align*}
        \item if $g, h \in \Group$, then we have $\xi(g h) = \xi(g) \xi(h)$.
    \end{enumerate}

    The \emph{dimension} of $\xi$ is that of $\Hil$.
    If $\Hil$ is finite-dimensional,
    we let $\dimRep{\xi} \defeq \dim{\Hil}$ denote the dimension of $\xi$.
\end{definition}

\begin{example}[Right-regular representation]
    Let $\Group$ be a unimodular topological group.
    The \emph{right-regular representation} is the representation
    \begin{align*}
        \RightRegularRepresentation : \Group \to \Hom(\Lebesgue{2}{\Group})
    \end{align*}
    defined via
    \begin{align*}
        \RightRegularRepresentation(h) f(g) = f(g h)
    \end{align*}
    for every $g, h \in \Group$.
\end{example}

\begin{definition}[Invariant subspaces]
\label{definition:invariant_subspaces}
    Let $\Group$ be a group and $\xi$ be a unitary representation of $\Group$ on a Hilbert space $\Hil$.
    We shall say that a vector subspace $W \subset \Hil$ is \emph{invariant} under $\xi$
    if for each $g \in \Group$, we have $\xi(g) W \subset W$.
\end{definition}

\begin{definition}[Irreducibility]
\label{definition:irreducible_representations}
    Let $\Group$ be a group and $\xi$ be a unitary representation of $\Group$ on a Hilbert space $\Hil$.
    \begin{enumerate}
        \item If the only invariant subspaces of $\xi$ are $\{0\}$ and $\Hil$,
            then $\xi$ is said to be \emph{irreducible}.
        \item Otherwise, if there exists a non-trivial invariant subspace,
            then $\xi$ is \emph{reducible}.
    \end{enumerate}
\end{definition}

\begin{definition}[Equivalent representations]
\label{definition:equivalent_representations}
    Let $\Group$ be a group and $\Hil_1, \Hil_2$ be Hilbert spaces.
    Suppose $\xi_1$, $\xi_2$ are representations of $\Group$ on $\Hil_1$ and $\Hil_2$ respectively.
    We shall say that $\xi_1$ and $\xi_2$ are \emph{equivalent}
    if there exists an invertible linear map
    \begin{align*}
        A : \Hil_1 \to \Hil_2
    \end{align*}
    such that for each $g \in \Group$, we have
    \begin{align*}
        \xi_2(g) = A \circ \xi_1(g) \circ A^{-1}.
    \end{align*}
    In that case, $A$ is called an \emph{intertwining operator}.
\end{definition}

\begin{definition}[Strongly continuous representations]
\label{definition:strongly_continuous_representation}
\index{strongly continuous}
    Let $\Group$ be a group and $\Hil$ be a Hilbert space.
    Suppose further that $\xi$ is a representation of $\Group$ on $\Hil$.
    We shall say that $\xi$ is \emph{strongly continuous}
    if for each $x \in \Hil$,
    the map
    \begin{align*}
        \Group \to \Hil : g \mapsto \xi(g) v
    \end{align*}
    is continuous.
\end{definition}

\begin{definition}[Unitary dual]
\label{definition:unitary_dual}
    Let $\Group$ be a locally compact topological group.
    The \emph{unitary dual} of $\Group$, denoted by $\dualGroup\Group$,
    is the set of all equivalence classes of
    \emph{strongly continuous, irreducible, unitary representations} of $\Group$.
\end{definition}

\begin{remark}
    Let $\Group$ be a locally compact topological group.
    We shall often abuse the notation and use $\dualGroup\Group$ to denote a set consisting of
    exactly one representation in each equivalence class of the actual unitary dual.
\end{remark}

\begin{example}[$\dualGroup\VectorSpace$]
    For each $\lambda \in \VectorSpace$,
    define
    \begin{align}
        \xi_\lambda : \VectorSpace \to \UnitaryGroup{1} : x \mapsto \e^{\i \turn \ip{\lambda}{x}}.
        \label{eq:elements_of_dual_of_vector_space}
    \end{align}

    It can be shown that
    \begin{align*}
        \dualGroup\VectorSpace = \{ \xi_\lambda : \lambda \in \VectorSpace \}.
    \end{align*}

    Therefore, the map
    \begin{align}
        \lambda \mapsto \xi_\lambda
        \label{eq:isomorphism_between_vector_space_and_its_dual_group}
    \end{align}
    is a group isomorphism which allows us to give $\dualGroup\VectorSpace$ a vector space structure.
\end{example}

\begin{definition}
    Let $\tau \in \dualGroup\Group$.
    We define the sets
    \begin{align*}
        \HilbertCompactGroupColumn{\tau}{i} \defeq \span \{ \tau_{i, j} : j = 1, \dots, \dimRep\tau \}
        \subset \Lebesgue{2}\CompactGroup, \quad
        \HilbertCompactGroup{\tau} = \bigoplus_{i = 1}^\dimRep\tau \HilbertCompactGroupColumn{\tau}{i}.
    \end{align*}
\end{definition}

\begin{theorem}[Peter-Weyl theorem]
\label{theorem:Peter-Weyl_theorem}
    Let $\Group$ be a compact topological group.
    The set
    \begin{align*}
        \left\{
            \sqrt{\dimRep\tau} \tau_{ij} : \tau \in \dualGroup\Group,\ i, j = 1, \dots, \dimRep\tau
        \right\}
    \end{align*}
    is an orthonormal basis of $\Lebesgue{2}{\Group}$ with respect to the \emph{normalised} Haar measure.

    Therefore, we obtain the following decomposition:
    \begin{align*}
        \Lebesgue{2}{\Group} \defeq
        \bigoplus_{\tau \in \dualGroup\Group} \bigoplus_{j = 1}^{\dimRep \tau} \Hilbert{\tau}{j}.
    \end{align*}

    Since $\Hilbert{\tau}{j}$ is an invariant subspace of $\RightRegularRepresentation$
    and since $\eval{\RightRegularRepresentation}{\Hilbert{\tau}{j}} \sim \tau$,
    we have
    \begin{align*}
        \RightRegularRepresentation \sim
        \bigoplus_{\tau \in \dualGroup\Group} \dimRep\tau \tau.
    \end{align*}
    In particular, the unitary dual can be generated by the right-regular representation.
\end{theorem}

\section{Distributions}

For this section,
we assume that $\VectorSpace$ is a vector space
and $\CompactGroup$ is a compact Lie group.
Also, when we integrate on $\VectorSpace$, $\CompactGroup$ or $\GroupDirect$,
we always mean with respect to a \emph{Haar measure}.

We shall denote by $\LieAlgebraCompactGroup$ the Lie algebra of $\CompactGroup$.
We fix a basis $Y_1, \dots, Y_{\dim \CompactGroup}$ of $\LieAlgebraCompactGroup$ and write
\begin{align*}
    Y^\beta \defeq Y_1^{\beta_1} \dots Y_{\dim \CompactGroup}^{\beta_{\dim \CompactGroup}},
    \quad \beta \in \N^{\dim \CompactGroup}.
\end{align*}

\begin{definition}[Schwartz space]
    We shall say that $f \in \SmoothFunctions \GroupDirect$ is \emph{rapidly decaying}
    if for every $N \in \N$,
    \begin{align*}
        \seminorm [\Schwartz \GroupDirect] N {f}
        \defeq
        \sup_{\abs \alpha, \abs \beta \leq N}
        \abs{%
            {(1 + \abs x)}^N \D{\abs \alpha}{x^\alpha} Y^{\beta}_k f(x, k)
        } < \infty.
    \end{align*}
    The set of all rapidly decaying functions will denoted by $\Schwartz \GroupDirect$.
    The family $\{\seminorm [\Schwartz \GroupDirect] N \dummy : N \in \N\}$ gives $\Schwartz \GroupDirect$
    the structure of a Fr\'echet space.
\end{definition}

\begin{definition}[Tempered distributions]
    We shall say that
    \begin{align*}
        \kappa : \Schwartz \GroupDirect \to \C
    \end{align*}
    is a \emph{tempered distribution} if it is linear and continuous.
    The set of all tempered distributions will be denoted by $\TemperedDistributions \GroupDirect$.
\end{definition}

As usual, if $\kappa \in \TemperedDistributions \GroupDirect$ and $\phi \in \Schwartz \GroupDirect$
\begin{align*}
    \dualBracket [\GroupDirect] \kappa \phi \defeq \kappa(\phi)
\end{align*}
We shall also write
\begin{align*}
    \int_{\GroupDirect} \kappa(x, k) \phi(x, k) d(x, k) \defeq \dualBracket [\GroupDirect] \kappa \phi,
\end{align*}
motivated by the inclusion $\Lebesgue 1 \GroupDirect \subset \TemperedDistributions \GroupDirect$,
and shall say that the above integral is interpreted \emph{in the sense of distributions}.

\begin{theorem}[Schwartz Kernel Theorem]
\label{theorem:Schwartz_Kernel_Theorem}
\index{Schwartz Kernel Theorem}
    If $T : \Schwartz \GroupDirect \to \TemperedDistributions \GroupDirect$ is a continuous linear operator,
    then there exists a unique distribution
    $\kappa \in \TemperedDistributions {\GroupDirect \times \GroupDirect}$ such that
    \begin{align*}
        T \phi(x, k) = \int_{\GroupDirect} \kappa(x, k; y, l) \phi(y, l) \dd (y, l).
    \end{align*}
\end{theorem}

\section{Convolutions}

In this section,
$\Group$ is a locally compact Lie group.

\begin{definition}[Convolution]
    Let $f_1, f_2 \in \Lebesgue 1 \Group$.
    We define the convolution of $f_1$ and $f_2$,
    denoted by $\conv {f_1} {f_2}$, via
    \begin{align*}
        \conv {f_1} {f_2} (g)
        \defeq \int_\Group f_1(h) f_2(h^{-1} g) \dd h.
    \end{align*}
\end{definition}

\section{Fourier Transform}

\begin{definition}[Fourier coefficient]
    Let $\xi \in \dualGroup\Group$.
    If $f \in \Lebesgue{1}{\Group}$,
    we define the \emph{Fourier coefficient of $f$ at $\xi$}, $\Fourier f(\xi)$, via
    \begin{align*}
        \Fourier f(\xi) \defeq \int_\Group f(g) \adj{\xi(g)} \dd g.
    \end{align*}

    The map
    \begin{align*}
        \Fourier f : \xi \mapsto \Fourier f(\xi)
    \end{align*}
    is called the \emph{Fourier transform of $f$}.
\end{definition}

\begin{definition}
    Let $\xi \in \VectorSpace$.
    If $f \in \Lebesgue{1}\VectorSpace$,
    we define the \emph{Fourier coefficient of $f$ at $\lambda$}, $\Fourier[\VectorSpace] f(\xi)$, via
    \begin{align*}
        \Fourier[\VectorSpace] f(\xi) \defeq \int_\VectorSpace f(x) \e^{-\i \turn \ip{x}\xi} \dd x.
    \end{align*}
\end{definition}

\begin{proposition}
\label{proposition:elementary_properties_of_the_Fourier_transform}
    Let $f, f_1, f_2 \in \Lebesgue{1}{\Group}$ and $\xi \in \dualGroup\Group$.
    The Fourier Transform satisfies the following properties:
    \begin{enumerate}
        \item For each $g \in \Group$, we have
            \begin{align*}
                \Fourier \{f(\dummy g)\} (\xi)
                = \xi(g) \Fourier f(\xi), \quad
                \Fourier \{f(g \dummy)\} (\xi)
                = \Fourier f(\xi) \xi(g).
            \end{align*}
        \item We have
            \begin{align*}
                \Fourier \{\conv{f_1}{f_2}\}(\xi)
                = \Fourier f_2(\xi) \Fourier f_1(\xi).
            \end{align*}
        \item If $f \in \SmoothFunctions\Group$ and $X \in \LieAlgebra$,
            \begin{align*}
                \Fourier \{\LeftDifferentialOperatorFirstOrder{X} f\}(\xi)
                = \xi(X) \Fourier f(\xi), \quad
                \Fourier \{\RightDifferentialOperatorFirstOrder{X} f\}(\xi)
                = \Fourier f(\xi) \xi(X).
            \end{align*}
    \end{enumerate}
\end{proposition}

\begin{proposition}
    There exists a measure $\Plancherel\Group$ on $\dualGroup\Group$ such that the following property holds:
    if $f \in \Schwartz\Group$, we have
    \begin{align*}
        \int_\Group \abs{f}^2 \dd g
        = \int_{\dualGroup\Group} \tr \left( \Fourier f(\xi) \adj{\Fourier f(\xi)} \right) \dd \Plancherel\Group(\xi).
    \end{align*}
\end{proposition}

\begin{example}[Plancherel Measure on $\dualGroup\CompactGroup$]
    If $f \in \SmoothFunctions{\CompactGroup}$,
    then the Peter-Weyl theorem implies:
    \begin{align*}
        \int_\CompactGroup \abs{f}^2 \dd g
        = \sum_{\tau \in \dualGroup\CompactGroup}
            \dimRep\tau
            \tr \left(
                \Fourier[\CompactGroup] f(\tau)
                \adj{\Fourier[\CompactGroup] f(\tau)}
            \right)
    \end{align*}
\end{example}

\begin{proposition}[Inverse Fourier Transform]
    Let $g \in \Group$.
    If $f \in \Schwartz\Group$,
    then we have
    \begin{align*}
        f(g) =
        \int_\dualGroup\Group
            \tr\left(
                \xi(g)
                \Fourier f(\xi)
            \right)
        \dd \Plancherel\Group(\xi).
    \end{align*}
\end{proposition}

\section{The $\GroupDirect$ case }

As our representations on the motion group will act on $\Lebesgue{2}\CompactGroup$,
we will define our Fourier Transform on $\CompactGroup$ so that it acts on $\Lebesgue{2}\CompactGroup$ as well.
This will allow useful comparisons later on.

\begin{definition}[Fourier Transform]
    Let $f \in \Lebesgue 1 \GroupDirect$ and $\lambda \in \VectorSpace$.
    We define the \emph{Fourier coefficient of $f$} at $\lambda$,
    denoted via $\Fourier[\GroupDirect] f(\lambda)$, via
    \begin{align*}
        \Fourier [\GroupDirect] f(\lambda) \defeq \int_\CompactGroup f(x, k) \e^{\i \turn \ip \lambda x} \adj{\RightRegularRepresentation(k)} \dd (x, k).
    \end{align*}
    The map
    \begin{align*}
        \lambda \in \VectorSpace \mapsto \Fourier [\GroupDirect] f(\lambda)
    \end{align*}
    is called the \emph{Fourier Transform} of $f$ on $\GroupDirect$.
\end{definition}

\begin{proposition}[Inverse formula]
    Let $f \in \SmoothFunctions \GroupDirect$.
    If $(x, k) \in \GroupDirect$,
    then $f$ can be recovered via the formula
    \begin{align*}
        f(x, k) = \int_\VectorSpace \tr\left(\e^{\i \turn \ip x \lambda} \RightRegularRepresentation(k) \Fourier[\GroupDirect] f(\lambda) \right) \dd \lambda.
    \end{align*}
\end{proposition}

\begin{lemma}
    For each $\tau \in \dualGroup\CompactGroup$,
    the space $\HilbertCompactGroup{\tau}$ is an eigenspace of the operator $\Laplacian[\CompactGroup]$.
    Denoting by $\JapaneseBracket{\CompactGroup}{\tau}$ the eigenvalue associated with the operator $\BesselPotential[\CompactGroup]{1}$ on the eigenspace $\HilbertCompactGroup\tau$, we obtain
    \begin{align*}
        \eval{\RightRegularRepresentation\left(\BesselPotential[\CompactGroup]{1}\right)}{\HilbertCompactGroup\tau}
        = \JapaneseBracket{\CompactGroup}{\tau} \Id{\HilbertCompactGroup\tau}
    \end{align*}
\end{lemma}

%\section{Difference operators}
%
%\begin{proposition}
%    Let $\Group$ be a compact topological group.
%    There exists a finite subfamily $\{q_1, \dots, q_M\}$ of
%    \begin{align*}
%        \{ \tau_{ij} - \Kronecker{i}{j} : \tau \in \dualGroup\Group,\ i, j = 1, \dots, \dimRep\tau \}
%    \end{align*}
%    which is \emph{strongly admissible}.
%\end{proposition}
%
%\section{Pseudo-differential calculus}

\chapter{Survey}

\section{The compact case}

In this section,
$\CompactGroup$ denotes a \emph{connected, compact Lie group}.
We equip $\CompactGroup$ with the unique normalized bi-invariant Riemannian metric,
and denote the corresponding Laplace-Beltrami operator by $\Laplacian [\CompactGroup]$.

The description of the unitary dual is given by the following \emph{Peter-Weyl theorem}.

\subsection{Fourier analysis on Lie groups}

\section{The graded case}

\chapter{Motion Groups}

From now on,
$\VectorSpace$ will denote a finite-dimensional vector space over $\R$,
while $\CompactGroup$ is a compact Lie subgroup of $\GeneralLinear\VectorSpace$.

\section{Motion groups}

\begin{definition}[Motion group]
\label{definition:motion_group}
\index{motion group}
    Let $\Group$ be a group.
    We shall say that $\Group$ is a \emph{motion group}
    if there exists a finite dimensional real vector space $\VectorSpace$
    and a compact connected Lie group $\CompactGroup \subset \GeneralLinear\VectorSpace$
    such that $\Group = \VectorSpace \rtimes \CompactGroup$.
\end{definition}

\begin{remark}
\label{remark:notation_kx}
    Let $x \in \VectorSpace$ and $k \in \CompactGroup$.
    We will never identify $x$ with $(x, \Id{\VectorSpace}) \in \VectorSpace \rtimes \CompactGroup$,
    or $k$ with $(0, k) \in \VectorSpace \rtimes \CompactGroup$.
    Therefore, when we write $k x$, it will \emph{always} mean the vector obtained by rotating $x$ by $k$, i.e.\ $k(x)$.
\end{remark}

\begin{example}[Vector spaces]
\label{example:trivial_case_of_motion_groups}
    Suppose that $\CompactGroup = \{\Id\VectorSpace\}$.
    It follows that $\Group \defeq \VectorSpace \rtimes \CompactGroup$ is isomorphic to $\VectorSpace$.
\end{example}

\begin{example}[Euclidean Motion Groups]
\label{example:Euclidean_motion_groups}
\index{Euclidean motion group}
    For each $n \in \N$, let
    \begin{align*}
        \MotionGroup{n} \defeq \{g \in \AffineTransformations{\R^n} : \det g = 1\}.
    \end{align*}
    The elements of $\MotionGroup{n}$ are called \emph{rigid motions},
    while $\MotionGroup{n}$ is called the \emph{Euclidean motion group}.

    It is easily shown that associating $(x, k) \in \R^n \rtimes \SpecialOrthogonalGroup{n}$ to the motion
    \begin{align*}
        g_{(x, k)} : \R^n \to \R^n : y \mapsto x + ky
    \end{align*}
    defines a group isomorphism between $\R^n \rtimes \SpecialOrthogonalGroup{n}$ and $\MotionGroup{n}$.
    We shall therefore identify $\MotionGroup{n}$ with $\R^n \rtimes \SpecialOrthogonalGroup{n}$ from now on.
\end{example}

\begin{example}
    \label{example:complex_motion_groups}
    Let $n \in \N$.
    Consider the group
    \begin{align*}
        \{g \in \AffineTransformations{\C^n} : \det_{\C^n} g = 1\}
    \end{align*}
    where the law is the composition of functions.

    Arguing like in Example~\ref{example:Euclidean_motion_groups},
    the above group can be identified with $\C^n \rtimes \SpecialUnitaryGroup{n}$.
\end{example}

\begin{remark}
    Since in our examples (e.g. Example \ref{example:complex_motion_groups}) our vector space might be $\C^n$,
    we choose to use $\VectorSpace$ to denote the vector space instead of simply $\R^n$ to avoid any confusion.
\end{remark}

\begin{lemma}[$\CompactGroup$-invariant inner product]
\label{lemma:existence_of_K-invariant_inner_product}
    Let $\VectorSpace$ be a vector space,
    and $\CompactGroup$ be a compact Lie group acting acting on $\VectorSpace$.
    There exists an inner product $\ip{\dummy}{\dummy} : \VectorSpace \times \VectorSpace \to \R$ which is $\CompactGroup$-invariant,
    i.e.\ for each $k \in \CompactGroup$ and every $x, y \in \VectorSpace$, we have
    \begin{align*}
        \ip{x}{y} = \ip{k x}{k y}.
    \end{align*}
\end{lemma}
\begin{proof}
    Let $Q : \VectorSpace \times \VectorSpace \to \R$ be an arbitrary inner product.
    Given $x, y \in \VectorSpace$, we let
    \begin{align*}
        \ip{x}{y} \defeq \int_\CompactGroup Q(k x, k y) \dd k,
    \end{align*}
    where $\dd k$ is the Haar measure on $\CompactGroup$.

    It follows that if $k \in \CompactGroup$,
    then using the right-invariance of the Haar measure on compact groups,
    (for this, see for example \cite[Theorem 7.4.21]{RuzhanskyTurunen10})
    we obtain
    \begin{align*}
        \ip{k x}{k y}
        = \int_\CompactGroup Q(h k x, h k y) \dd h
        = \int_\CompactGroup Q(h x, h y) \dd h
        = \ip{x}{y},
    \end{align*}
    i.e.\ $\ip{\dummy}{\dummy}$ is $\CompactGroup$-invariant.

    The fact that $\ip{\dummy}{\dummy}$ is bilinear and nonnegative definite follows immediately from the fact that $Q$ has those properties.
    Now, if $x, y \in \VectorSpace$ are such that $\ip{x}{x} = 0$.
    It follows that $Q(k x, k x) = 0$ for almost every $k \in \CompactGroup$, hence for at least one such $k$.
    However, that means that $k x = 0$, hence $x = 0$ as $k$ is invertible.
\end{proof}

From now on, $\VectorSpace$ will be given the structure of a \emph{Euclidean space} with a $\CompactGroup$-invariant inner product
whose existence is ensured by Lemma~\ref{lemma:existence_of_K-invariant_inner_product}.

\begin{definition}[Lebesgue measure]
    We call the \emph{Lebesgue measure} on $\VectorSpace$
    the unique Haar measure $\mu$ on $\VectorSpace$ such that
    \begin{align*}
        \mu\left(\{x \in \VectorSpace : \ip{x}{x} \leq 1\}\right) = \frac{\pi^{\dim \VectorSpace/2}}{\Gamma(\frac{\dim \VectorSpace}{2} + 1)}.
    \end{align*}
\end{definition}

From now on, integration on $\VectorSpace$ will always be performed with respect to the above Lebesgue measure.

\begin{lemma}
\label{lemma:K_subset_of_SO_and_invariance_of_the_induced_Lebesgue_measure}
    With the inner product defined in Lemma~\ref{lemma:existence_of_K-invariant_inner_product},
    $\CompactGroup$ is a Lie subgroup of $\SpecialOrthogonalGroup\VectorSpace$.

    Therefore, any Haar measure on $\VectorSpace$ is also invariant under the action of $\CompactGroup$.
\end{lemma}

\begin{lemma}[Haar measure]
\label{lemma:Haar_measure}
    If $\dd x$ is the Lebesgue measure on $\VectorSpace$ and $\dd k$ is the normalised Haar measure on $\CompactGroup$,
    then the the product measure $\dd x \dd k$ is a Haar measure on $\Group = \VectorSpace \rtimes \CompactGroup$,
    which is both left and right-invariant.
\end{lemma}
\begin{proof}
    Let $(x, k) \in \Group$.
    \begin{align*}
        \int_\Group f((x, k) (y, l)) \dd (y, l)
        = \int_\VectorSpace \int_\CompactGroup f(x + ky, k l) \dd l \dd y
    \end{align*}

    Now, let us substitute $y$ for $k^{-1}(y - x)$ and $l$ for $k^{-1} l$ in the above.
    As the Lebesgue measure is invariant under $\OrthogonalGroup{\VectorSpace}$ and under translations,
    and because the Haar measure $\dd l$ is left-invariant,
    we obtain
    \begin{align*}
        \int_\Group f((x, k) (y, l)) \dd (y, l)
        &= \int_\VectorSpace \int_\CompactGroup f(y, l) \dd l \dd y\\
        &= \int_\Group f(y, l) \dd (y, l),
    \end{align*}
    showing that $\dd y \dd l$ is indeed a Haar measure on $\Group$.

    Since by Proposition~\ref{proposition:sufficient_conditions_to_be_unimodular} $\dd l$ is also right-invariant,
    arguing similarly shows that $\dd y \dd l$ is also right-invariant.
\end{proof}

\begin{proposition}
    Let $\VectorSpace$ be a finite dimensional vector space
    and $\CompactGroup$ be a subgroup of $\SpecialOrthogonalGroup\VectorSpace$.
    The following properties hold:
    \begin{enumerate}
        \item The Lebesgue measure on $\VectorSpace$ is invariant under $\CompactGroup$,
            i.e.\ for every $k \in \CompactGroup$ and each Borel set $A \subset \VectorSpace$, we have
            \begin{align*}
                \int_A 1 \dd x = \int_{kA} 1 \dd x;
            \end{align*}
        \item The Laplacian on $\VectorSpace$ is invariant under $\CompactGroup$,
            i.e.\ for every $k \in \CompactGroup$ and every $\phi \in \SmoothFunctions\VectorSpace$, we have
            \begin{align*}
                \Laplacian[\VectorSpace] (\phi \circ k)(x) = \Laplacian[\VectorSpace] \phi(k x);
            \end{align*}
        \item The action of $\CompactGroup$ on $\VectorSpace$ commutes with the dilation structure of $\VectorSpace$.
    \end{enumerate}
\end{proposition}
\begin{proof}
    \begin{enumerate}
        \item This follows easily from the change of variables formula,
            \begin{align*}
                \int_{k A} 1 \dd x
                = \int_A (1 \circ k) \det k \dd x
                = \int_A 1 \dd x,
            \end{align*}
            where we used the fact that $\det k = 1$ since $k \in \SpecialOrthogonalGroup\VectorSpace$.
        \item First, observe that
            \begin{align*}
                \dd (\phi \circ k)(x) = \dd \phi(k x) k
            \end{align*}
            which implies that $\grad (\phi \circ k)(x) = k^{-1} \grad \phi(k x)$.

            From there, using the fact that $\Hessian = \dd \grad$,
            \begin{align*}
                \Hessian (\phi \circ k)(x) = (\dd \grad(\phi \circ k))(x) = k^{-1} (\Hessian \phi(k x)) k.
            \end{align*}

            Therefore, we conclude by observing that
            \begin{align*}
                \Laplacian[\VectorSpace] (\phi \circ k)(x)
                = \tr (\Hessian (\phi \circ k)(x))
                = \tr (\Hessian \phi (k x))
                = \Laplacian[\VectorSpace] \phi(k x).
            \end{align*}
    \end{enumerate}
\end{proof}

\section{Lie algebra structure}

We shall denote by $\LieAlgebraCompactgroup$ the Lie algebra of $\CompactGroup$.

\begin{definition}[Lie algebra of $\Group$]
    We shall call the set
    \begin{align*}
        \LieAlgebra \defeq
        \VectorSpace \oplus \LieAlgebraCompactgroup
    \end{align*}
    the \emph{Lie algebra} of $\Group$.  
    Given $(X_1, Y_1), (X_2, Y_2) \in \LieAlgebra$,
    we define its \emph{Lie bracket} via
    \begin{align*}
        \LieBracket{(X_1, Y_1)}{(X_2, Y_2)}
        \defeq \left(Y_1 X_2 - Y_2 X_1, \LieBracket[\LieAlgebraCompactGroup]{Y_1}{Y_2}\right).
    \end{align*}
\end{definition}

\begin{lemma}[Commutation relations]
    Let $X_1, X_2, X \in \VectorSpace$ and $Y_1, Y_2, Y \in \LieAlgebraCompactgroup$.
    We have the following \emph{commutation relations}
    \begin{align*}
        \LieBracket{X_1}{X_2} = 0,\quad
        \LieBracket{X}{Y} = (-Y X, 0),\quad
        \LieBracket{Y_1}{Y_2} = \LieBracket[\LieAlgebraCompactgroup]{Y_1}{Y_2}.
    \end{align*}
\end{lemma}

\begin{definition}[Exponential map]
\label{definition:exponential_map}
\index{motion group!exponential map}
    The \emph{exponential map} on $\Group$ is the map
    \begin{align*}
        \exp_\Group : \LieAlgebra \to \Group : (X, Y) \mapsto \left(\D*{t}<t = 1>(\exp_\CompactGroup tY) X, \exp_\CompactGroup Y\right)
    \end{align*}
\end{definition}

\begin{definition}[Left and right-invariant vector fields]
\label{definition:invariant_differential_operators}
    Let $X \in \LieAlgebra$.
    We define $\LeftDifferentialOperatorFirstOrder{X}$,
    the \emph{left-invariant differential operator associated to $X$}, via
    \begin{align*}
        \LeftDifferentialOperatorFirstOrder{X} f(g)
            \defeq \D*{t}<t = 0> f(g \exp_\Group(t X)),
    \end{align*}
    for each $f \in \SmoothFunctions{\Group}$.

    Similarly,
    we define $\RightDifferentialOperatorFirstOrder{X}$,
    the \emph{right-invariant differential operator associated to $X$}, via
    \begin{align*}
        \RightDifferentialOperatorFirstOrder{X} f(g)
            \defeq \D*{t}<t = 0> f(\exp_\Group(t X) g),
    \end{align*}
    where $f \in \SmoothFunctions{\Group}$.
\end{definition}

\begin{proposition}
    Let $X \in \LieAlgebra$.
    The differential operator $\LeftDifferentialOperatorFirstOrder{X}$ is the only differential operator satisfying the following properties:
    \begin{enumerate}
        \item $\LeftDifferentialOperatorFirstOrder{X}$ is \emph{left-invariant},
            i.e. for every $h \in \Group$, we have
            \begin{align*}
                (X f(h \dummy))(g) = (X f)(h g);
            \end{align*}
        \item The vector in $T_e \Group$ corresponding to the differentiation by $\LeftDifferentialOperatorFirstOrder{X}$ at $e$ is precisely $X$.
        \item Given $X, Y \in \LieAlgebra$, we have
            \begin{align*}
                \eval{%
                    (\LeftDifferentialOperatorFirstOrder{X} \LeftDifferentialOperatorFirstOrder{Y} - \LeftDifferentialOperatorFirstOrder{Y} \LeftDifferentialOperatorFirstOrder{X})
                }{e}
                = \LeftDifferentialOperatorFirstOrder{\LieBracket{X}{Y}}.
            \end{align*}
    \end{enumerate}
\end{proposition}

\begin{example}[$2$-dimensional Euclidean motion group]
\label{example:Lie_Algebra_of_2-dimensional_Euclidean_motion_group}
    Assume $\Group = \R^2 \rtimes \T$.
    The Lie Algebra is the vector space
    \begin{align*}
        \LieAlgebra \defeq \R^2 \oplus \SkewSymmetric{\R^2}.
    \end{align*}
    
    The vectors
    \begin{align*}
        X_1 = (1, 0),\quad
        X_2 = (0, 1)\quad
        X_3 =
            \begin{pmatrix}
                0 & -1\\
                1 &  0
            \end{pmatrix},
    \end{align*}
    form a basis of $\LieAlgebra$
    which satisfies the commutation relations
    \begin{align*}
        \LieBracket{X_1}{X_2} = 0,\quad
        \LieBracket{X_2}{X_3} = X_1,\quad
        \LieBracket{X_3}{X_1} = X_2.
    \end{align*}

    Moreover, if $f \in \SmoothFunctions{\Group}$,
    then the associated left-invariant operators act via
    \begin{align*}
        \LeftDifferentialOperatorFirstOrder{X_1} f(x, t)
            &= \cos(\turn t) \D[f]{x_1}(x, t) + \sin(\turn t) \D[f]{x_2}(x, t)\\
        \LeftDifferentialOperatorFirstOrder{X_2} f(x, t)
            &= -\sin(\turn t) \D[f]{x_1}(x, t) + \cos(\turn t) \D[f]{x_2}(x, t)\\
        \LeftDifferentialOperatorFirstOrder{X_3} f(x, t)
            &= \D[f]{t}(x, t),
    \end{align*}
    where $(x, t) \in \R^2 \rtimes \T$.
\end{example}

\begin{lemma}
    Let $j \in \{1, \dots, \dim \VectorSpace\}$.
    If $\phi \in \SmoothFunctions\Group$,
    \begin{align*}
        \LeftDifferentialOperatorFirstOrder{X_j} \phi(x, k)
        = \ip{k^{-1} \grad \phi(x, k)}{e_j}
    \end{align*}
\end{lemma}
\begin{proof}
    By a simple calculation,
    \begin{align*}
        \LeftDifferentialOperatorFirstOrder{X_j} \phi(x, k)
        =& \D*{t}<t = 0> \phi((x, k) (t e_j, \Id\VectorSpace))\\
        =& \D*{t}<t = 0> \phi(x + t k e_j, k)\\
        =& \ip{\grad \phi(x, k)}{k e_j}.
    \end{align*}

    From there, it follows that:
    \begin{align*}
        \LeftDifferentialOperatorFirstOrder{X_j} \phi(x, k)
        = \ip{k^{-1} \grad \phi(x, k)}{e_j}.
    \end{align*}
\end{proof}

Since $\CompactGroup$ is compact,
$\LieAlgebraCompactgroup$ is \emph{reductive}.
Therefore, it follows that
\begin{align*}
    \LieAlgebra = \VectorSpace \oplus \mathfrak a \oplus \mathfrak s,
\end{align*}
where $\mathfrak a$ is \emph{abelian} and $\mathfrak s$ is \emph{semisimple}.
In particular, we can naturally define an inner product $\ip[\LieAlgebra]\dummy\dummy$ on $\LieAlgebra$ via the Killing form of $\mathfrak s$.

From now on,
$X_1, \dots, X_{\dim \Group}$ will denote a basis such that
\begin{enumerate}
    \item the collection is orthonormal with respect to $\ip[\LieAlgebra]\dummy\dummy$;
    \item If $1 \leq j \leq \dim \VectorSpace$, then $X_j \in \VectorSpace$,
        otherwise $X_j \in \LieAlgebraCompactgroup$.
\end{enumerate}

\begin{definition}
    Let $\alpha \in \N^{\dim \Group}$.
    We define the left-invariant differential operator $\LeftDifferentialOperator{\alpha}$ via
    \begin{align*}
        \LeftDifferentialOperator{\alpha} =
        \LeftDifferentialOperatorFirstOrder{X_1}^{\alpha_1} \dots
        \LeftDifferentialOperatorFirstOrder{X_{\dim \Group}}^{\alpha_{\dim \Group}}
    \end{align*}
\end{definition}

\begin{definition}[Left-invariant Laplacian]
\label{definition:left-invariant_Laplacian}
\index{Laplacian}
    The \emph{left-invariant Laplacian} $\Laplacian$ is the left-invariant differential operator
    \begin{align*}
        \Laplacian \defeq \sum_{j = 1}^{\dim \VectorSpace} \LeftDifferentialOperatorFirstOrder{X_j}^2
    \end{align*}
\end{definition}

\begin{proposition}
    Let $f \in \SmoothFunctions\Group$.
    If $(x, k) \in \Group$, we check that
    \begin{align*}
        \Laplacian f(x, k) = \Laplacian[\VectorSpace] f(x, k) + \Laplacian[\CompactGroup] f(x, k)
    \end{align*}
\end{proposition}
\begin{proof}
    Let $(x, k) \in \Group$.
    By ,
    we check that for each $j \in \{1, \dots, \dim \VectorSpace\}$,
    \begin{align*}
        \LeftDifferentialOperatorFirstOrder{X_j} \LeftDifferentialOperatorFirstOrder{X_j} f(x, k)
        &= \LeftDifferentialOperatorFirstOrder{X_j} \ip{k^{-1} \grad f(x, k)}{e_j}\\
        &= \ip{k^{-1} \dd \grad f(x, k)[k e_j]}{e_j}\\
        &= \ip{k^{-1} \Hessian f(x, k) k e_j}{e_j}.
    \end{align*}

    Summing with respect to $j$, we get that
    \begin{align*}
        \sum_{j = 0}^{\dim \VectorSpace} \LeftDifferentialOperatorFirstOrder{X_j} \LeftDifferentialOperatorFirstOrder{X_j} f(x, k)
        = \tr \Hessian f(x, k) = \Laplacian[\VectorSpace] f(x, k).
    \end{align*}
\end{proof}

\section{Unitary representations}

\begin{definition}
\label{definition:reducible_representation}
    Let $\lambda \in \dualGroup{\VectorSpace}$.
    We define a unitary representation $\Rep{\lambda} \in \Hom(\Group, \End(\Lebesgue{2}{\CompactGroup}))$ of $\Group$ via
    \begin{align}
        \Rep{\lambda} (x, k) F(u) \defeq (u \lambda)(x) F(k^{-1} u),
        \label{eq:reducible_representations_on_the_motion_groups}
    \end{align}
    where $(x, k) \in \CompactGroup$, $F \in \Lebesgue{2}{\CompactGroup}$ and $u \in \CompactGroup$.
\end{definition}

Unfortunately, the above representation is often reducible.
However, as we shall see later, the Fourier Transform on $\Group$ can be written exclusively with those representations.

\begin{example}[$2$-dimensional Euclidean motion group]
    Let $\lambda \in \R^2$.
    Let $(x, t) \in \Group = \R^2 \rtimes \SpecialOrthogonalGroup{2}$.
    If $\lambda \neq 0$, then $\Rep{\lambda}$ is irreducible.

    Using the isomorphism $\SpecialOrthogonalGroup{2} \sim \T$,
    \eqref{eq:reducible_representations_on_the_motion_groups} takes the form
    \begin{align*}
        \Rep\lambda(x, t) F(u)
        = \lambda(-\e^{\i \turn u}x) F(u - t),
    \end{align*}
    where $F \in \Lebesgue{2}{\T}$, $(x, t) \in \Group$, and $u \in \T$.

    Defining ${\Rep\lambda(x, t)}_{m n} \defeq \ip[\Lebesgue{2}\T]{\Rep\lambda(x, t) \e^{\i \turn n \dummy}}{\e^{\i \turn m \dummy}}$ for every $m, n \in \Z$,
    it follows that
    \begin{align*}
        {\Rep\lambda(x, t)}_{m n}
        = \e^{-\i \turn n t} \int_\T \lambda(-\e^{\i \turn u}x) \e^{\i \turn (n - m)u} \dd u.
    \end{align*}
\end{example}

\begin{lemma}[Invariance property]
    Let $\lambda \in \VectorSpace$.
    For each $k \in \CompactGroup$, we have
    \begin{align*}
        \Rep{k \lambda}(y, l) = R_k \Rep{\lambda}(y, l) R_{k^{-1}}
    \end{align*}
\end{lemma}
\begin{proof}
    Let $F \in \Lebesgue{2}\CompactGroup$.
    It follows that
    \begin{align*}
        R_k \Rep\lambda(y, l) R_{k^{-1}} F(u)
        &= R_k (u \lambda)(x) F(l^{-1} u k^{-1})\\
        &= (u k \lambda)(x) F(l^{-1} u)
        = \Rep{k \lambda}(y, l) F(u).
    \end{align*}
\end{proof}

\begin{lemma}
    Suppose $\Group = \R^2 \rtimes \SpecialOrthogonalGroup{2}$.
    If $(x, t) \in \Group$, then for each $m, n \in \Z$, we have
    \begin{align*}
        (m - n) {\Rep\lambda(x, t)}_{m n}
        = - \conj{x} \D{\conj{x}} {\Rep\lambda(x, t)}_{mn}
    \end{align*}
\end{lemma}
\begin{proof}
    Using integration by parts, we obtain:
    \begin{align}
        (m - n) {\Rep\lambda(x, t)}_{m n}
    = -\frac{\e^{-\i \turn n t}}{\i \turn} \int_\T \lambda(-\e^{\i \turn u}x) \D*{u} \e^{\i \turn (n - m)u} \dd u \notag\\
        = \frac{\e^{-\i \turn n t}}{\i \turn} \int_\T \D*{u} \lambda(-\e^{\i \turn u}x) \e^{\i \turn (n - m)u} \dd u.
        \label{eq:lemma:off_diagonal_decay_in_dimension_2}
    \end{align}

    By simple calculations, we can show that
    \begin{align*}
        \D*{u} \lambda(-\e^{\i \turn u} x) = {(-\i \turn)} \conj{x} \D{\conj{x}} \lambda(-\e^{\i \turn u} x).
    \end{align*}
    From there, it follows that~\eqref{eq:lemma:off_diagonal_decay_in_dimension_2} becomes
    \begin{align*}
        (m - n) {\Rep\lambda(x, t)}_{m n}
        = - \conj{x} \D{\conj{x}} {\Rep\lambda(x, t)}_{mn}
    \end{align*}
\end{proof}

\subsection{Unitary dual}

Our description of the unitary dual comes from \cite{Kumahara73},
which itself is a minor adaptation of the one in \cite{Ito52}.
Although both articles only specifically mention the case of the Euclidean motion group,
the author of \cite{Ito52} mentions that the arguments generalise verbatim to motion groups.

Throughout this section, fix $\lambda \in \dualGroup{\VectorSpace}$
and denote by $\IsotropySubgroup{\CompactGroup}{\lambda}$ its isotropy subgroup.

Let $\tau \in \dualGroup{\IsotropySubgroup{\CompactGroup}{\lambda}}$ and denote by $\dimRep{\tau}$ its dimension.
For $q = 1, \dots, \dimRep{\tau}$, let
\begin{align}
    P^\tau_q F(u) \defeq \dimRep{\tau} \int_\IsotropySubgroup{\CompactGroup}{\lambda} \conj{\tau_{qq}(m)} F(u m) \dd m,
    \quad F \in \Lebesgue{2}{\CompactGroup}, u \in \CompactGroup.
    \label{eq:projection_on_L2_of_the_compact_group}
\end{align}

By the Inverse Fourier Transform at $e \in \IsotropySubgroup{\CompactGroup}{\lambda}$,
if $F \in \Lebesgue{2}{\CompactGroup}$ and $u \in \CompactGroup$, then
\begin{align*}
    F(u)
    &= F(u e) = \sum_{\tau \in \dualGroup{\IsotropySubgroup{\CompactGroup}{\lambda}}} \dimRep{\tau} \tr\left(\Rep[\IsotropySubgroup{\CompactGroup}{\lambda}]{\tau}(e) \Fourier[\IsotropySubgroup{\CompactGroup}{\lambda}]{F(u \dummy)}\right)\\
    &= \sum_{\tau \in \dualGroup{\IsotropySubgroup{\CompactGroup}{\lambda}}} \dimRep{\tau} \sum_{q = 1}^{\dimRep{\tau}} {\Fourier[\IsotropySubgroup{\CompactGroup}{\lambda}]{F(u \dummy)}}_{qq}
    = \sum_{\tau \in \dualGroup{\IsotropySubgroup{\CompactGroup}{\lambda}}} \sum_{q = 1}^\dimRep\tau P^\tau_q F(u).
\end{align*}
Now, writing $\Hilbert{\tau}{q} \defeq P^\tau_q \Lebesgue{2}{\CompactGroup}$,
it then follows that
\begin{align*}
    \Lebesgue{2}{\CompactGroup}
    = \sum_{\tau \in \dualGroup{\IsotropySubgroup{\CompactGroup}{\lambda}}} \sum_{q = 1}^\dimRep\tau \Hilbert{\tau}{q}.
\end{align*}

In fact, we shall see that the $P^\tau_q$ are \emph{orthogonal projections}.

\begin{lemma}
    Let $\mu, \tau \in \dualGroup{\IsotropySubgroup{\CompactGroup}{\lambda}}$, $q \in \{1, \dots, \dimRep{\tau}\}$ and $m, n \in \{1, \dots, \dimRep{\tau} \}$.
    If $f \in \Lebesgue{2}{\RightQuotient{\CompactGroup}{\IsotropySubgroup{\CompactGroup}{\lambda}}}$, then
    \begin{align*}
        P^\mu_q (f \otimes \tau_{m n}) = \Kronecker{\mu}{\tau} \Kronecker{n}{q} (f \otimes \tau_{m n}).
    \end{align*}

    In particular, the following properties hold:
    \begin{enumerate}
        \item $P^\mu_q$ is an orthogonal projection onto
            \begin{align*}
                \Hilbert{\mu}{q} =
                    \Lebesgue{2}{\RightQuotient{\CompactGroup}{\IsotropySubgroup{\CompactGroup}{\lambda}}}
                    \otimes
                    \Span \{\mu_{p q} : p = 1, \dots, \dimRep{\mu}\};
            \end{align*}
        \item The Hilbert spaces
            \begin{align*}
                \{\Hilbert{\tau}{q} : \tau \in \dualGroup{\IsotropySubgroup{\CompactGroup}{\lambda}}, q = 1, \dots, \dimRep{\tau} \}
            \end{align*}
            are mutually orthogonal;
        \item We have the decomposition
            \begin{align*}
                \Lebesgue{2}{\CompactGroup} = \bigoplus_{\tau \in \dualGroup{\IsotropySubgroup{\CompactGroup}{\lambda}}} \bigoplus_{q = 1}^{\dimRep{\tau}} \Hilbert{\tau}{q}.
            \end{align*}
    \end{enumerate}
\end{lemma}
\begin{proof}
    Fix $u \in \CompactGroup$ and write $u = u' u''$,
    where $u' \in \RightQuotient{\CompactGroup}{\IsotropySubgroup{\CompactGroup}{\lambda}}$
    and $u'' \in \IsotropySubgroup{\CompactGroup}{\lambda}$.
    It follows that
    \begin{align*}
        P^\mu_q (f \otimes \tau_{m n}) (u)
        &= \dimRep{\mu}
            \int_\IsotropySubgroup{\CompactGroup}{\lambda}
                \conj{\mu_{q q}(m)}
                f(u')
                \tau_{m n}(u'' m)
            \dd m\\
        &= \sum_{p = 1}^\dimRep{\tau}
                f(u')
                \tau_{m p}(u'')
                \dimRep{\mu}
                \int_\IsotropySubgroup{\CompactGroup}{\lambda}
                    \conj{\mu_{q q}(m)}
                    \tau_{p n}(m)
                \dd m.
    \end{align*}

    Using the Peter-Weyl Theorem, we conclude that in fact
    \begin{align*}
        P^\mu_q (f \otimes \tau_{m n}) (u)
        &= \sum_{p = 1}^\dimRep{\tau}
            \Kronecker{\mu}{\tau}
            \Kronecker{q}{p}
            \Kronecker{q}{n}
            f(u')
            \tau_{m p}(u'')\\
        &= \Kronecker{\mu}{\tau}
            \Kronecker{q}{n}
            f(u')
            \tau_{m q}(u'')\\
        &= \Kronecker{\mu}{\tau}
            \Kronecker{q}{n}
            (f \otimes \tau_{m n})(u),
    \end{align*}
    which is what we wanted to show.
\end{proof}

The following result and its proof can be found in \cite[Theorem 1.1, 1.2, 1.3]{Ito52}.
Although the paper specifically treats the case of the Euclidean motion groups,
the author remarks (\cite[Remark p. 84]{Ito52}) that the argument works for motion groups.

\begin{proposition}[Unitary dual]
\label{proposition:unitary_dual}
    Let $\lambda, \lambda' \in \dualGroup{\VectorSpace} \setminus \{0\}$
    and $\tau \in \dualGroup{\IsotropySubgroup{\CompactGroup}{\lambda}}$,
    $\tau' \in \dualGroup{\IsotropySubgroup{\CompactGroup}{\lambda'}}$.
    The following properties hold
    \begin{enumerate}
        \item $\Rep{\lambda}$ restricts to an infinite-dimensional irreducible unitary representation on each $\Hilbert{\tau}{q}$;
        \item $(\Hilbert{\tau}{q}, \Rep{\lambda})$ and $(\Hilbert{\tau'}{q'}, \Rep{\lambda'})$ are equivalent if and only if
            \begin{align*}
                \lambda' = k \lambda \quad \text{and} \quad \EquivalenceClass{\dualGroup{\IsotropySubgroup{\CompactGroup}{\lambda}}}{\tau} = \EquivalenceClass{\dualGroup{\IsotropySubgroup{\CompactGroup}{\lambda}}}{\tau'(k \dummy k^{-1})}
            \end{align*}
            for some $k \in \CompactGroup$.
            In particular, if $q_1, q_2 \in \{1, \dots, \dimRep\tau\}$,
            then $(\Hilbert{\tau}{q_1}, \Rep{\lambda})$ and $(\Hilbert{\tau}{q_2}, \Rep{\lambda})$ are equivalent.
    \end{enumerate}
\end{proposition}

\begin{definition}[Unitary dual]
\label{definition:unitary_dual_of_motion_group}
\index{motion group!unitary dual}
    Fix $\lambda_0 \in \dualGroup{\VectorSpace} \setminus \{0\}$.
    We define the \emph{unitary dual} of $\Group$, denoted by $\dualGroup{\Group}$, via
    \begin{align*}
        \dualGroup{\Group} \defeq \{ (\Hilbert{\tau}{1}, \Rep{\lambda \lambda_0}) : \lambda \in \R^+, \tau \in \dualGroup{\IsotropySubgroup{\CompactGroup}{\lambda}} \}.
    \end{align*}
\end{definition}

\begin{definition}
    A \emph{measurable field of operators on $\dualGroup\Group$} is a map
    \begin{align*}
        \sigma : \dualGroup\VectorSpace \to \End(\SmoothFunctions{\CompactGroup})
    \end{align*}
    satisfying the following properties.
    \begin{enumerate}
        \item For each $\lambda \in \dualGroup\VectorSpace$,
            each $\tau \in \dualGroup{\IsotropySubgroup{\CompactGroup}{\lambda}}$, and each $q \in \{1, \dots, \dimRep \tau\}$, we have
            \begin{align*}
                \sigma(\lambda) \SmoothVectors{\Hilbert{\tau}{q}} \subset \SmoothVectors{\Hilbert{\tau}{q}}
            \end{align*}
        \item If $\lambda_1, \lambda_2 \in \dualGroup\VectorSpace$ and if $H_1, H_2$ are two Hilbert subspaces of $\Lebesgue{2}{\CompactGroup}$ such that
            $(H_1, \Rep{\lambda_1})$ and $(H_2, \Rep{\lambda_2})$ are equivalent,
            then if $T : H_1 \to H_2$ is the intertwining operator,
            we have
            \begin{align*}
                \eval{\sigma(\lambda_1)}{H_1}  = T \sigma(\lambda_2) T^{-1}
            \end{align*}
    \end{enumerate}
\end{definition}

\subsection{Infinitesimal representations}

\begin{definition}[Infinitesimal Representation]
\label{definition:infinitesimal_representation}
\index{motion group!infinitesimal representation}
    Let $X \in \g$.
    We define the infinitesimal representation of $X$ as the operator
    \begin{align*}
        \Rep{\lambda}(X) : \SmoothFunctions{\CompactGroup} \to \SmoothFunctions{\CompactGroup}
    \end{align*}
    defined via
    \begin{align*}
        \Rep{\lambda}(X) F(u) \defeq \D*{t}<t=0> \Rep{\lambda}(\exp(t X)) F(u),
    \end{align*}
    where $F \in \SmoothFunctions{\CompactGroup}$.
\end{definition}

\begin{proposition}[Infinitesimal representations]
\label{proposition:infinitesimal_representations_of_differential_operators}
    Let $\lambda \in \dualGroup{\VectorSpace}$ and let $j \in \{1, \dots, \dim \Group\}$.
    The infinitesimal representation of $X_j$ has the following expression:
    \begin{enumerate}
        \item if $j \leq \dim \VectorSpace$, then
            \begin{align*}
                \Rep{\lambda}(X_j) F(u) = \directionalDerivative{u^{-1} e_i} \lambda(0) F(u)
            \end{align*}
        \item if $j > \dim \VectorSpace$, then
            \begin{align*}
                \Rep{\lambda}(X_j) F(u) = -\RightDifferentialOperatorFirstOrder{Y_j} F(u),
            \end{align*}
            where $Y_j$ is like in~\eqref{eq:Lie_algebra_vector_coming_from_compact_group} and
            $\RightDifferentialOperatorFirstOrder{Y_j}$ is the right-invariant differential operator on $\CompactGroup$ associated with $Y_j$.
    \end{enumerate}
    In the above, $F$ is an arbitrary function in $\SmoothFunctions{\CompactGroup}$.
\end{proposition}
\begin{proof}
    \begin{enumerate}
        \item Fix $j \in \{1, \dots, \dim \VectorSpace\}$.
            Since $\exp_\Group(t X_j) = (t e_i, \Id{\VectorSpace})$.
            It follows that
            \begin{align*}
                \Rep{\lambda}(X) F(u) = \D*{t}<t = 0> \lambda(t u^{-1} e_i) F(u)
                = \directionalDerivative{u^{-1} e_i} \lambda(0) F(u)
            \end{align*}
            which is what we wanted to show.
        \item Let $Y_j$ be like in~\eqref{eq:Lie_algebra_vector_coming_from_compact_group} so that
            \begin{align*}
                \exp_\Group(t X_j) = \Phi^{-1}
                    \begin{pmatrix}
                        \exp_\CompactGroup(t Y_j) & 0\\
                        0 & 1
                    \end{pmatrix}
                    = (0, \exp_\CompactGroup(t Y_j)).
            \end{align*}

            From there, it immediately follows that
            \begin{align*}
                \Rep{\lambda}(X_j) F(u)
                = \D*{t}<t = 0> F({(\exp_\CompactGroup(t Y_j))}^{-1} u)
                = -\RightDifferentialOperatorFirstOrder{Y_j} F(u).
            \end{align*}
    \end{enumerate}
\end{proof}

\begin{corollary}[Infinitesimal representation of $\Laplacian$]
\label{corollary:infinitesimal_representation_of_the_Laplacian}
    Let $\lambda \in \dualGroup{\VectorSpace}$.
    The infinitesimal representation of $\Laplacian$ is given by
    \begin{align*}
        \Rep{\lambda}(\Laplacian) = - \norm[\dualGroup{\VectorSpace}]{\lambda}^2 \Id{\Lebesgue{2}{\CompactGroup}} + \RightLaplacian[\CompactGroup].
    \end{align*}
\end{corollary}
\begin{proof}
    By Proposition~\ref{proposition:infinitesimal_representations_of_differential_operators},
    we know that
    \begin{align*}
        \Rep{\lambda}(\Laplacian) F(u) =
        \left(
            \sum_{j = 1}^{\dim \VectorSpace}
                \directionalDerivative{u^{-1} e_i}^2 \lambda(0)
                + \RightLaplacian[\CompactGroup]
        \right)
        F(u).
    \end{align*}

    Since the Laplacian on $\VectorSpace$ is invariant under $\CompactGroup$,
    we have
    \begin{align*}
        \Laplacian[\VectorSpace] \lambda(0)
        = (\Laplacian[\VectorSpace] \lambda \circ u)(0)
        = \Laplacian (\lambda \circ u)(0).
    \end{align*}

    Since the right-hand side is exactly equal to
    \begin{align*}
        \sum_{j = 1}^{\dim \VectorSpace} \directionalDerivative{u^{-1} e_j}^2 \lambda(0),
    \end{align*}
    it follows that we have
    \begin{align*}
        \Rep{\lambda}(\Laplacian) F(u)
        = (\Laplacian[\VectorSpace] \lambda(0) + \RightLaplacian[\CompactGroup]) F(u),
    \end{align*}
    which concludes the proof.
\end{proof}

\section{Fourier Transform}

\subsection{Definition and elementary properties}

\begin{definition}[Fourier transform]
\label{definition:Fourier_Transform}
\index{motion group!Fourier transform}
    Let $f \in \Lebesgue{1}{\Group}$ and $\lambda \in \dualGroup{\VectorSpace}$.
    We define its \emph{Fourier coefficient} at $\lambda$ via
    \begin{align*}
        \Fourier{f}(\lambda) \defeq \int_\Group f(g) \adj{\Rep{\lambda}(g)} \dd g.
    \end{align*}

    Moreover, the map
    \begin{align*}
        \Fourier{f} : \dualGroup{\VectorSpace} \to \End(\Lebesgue{2}{\CompactGroup}) :
        \lambda \mapsto \Fourier{f}(\lambda)
    \end{align*}
    is called the \emph{Fourier Transform} of $f$.
\end{definition}

\begin{lemma}
\label{lemma:kernels_of_Fourier_coefficients}
    Let $\lambda \in \VectorSpace$, and $f \in \Lebesgue{1}\Group$.
    The \emph{integral kernel} of the operator $\Fourier f(\lambda)$ is given by
    \begin{align}
        K_{f}(\lambda; u, k) \defeq \int_\CompactGroup \Fourier[\VectorSpace] f(k \lambda, k u^{-1}),
        \label{integral_kernel_of_Fourier_coefficient}
    \end{align}
    i.e. for every $F \in \Lebesgue{2}\CompactGroup$ and every $u \in \CompactGroup$, we have
    \begin{align*}
        \Fourier f(\lambda) F(u) = \int_\CompactGroup K_f(\lambda; u, k) F(k) \dd k.
    \end{align*}

    In particular, the following properties hold:
    \begin{enumerate}
        \item if $f \in \Schwartz\Group$, then $K_f$ is smooth,
            $\Fourier f(\lambda)$ is trace class and
            \begin{align*}
                \tr(\Fourier f(\lambda)) = \int_\CompactGroup \Fourier[\VectorSpace] f(k \lambda, e) \dd k;
            \end{align*}
        \item if $f \in \Lebesgue{1}\Group \cap \Lebesgue{2}\Group$, then for almost every $\lambda \in \VectorSpace$,
            $\Fourier f(\lambda)$ is Hilbert-Schmidt and
            \begin{align*}
                \norm[\SchattenClasses{2}{\Lebesgue{2}\CompactGroup}]{\Fourier f(\lambda)}^2
                = \int_\CompactGroup \int_\CompactGroup \abs{\Fourier[\VectorSpace] f(k \lambda, u)}^2 \dd u \dd k
            \end{align*}
    \end{enumerate}
\end{lemma}
\begin{proof}
    Let $F \in \Lebesgue{2}\CompactGroup$ and $u \in \CompactGroup$.
    By definition of the Fourier Transform,
    \begin{align*}
        \Fourier f(\lambda) F(u) =
        \int_\VectorSpace
            \int_\CompactGroup
                f(x, k) \e^{-\i \turn \ip{k u \lambda}{x}} F(k u).
            \dd k
        \dd x
    \end{align*}

    Recognising the Fourier Transform on $\VectorSpace$ in the above, we obtain
    \begin{align*}
        \Fourier f(\lambda) F(u) =
        \int_\CompactGroup
            \Fourier[\VectorSpace] f(k u \lambda, k) F(k u)
        \dd k
        =
        \int_\CompactGroup
            \Fourier[\VectorSpace] f(k \lambda, k u^{-1}) F(k),
        \dd k
    \end{align*}
    where we substituted $k$ for $k u^{-1}$ to obtain the last line.

    From there, it follows that the kernel is indeed given by~\eqref{integral_kernel_of_Fourier_coefficient}.
    Let us now prove the two remaining claims.

    \begin{enumerate}
        \item If $f \in \Schwartz\Group$, it follows that the integral kernel is smooth.
            Using~\cite[Corollary 4.1]{DelgadoRuzhansky14}, it follows that $\Fourier f(\lambda)$ is trace-class, and
            \begin{align*}
            \tr(\Fourier f(\lambda))
            = \int_\CompactGroup K_f(\lambda; k, k) \dd k
            = \int_\CompactGroup \Fourier[\VectorSpace] f(k \lambda, e) \dd k.
        \end{align*}
    \item Now, if $f \in \Lebesgue{1}\Group \cap \Lebesgue{2}\Group$,
        then $K_{f} \in \Lebesgue{2}{\CompactGroup \times \CompactGroup}$ for almost every $\lambda \in \VectorSpace$.
        For such $\lambda$, it follows by~\cite[Theorem VI.23]{Reed72} that $\Fourier f(\lambda)$ is Hilbert-Schmidt and
        \begin{align*}
            \norm[\SchattenClasses{2}{\Lebesgue{2}\CompactGroup}]{\Fourier f(\lambda)}^2
            &= \int_\CompactGroup \int_\CompactGroup \abs{K_f(\lambda; u, k)}^2 \dd k \dd u\\
            &= \int_\CompactGroup \int_\CompactGroup \abs{\Fourier[\VectorSpace] f(k \lambda, k u^{-1})}^2 \dd k \dd u.
        \end{align*}
        Substituing $u$ for $u^{-1} k$ in the above, we obtain
        \begin{align*}
            \norm[\SchattenClasses{2}{\Lebesgue{2}\CompactGroup}]{\Fourier f(\lambda)}^2
            &= \int_\CompactGroup \int_\CompactGroup \abs{\Fourier[\VectorSpace] f(k \lambda, u)}^2 \dd k \dd u,
        \end{align*}
        as required.
    \end{enumerate}
\end{proof}

\begin{definition}
    We shall say that $L \in \SmoothFunctions{\VectorSpace \times \CompactGroup \times \CompactGroup}$ belongs to $\ScalarImageSchwartz\Group$ if and only if
    \begin{enumerate}
        \item $L$ is rapidly decaying in $x$, i.e.\ for each $N \in \N$,
            \begin{align*}
                \seminorm[\ScalarImageSchwartz\Group]{N}{L}
                \defeq
                \sup_{\abs\alpha, \abs\beta, \abs{\beta'} \leq N}
                \abs{%
                    {(1 + \abs{x})}^N
                    \LeftDifferentialOperatorOnCompactGroup[k_1]{\beta}
                    \LeftDifferentialOperatorOnCompactGroup[k_2]{\beta'}
                    \D[L]{x^\alpha}(x, k_1, k_2)
                }
                < \infty.
            \end{align*}
        \item for each $k \in \CompactGroup$, we have
            \begin{align*}
                L(k \lambda; k_1, k_2) = L(\lambda, k_1 k, k_2 k).
            \end{align*}
    \end{enumerate}
\end{definition}

The proof of the following result can be found in~\cite{Kumahara76}.
\begin{proposition}
    The map
    \begin{align*}
        f \in \Schwartz\Group \to K_f \in \ScalarImageSchwartz\Group
    \end{align*}
    is a topological isomorphism from $\Schwartz\Group$ onto $\ScalarImageSchwartz\Group$.
\end{proposition}

\subsection{Plancherel formula}

\begin{proposition}[Plancherel formula]
\label{proposition:Plancherel_formula}
\index{motion group!Fourier transform!Plancherel formula}
    Let $f \in \Lebesgue{1}{\Group} \cap \Lebesgue{2}{\Group}$.
    The following formula holds
    \begin{align}
        \int_G \abs{f}^2 \dd g = \int_\dualGroup{\VectorSpace} \norm[\HilbertSchmidt{\Lebesgue{2}{\CompactGroup}}]{\Fourier{f}(\lambda)}^2 \dd \Plancherel{\VectorSpace}(\lambda).
        \label{proposition:Plancherel_formula:formula}
    \end{align}
\end{proposition}
\begin{proof}
    It follows from Lemma~\ref{lemma:kernels_of_Fourier_coefficients} that for almost every $\lambda \in \VectorSpace$,
    $\Fourier f(\lambda)$ is trace class and
    \begin{align*}
        \norm[\SchattenClasses{2}{\Lebesgue{2}\CompactGroup}]{\Fourier f(\lambda)}^2
        = \int_\CompactGroup \int_\CompactGroup \abs{\Fourier[\VectorSpace] f(k \lambda, u)}^2 \dd u \dd k.
    \end{align*}

    Now, integrating with respect to $\lambda$,
    we obtain
    \begin{align*}
        \int_\dualGroup{\VectorSpace} \norm[\HilbertSchmidt{\Lebesgue{2}{\CompactGroup}}]{\Fourier{f}(\lambda)}^2 \dd \Plancherel{\VectorSpace}(\lambda)
        &= \int_\dualGroup{\VectorSpace} \int_\CompactGroup \abs{\Fourier[\VectorSpace]{f}(\lambda, k)}^2 \dd k \dd \Plancherel{\VectorSpace}(\lambda)\\
        &= \int_\VectorSpace \int_\CompactGroup \abs{f(x, k)}^2 \dd u \dd k,
    \end{align*}
    where the last line was obtained by applying the Plancherel formula on $\VectorSpace$.
\end{proof}

%\begin{lemma}
%    Let $\phi \in \Schwartz\Group$.
%    For each $\lambda \in \VectorSpace$,
%    the operator $\Fourier \phi(\lambda)$ is trace class.
%    Moreover, for each $N \in \N$, there exists $C \geq 0$ such that
%    \begin{align*}
%        \norm[\SchattenClasses{1}{\Lebesgue{2}{\CompactGroup}}]{\Fourier \phi(\lambda)}
%        &\leq C {(1 + \abs\lambda)}^{-N}.
%    \end{align*}
%    In particular, the map
%    \begin{align*}
%        \lambda \in \VectorSpace \mapsto \norm[\SchattenClasses{1}{\Lebesgue{2}{\CompactGroup}}]{\Fourier \phi(\lambda)}
%    \end{align*}
%    is integrable.
%\end{lemma}
%\begin{proof}
%    Let $\alpha > \dim \CompactGroup$.
%    It follows by \cite[Proposition 3.3]{DelgadoRuzhansky14} that
%    \begin{align}
%        \BesselPotential[\CompactGroup]{-\alpha} \in \SchattenClasses{1}{\Lebesgue{2}\CompactGroup}
%        \label{lemma:preparation_for_inverse_formula:Bessel_potential_in_trace_class}
%    \end{align}
%
%    Let $\lambda \in \VectorSpace$.
%    We check that for each $F \in \Lebesgue{2}\CompactGroup$,
%    \begin{align*}
%        \BesselPotential[\CompactGroup]{\alpha} \Fourier \phi(\lambda) F(u)
%        = \int_\CompactGroup \BesselPotential[\CompactGroup]{\alpha}_u K_{\phi, \lambda}(u, k) F(k) \dd k,
%    \end{align*}
%    where $K_{\phi, \lambda}(u, k)$ represents the kernel of $\Fourier \phi(\lambda)$.
%
%    Since $K_{\phi, \lambda}$ is smooth with respect to $(u, k)$ and is rapidly decaying in $\lambda$,
%    it follows that for each $N \in \N$, there exists $C \geq 0$ such that
%    \begin{align*}
%        \norm[\Lin{\Lebesgue{2}\CompactGroup}]{\BesselPotential[\CompactGroup]{\alpha} \Fourier \phi(\lambda)}
%        \leq C {(1 + \abs\lambda)}^{-N}.
%    \end{align*}
%
%    Combining the above with~\eqref{lemma:preparation_for_inverse_formula:Bessel_potential_in_trace_class},
%    we obtain
%    \begin{align*}
%        \norm[\SchattenClasses{1}{\Lebesgue{2}{\CompactGroup}}]{\Fourier \phi(\lambda)}
%        &\leq
%        \norm[\SchattenClasses{1}{\Lebesgue{2}\CompactGroup}]{\BesselPotential[\CompactGroup]{-\alpha}}
%            \norm[\Lin{\Lebesgue{2}\CompactGroup}]{\BesselPotential[\CompactGroup]{\alpha} \Fourier \phi(\lambda)}\\
%        &\leq C {(1 + \abs\lambda)}^{-N},
%    \end{align*}
%    which is the desired estimate.
%\end{proof}

\begin{proposition}[Inverse Fourier Transform]
\label{proposition:inverse_Fourier_Transform}
\index{motion group!Fourier transform!inverse formula}
    Let $\phi \in \Schwartz{\Group}$.
    For each $g \in \Group$,
    we have
    \begin{align*}
        \phi(g)
        = \int_\dualGroup{\VectorSpace}
        \tr \left( \Rep{\lambda}(g) \Fourier \phi(\lambda) \right) \dd \Plancherel{\VectorSpace}(\lambda).
    \end{align*}
\end{proposition}
\begin{proof}
    Let us assume that $g = e$.
    By Lemma~\ref{lemma:kernels_of_Fourier_coefficients}, we know that $\Fourier \phi(\lambda)$ is trace class and
    \begin{align*}
        \tr(\Fourier \phi(\lambda))
        = \int_\CompactGroup \Fourier[\VectorSpace] \phi(k \lambda, e) \dd k.
    \end{align*}

    Integrating with respect to $\lambda$, we obtain
    \begin{align*}
        \int_\VectorSpace \tr(\Fourier \phi(\lambda)) \dd \lambda
        &= \int_\VectorSpace \int_\CompactGroup \Fourier[\VectorSpace] \phi(k \lambda, e) \dd k \dd \lambda\\
        &= \int_\VectorSpace \Fourier[\VectorSpace] \phi(\lambda, e) \dd \lambda,
    \end{align*}
    where the last line was obtained by a change of variables after permuting the integrals.

    Recognising an inverse Fourier Transform in the right-hand side of the above, we obtain
    \begin{align*}
        \int_\VectorSpace \tr(\Fourier \phi(\lambda)) \dd \lambda
        = \phi(0, e),
    \end{align*}
    concluding the case $g = e$.

    The general case follows immediately, since
    \begin{align*}
        \phi(g) = \phi(e g) = \int_\VectorSpace \tr(\Fourier \{\phi(\dummy g)\}(\lambda)) \dd \lambda
        = \int_\VectorSpace \tr(\Rep\lambda(g) \Fourier \phi(\lambda)) \dd \lambda,
    \end{align*}
    where the last equality was obained by Proposition~\ref{proposition:elementary_properties_of_the_Fourier_transform}.
\end{proof}

\section{Fourier Transform of distributions}

\begin{definition}[Image of the Schwartz space]
    We shall denote by $\Schwartz{\dualGroup\Group}$
    the set of all functions $F \in \SmoothFunctions{\VectorSpace, \Lin{\Lebesgue{2}\CompactGroup}}$ such that:
    \begin{enumerate}
        \item For each $\alpha \in \N^{\dim \VectorSpace}$,
            $\D[F]{\lambda^\alpha}$ leaves $\SmoothFunctions\CompactGroup$ stable;
        \item For each $N \in \N$, the quantity
            \begin{align*}
                \seminorm[\Schwartz{\dualGroup\Group}]{N}{F} \defeq
                \sup_{\abs\alpha, \abs\beta, \abs{\beta'} \leq N}
                \sup_{\lambda \in \VectorSpace}
                {(1 + \abs\lambda)}^N
                \norm[\Lin{\Lebesgue{2}\CompactGroup}]{%
                    \Rep\lambda(\LeftDifferentialOperator\beta)
                    \D[F]{\lambda^\alpha}(\lambda)
                    \Rep\lambda(\LeftDifferentialOperator{\beta'})
                }
            \end{align*}
            is finite;
        \item For every $k \in \VectorSpace$ and each $\lambda \in \VectorSpace$,
            \begin{align*}
                F(\lambda) = R_k^{-1} F(k \lambda) R(k).
            \end{align*}
    \end{enumerate}
\end{definition}

\begin{lemma}
    The space $\Schwartz{\dualGroup\Group}$ is a Fr\'echet space
    whose topology is given by the seminorms $\seminorm[\Schwartz{\dualGroup\Group}]{N}{\dummy}$, $N \in \N$.
\end{lemma}

\begin{proposition}[Fourier transform and duality]
    Let $f \in \Schwartz\Group$, and $H \in \Schwartz{\dualGroup\Group}$.
    We have the identities
    \begin{align*}
        \ip[\dualGroup\Group]{\Fourier f}{H}
        = \ip[\Group]{f}{\iota \circ \InverseFourier H},\quad
        \ip[\Group]{\InverseFourier H}{f}
        = \ip[\dualGroup\Group]{H}{\Fourier (\iota \circ f)},
    \end{align*}
    where $(\iota \circ f)(g) = f(g^{-1})$.
\end{proposition}

\begin{definition}[Fourier Transform on distributions]
    Let $f \in \TemperedDistributions\Group$.
    We define the \emph{Fourier Transform} of $f$, $\Fourier f$,
    an element of $\TemperedDistributions{\dualGroup\Group}$, via
    \begin{align*}
        \ip[\dualGroup\Group]{\Fourier f}{H}
        \defeq \ip[\Group]{f}{\iota \circ \InverseFourier H},\quad
    \end{align*}
    for any $H \in \Schwartz{\dualGroup\Group}$.

    Similarly, let $H \in \TemperedDistributions{\dualGroup\Group}$.
    We define the tempered distribution $\InverseFourier H \in \TemperedDistributions\Group$,
    called the \emph{inverse Fourier Transform} of $H$, via
    \begin{align*}
        \ip[\Group]{\InverseFourier H}{f}
        = \ip[\dualGroup\Group]{H}{\Fourier (\iota \circ f)},
    \end{align*}
    for any $f \in \TemperedDistributions\Group$.
\end{definition}

\begin{proposition}
    The Fourier Transform is a topological linear isomorphism
    of $\TemperedDistributions\Group$ onto $\TemperedDistributions{\dualGroup\Group}$.
\end{proposition}

%\begin{definition}[$\LebesgueDual{2}{\Group}$]
%    We shall say that a map
%    \begin{align*}
%        \sigma : \dualGroup{\VectorSpace} \to \HilbertSchmidt{\Lebesgue{2}{\CompactGroup}}
%    \end{align*}
%    belongs to $\LebesgueDual{2}{\Group}$ if and only if the following conditions are met:
%    \begin{enumerate}
%        \item $\sigma$ is measurable;
%        \item for each $k \in \CompactGroup$, we have
%            \begin{align*}
%                \sigma(k \lambda) = R_k \sigma(\lambda) R_k^{-1}
%            \end{align*}
%        \item the quantity
%            \begin{align*}
%                \norm[\LebesgueDual{2}{\Group}]{\sigma} \defeq
%                    \left(
%                        \int_\dualGroup{\VectorSpace}
%                            \norm[\HilbertSchmidt{\Lebesgue{2}{\CompactGroup}}]{\sigma(\lambda)}^2
%                        \dd \Plancherel{\VectorSpace}(\lambda)
%                    \right)^{\frac{1}{2}}
%            \end{align*}
%            is finite.
%    \end{enumerate}
%
%    If $\sigma_1, \sigma_2 \in \LebesgueDual{2}{\Group}$, then we let
%    \begin{align*}
%        \ip[\LebesgueDual{2}{\Group}]{\sigma_1}{\sigma_2} \defeq
%        \int_\dualGroup{\VectorSpace}
%            \tr\left(
%                \sigma_1(\lambda) \adj{\sigma_2(\lambda)}
%            \right)
%        \dd \Plancherel{\VectorSpace}(\lambda).
%    \end{align*}
%    If we quotient $\LebesgueDual{2}{\Group}$ by $\Plancherel{\VectorSpace}$-almost everywhere equality,
%    which we shall do from now onwards,
%    then the above gives $\LebesgueDual{2}{\Group}$ the structure of a Hilbert space.
%\end{definition}
%
%\begin{definition}[$\Kernels{\Group}$]
%    We shall say that a tempered distribution $\kappa \in \TemperedDistributions{\Group}$ belongs to $\Kernels{\Group}$
%    if and only if the map
%    \begin{align}
%        T_\kappa : \Schwartz{\Group} \to \TemperedDistributions{\Group} : f \mapsto \conv{f}{\kappa}
%    \end{align}
%    extends to a continuous map from $\Lebesgue{2}{\Group}$ into itself.
%    In this case, we let
%    \begin{align*}
%        \norm[\Kernels{\Group}]{\kappa} \defeq \norm[\Lin{\Lebesgue{2}{\Group}}]{T_\kappa}.
%    \end{align*}
%\end{definition}
%
%\begin{definition}
%    We shall say that a map
%    \begin{align*}
%        \sigma : \dualGroup{\VectorSpace} \to \Lin{\Lebesgue{2}{\CompactGroup}}
%    \end{align*}
%    belongs to $\LebesgueDual{\infty}{\Group}$ if and only if it is invariant under $\CompactGroup$ and the quantity
%    \begin{align*}
%        \norm[\LebesgueDual{\infty}{\Group}]{\sigma} \defeq
%            \esssup_{\lambda \in \dualGroup{\VectorSpace}}
%                \norm[\Lin{\Lebesgue{2}{\CompactGroup}}]{\sigma(\lambda)}
%    \end{align*}
%    is finite.
%    The essential supremum is taken with respect to the Plancherel measure.
%\end{definition}
%
%\begin{theorem}[Abstract Plancherel formula]
%    The Fourier Transform can be extended to a \emph{surjective} isometry
%    \begin{align*}
%        \Fourier : \Lebesgue{2}{\Group} \to \LebesgueDual{2}{\Group}.
%    \end{align*}
%
%    Moreover, for every left-invariant operator $T \in \Lin{\Lebesgue{2}{\Group}}$,
%    there exists a unique element $\sigma \in \LebesgueDual{\infty}{\Group}$ such that
%    \begin{align*}
%        \Fourier\{T f\}(\lambda) = \sigma(\lambda) \Fourier f(\lambda)
%    \end{align*}
%    holds for $\Plancherel{\Group}$-almost every $\lambda \in \dualGroup{\VectorSpace}$.
%\end{theorem}

\subsection{Sobolev spaces}

\begin{definition}[Sobolev norm]
    Let $s \in \R$.
    If $\phi \in \Schwartz\Group$, we let
    \begin{align*}
        \norm[\Sobolev{s}]{\phi} \defeq
        \left(
            \int_\dualGroup\VectorSpace
                \norm[\HilbertSchmidt{\Lebesgue{2}{\CompactGroup}}]{%
                    \Rep\lambda \BesselPotential{s}
                    \Fourier \phi(\lambda)
                    }^2
            \dd \Plancherel\VectorSpace(\lambda)
        \right)^{1 / 2}.
    \end{align*}
\end{definition}

\begin{definition}[Sobolev spaces]
\label{definition:Sobolev_spaces}
    Let $s \in \R$.
    We define the \emph{Sobolev space of order $s$} to be the completion of $\Schwartz\Group$ with the norm $\norm[\Sobolev{s}]{\dummy}$.
\end{definition}

\subsubsection{Sobolev embeddings}

\begin{proposition}[Sobolev embedding]
\label{proposition:Sobolev_embedding}
    If $s > \dim \Group / 2$, then we have the following continuous inclusion
    \begin{align*}
        \Sobolev{s} \subset \ContinuousFunctions\Group \cap \Lebesgue{\infty}{\Group}.
    \end{align*}
    More precisely, there exists $C \geq 0$ such that the following property holds:
    for every $f \in \Sobolev{s}$,
    there exists a continuous function $\tilde{f} \in \ContinuousFunctions\Group$ such that $f = \tilde{f}$ almost everywhere and
    \begin{align*}
        \norm[\ContinuousFunctions\Group]{\tilde{f}} \leq C \norm[\Sobolev{s}]{f}.
    \end{align*}
\end{proposition}

\subsection{Fourier Transform of distributions}

For the sequel, we need to be able to take the Fourier Transform of certain distributions.
To this end, we follow the ideas of \cite{FischerRuzhansky15}.

\begin{definition}
    Let $a, b \in \R$.
    We shall say that $f \in \TemperedDistributions\Group$ belongs to $\KernelsSobolev{a}{b}$ if and only if the map
    \begin{align*}
        \Schwartz\Group \to \Schwartz\Group : \phi \to \conv{\phi}{f}
    \end{align*}
    is a continuous map in $\Lin{\Sobolev{a}, \Sobolev{b}}$.
\end{definition}

\begin{definition}
    Let $a, b \in \R$.
    We shall say that a map
    \begin{align*}
        \sigma : \dualGroup\VectorSpace \to \End(\SmoothFunctions{\CompactGroup})
    \end{align*}
    belongs to $\LebesgueDual[a, b]{\infty}{\Group}$ if and only if
    \begin{enumerate}
        \item TODO: Invariance condition
        \item The map
            \begin{align*}
                \lambda \in \dualGroup\VectorSpace \mapsto
                \Rep\lambda \BesselPotential{b} \sigma(\lambda) \Rep\lambda \BesselPotential{-a}
            \end{align*}
            belongs to $\LebesgueDual{\infty}{\Group}$.
    \end{enumerate}
\end{definition}

\begin{proposition}[Extension of the Fourier Transform]
    Define the sets
    \begin{align*}
        K \defeq \bigcup_{a, b \in \R} \KernelsSobolev{a}{b}, \quad
        L \defeq \bigcup_{a, b \in \R} \LebesgueDual[a, b]{\infty}{\Group}.
    \end{align*}

    The Fourier Transform can be extended as a bijective map
    \begin{align*}
        \Fourier : K \to L,
    \end{align*}
    and preserve the following properties.
    \begin{enumerate}
        \item If $f \in \Lebesgue{1}{\Group}$, then it coincides with Definition~\ref{definition:Fourier_Transform}.
        \item If $f_1, f_2 \in K$ are such that $\conv{f_1}{f_2} \in K$, then
            \begin{align*}
                \Fourier\{\conv{f_1}{f_2}\} = \Fourier f_2 \Fourier f_1.
            \end{align*}
        \item If $f \in \SmoothFunctions{\Group} \cap K$, $X \in \LieAlgebra$ and $\LeftDifferentialOperatorFirstOrder{X} f \in K$, then
            \begin{align*}
                \Fourier\{\LeftDifferentialOperatorFirstOrder{X} f\}(\lambda) = \Rep\lambda(X) \Fourier f(\lambda).
            \end{align*}
    \end{enumerate}
\end{proposition}

\section{Taylor formula}

\begin{proposition}[Taylor remainder]
    There exists an admissible collection of smooth functions $q_1, \dots, q_M \in \SmoothFunctions\Group$
    and a collection $\{\TaylorLeftDifferentialOperator{\alpha}\}_{\alpha \in \N^M}$ of left-invariant differential operators satisfying the following properties:
    \begin{enumerate}
        \item for each $\alpha \in \N^M$, $\TaylorLeftDifferentialOperator\alpha$'s order is less than $\abs\alpha$;
        \item if $f \in \SmoothFunctions\Group$ and $(x, k) \in \Group$,
            we have the following Taylor development
            \begin{align*}
                f(x, k) &= \sum_{\abs\alpha \leq N} \frac{1}{\alpha!} q^\alpha({(x, k)}^{-1}) \TaylorLeftDifferentialOperator\alpha f(0, e) + \BigO(h(x, k)^N),
            \end{align*}
            where $h : \Group \to \R^+$ denotes the geodesic distance to the identity.
    \end{enumerate}
\end{proposition}
\begin{proof}
    Let $f \in \SmoothFunctions\Group$.
    Choose an embedding
    \begin{align*}
        \iota : \CompactGroup \to \R^D.
    \end{align*}

    By REFERENCE, there exists an open neighbourhood $O \subset \R^D$ containing $\iota(\CompactGroup)$
    such that an orthogonal projection $p : O \to \iota(\CompactGroup)$ is defined and smooth.
    We can therefore extend $f$ onto $O$ via:
    \begin{align*}
        F : \VectorSpace \times O : (x, y) \mapsto f(x, (\iota^{-1} \circ p)(y)).
    \end{align*}

    By the Taylor Theorem on $\VectorSpace \times O$ at $(0, \iota(e))$, we get
    \begin{align*}
        F(x, y) =
        \sum_{\substack{\alpha = (\alpha_1, \alpha_2)\\ \abs\alpha \leq N}}
            \frac{1}{\alpha!}
            &x^{\alpha_1} {(y - \iota(e))}^{\alpha_2}
            \D[F]{x^{\alpha_1}, y^{\alpha_2}}(0, \iota(e))\\
            &+ \BigO\left((\norm{x} + \norm[\R^N]{y - \iota(e)})^N\right).
    \end{align*}
    In particular, denoting by $h(g)$ the geodesic distance between $g \in \Group$ and $e \in \Group$,
    we observe that
    \begin{align}
        f(x, k) =
        \sum_{\substack{\alpha = \alpha_1 + \alpha_2\\ \abs\alpha \leq N}}
            \frac{1}{\alpha!}
            x^{\alpha_1} {(\iota(k) - \iota(e))}^{\alpha_2}
            \D[F]{x^{\alpha_1}, y^{\alpha_2}}(0, \iota(e))
            + \BigO\left({h(x, k)}^N\right).
        \label{proposition:Taylor_remainder_theorem:Taylor_development}
    \end{align}

    Now, let
    \begin{align*}
        q_j(x, k) &= -\ip{x}{k e_i}, \quad &1 \leq &j \leq \dim \VectorSpace\\
        q_j(x, k) &= \ip[\R^N]{\iota(k^{-1}) - \iota(e)}{e_{j}}, \quad &\dim \VectorSpace < &j \leq \dim \VectorSpace + D
    \end{align*}
    where $e_1, \dots, e_{\dim V}$ is an orthonormal basis of $\VectorSpace$
    and $e_{\dim V + 1}, \dots, e_{\dim \VectorSpace + D}$ is an orthonormal basis of $\R^D$.
    Moreover, we know that for each $\alpha = (\alpha_1, \alpha_2)$,
    there exists a left-invariant differential operator $\TaylorLeftDifferentialOperator{\alpha}$ of order at most $\abs\alpha$ such that
    \begin{align*}
        \TaylorLeftDifferentialOperator{\alpha} f(e) = \D[F]{x^{\alpha_1}, y^{\alpha_2}}(0, \iota(e)),
    \end{align*}
    since $F$ doesn't locally vary in the directions perpendicular to the tangent plane of $\iota(\CompactGroup)$.

    Now, let us check that if $\alpha = (\alpha_1, \alpha_2)$, then
    \begin{align*}
        q^\alpha({(x, k)}^{-1})
        = q^\alpha(-k^{-1} x, k^{-1})
        = x^{\alpha_1} {(\iota(k) - \iota(e))}^{\alpha_2}.
    \end{align*}

    It follows that~\eqref{proposition:Taylor_remainder_theorem:Taylor_development} becomes
    \begin{align*}
        f(x, k) &= \sum_{\abs\alpha \leq N} \frac{1}{\alpha!} q^\alpha({(x, k)}^{-1}) \TaylorLeftDifferentialOperator\alpha f(0, e) + \BigO(h(x, k)^N),
    \end{align*}
    which concludes our proof.
\end{proof}

\chapter{Symbols}
\label{chapter:symbols}

\section{Difference operators}

\begin{definition}
\label{definition:difference_operators}
    Let $q \in \SmoothFunctions{\Group}$.
    The \emph{difference operator} associated with $q$, $\DifferenceOperator{q}$ is defined via
    \begin{align*}
        \DifferenceOperator{q} \Fourier f \defeq \Fourier\{q f\},
    \end{align*}
    where $f \in \Schwartz{\Group}$.

    Moreover, if $q$ vanishes at order $k \in \N$,
    we shall say that $\DifferenceOperator{q}$ is a \emph{difference operator of order $k$}.
\end{definition}

\begin{definition}
\label{definition:admissibility_of_difference_operators}
\index{difference operators!admissibility}
    A finite collection $q_1$, \dots, $q_M \in \SmoothFunctions{\Group}$ of smooth functions is said to be \emph{admissible}
    if $\dd q_j(e) \neq 0$ for each $j \in \{1, \dots, M\}$
    and if
    \begin{align*}
        \rank(\dd q_1(e), \dots, \dd q_m(e)) = \dim \Group.
    \end{align*}

    Moreover, if
    \begin{align*}
        \bigcap_{j = 1}^M \{ q_j = 0 \} = \{e\},
    \end{align*}
    we shall say that the collection is \emph{strongly admissible}.

    A collection of \emph{difference operators} is called \emph{(strongly) admissible}
    if the associated smooth functions form a \emph{(strongly) admissible} collection.
\end{definition}

Using \cite[Lemma 4.4]{RuzhanskyTurunenWirth10}, we can show

\begin{lemma}
    There exists a strongly admissible family $q_1$, \dots, $q_M \in \SmoothFunctions{\Group}$ on $\Group$.
\end{lemma}

\begin{remark}
    From now on, we fix an admissible family $q_1, \dots, q_M$ on $\Group$.
    Given $\alpha \in \N^M$, we let
    \begin{align*}
        q^\alpha = \prod_{j = 1}^M q_j^{\alpha_j}.
    \end{align*}
    Moreover, we let $\DifferenceOperatorOrder{\alpha}$ be the difference operator associated with the smooth function
    \begin{align*}
        \Group \to \C : g \mapsto q^\alpha(g^{-1}).
    \end{align*}
\end{remark}

\section{Symbols and Kohn-Nirenberg quantization}

\section{Symbol classes}

\begin{definition}[Symbol classes]
\label{definition:symbol_classes}
    Let $m \in \R$ and fix $\rho, \delta \in \R$ such that $0 \leq \rho \leq \delta \leq 1$.
    We shall say that a map
    \begin{align*}
        \sigma : \Group \times \VectorSpace \mapsto
    \end{align*}
    is a \emph{symbol of order $m$ and of type $(\rho, \delta)$} if the following condition is satisfied:
    for each $\alpha \in \N^{\dim \Group}$, $\beta \in \N^m$, and each $\gamma \in \R$, the quantity
    \begin{align*}
        \sup_{g \in \Group} \esssup_{\lambda \in \dualGroup\VectorSpace}
        \norm[\Lin{\Lebesgue{2}{\CompactGroup}}]{\Rep{\lambda} \BesselPotential{\rho \abs\alpha - m - \delta \abs\beta + \gamma} \LeftDifferentialOperator{\beta} \DifferenceOperatorOrder{\alpha} \sigma(g, \lambda) \Rep{\lambda} \BesselPotential{-\gamma}}
    \end{align*}
    is finite.

    The set $\SymbolClass{m}{\rho, \delta}$ will be used to denote the set of all symbols of order $m$ and type $(\rho, \delta)$.
\end{definition}

\begin{definition}[Smoothing symbols]
\label{definition:smoothing_symbols}
    We let
    \begin{align*}
        \SmoothingSymbols \defeq \bigcap_{m \in \R} \SymbolClass{m}{1, 0}.
    \end{align*}
    The elements of $\SmoothingSymbols$ will be called \emph{smoothing symbols}.
\end{definition}

\begin{definition}[Operator classes]
\label{definition:operator_classes}
    Let $\sigma \in \SymbolClass{m}{\rho, \delta}$.
    We define the operator $\Op(\sigma)$ via
    \begin{align*}
        \Op(\sigma) \phi(g) \defeq
        \int_\dualGroup\VectorSpace
            \tr\left(\Rep\lambda(g) \sigma(g, \lambda) \Fourier f(\lambda)\right)
        \dd \Plancherel{\VectorSpace}(\lambda),
    \end{align*}
    where $\phi \in \Schwartz\Group$, and $g \in \Group$.


    If $T = \Op(\sigma)$ for a certain $\sigma \in \SymbolClass{m}{\rho, \delta}$,
    we shall say that $T$ is an \emph{operator of order $m$ and of type $(\rho, \delta)$}.

    The set of all such operators will be denoted by
    \begin{align*}
        \OperatorClass{m}{\rho, \delta} \defeq \Op(\SymbolClass{m}{\rho, \delta}).
    \end{align*}
    Naturally, an operator in
    \begin{align*}
        \SmoothingOperators \defeq \Op(\SmoothingSymbols)
    \end{align*}
    is called \emph{smoothing}.
\end{definition}

\begin{definition}[Kernel of a symbol]
\label{definition:kernel_of_symbol}
    Let $\sigma \in \SymbolClass{m}{\rho, \delta}$.
    For each $g \in \Group$, we let
    \begin{align*}
        \kappa_g \defeq \InverseFourier\{\sigma(g, \dummy)\} \in \TemperedDistributions\Group.
    \end{align*}
    The map
    \begin{align*}
        \kappa : \Group \to \TemperedDistributions\Group : g \mapsto \kappa_g
    \end{align*}
    is called the \emph{kernel} of $\sigma$.
\end{definition}

\begin{proposition}[Quantisation]
    Let $\sigma \in \SymbolClass{m}{\rho, \delta}$,
    and denote by $\kappa$ its associated kernel.
    If $\phi \in \Schwartz\Group$, then for each $g \in \Group$, we have
    \begin{align*}
        \Op(\sigma) \phi(g) = \conv{\phi}{\kappa_g}.
    \end{align*}

    In other words, $\kappa$ is the right convolution kernel associated with $\Op(\sigma)$.
\end{proposition}

\section{Link with the H\"ormander classes}

\begin{definition}[Rotation of symbols]
    Let $\tilde \sigma \in \SymbolClass[\GroupDirect]{m}{\rho, \delta}$.
    We define the operator
    \begin{align*}
        \Rotation {\tilde \sigma} : \SmoothFunctions \CompactGroup \to \SmoothFunctions \CompactGroup
    \end{align*}
    via the formula
    \begin{align*}
        \Rotation {\tilde \sigma}(x, k; \lambda) F(u) \defeq \tilde \sigma(x, k; k u^{-1} \lambda) F(u),
    \end{align*}
    where $x, \lambda \in \VectorSpace$, $k, u \in \CompactGroup$, and $F \in \SmoothFunctions \CompactGroup$.

    Similarly, given $\sigma \in \SymbolClass m {\rho, \delta}$,
    we define the operator
    \begin{align*}
        \InverseRotation \sigma : \SmoothFunctions \CompactGroup \to \SmoothFunctions \CompactGroup
    \end{align*}
    via the formula
    \begin{align*}
        \InverseRotation \sigma (x, k; \lambda) F(u) \defeq \sigma(x, k; u k^{-1} \lambda) F(u),
    \end{align*}
    where again $x, \lambda \in \VectorSpace$, $k, u \in \CompactGroup$, and $F \in \SmoothFunctions \CompactGroup$.
\end{definition}

\begin{lemma}
\label{lemma:Y_derivative_on_lambda_variable_of_symbols}
    Let $Y \in \LieAlgebraCompactGroup$, $\sigma \in \SymbolClass m {\rho, \delta}$, $\tilde \sigma \in \SymbolClass[\GroupDirect] m {\rho, \delta}$.
    We have the following expression
    \begin{align*}
        \LeftDifferentialOperatorFirstOrder Y_l \sigma(x, k; l \lambda)
        &= \sum_{j = 1}^{\dim \VectorSpace} \DifferenceOperator{j} \sigma(x, k; l \lambda) \Rep {l Y \lambda} (X_j)\\
        \LeftDifferentialOperatorFirstOrder Y_l \tilde \sigma(x, k; l \lambda)
        &= \sum_{j = 1}^{\dim \VectorSpace} \DifferenceOperator[\VectorSpace]{j} \tilde \sigma(x, k; l \lambda) \Rep[\GroupDirect] {l Y \lambda} (\partial_j),
    \end{align*}
    where $(x, k) \in \GroupDirect$, $\lambda \in \VectorSpace$ and $l \in \CompactGroup$.
\end{lemma}
\begin{proof}
    By definition, we know that
    \begin{align*}
        \LeftDifferentialOperatorFirstOrder Y_l \sigma(x, k; l \lambda)
        &= \eval{\D*{1}{t}}{t = 0} \sigma(x, k; l \exp_\CompactGroup (t Y) \lambda),
    \end{align*}
    which after applying the chain rule, becomes
    \begin{align}
        &\LeftDifferentialOperatorFirstOrder Y_l \sigma(x, k; l \lambda) \notag\\
        &\quad = \sum_{j = 1}^{\dim \VectorSpace} \eval{\D*{1}{s}}{s = 0} \sigma(x, k; l \lambda + s u e_j) \eval{\D*{1}{t}}{t = 0} \ip {l \exp_\CompactGroup (t Y) \lambda} {u e_j}.
        \label{eq:k_differentiation_of_lambda_variable_in_symbol}
    \end{align}

    Now, we observe that
    \begin{align*}
        \eval{\D*{1}{t}}{t = 0} \ip {l \exp_\CompactGroup (t Y) \lambda} {u e_j}
        = \ip {l Y \lambda} {u e_j} = \frac{1}{\i \turn} \Rep {l Y \lambda} (X_j),
    \end{align*}
    while at the same time
    \begin{align*}
        \eval{\D*{1}{s}}{s = 0} \sigma(x, k; l \lambda + s u e_j)
        = \i \turn \DifferenceOperator{j} \sigma(x, k; l \lambda).
    \end{align*}

    Therefore, it follows that~\eqref{eq:k_differentiation_of_lambda_variable_in_symbol} becomes
    \begin{align*}
        \LeftDifferentialOperatorFirstOrder Y_l \sigma(x, k; l \lambda)
        = \sum_{j = 1}^{\dim \VectorSpace} \DifferenceOperator{j} \sigma(x, k; l \lambda) \Rep {l Y \lambda} (X_j).
    \end{align*}

    Now, we turn to the case of symbols on the direct product.
    Again, we start with
    \begin{align*}
        \LeftDifferentialOperatorFirstOrder Y_l \tilde \sigma(x, k; l \lambda)
        &= \eval{\D*{1}{t}}{t = 0} \tilde \sigma(x, k; l \exp_\CompactGroup (t Y) \lambda).
    \end{align*}
    Applying the chain rule, we obtain
    \begin{align}
        \LeftDifferentialOperatorFirstOrder Y_l \tilde \sigma(x, k; l \lambda)
        &= \sum_{j = 1}^{\dim \VectorSpace} \eval{\D*{1}{s}}{s = 0} \tilde \sigma(x, k; l \lambda + s e_j) \eval{\D*{1}{t}}{t = 0} \ip {l \exp_\CompactGroup (t Y) \lambda} {e_j}.
        \label{eq:k_differentiation_of_lambda_variable_in_symbol_2}
    \end{align}

    By definition, it is clear that
    \begin{align*}
        \DifferenceOperator[\VectorSpace]{j} \tilde \sigma(x, k; l \lambda) = \frac 1 {\i \turn} \D{1}[\tilde \sigma]{{\lambda_j} } (x, k, l \lambda),
    \end{align*}
    while
    \begin{align*}
        \eval{\D*{1}{t}}{t = 0} \ip {l \exp_\CompactGroup (t Y) \lambda} {e_j}
        &= \ip {l Y \lambda} {e_j}
        = \frac{1}{\i \turn} \Rep[\GroupDirect] {l Y \lambda} (\partial_j).
    \end{align*}

    Therefore, it follows that \eqref{eq:k_differentiation_of_lambda_variable_in_symbol_2} becomes
    \begin{align}
        \LeftDifferentialOperatorFirstOrder Y_l \tilde \sigma(x, k; l \lambda)
        &= \sum_{j = 1}^{\dim \VectorSpace} \DifferenceOperator[\VectorSpace]{j} \tilde \sigma(x, k; l \lambda) \Rep[\GroupDirect] {l Y \lambda} (\partial_j).
    \end{align}
\end{proof}

\begin{lemma}
\label{lemma:link_between_symbols}
    Let $\sigma$ and $\tilde \sigma$ be symbols on $\Group$ and $\GroupDirect$ respectively be such that
    \begin{align*}
        \Op[\Group] (\sigma) = \Op[\GroupDirect] (\tilde \sigma).
    \end{align*}
    \begin{enumerate}
        \item
            \label{item:action_of_difference_operators}
            If $q \in \SmoothFunctions \Group$,
            then defining $\tilde q(y, l) = q(l y, l)$ yields
            \begin{align*}
                \DifferenceOperator{q} \sigma = \Rotation {\DifferenceOperator[\GroupDirect]{\tilde q} \tilde \sigma}
                \quad \text{and} \quad
                \DifferenceOperator[\GroupDirect]{\tilde q} \tilde \sigma = \InverseRotation {\DifferenceOperator{q} \sigma}.
            \end{align*}
            In particular, $\sigma = \Rotation {\tilde \sigma}$ and $\tilde \sigma = \InverseRotation \sigma$.
        \item
            \label{item:action_of_Euclidean_derivative}
            If $X \in \LieAlgebra \cap \VectorSpace$, then
            \begin{align*}
                \LeftDifferentialOperatorFirstOrder{X} \sigma
                = \Rotation {\LeftDifferentialOperatorFirstOrder{X} \tilde \sigma}
                \quad \text{and} \quad
                \LeftDifferentialOperatorFirstOrder{X} \tilde \sigma
                = \InverseRotation {\LeftDifferentialOperatorFirstOrder{X} \sigma}.
            \end{align*}
        \item
            \label{item:action_of_K-derivative}
            If $Y \in \LieAlgebraCompactGroup$, we have
            \begin{align*}
                \LeftDifferentialOperatorFirstOrder{Y} \sigma(x, k; \lambda)
                &= \Rotation {
                    \LeftDifferentialOperatorFirstOrder{Y} \tilde \sigma
                    + \sum_{j = 1}^{\dim \VectorSpace} (\DifferenceOperator[\VectorSpace]{j} \tilde \sigma) \Rep[\GroupDirect] {k Y k^{-1} \lambda} (\partial_j)
                }(x, k; \lambda)\\
                \LeftDifferentialOperatorFirstOrder Y \tilde \sigma(x, k; \lambda)
                &= \InverseRotation {\LeftDifferentialOperatorFirstOrder Y \sigma
                - \sum_{j = 1}^{\dim \VectorSpace} \DifferenceOperator{j} \sigma \ \Rep \lambda (Y^t X_j)
                }(x, k; \lambda).
            \end{align*}
    \end{enumerate}
\end{lemma}
\begin{proof}
    Let us write $T = \Op (\sigma)$.
    It follows that by
    % TODO: Reference both quantisations
    \begin{align*}
        T \phi(x, k)
        &= \int_{\GroupDirect} \phi(y, l) {\tilde \kappa}_{x, k}(x - y, l^{-1} k) \dd (y, l)\\
        &= \int_{\GroupDirect} \phi(y, l) {\kappa}_{x, k}({(y, l)}^{-1} (x, k)) \dd (y, l)
    \end{align*}
    in the sense of distributions.
    Therefore, it follows that we have
    \begin{align}
        {\tilde \kappa}_{x, k}(x - y, l^{-1} k) =
        {\kappa}_{x, k}({(y, l)}^{-1} (x, k)),
        \label{eq:link_between_the_kernels}
    \end{align}
    again in the sense of distributions.

    \begin{enumerate}
        \item
            Suppose first that $\sigma \in \SmoothingSymbols$ so that its kernel $\kappa_{x, k} \in \Schwartz \Group$.
            By definition, $\DifferenceOperator{q} \sigma(x, k; \lambda)$ is equal to
            \begin{align*}
                &\quad \int_\Group q((y, l)^{-1}) \kappa_{x, k}((y, l)^{-1}) \e^{\i \turn \ip {u^{-1} \lambda} y} \RightRegularRepresentation(l) \dd (y, l)\\
                &= \int_\Group q((y, l)^{-1} (x, k)) \kappa_{x, k}((y, l)^{-1} (x, k))\\
                &\qquad \qquad \e^{\i \turn \ip {u^{-1} \lambda} {k^{-1} (y - x)}} \RightRegularRepresentation(k^{-1} l) \dd (y, l)
            \end{align*}
            where we substituted $(y, l)$ for $(x, k)^{-1} (y, l)$ to obtain the last line.

            Using
            \begin{align*}
                (y, l)^{-1} (x, k) = (l^{-1}(x - y), l^{-1} k)
            \end{align*}
            and~\eqref{eq:link_between_the_kernels},
            if follows that $\DifferenceOperator{q} \sigma(x, k; \lambda)$ becomes
            \begin{align*}
                \int_\Group q(l^{-1}(x - y), l^{-1} k) \tilde \kappa_{x, k}(x - y, l^{-1} k) \e^{\i \turn \ip {k u^{-1} \lambda} {(y - x)}} \RightRegularRepresentation(k^{-1} l) \dd (y, l).
            \end{align*}

            We can now substitute $y$ for $y + x$ and $l$ for $k l$ to obtain that
            \begin{align*}
                &\DifferenceOperator{q} \sigma(x, k; \lambda) F(u) =\\
                &\quad \int_\Group q(-l^{-1} y, l^{-1}) \tilde \kappa_{x, k}(-y, l^{-1}) \e^{\i \turn \ip {k u^{-1} \lambda} y} \RightRegularRepresentation(l) \dd (y, l) F(u)
            \end{align*}
            which we recognise to be exactly $\DifferenceOperator {\tilde q} \tilde \sigma(x, k, k u^{-1} \lambda) F(u)$.
            Therefore, we have shown that
            \begin{align*}
                \DifferenceOperator{q} \sigma(x, k; \lambda) F(u)
                = \DifferenceOperator {\tilde q} \tilde \sigma(x, k, k u^{-1} \lambda) F(u),
            \end{align*}
            or in other words $\DifferenceOperator q \sigma = \Rotation {\DifferenceOperator[\GroupDirect] {\tilde q} \tilde \sigma}$.

            The other identity is proven by observing that $\InverseRotation \dummy$ is the inverse of $\Rotation \dummy$.
        \item
            This identity follows easily from the previous point.
        \item
            Applying $\LeftDifferentialOperatorFirstOrder Y$ on both sides of $\sigma = \Rotation {\tilde \sigma}$,
            we obtain
            \begin{align}
                \LeftDifferentialOperatorFirstOrder Y_k \sigma(x, k; \lambda)
                &= \LeftDifferentialOperatorFirstOrder Y_{k' = k} \tilde \sigma(x, k'; k u^{-1} \lambda) + \LeftDifferentialOperatorFirstOrder Y_{k' = k} \tilde \sigma(x, k; k' u^{-1} \lambda)\\
                &= \Rotation {\LeftDifferentialOperatorFirstOrder Y \tilde \sigma}(x, k; \lambda) + \LeftDifferentialOperatorFirstOrder Y_{k' = k} \tilde \sigma(x, k; k' u^{-1} \lambda).
                \label{eq:Y_derivative_of_rotated_symbol_on_direct_product}
            \end{align}

            The second term on the right-hand side can be computed via Lemma \ref{lemma:Y_derivative_on_lambda_variable_of_symbols} to be
            \begin{align*}
                \LeftDifferentialOperatorFirstOrder Y_{k' = k} \tilde \sigma(x, k; k' u^{-1} \lambda)
                &= \sum_{j = 1}^{\dim \VectorSpace} \DifferenceOperator[\VectorSpace]{j} \tilde \sigma(x, k; k u^{-1} \lambda) \Rep[\GroupDirect]{k Y u^{-1} \lambda} (\partial_j)\\
                &= \sum_{j = 1}^{\dim \VectorSpace} \DifferenceOperator[\VectorSpace]{j} \tilde \sigma(x, k; k u^{-1} \lambda) \Rep[\GroupDirect]{k Y k^{-1} (k u^{-1} \lambda)} (\partial_j)\\
                &= \sum_{j = 1}^{\dim \VectorSpace} \Rotation {\DifferenceOperator[\VectorSpace]{j} \tilde \sigma} (x, k; \lambda) \Rotation {\Rep[\GroupDirect]{k Y k^{-1} \lambda} (\partial_j)}.
            \end{align*}

            Plugging the above into \eqref{eq:Y_derivative_of_rotated_symbol_on_direct_product}, we obtain
            \begin{align*}
                \LeftDifferentialOperatorFirstOrder Y \sigma(x, k; \lambda)
                &= \Rotation {\LeftDifferentialOperatorFirstOrder Y \tilde \sigma}(x, k; \lambda)
                + \sum_{j = 1}^{\dim \VectorSpace} \Rotation {\DifferenceOperator[\VectorSpace]{j} \tilde \sigma \ \Rep[\GroupDirect]{k Y k^{-1} \lambda} (\partial_j)}(x, k; \lambda)\\
                &= \Rotation {\LeftDifferentialOperatorFirstOrder Y \tilde \sigma
                    + \sum_{j = 1}^{\dim \VectorSpace} \DifferenceOperator[\VectorSpace]{j} \tilde \sigma \ \Rep[\GroupDirect]{k Y k^{-1} \lambda} (\partial_j)
                }(x, k; \lambda),
            \end{align*}
            which is what we wanted to show.

            To show the second identity,
            we apply $\LeftDifferentialOperatorFirstOrder Y$ on both sides of $\tilde \sigma = \InverseRotation {\sigma}$ to obtain
            \begin{align*}
                \LeftDifferentialOperatorFirstOrder Y_k \tilde \sigma(x, k; \lambda)
                = \LeftDifferentialOperatorFirstOrder Y_{k' = k} \sigma(x, k'; u k^{-1} \lambda) + \LeftDifferentialOperatorFirstOrder Y_{k' = k} \sigma(x, k; u {k'}^{-1} \lambda) \notag\\
                = \InverseRotation {\LeftDifferentialOperatorFirstOrder Y \sigma}(x, k; \lambda) - \LeftDifferentialOperatorFirstOrder Y_{u} \tilde \sigma(x, k; u k^{-1} \lambda).
                \label{eq:Y_derivative_of_rotated_symbol_on_semi-direct_product}
            \end{align*}

            Using Lemma~\ref{lemma:Y_derivative_on_lambda_variable_of_symbols} again to compute the second term of the right-hand side,
            \begin{align*}
                \LeftDifferentialOperatorFirstOrder Y_u \tilde \sigma(x, k; u k^{-1} \lambda)
                &= \sum_{j = 1}^{\dim \VectorSpace} \DifferenceOperator{j} \sigma (x, k; u k^{-1} \lambda) \Rep {u Y k \lambda} (X_j)\\
                &= \sum_{j = 1}^{\dim \VectorSpace} \DifferenceOperator{j} \sigma (x, k; u k^{-1} \lambda) \Rep {u Y u^{-1} u k \lambda} (X_j)\\
                &= \sum_{j = 1}^{\dim \VectorSpace} \InverseRotation {\DifferenceOperator{j} \sigma \Rep {u Y u^{-1} \lambda} (X_j)}.
            \end{align*}

            Observing that in the above,
            \begin{align*}
                \Rep {u Y u^{-1} \lambda} (X_j)
                = \i \turn \ip {u Y u^{-1} \lambda} {u X_j}
                = \i \turn \ip {\lambda} {u Y^t X_j}
                = \Rep \lambda (Y^t X_j)
            \end{align*}
            so that \eqref{eq:Y_derivative_of_rotated_symbol_on_direct_product} becomes
            \begin{align*}
                \LeftDifferentialOperatorFirstOrder Y_k \tilde \sigma(x, k; \lambda)
                = \InverseRotation {\LeftDifferentialOperatorFirstOrder Y \sigma
                - \sum_{j = 1}^{\dim \VectorSpace} \DifferenceOperator{j} \sigma \ \Rep \lambda (Y^t X_j)
                }(x, k; \lambda),
            \end{align*}
            concluding the proof.
    \end{enumerate}
\end{proof}

\begin{lemma}
\label{lemma:inclusion_in_zero_class}
    Let $m \defeq \frac {-\dim \CompactGroup} 2 (1 - \rho)$.
    % TODO: Can we prove the result for $m = 0$ or improve?
    \begin{enumerate}
        \item
            If $\sigma \in \SymbolClass m {\rho, \delta}$,
            then $\tilde \sigma = \InverseRotation \sigma$ satisfies
            \begin{align*}
                \sup_{(x, k) \in \Group} \esssup_{\lambda \in \VectorSpace}
                \norm [\Lin {\Lebesgue 2 \CompactGroup}] {\tilde \sigma(x, k; \lambda)} < \infty.
            \end{align*}
        \item
            If $\tilde \sigma \in \SymbolClass [\GroupDirect] m {\rho, \delta}$,
            then $\sigma = \Rotation {\tilde \sigma}$ satisfies
            \begin{align*}
                \sup_{(x, k) \in \Group} \esssup_{\lambda \in \VectorSpace}
                \norm [\Lin {\Lebesgue 2 \CompactGroup}] {\sigma(x, k; \lambda)} < \infty.
            \end{align*}
    \end{enumerate}
\end{lemma}
\begin{proof}
    Let $F \in \Lebesgue 2 \CompactGroup$ and $u \in \CompactGroup$.
    By the Sobolev Inequality,
    we know that
    \begin{align*}
        \int_\CompactGroup \sup_{v \in \CompactGroup} \abs {\sigma(x, k; v \lambda) F(u)}^2 \dd u
        \leq C \sum_{\beta} \int_\CompactGroup \int_\CompactGroup \abs{Y^\beta_v \sigma(x, k; v \lambda) F(u)}^2 \dd v \dd u
    \end{align*}
    Using Lemma~\ref{lemma:Y_derivative_on_lambda_variable_of_symbols},
    we know that each $Y^\beta_v \sigma \in \SymbolClass {m + \abs \beta (1 - \rho)} {\rho, \delta} \subset \SymbolClass 0 {\rho, \delta}$ so that the above becomes
    \begin{align*}
        \sup_{(x, k) \in \Group} \esssup_{\lambda \in \VectorSpace}
        \int_\CompactGroup \sup_{v \in \CompactGroup} \abs {\sigma(x, k; v \lambda) F(u)}^2 \dd u
        \leq C \norm [\Lebesgue 2 \CompactGroup] {F}^2.
    \end{align*}

    From the above inequality, we easily derive that
    \begin{align*}
        \sup_{(x, k) \in \Group}& \esssup_{\lambda \in \VectorSpace} \norm [\Lebesgue 2 \CompactGroup] {\tilde \sigma(x, k; \lambda) F}^2\\
        &= \sup_{(x, k) \in \Group} \esssup_{\lambda \in \VectorSpace} \int_\CompactGroup \abs {\sigma(x, k; u k^{-1} \lambda) F(u)}^2 \dd u\\
        &\leq \sup_{(x, k) \in \Group} \esssup_{\lambda \in \VectorSpace} \int_\CompactGroup \sup_{v \in \CompactGroup} \abs {\sigma(x, k; v \lambda) F(u)}^2 \dd u\\
        &\leq C \norm [\Lebesgue 2 \CompactGroup] {F}^2,
    \end{align*}
    which concludes the proof.

    The second bound is obtained with an identical argument.
\end{proof}

\begin{theorem}
    Let $\delta' = \max \{\delta, 1 - \rho\}$ and $m' = m + \frac {\dim \CompactGroup} 2 (1 - \rho)$.

    \begin{enumerate}
        \item
            For each $\sigma \in \SymbolClass m {\rho, \delta}$,
            the symbol $\tilde \sigma = \InverseRotation \sigma$ belongs to $\SymbolClass [\GroupDirect] {m'} {\rho, \delta'}$
            and satisfies
            \begin{align*}
                \Op[\Group] (\sigma) = \Op[\GroupDirect] (\tilde \sigma).
            \end{align*}
        \item
            Reciprocally, for each $\tilde \sigma \in \SymbolClass [\GroupDirect] m {\rho, \delta}$,
            the symbol $\sigma = \Rotation {\tilde \sigma}$ belongs to $\SymbolClass {m'} {\rho, \delta'}$
            and satisfies
            \begin{align*}
                \Op[\GroupDirect] (\tilde \sigma) = \Op[\Group] (\sigma).
            \end{align*}
    \end{enumerate}

    In particular, if $\rho = 1$, the symbol classes coincide
    \begin{align*}
        \SymbolClass m {1, \delta} = \SymbolClass [\GroupDirect] m {1, \delta}.
    \end{align*}
\end{theorem}
\begin{proof}
    Let us prove the first claim,
    as the proof of the second uses an identical argument.

    Let $\sigma \in \SymbolClass m {\rho, \delta}$,
    and write $\tilde \sigma \defeq \InverseRotation \sigma$.

    \begin{claim}
        Given $\tilde{q} \in \CompactGroup$ and $\beta \in \N^{\dim \Group}$,
        there exists a symbol $\tau \in \SymbolClass {m - \rho \order(q) + \delta' \abs \beta} {\rho, \delta'}$ such that
        \begin{align*}
            \DifferenceOperator [\GroupDirect] {\tilde q} X^\beta \tilde \sigma = \InverseRotation \tau.
        \end{align*}
    \end{claim}
    \begin{proof}[Proof of the claim]
        Let us prove the claim by induction on $\abs \beta$.
        The initial case $\beta = 0$ follows easily from Lemma~\ref{lemma:link_between_symbols} (\ref{item:action_of_difference_operators})
        and the inclusion
        \begin{align*}
            \SymbolClass {m - \rho \order(q)} {\rho, \delta} \subset \SymbolClass {m - \rho \order(q)} {\rho, \delta'}.
        \end{align*}

        Now, let us assume the claim holds for a certain $\beta$,
        so that there exists $\tau \in \SymbolClass {m - \rho \order(q) + \delta' \abs \beta} {\rho, \delta'}$ such that
        \begin{align*}
            \DifferenceOperator [\GroupDirect] {\tilde q} X^\beta \tilde \sigma = \InverseRotation \tau.
        \end{align*}

        If $X \in \LieAlgebra \cap \VectorSpace$,
        then by (\ref{item:action_of_Euclidean_derivative}) of Lemma~\ref{lemma:link_between_symbols} we have
        \begin{align*}
            \DifferenceOperator [\GroupDirect] {\tilde q} X X^\beta \tilde \sigma = \InverseRotation {X \tau},
        \end{align*}
        where $X \tau \in \SymbolClass {m - \rho \order(q) + \delta' (\abs \beta + 1)} {\rho, \delta'}$.

        If $X \in \LieAlgebraCompactGroup$,
        then by (\ref{item:action_of_K-derivative}) of Lemma~\ref{lemma:link_between_symbols} we have
        \begin{align*}
            \DifferenceOperator [\GroupDirect] {\tilde q} X X^\beta \tilde \sigma = \InverseRotation {X \tau - \sum_{j = 1}^{\dim \VectorSpace} \DifferenceOperator{j} \tau \ \Rep {\lambda} (X^t X_j) }
        \end{align*}
        where $X \tau - \sum_{j = 1}^{\dim \VectorSpace} \DifferenceOperator{j} \tau \ \Rep {\lambda} (X^t X_j) \in \SymbolClass {m - \rho \order(q) + \delta' (\abs \beta + 1)} {\rho, \delta'}$.
        To see this,
        we observe the condition $\delta' \geq 1 - \rho$ implies that $X \tau$ is the term with higher order.

        Either way, this concludes the induction, and thus the proof of the claim.
    \end{proof}

    By our claim,
    there exists a symbol $\tau \in \SymbolClass {m - \rho \order(q) + \delta' \abs \beta} {\rho, \delta'}$ such that
    \begin{align*}
        \DifferenceOperator [\GroupDirect] {\tilde q} X^\beta \tilde \sigma = \InverseRotation \tau.
    \end{align*}

    Note that the above implies that
    \begin{align*}
        \Rep[\GroupDirect] \lambda \BesselPotential{-m' + \rho \order(q) - \delta' \abs \beta + \gamma} \DifferenceOperator [\GroupDirect] {\tilde q} X^\beta \tilde \sigma(x, k; \lambda) \Rep [\GroupDirect] \lambda \BesselPotential{-\gamma}\\
        = \InverseRotation {\Rep \lambda \BesselPotential{-m' + \rho \order(q) - \delta' \abs \beta + \gamma} \tau \Rep \lambda \BesselPotential{-\gamma}} (x, k; \lambda).
    \end{align*}

    Since the operator inside $\InverseRotation \dummy$ on the second line belongs to
    \begin{align*}
        \SymbolClass {-\frac {\dim \CompactGroup} 2 (1 - \rho) } {\rho, \delta'},
    \end{align*}
    then Lemma~\ref{lemma:inclusion_in_zero_class} implies that
    \begin{align*}
        &\sup_{g \in \GroupDirect} \esssup_{\lambda \in \VectorSpace}\\
        &\quad
        \norm [\Lin {\Lebesgue 2 \CompactGroup}] {%
            \Rep[\GroupDirect] \lambda \BesselPotential{-m' + \rho \abs \alpha  - \delta' \abs \beta} \DifferenceOperator [\GroupDirect] {\tilde q} X^\beta \tilde \sigma(x, k; \lambda) \Rep [\GroupDirect] \lambda
        }
        % TODO: show the direct product doesn't require the gamma trick
    \end{align*}
    is finite,
    which by definition means that $\tilde \sigma \in \SymbolClass [\GroupDirect] {m'} {\rho, \delta'}$.
\end{proof}

\section{Littlewood-Paley decomposition}

\begin{lemma}
\label{lemma:derivatives_of_radial_functions}
    Let $\alpha \in \N^n$,
    and fix a radial function $\chi \in \SmoothFunctions{\R^n}$.
    If $\alpha \in \N^n$, then
    \begin{align}
        \D{\abs \alpha}[\chi]{x^\alpha}(x)
        = \sum_{r = 1}^{C_\alpha} f_r(\norm[\R^n]{x}) P_r(x),
    \end{align}
    where $P_r$ is a polynomial depending only on $\alpha$.

    Moreover, if $\supp \chi$ is compact
    and if there exists $\delta > 0$ such that the radial derivative $\D{\abs \alpha}[\chi]{\lambda}$ vanishes on on $\Ball[\R^n]{0}{\delta}$,
    then we have
    \begin{align*}
        \sup_r \sup_{\lambda \in \R^+} \abs{f_r} < \infty
    \end{align*}
\end{lemma}
\begin{proof}
    Using the chain rule, we know that for a purely radial function $f$
    \begin{align}
        \D{1}[f]{{x_i} } = \D{1}[\lambda]{{x_i} } \D{1}[f]{\lambda} = \frac{\D{1}[f]{\lambda}}{\norm[\R^n]{x}} x_i.
    \end{align}

    We know proceed to show the claim by induction on $\alpha$.
    The result is clearly true when $\abs{\alpha} = 0$.
    If we assume it is true for some $\alpha \in \N^n$, then by the above,
    \begin{align}
        \D{\abs \alpha + 1}[\chi]{{x_i} , x^\alpha}(x)
        &= \D{1}{{x_i} } \sum_{r = 1}^{C_\alpha} f_r(\norm[\R^n]{x}) P_r(x)\\
        &= \sum_{r = 1}^{C_\alpha} \frac{\D{1}[f_r]{\lambda}}{\norm[\R^n]{x}}(\norm[\R^n]{x}) x_i P_r(x)
        + \sum_{r = 1}^{C_\alpha} f_r(\norm[\R^n]{x}) \D{1}[P_r]{{x_i} }(x),
    \end{align}
    which concludes the proof.
\end{proof}

\begin{lemma}
\label{lemma:left_regular_representation_of_polynomials}
    Let $P \in \SmoothFunctions{\dualGroup{\VectorSpace}}$ be a polynomial.
    We can find functions $q_i \in \Polynomials{\CompactGroup}$, $f_i \in \Lebesgue{2}{\dualGroup{\VectorSpace}}$, $i = 1, \dots, N$ such that
    \begin{align*}
        P(k \lambda) = \sum_{i = 1}^N q_i(k) f_i(\lambda)
    \end{align*}
    for each $k \in \CompactGroup$ and each $\lambda \in \dualGroup{\VectorSpace}$.

    Moreover, the $q_i$ satisfy the bound
    \begin{align*}
        \sup_i \sup_\CompactGroup \abs{q_i} < \infty.
    \end{align*}
\end{lemma}

\begin{theorem}[Littlewood-Paley decomposition]
\label{theorem:Littlewood-Paley_decomposition}
\index{Littlewood-Paley decomposition}
    Let $\AbelianGroup$ be a \emph{locally compact abelian Lie group},
    and suppose that $\CompactGroup \subset \Aut(\AbelianGroup)$ is \emph{compact}.

    We now consider the group $\Group = \AbelianGroup \ltimes \CompactGroup$.

    Suppose further that the following conditions are satisfied.
    \begin{enumerate}
        \item The Haar measure of $\AbelianGroup$ is invariant under $\CompactGroup$,
            i.e.\ for each $f \in \Lebesgue{1}{\Group}$ and each $k \in \CompactGroup$, we have
            \begin{align*}
                \int_\AbelianGroup f(a) \dd a = \int_\AbelianGroup f(k a) \dd a.
            \end{align*}
        \item The Laplacian on $\AbelianGroup$ is invariant under $\CompactGroup$,
            i.e.\ for every $\phi \in \Schwartz\AbelianGroup$ and every $k \in \CompactGroup$, we have
            \begin{align*}
                (\Laplacian[\AbelianGroup] \phi(k \dummy))(a)
                = (\Laplacian[\AbelianGroup] \phi)(k a).
            \end{align*}
        \item There exists a Littlewood-Paley decomposition on $\AbelianGroup$ invariant under $\CompactGroup$,
            i.e.\ a sequence $\chi_j \in \Fourier(\Schwartz\AbelianGroup), j \in \N$ such that:
            \begin{enumerate}
                \item they sum to 1, i.e.\ we have $\sum_{j = 0}^\infty \chi_j = 1$.
                \item the functions $\chi_j$, $j \in \N$, are invariant under $\CompactGroup$:
                    for each $\lambda \in \dualGroup\AbelianGroup$ and each $k \in \CompactGroup$, we have $\chi_j(k \lambda) = \chi_j(\lambda)$.
                \item There exists $C > 0$ such that for every $j \in \N$, we have
                    \begin{align*}
                        \chi_j(\lambda) = 0 \quad \text{if} \quad \JapaneseBracket{\AbelianGroup}{\lambda} \geq C 2^j.
                    \end{align*}
                \item for each $q \in \Polynomials\AbelianGroup$,
                    there exists a finite family $q_1, \dots, q_{C_q} \in \Polynomials\CompactGroup$ such that
                    \begin{align*}
                        q(k a) = \sum_{r = 1}^{C_q} f_r(a) q_j(k)
                    \end{align*}
                    for some bounded functions $f_r : A \to \C$, $r \in \{1, \dots, C_q\}$.
            \end{enumerate}
    \end{enumerate}

    If all the above conditions hold,
    there exists a sequence $\eta_l \in \SmoothingSymbols$, $l \in \N$ of smoothing symbols satisfying the following properties
    \begin{enumerate}
        \item the semi-norms decay in the following way
            \begin{align}
                \SymbolSemiNorm{m}{\rho, \delta}{\eta_l} \leq C 2^{-lm}
            \end{align}
        \item the associated kernels $\kappa_l$ satisfy
            \begin{align*}
                \sum_{l = 0}^\infty \kappa_l = \DiracDelta{e_\Group}
            \end{align*}
            in the sense of distributions.
    \end{enumerate}
\end{theorem}
\begin{proof}
    \begin{description}
        \item[Step 1] Constructing the dyadic decomposition.

            First, let us find a smooth function $\chi_0 \in \SmoothFunctions{\dualGroup{\VectorSpace}}$ invariant under $\CompactGroup$ such that
            \begin{align*}
                \chi_0(\lambda) = 1 \  \text{if}\  \norm[\dualGroup{\VectorSpace}]{\lambda} \leq 1, \quad \text{and} \quad
                \chi_0(\lambda) \equiv 0 \ \text{if}\  \norm[\dualGroup{\VectorSpace}]{\lambda} \geq 2.
            \end{align*}

            Then, for each $l \in \N$ satisfying $l \geq 1$, let
            \begin{align*}
                \chi_l = \chi_0(2^{-l} \dummy) - \chi_0(2^{-l + 1} \dummy).
            \end{align*}
            so that $\supp \chi_l \subset \Ball{0}{2^{l + 1}}$.

            In particular, it should be clear that
            \begin{align*}
                \sum_{l = 0}^N \chi_l = \chi_0(2^{-N} \dummy)
            \end{align*}
            so that in fact
            \begin{align}
                \sum_{l = 0}^\infty \chi_l = 1.
                \label{eq:theorem:Littlewood-Paley_decomposition:partition_of_unity}
            \end{align}

            Fix $l \in \N$.
            We define our symbol $\eta_l$ as follows.
            For each $\tau \in \dualGroup{\CompactGroup}$, we let
            \begin{align*}
                \eta_l(\lambda)
                = \sum_{\JapaneseBracket{\CompactGroup}{\tau} \leq 2^l}
                \chi_{l - \Ceiling{\log_2 \JapaneseBracket{\CompactGroup}{\tau}}}(\lambda) \Id{V_\tau},
            \end{align*}
            where $V_\tau = \Span \{ \tau_{ij} : i, j = 1, \dots, \dimRep{\tau} \}$.

            Note that since $\supp \chi_l \subset \Ball{0}{2^{l + 1}}$,
            we get that
            \begin{align}
                \eta_l(\lambda) = 0 \quad \text{if } \norm[\dualGroup{\CompactGroup}]{v} \geq 2^{l + 1}
                \label{eq:theorem:Littlewood-Paley_decomposition:cancellation_condition}
            \end{align}

            We check that
            \begin{align*}
                \sum_{l = 0}^\infty \eta_l
                &= \sum_{l = 0}^\infty
                    \sum_{\JapaneseBracket{\CompactGroup}{\tau} \leq 2^l}
                        \chi_{l - \Ceiling{\log_2 \JapaneseBracket{\CompactGroup}{\tau}}} \Id{V_\tau}\\
                &= \sum_{\tau \in \dualGroup{\CompactGroup}}
                    \sum_{l = \Ceiling{\log_2 \JapaneseBracket{\CompactGroup}{\tau}}}^\infty
                        \chi_{l - \Ceiling{\log_2 \JapaneseBracket{\CompactGroup}{\tau}}} \Id{V_\tau},
            \end{align*}
            where the last line was obtained by commuting the two sums.

            Substituing $l$ for $l + \Ceiling{\log_2 \JapaneseBracket{\CompactGroup}{\tau}}$ in the inner sum,
            the above becomes
            \begin{align*}
                \sum_{l = 0}^\infty \eta_l
                = \sum_{\tau \in \dualGroup{\CompactGroup}}
                    \sum_{l = 0}^\infty
                        \chi_l \Id{V_\tau}
                = \sum_{\tau \in \dualGroup{\CompactGroup}}
                    \Id{V_\tau}
                = \Id{\Lebesgue{2}{\CompactGroup}},
            \end{align*}
            where the second to last inequality was obtained from~\eqref{eq:theorem:Littlewood-Paley_decomposition:partition_of_unity}.

        \item[Step 2] Computing the associated kernels $\kappa_l$.

            By applying the inverse Fourier Transform (Proposition~\ref{proposition:inverse_Fourier_Transform})
            we obtain that the kernel is given by
            \begin{align}
                \kappa_l(x, k)
                = \sum_{\JapaneseBracket{\CompactGroup}{\tau} \leq 2^l}
                    \int_\dualGroup{\VectorSpace}
                        \chi_{l - \Ceiling{\log_2 \JapaneseBracket{\CompactGroup}{\tau}}}(\lambda) \tr( \left. \Rep{\lambda}(x, k) \right|_{V_\tau} )
                    \dd \Plancherel{\VectorSpace}(\lambda)
                \label{eq:theorem:Littlewood-Paley_decomposition:computing_kernel}
            \end{align}

            By the Peter-Weyl Theorem,
            $\{ \sqrt{\dimRep{\tau}} \tau_{pq} : p, q = 1, \dots, \dimRep{\tau} \}$
            is an orthonormal basis of $V_\tau$,
            allowing us to compute the trace as
            \begin{align*}
                \tr( \left. \Rep{\lambda}(x, k) \right|_{V_\tau})
                &= \sum_{p = 1}^\dimRep{\tau}
                    \dimRep{\tau}
                    \int_\Lebesgue{2}{\CompactGroup}
                    (u \lambda)(x) \tau_{pp}(k^{-1} u) \conj{\tau_{pp}(u)}
                    \dd u\\
                &= \sum_{p,q = 1}^\dimRep{\tau}
                    \dimRep{\tau}
                    \int_\Lebesgue{2}{\CompactGroup}
                        (u \lambda)(x) \tau_{pq}(k^{-1}) \tau_{q p}(u) \conj{\tau_{pp}(u)}
                    \dd u.
            \end{align*}

            Using the above in~\eqref{eq:theorem:Littlewood-Paley_decomposition:computing_kernel},
            and substituing $\lambda$ for $u^{-1} \lambda$,
            we obtain
            \begin{align}
                \kappa_l (x, k)
                = &\sum_{\JapaneseBracket{\CompactGroup}{\tau} \leq 2^l}
                        \sum_{p,q = 1}^\dimRep{\tau}
                        \dimRep{\tau}
                        \int_\dualGroup{\VectorSpace}
                                \int_\Lebesgue{2}{\CompactGroup} \notag\\
                                    &\chi_{l - \Ceiling{\log_2 \JapaneseBracket{\CompactGroup}{\tau}}}(u^{-1}\lambda) \lambda(x) \tau_{pq}(k^{-1}) \tau_{qp}(u) \conj{\tau_{pp}(u)}
                                \dd u
                            \dd \Plancherel{\VectorSpace}(\lambda)
                    \label{eq:theorem:Littlewood-Paley_decomposition:computing_kernel:2}
            \end{align}

            Using the invariance of $\chi_{k}$ under $\CompactGroup$ and
            \begin{align*}
                \dimRep{\tau} \int_\Lebesgue{2}{\CompactGroup} \tau_{qp}(u) \conj{\tau_{pp}(u)} \dd u = \Kronecker{p}{q},
            \end{align*}
            then~\eqref{eq:theorem:Littlewood-Paley_decomposition:computing_kernel:2} becomes
            \begin{align*}
                \kappa_l (x, k)
                = &\sum_{\JapaneseBracket{\CompactGroup}{\tau} \leq 2^l}
                    \int_\dualGroup{\VectorSpace}
                        \chi_{l - \Ceiling{\log_2 \JapaneseBracket{\CompactGroup}{\tau}}}(\lambda) \lambda(x)
                    \dd \Plancherel{\VectorSpace}(\lambda)
                    \conj{\Character{\tau}(k)}
            \end{align*}
            which, after recognising the inverse Fourier Transform on $\dualGroup{\VectorSpace}$,
            yields the following expression for the kernel
            \begin{align}
                \kappa_l (x, k)
                = &\sum_{\JapaneseBracket{\CompactGroup}{\tau} \leq 2^l}
                    \InverseFourier[\VectorSpace]{\chi_{l - \Ceiling{\log_2 \JapaneseBracket{\CompactGroup}{\tau}}}}(x) \conj{\Character{\tau}(k)}.
                \label{eq:theorem:Littlewood-Paley_decomposition:kernel}
            \end{align}

        \item[Step 3] We show that for every $q \in \Polynomials{\Group}$
            and every $\lambda \in \dualGroup{\VectorSpace}$,
            \begin{align}
                \sup_{l \in \N} \norm[\Lin{\Lebesgue{2}{\CompactGroup}}]{\DifferenceOperator{q} \eta_l(\lambda)} < \infty.
            \end{align}

            Fix $q_1 \in \Polynomials{\VectorSpace}$, $q_2 \in \Polynomials{\CompactGroup}$
            and write $q(x, k) = q_1(x) q_2(k)$.
            We also choose an arbitrary function $F \in \Lebesgue{2}{\CompactGroup}$,
            and an element $u \in \CompactGroup$.

            Informally, the idea behind the proof of this step is the following
            if we can write
            \begin{align*}
                \DifferenceOperator{q} \eta_l(\lambda) F(u)
                = \sum_{\tau \in \dualGroup{\CompactGroup}}
                    \dimRep{\tau}
                    \tr\left( \tau(u) \sigma_{l, \lambda, q, \lambda}(u, \tau) \Fourier[\CompactGroup] F(\tau) \right),
            \end{align*}
            then a bound on the operator norm of $\DifferenceOperator{q} \eta_l (\lambda)$ can be obtained by finding an appropriate bound on $\sigma_{l, q, \lambda}$.
            Looking at~\eqref{eq:theorem:Littlewood-Paley_decomposition:kernel},
            we can see the latter is the right approach
            as we have a sum on $\dualGroup{\CompactGroup}$ already.

            Multiplying $\kappa_l$ par $q$ and taking the Fourier Transform, we get
            \begin{align}
                \DifferenceOperator{q} \eta_l (\lambda) F(u)
                = &\sum_{\JapaneseBracket{\CompactGroup}{\tau} \leq 2^l}
                    \int_\VectorSpace
                        \int_\CompactGroup
                            q_1(x) \InverseFourier[\VectorSpace]{\chi_{l - \Ceiling{\log_2 \JapaneseBracket{\CompactGroup}{\tau}}}}(x) (k u \lambda)(-x) \notag\\
                            &\quad q_2(k) \conj{\Character{\tau}(k)} F(k u)
                        \dd k
                    \dd x\notag\\
                = &\sum_{\JapaneseBracket{\CompactGroup}{\tau} \leq 2^l}
                    \int_\CompactGroup
                        \DifferenceOperator[\VectorSpace]{q_1} \chi_{l - \Ceiling{\log_2 \JapaneseBracket{\CompactGroup}{\tau}}}(k u \lambda)
                        q_2(k) \conj{\Character{\tau}(k)} F(k u)
                    \dd k \label{eq:theorem:Littlewood-Paley_decomposition:rho_condition},
            \end{align}
            where the second line was obtained by integrating with respect to $x$.

            Substituing $k$ for $k u^{-1}$ in the above,
            and using the Leibniz rule for polynomials on $q_2$, we obtain
            \begin{align*}
                \DifferenceOperator{q} \eta_l (\lambda) F(u)
                = &\sum_{\JapaneseBracket{\CompactGroup}{\tau} \leq 2^l}
                    \sum_{p = 1}^{C_q}
                        \int_\CompactGroup
                            \DifferenceOperator[\VectorSpace]{q_1} \chi_{l - \Ceiling{\log_2 \JapaneseBracket{\CompactGroup}{\tau}}}(k \lambda)\\
                            &\quad q_{2, p}(k) {q'}_{2, p}(u^{-1}) \conj{\Character{\tau}(k u^{-1})} F(k)
                        \dd k,
            \end{align*}

            \begin{claim}
                We have the decomposition
                \begin{align*}
                    \DifferenceOperator[\VectorSpace]{q_1} \chi_{l - \Ceiling{\log_2 \JapaneseBracket{\CompactGroup}{\tau}}}(k \lambda) = \sum_{r = 1}^{C_q} f_{l, r}(\tau, \lambda) q_r(k),
                \end{align*}
                where $q_r$ and $f_{l, r}$ satisfies the following bound
                \begin{align}
                    \sup_{l \in \N} \sup_{\tau \in \dualGroup{\CompactGroup}} \sup_{\lambda \in \dualGroup{\VectorSpace}} \abs{f_{l, r}(\tau, \lambda)} < \infty,\quad
                    \sup_{r} \sup_{k \in \CompactGroup} \abs{q(k)} < \infty
                    \label{eq:theorem:Littlewood-Paley_decomposition:claim_bound}
                \end{align}
            \end{claim}
            \begin{proof}[Proof of the claim]
                Assume first that $l - \Ceiling{\log_2 \JapaneseBracket{\CompactGroup}{\tau}} \neq 0$.
                Since
                \begin{align*}
                    \chi_{l - \Ceiling{\log_2 \JapaneseBracket{\CompactGroup}{\tau}}}(\lambda)
                    = \chi_1(2^{-l + \Ceiling{\log_2 \JapaneseBracket{\CompactGroup}{\tau} + 1}} \lambda),
                \end{align*}
                it follows that
                \begin{align*}
                    \DifferenceOperator[\VectorSpace]{q_1} &\chi_{l - \Ceiling{\log_2 \JapaneseBracket{\CompactGroup}{\tau}}}(k \lambda)
                    =\\
                    &\qquad 2^{(-l + \Ceiling{\log_2 \JapaneseBracket{\CompactGroup}{\tau} + 1}) \order{q_1}}
                    \DifferenceOperator[\VectorSpace]{q_1} \chi_1(2^{-l + \Ceiling{\log_2 \JapaneseBracket{\CompactGroup}{\tau} + 1}} k \lambda).
                \end{align*}

                Using Lemma~\ref{lemma:derivatives_of_radial_functions} and Lemma~\ref{lemma:left_regular_representation_of_polynomials}
                we obtain
                \begin{align*}
                    \DifferenceOperator[\VectorSpace]{q_1} &\chi_{l - \Ceiling{\log_2 \JapaneseBracket{\CompactGroup}{\tau}}}(k \lambda) =
                    2^{(-l + \Ceiling{\log_2 \JapaneseBracket{\CompactGroup}{\tau} + 1}) \order{q_1}}\\
                    &\sum_{r = 1}^{C_q} f_r(2^{-l + \Ceiling{\log_2 \JapaneseBracket{\CompactGroup}{\tau} + 1}} \norm[\dualGroup{\VectorSpace}]{\lambda}) q_r(k) P_r(2^{-l + \Ceiling{\log_2 \JapaneseBracket{\CompactGroup}{\tau} + 1}} \lambda),
                \end{align*}
                where each $f_r$, $q_r$ and each $P_r$ is independent of $l$ and $\tau$.
                Now, writing
                \begin{align*}
                    f_{l, r}(\tau, \lambda) =
                    &2^{(-l + \Ceiling{\log_2 \JapaneseBracket{\CompactGroup}{\tau} + 1}) \order{q_1}}\\
                    &f_r(2^{-l + \Ceiling{\log_2 \JapaneseBracket{\CompactGroup}{\tau} + 1}} \norm[\dualGroup{\VectorSpace}]{\lambda}) P_r(2^{-l + \Ceiling{\log_2 \JapaneseBracket{\CompactGroup}{\tau} + 1}} \lambda),
                \end{align*}
                we obtain the desired formula.
                The bound comes from the bounds in Lemma~\ref{lemma:derivatives_of_radial_functions} and~\ref{lemma:left_regular_representation_of_polynomials}.

                The case $l - \Ceiling{\log_2 \JapaneseBracket{\CompactGroup}{\tau}} = 0$ can be treated similarly.
            \end{proof}

            Using the above claim,
            and the identity
            \begin{align*}
                \conj{\Character{\tau}(k u^{-1})} = \sum_{i, j = 1}^\dimRep{\tau} \tau_{ij}(u) \conj{{\tau_{ij}(k)}},
            \end{align*}
            we observe that
            \begin{align}
                \DifferenceOperator{q} \eta_l (\lambda) F(u)
                = &\sum_{p, r = 1}^{C_q}
                    \sum_{\JapaneseBracket{\CompactGroup}{\tau} \leq 2^l}
                        \sum_{i, j = 1}^\dimRep{\tau}\notag
                            \tau_{ij}(u) {q'}_{2, p}(u^{-1})\\
                            &\quad f_{l, r}(\tau, \lambda)
                            \int_\CompactGroup
                                q_{r}(k) q_{2, p}(k) F(k) \conj{\tau_{ij}(k)}
                            \dd k
                            \label{eq:theorem:Littlewood-Paley_decomposition:exact_expression_for_operator}\\
                = &\sum_{p, r = 1}^{C_q}
                    \sum_{\JapaneseBracket{\CompactGroup}{\tau} \leq 2^l}
                        \sum_{i, j = 1}^\dimRep{\tau}
                            \tau_{ij}(u) {q'}_{2, p}(u^{-1})\notag\\
                            &\quad f_{l, r}(\tau, \lambda)
                            \Fourier[\CompactGroup]{} {\left\{ q_{r} q_{2, p} F\right\}}_{j i}(\tau).\notag
            \end{align}

            For $p, r = 1, \dots, \dimRep{\tau}$, defining the symbols
            \begin{align}
                \sigma_{l, \lambda, p, r}(u, \tau) =
                \begin{cases}
                    \frac{1}{\dimRep{\tau}} {q'}_{2, p}(u^{-1}) f_{l, r}(\tau, \lambda) \Id{\dimRep{\tau}} & \text{if } \JapaneseBracket{\CompactGroup}{\tau} \leq 2^l\\
                    0 & \text{otherwise}
                \end{cases}
            \end{align}
            and denoting by $T_{l, \lambda, p, r}$ the corresponding operators,
            we see that in fact,
            \begin{align*}
                \DifferenceOperator{q} \eta_l (\lambda) F(u)
                = &\sum_{p, r = 1}^{C_q}
                    \sum_{\tau \in \dualGroup{\CompactGroup}}
                        \dimRep{\tau}
                        \tr \left(
                            \tau(u)
                            \sigma_{l, \lambda, p, r}(u, \tau)
                            \Fourier[\CompactGroup]{} \left\{ q_r q_{2, p} F\right\}(\tau)
                        \right)\\
                = &\sum_{p, r = 1}^{C_q}
                        T_{l, \lambda, p, r} (q_r q_{2, p} F)(u).
            \end{align*}

            By~\eqref{eq:theorem:Littlewood-Paley_decomposition:claim_bound},
            we obtain
            \begin{align*}
                \norm[\Lin{\Lebesgue{2}{\CompactGroup}}]{T_{l, \lambda, p, r}}
                &\leq C \sup_{\tau \in \dualGroup{\CompactGroup}} \sup_{u \in \CompactGroup} \norm[\Lin{\HilbertRep{\tau}}]{\sigma_{l, \lambda, p, r}(u, \tau)}\\
                &\leq C_q < \infty
            \end{align*}
            From there, it follows that
            \begin{align*}
                \norm[\Lebesgue{2}{\CompactGroup}]{\DifferenceOperator{q} \eta_l (\lambda) F}
                &\leq C_q \sum_{p, r = 1}^{C_q} \norm[\Lebesgue{2}{\CompactGroup}]{q_r q_{2, p} F}
            \end{align*}
            where $C_q$ does not depend on $l$.

            Now, using the fact that
            \begin{align*}
                \sup_\CompactGroup \abs{q_r q_{2, p}} \leq C_q < \infty
            \end{align*}
            in the above, this concludes the step.

        \item[Step 4] $\ip[\Lebesgue{2}{\CompactGroup}]{\DifferenceOperator{q} \eta_l(\lambda) \mu_{mn}}{\nu_{kl}}$ is non-zero
            only if $\norm[\dualGroup{\VectorSpace}]{\lambda}, \JapaneseBracket{\CompactGroup}{\mu}, \JapaneseBracket{\CompactGroup}{\nu} \leq C_q 2^l$.

            Choose $C_q \geq 2$ so that
            \begin{align*}
                q_r q_{2, p}, q'_{2, p}(\dummy^{-1})
            \end{align*}
            can be generated by the representations $\{ \tau \in \dualGroup{\CompactGroup} : \JapaneseBracket{\CompactGroup}{\tau} \leq \frac{C_q}{2} \}$.

            Suppose now that $\max\{\JapaneseBracket{\CompactGroup}{\mu}, \JapaneseBracket{\CompactGroup}{\nu}\} > C_q 2^l$.
            It follows from our choice of $C_q$ that if $\JapaneseBracket{\CompactGroup}{\tau} \leq 2^l$,
            either of the following equations hold
            \begin{align*}
                \int_\CompactGroup q_r(k) q_{2, p}(k) \mu_{mn}(k) \conj{\tau_{ij}(k)} \dd k &= 0\\
                \int_\CompactGroup \tau_{ij}(u) q'_{2, p}(u) \conj{\nu_{mn}(k)} \dd k &= 0.
            \end{align*}

            From~\eqref{eq:theorem:Littlewood-Paley_decomposition:exact_expression_for_operator} with $F = \mu_{mn}$,
            we can see that the above implies
            \begin{align*}
                \ip[\Lebesgue{2}{\CompactGroup}]{\DifferenceOperator{q} \eta_l(\lambda) \mu_{mn}}{\nu_{kl}} = 0.
            \end{align*}

            The condition on $\lambda$ is obvious by~\eqref{eq:theorem:Littlewood-Paley_decomposition:rho_condition}
            keeping in mind that $\supp \chi_k \subset \Ball[\dualGroup{\VectorSpace}]{0}{2^{k + 1}}$.

        \item[Step 5] Conclusion.

            Let
            \begin{align*}
                L_l(\lambda) =
                \begin{cases}
                    {\left. \Rep{\lambda} \BesselPotential{1} \right|}_{\oplus_{\JapaneseBracket{\CompactGroup}{\tau} \leq C_q 2^l} V_\tau}
                    & \text{if } \norm[\dualGroup{\VectorSpace}]{\lambda} \leq C_q 2^l\\
                    0 & \text{otherwise},
                \end{cases}
            \end{align*}
            where $C_q$ is given by Step 4.

            Observe that $L_l(\lambda)$ is a bounded operator in $\Lebesgue{2}{\CompactGroup}$,
            and the operator norm is bounded by $C_q 2^l$ uniformly in $\lambda$.

            By Step 4,
            \begin{align*}
                \Rep{\lambda} \BesselPotential{- m - \order(q) + \gamma}&
                \DifferenceOperator{q} \eta_l(\lambda)
                \Rep{\lambda} \BesselPotential{-\gamma}\\
                &= {L_l(\lambda)}^{-m - \order(q) + \gamma}
                \DifferenceOperator{q} \eta_l(\lambda)
                {L_l(\lambda)}^{-\gamma},
            \end{align*}
            from which it follows that
            \begin{align*}
                \norm[\Lin{\Lebesgue{2}{\CompactGroup}}]{\Rep{\lambda} \BesselPotential{- m - \order(q) + \gamma} \DifferenceOperator{q} \eta_l(\lambda) \Rep{\lambda} \BesselPotential{-\gamma}}
                \leq C_q 2^{-l m},
            \end{align*}
            which is what we wanted to show.
    \end{description}
\end{proof}

\section{Kernel estimates}

\begin{theorem}[Kernel estimates]
\label{theorem:kernel_estimates}
    Let $\sigma \in \SymbolClass{m}{\rho, \delta}$, and denote by $\kappa$ its associated kernel.
    \begin{enumerate}
        \item For every $N \in \N$, there exists $C > 0$ such that for every $g, (y, l) \in \Group$ with $\norm{y} \geq 1$, we have
            \begin{align*}
                \abs{\kappa_g(y, l)} \leq C \norm{y}^{-N}
            \end{align*}
        \item If $\dim \Group + m > 0$, there exists $C > 0$ such that for every $g, (y, l) \in \Group$ with $y \neq 0$, we have
            \begin{align*}
                \abs{\kappa_g(y, l)} \leq C \norm{y}^{- \frac{\dim \Group + m}{\rho}}
            \end{align*}
        \item If $\dim \Group + m = 0$, there exists $C > 0$ such that for every $g, (y, l) \in \Group$ with $y \neq 0$, we have
            \begin{align*}
                \abs{\kappa_g(y, l)} \leq C \log \norm{y}
            \end{align*}
        \item If $\dim \Group + m < 0$, $\kappa_g$ is continuous on $\Group$ and is bounded
            \begin{align*}
                \sup_{g, h \in \Group} \abs{\kappa_g(h)} \leq C < \infty.
            \end{align*}
    \end{enumerate}
\end{theorem}

\section{Adjoint and composition formulas}

\section{\texorpdfstring{$L^2$}{L2} boundedness}

The aim of this section is to prove
that symbols of order $0$ induce bounded operators in $\Lebesgue 2 \Group$
under one of the following conditions

\begin{enumerate}
    \item $\rho > \delta$;
    \item $\rho = \delta = 0$.
\end{enumerate}

These results would naturally generalise to symbols of order $m$ as follows:
if $\sigma \in \SymbolClass m {\rho, \delta}$,
where $\rho, \delta$ satisfy either of the conditions above,
then $\Op(\sigma)$ extends continuously to a continuous operator
\begin{align*}
    \Op(\sigma) : \Sobolev s \to \Sobolev {s - m}
\end{align*}
for every $s \in \R$.

\subsection{The case $\rho > \delta$}

Suppose that $\sigma_T \in \SymbolClass 0 {\rho, \delta}$ for $0 \leq \delta < \rho \leq 1$.
One way of showing that $T \defeq \Op(\sigma_T)$ is bounded would be to find $S \in \OperatorClass 0 {\rho, \delta}$ such that
\begin{align}
    \adj T T = C \Id {\Schwartz \Group} - \adj S S + R,
    \label{eq:L2_boundedness_decomposition}
\end{align}
where $R \in \OperatorClass m {\rho, \delta}$ for some $m < 0$.

It would then follow that
\begin{align*}
    \norm [\Lebesgue 2 \Group] {T \phi}^2
    = C \norm [\Lebesgue 2 \Group] {\phi}^2 - \norm [\Lebesgue 2 \Group] {S \phi}^2 + \ip [\Lebesgue 2 \Group] {R \phi} \phi,
\end{align*}
reducing the boundedness of $T$ to that of $R$,
which is much easier to prove.

A sufficient condition to have \eqref{eq:L2_boundedness_decomposition} would be to have
\begin{align*}
    \adj {\sigma_T} \sigma_T = C \Id {\Lebesgue 2 \CompactGroup} - \adj {\sigma_S} \sigma_S,
\end{align*}
on the symbol side,
with $\sigma_S \defeq \Op^{-1}(S)$.
A good candidate for $\sigma_S$ would therefore be
\begin{align}
    \sigma_S \defeq \sqrt{C \Id {\Lebesgue 2 \CompactGroup} - \adj {\sigma_T} \sigma_T}.
    \label{eq:symbol_to_prove_L2_boundedness}
\end{align}

The condition that $S \in \OperatorClass 0 {\rho, \delta}$ is equivalent to showing
that $\sigma_S \in \SymbolClass 0 {\rho, \delta}$.
The crucial part of the proof is therefore to show that~\eqref{eq:symbol_to_prove_L2_boundedness} is a symbol of order $0$.

\begin{definition}[Resolvent of a symbol]
    Let $\sigma \in \SymbolClass m {\rho, \delta}$.
    For every $z \in \C$ such that $\sigma - z \Id {\Lebesgue 2 \CompactGroup}$ is invertible,
    we let
    \begin{align*}
        R(z, \sigma) \defeq {(\sigma - z \Id {\Lebesgue 2 \CompactGroup})}^{-1}.
    \end{align*}
\end{definition}

\begin{lemma}
\label{lemma:inverse_of_square_root_of_a_symbol_of_order_zero}
    Let $\rho, \delta \in \R$ be such that $1 \geq \rho > \delta \geq 0$.
    If $\sigma \in \SymbolClass 0 {\rho, \delta}$ is \emph{positive definite} and \emph{elliptic},
    then the map
    \begin{align*}
        \tilde \sigma(x, k; \lambda)
        \defeq \int_{\Gamma} z^{-\frac 1 2} \Resolvent z {\sigma(x, k; \lambda)} \dd z
        % TODO specify what \gamma is
    \end{align*}
    also belongs to $\SymbolClass 0 {\rho, \delta}$.
\end{lemma}

\begin{corollary}
    Let $\rho, \delta \in \R$ be such that $1 \geq \rho > \delta \geq 0$.
    If $\sigma \in \SymbolClass 0 {\rho, \delta}$ is \emph{positive definite} and \emph{elliptic},
    then its \emph{square root}
    \begin{align*}
        (x, k; \lambda) \in \Group \times \VectorSpace \mapsto \sqrt{\sigma(x, k; \lambda)}
    \end{align*}
    also belongs to $\SymbolClass 0 {\rho, \delta}$.
\end{corollary}
\begin{proof}
    Let $\tilde \sigma$ be the symbol defined in Lemma~\ref{lemma:inverse_of_square_root_of_a_symbol_of_order_zero}.
    By the Helffer-Sjostrand formula,
    % TODO Reference for Helffer-Sjostrand
    we get
    \begin{align*}
        \sqrt{\sigma(x, k; \lambda)} = \tilde \sigma(x, k; \lambda) \sigma(x, k; \lambda),
    \end{align*}
    which must be in $\SymbolClass 0 {\rho, \delta}$,
    since $\tilde \sigma, \sigma \in \SymbolClass 0 {\rho, \delta}$ by Lemma~\ref{lemma:inclusion_in_zero_class}.
\end{proof}

\begin{proposition}[$L^2$ boundedness]
    Let $\rho, \delta \in \R$ be such that $1 \geq \rho > \delta \geq 0$.
    If $\sigma \in \SymbolClass 0 {\rho, \delta}$,
    then its associated operator $T \defeq \Op(\sigma)$ is bounded in $\Lebesgue 2 \Group$,
    i.e.\ we have
    \begin{align*}
        \norm [\Lebesgue 2 \Group] {T \phi} \leq C \norm [\Lebesgue 2 \Group] \phi.
    \end{align*}
    for every $\phi \in \Schwartz \Group$.
\end{proposition}
\begin{proof}
    \begin{description}
        \item[Step 1.] If $\sigma \in \SymbolClass m {\rho, \delta}$ for $m < -1$, then
            \begin{align*}
                \norm [\Lin{\Lebesgue 2 \Group}] {\Op(\sigma)} < \infty.
            \end{align*}
        \item[Step 2.] If $\sigma \in \SymbolClass m {\rho, \delta}$ for $m < 0$, then
            \begin{align*}
                \norm [\Lin{\Lebesgue 2 \Group}] {\Op(\sigma)} < \infty.
            \end{align*}
        \item[Step 3.] Conclude.
    \end{description}
\end{proof}

\chapter{Applications of pseudo-differential calculus}

\section{Construction of parametrices}

Given $\Lambda \geq 0$,
we let
\begin{align*}
    E_\lambda(\Lambda)
    \defeq
    \begin{cases}
        \Id {\Lebesgue 2 \CompactGroup} & \text{if } \lambda \geq \Lambda\\
        0 & \text{otherwise}
    \end{cases}
\end{align*}

\begin{definition}[Ellipticity]
\label{definition:ellipticity}
    Let $\sigma \in \SymbolClass m {\rho, \delta}$
    We shall say that that $\sigma$ is \emph{elliptic} and of \emph{elliptic order} $m$
    if there exists $\Lambda \geq 0$ such that the following property holds:
    for each $\gamma \in \R$,
    there exists $C \geq 0$ such that
    \begin{align}
        \norm [\Lebesgue 2 \CompactGroup] {\Rep \lambda \BesselPotential \gamma \sigma(g, \lambda) F}
        \geq C
        \norm [\Lebesgue 2 \CompactGroup] {\Rep \lambda \BesselPotential {\gamma + m} F}
        \label{eq:definition_of_ellipticity}
    \end{align}
    holds for every $F \in E_\lambda(\Lambda) \SmoothFunctions \CompactGroup$
    and every $(g, \lambda) \in \Group \times \VectorSpace$.
\end{definition}

\begin{lemma}[Inverse of an elliptic symbol]
\label{lemma:inverse_of_elliptic_symbol}
    Let $\sigma \in \SymbolClass m {\rho, \delta}$ be an elliptic symbol of elliptic order $m$,
    with $\Lambda \geq 0$ given by Definition~\ref{definition:ellipticity}.

    For any $F \in E_\lambda(\Lambda) \SmoothFunctions \CompactGroup$,
    let
    \begin{align*}
        E_\lambda(\Lambda) \sigma(g, \lambda)^{-1} \left(\sigma(g, \lambda) F\right) \defeq F.
    \end{align*}
    and consider the unique extension of $E_\lambda(\Lambda) \sigma(g, \lambda)^{-1}$ onto $\SmoothFunctions \CompactGroup$ which satisfies
    \begin{align*}
        \eval {%
            E_\lambda(\Lambda) \sigma(g, \lambda)^{-1}
            } {(\sigma(g, \lambda) E_\lambda(\Lambda))^\perp} = 0.
    \end{align*}

    The map $E_\lambda(\Lambda) \sigma(g, \lambda)^{-1}$ is well-defined.
    Moreover,
    for each $\lambda \in \R$,
    we have
    \begin{align}
        \norm [\Lin {\Lebesgue 2 \CompactGroup}] {%
            \Rep \lambda \BesselPotential {m + \gamma}
            E_\lambda(\Lambda) \sigma(g, \lambda)^{-1}
            \Rep \lambda \BesselPotential {-\gamma}
        }
        < \infty
        \label{eq:boundedness_of_E_lambda_Lambda_sigma_inverse}
    \end{align}
\end{lemma}
\begin{proof}
    Suppose that $F_1, F_2 \in E_\lambda(\Lambda) \SmoothFunctions \CompactGroup$ are such that
    \begin{align*}
        \sigma(g, \lambda) F_1
        = \sigma(g, \lambda) F_2.
    \end{align*}
    By definition of ellipticity with $\gamma \defeq - m$,
    it follows that
    \begin{align*}
        \norm [\Lebesgue 2 \CompactGroup] {F_1 - F_2}
        &\leq C
        \norm [\Lebesgue 2 \CompactGroup] {\Rep \lambda \BesselPotential {-m} \sigma(g, \lambda) (F_1 - F_2)}\\
        &= 0,
    \end{align*}
    so that the map is indeed well-defined.

    Now, given $\gamma \in \R$,
    we easily check by ellipticity that
    \begin{align*}
        &\norm [\Lebesgue 2 \CompactGroup] {\Rep \lambda \BesselPotential {m + \gamma} E_\lambda(\Lambda) \sigma(g, \lambda)^{-1} F}\\
        &\quad \leq C
        \norm [\Lebesgue 2 \CompactGroup] {\Rep \lambda \BesselPotential {\gamma} \sigma(g, \lambda) E_\lambda(\Lambda) \sigma(g, \lambda)^{-1} F}\\
        &\quad \leq C
        \norm [\Lebesgue 2 \CompactGroup] {\Rep \lambda \BesselPotential {\gamma} F},
    \end{align*}
    which shows~\eqref{eq:boundedness_of_E_lambda_Lambda_sigma_inverse}.
\end{proof}

\begin{proposition}[Inverse of an elliptic symbol]
\label{proposition:inverse_of_elliptic_symbol}
    Assume that $1 \geq \rho > \delta \geq 0$.
    Suppose that $\sigma \in \SymbolClass m {\rho, \delta}$ is a symbol which is $(\Lambda, m)$-elliptic.

    Let $\chi \in \SmoothFunctions {\R^+, [0, 1]}$ be such that
    \begin{align*}
        \chi_\Lambda(r) =
        \begin{cases}
            1 & \text{if } r \geq \Lambda_2\\
            0 & \text{if } r \leq \Lambda_1,
        \end{cases}
    \end{align*}
    for some $\Lambda < \Lambda_1 < \Lambda_2$,
    and define the symbol
    \begin{align*}
        \psi_\Lambda(\lambda)
        \defeq \sum_{\tau \in \dualGroup \CompactGroup}
        \chi_\Lambda(\JapaneseBracket \VectorSpace \lambda + \JapaneseBracket \CompactGroup \lambda) \Id {\Span \{\tau_{i j} : 1 \leq i, j \leq \dimRep \tau\}}.
    \end{align*}

    The map
    \begin{align*}
        \sigma_\Lambda^{-1}(g, \lambda)
        \defeq \psi_\Lambda E_\lambda(\Lambda_1) \sigma(g, \lambda)^{-1}
    \end{align*}
    defines a symbol of $\SymbolClass {-m} {\rho, \delta}$.
\end{proposition}
\begin{proof}
    Let us first observe that
    \begin{align}
        \psi_\Lambda = \sigma^{-1}_\Lambda(g, \lambda) \sigma(g, \lambda).
        \label{eq:identity_satisfied_by_inverse_of_symbol}
    \end{align}

    Since $\psi_\Lambda$ commutes with the powers of $\Rep \lambda \BesselPotentialSquared {}$,
    \begin{align*}
        &\norm [\Lin {\Lebesgue 2 \CompactGroup}] {\Rep \lambda \BesselPotential m \sigma_\Lambda(g, \lambda)^{-1}}\\
        &\quad \leq
        \norm [\Lin {\Lebesgue 2 \CompactGroup}] {\psi_\Lambda}
        \norm [\Lin {\Lebesgue 2 \CompactGroup}] {\Rep \lambda \BesselPotential m E_\lambda(\Lambda) \sigma(g, \lambda)^{-1}}
        < \infty
    \end{align*}
    where the finiteness is given by Lemma~\ref{lemma:inverse_of_elliptic_symbol}.

    We have thus shown that
    \begin{align*}
        \sup_{g \in \Group}
        \esssup_{\lambda \in \VectorSpace}
        \norm [\Lin {\Lebesgue 2 \CompactGroup}] {\Rep \lambda \BesselPotential {m - \delta \abs {\beta_0}} \LeftDifferentialOperator {\beta_0} \sigma_\Lambda(g, \lambda)^{-1}}
        < \infty,
    \end{align*}
    for $\beta_0 = 0$.
    Let us show it also holds for $\beta_0$ of higher order.

    By the Leibniz rule,
    we have
    \begin{align*}
        0 =
        \LeftDifferentialOperator {\beta_0} \psi_\Lambda
        =
        \sum_{\beta_0 = \beta_1 + \beta_2}
        \left(
        \LeftDifferentialOperator {\beta_1} \sigma_\Lambda(g, \lambda)^{-1}
        \LeftDifferentialOperator {\beta_2} \sigma(g, \lambda)\right),
    \end{align*}
    which implies that
    \begin{align*}
        \LeftDifferentialOperator {\beta_0} \sigma_{\Lambda}(g, \lambda)^{-1} \sigma(g, \lambda)
        = -
        \sum_{\substack{
            \beta_0 = \beta_1 + \beta_2\\
            \abs {\beta_1} < \abs {\beta_0}}
        }
        \left(
        \LeftDifferentialOperator {\beta_1} \sigma_\Lambda(g, \lambda)^{-1}
        \LeftDifferentialOperator {\beta_2} \sigma(g, \lambda) \right).
    \end{align*}
    Since $\sigma(g, \lambda)$ is invertible on $E_\lambda(\Lambda_1) \SmoothFunctions \CompactGroup$,
    we obtain
    \begin{align*}
        &\LeftDifferentialOperator {\beta_0} \sigma_{\Lambda}(g, \lambda)^{-1}\\
        &\quad
        = -
        \sum_{\substack{
            \beta_0 = \beta_1 + \beta_2\\
            \abs {\beta_1} < \abs {\beta_0}}
        }
        \left(
        \LeftDifferentialOperator {\beta_1} \sigma_\Lambda(g, \lambda)^{-1}
        \LeftDifferentialOperator {\beta_2} \sigma(g, \lambda) \right)
        E(\Lambda_1) \sigma^{-1}(g, \lambda),
    \end{align*}
    which allows us to conclude by induction and Lemma~\ref{lemma:inverse_of_elliptic_symbol}.

    Let us now work with difference operators.
    The ideas are the same,
    but the proof is slightly trickier
    because we have a \emph{Leibniz-like} rule instead of the classical Leibniz rule.

    Applying the Leibniz-like rule to~\eqref{eq:identity_satisfied_by_inverse_of_symbol},
    we obtain
    \begin{align*}
        \DifferenceOperator j \psi_\Lambda(\lambda)
        =
        &\DifferenceOperator j \sigma_\Lambda(g, \lambda)^{-1} \sigma(g, \lambda)
        + \sigma_\Lambda(g, \lambda)^{-1} \DifferenceOperator j \sigma(g, \lambda) \\
        &+ \sum_{1 \leq k, l \leq \dimDifferenceOperators} c_j^{k l} \DifferenceOperator k \sigma_\Lambda(g, \lambda)^{-1} \DifferenceOperator l \sigma(g, \lambda).
    \end{align*}
    Reorganising the terms above,
    and using the fact that $\sigma$ is invertible on $E_\lambda(\Lambda_1) \SmoothFunctions \CompactGroup$,
    we obtain
    \begin{align}
        &\DifferenceOperator j \sigma_\Lambda(g, \lambda)^{-1}
        + \sum_{1 \leq k, l \leq \dimDifferenceOperators} c_j^{k l} \DifferenceOperator k \sigma_\Lambda(g, \lambda)^{-1} \DifferenceOperator l \sigma(g, \lambda) E_\lambda(\Lambda_1) \sigma^{-1}(g, \lambda) \notag\\
        &=
        \DifferenceOperator j \psi_\Lambda(\lambda) E_\lambda(\Lambda_1) \sigma^{-1}(g, \lambda)
        - \sigma_\Lambda(g, \lambda)^{-1} \DifferenceOperator j \sigma(g, \lambda) E_\lambda(\Lambda_1) \sigma^{-1}(g, \lambda).
        \label{eq:system_of_equations_of_difference_operators}
    \end{align}

    If $R(g, \lambda)$ is the second line of the above,
    we check easily that
    \begin{align*}
        \norm [\Lin {\Lebesgue 2 \CompactGroup}] {%
            \Rep \lambda \BesselPotential {m + \rho + \gamma}
            R(g, \lambda)
            \Rep \lambda \BesselPotential {-\gamma}
        } < \infty,
    \end{align*}
    we can therefore deduce that
    \begin{align*}
        \norm [\Lin {\Lebesgue 2 \CompactGroup}] {%
            \Rep \lambda \BesselPotential {m + \rho + \gamma}
            \DifferenceOperator j \sigma_\Lambda(g, \lambda)^{-1}
            \Rep \lambda \BesselPotential {-\gamma}
        } < \infty,
    \end{align*}
    by considering the equations~\eqref{eq:system_of_equations_of_difference_operators} for all $j$ simultaneously.
    Applying $\DifferenceOperatorOrder \alpha$ to~\eqref{eq:identity_satisfied_by_inverse_of_symbol},
    we can deduce inductively by arguing like above that
    \begin{align*}
        \norm [\Lin {\Lebesgue 2 \CompactGroup}] {%
            \Rep \lambda \BesselPotential {m + \rho \abs \alpha + \gamma}
            \DifferenceOperatorOrder \alpha \sigma_\Lambda(g, \lambda)^{-1}
            \Rep \lambda \BesselPotential {-\gamma}
        } < \infty.
    \end{align*}

    The general case can be deduced similarly by applying $\LeftDifferentialOperator \beta \DifferenceOperatorOrder \alpha$ to~\eqref{eq:identity_satisfied_by_inverse_of_symbol}.
\end{proof}

\begin{theorem}[Existence of left parametrices]
    Let $A \in \OperatorClass m {\rho, \delta}$ be an $m$-elliptic operator of order $m$ with $1 \geq \rho > \delta \geq 0$.
    The exists an operator $B \in \OperatorClass {-m} {\rho, \delta}$ which satisfies
    \begin{align*}
        B A = \Id {\Schwartz \Group} \qquad \text{mod} \quad \SmoothingOperators
    \end{align*}
\end{theorem}
\begin{proof}
    Since $A = \Op(\sigma)$ is elliptic,
    we know that $\sigma_\Lambda(g, \lambda)^{-1} \in \SymbolClass {-m} {\rho, \delta}$ and
    \begin{align*}
        \sigma_\Lambda(g, \lambda)^{-1} \sigma(g, \lambda)
        &= \psi_\Lambda - \sigma_\Lambda(g, \lambda)^{-1} \sigma(g, \lambda) (\Id {\Lebesgue 2 \CompactGroup} - \psi_\Lambda)\\
        &= \psi_\Lambda \quad \text{mod} \quad \SmoothingSymbols,\\
        &= \Id {\Lebesgue 2 \CompactGroup} \quad \text{mod} \quad \SmoothingSymbols,
    \end{align*}
    since $\Id {\Lebesgue 2 \CompactGroup} - \psi_\Lambda$ is smoothing.

    Therefore,
    letting $B_0 \defeq \Op(\sigma_\Lambda(g, \lambda)^{-1})$ yields
    \begin{align*}
        B_0 A = \Id {\Schwartz \Group} - E_0 \quad \text{where } E_0 = \Op(e_0) \in \OperatorClass {-(\rho - \delta)} {\rho, \delta}
    \end{align*}
    by the composition formula.
    Suppose by induction that we have defined $B_j$ and $E_j = \Op(e_j)$ for $j = 0, \dots, k$ such that
    \begin{align*}
        (B_0 + \dots + B_k) A = I - E_k \quad \text{where } E_k = \Op(e_k) \in \OperatorClass {-(k + 1)(\rho - \delta)} {\rho, \delta}
    \end{align*}

    Letting $B_{k + 1} \defeq \Op(E_k \sigma_\Lambda(g, \lambda)^{-1})$,
    we obtain
    \begin{align*}
        (B_0 + \dots + B_{k + 1}) A = I - E_k + B_{k + 1} A = I - E_{k + 1}
    \end{align*}
    where $E_{k + 1} \defeq E_k - B_{k + 1} A$ is an operator in $\OperatorClass {-(k + 2)(\rho - \delta)} {\rho - \delta}$,
    again by the composition formula.

    Since the order of $B_k$ strictly decreases as $k$ strictly increases,
    it follows that
    \begin{align*}
        B \sim \sum_{k = 0}^{+\infty} B_k
    \end{align*}
    defines an operator in $\OperatorClass {-m} {\rho, \delta}$.
    Moreover,
    for each $N \in \N$ we have
    \begin{align*}
        B A - I + E_{N}
        =
        (B - \sum_{k = 0}^N B_k) A \in \OperatorClass {-(N + 1) (\rho - \delta)  + m} {\rho, \delta},
    \end{align*}
    with $E_N \in \OperatorClass {-(N + 1)(\rho - \delta)} {\rho, \delta}$.
    Letting $N \to \infty$ on both sides of the above,
    we obtain that
    \begin{align*}
        B A - I \in \SmoothingOperators,
    \end{align*}
    concluding the proof.
\end{proof}

\chapter{Towards groups with polynomial growth}

\section{The setting}

In this chapter,
$\Group$ will denote a connected locally compact Lie group,
for which we fix a non-zero left-invariant Haar measure $\mu = \dd g$.
Moreover,
we shall assume that $\Group$ has \emph{polynomial growth},
in the sense of the following definition.

\begin{definition}[Group of polynomial growth]
    We shall say that $\Group$ has \emph{polynomial growth}
    if for every compact neighbourhood $U \subset \Group$
    containing the identity $e_\Group$,
    there exists $C \geq 0$ such that
    \begin{align*}
        \mu(U^n) \leq C n^C
    \end{align*}
    for every $n \in \N$.
\end{definition}

This implies that $\Group$ is unimodular.

\section{The heat kernel}

\begin{lemma}
\label{lemma:estimate_on_heat_kernel}
    Let $q \in \SmoothFunctions \Group$ which vanishes up to order $a \in \N$ at $e_\Group$.
    If $\beta \in \N^{\dim \Group}$,
    then there exists $C \geq 0$ such that
    \begin{align}
        \int_\Group \abs {q(x) \LeftDifferentialOperator \beta h_t(x)} \dd x
        \leq C_{q, \beta} t^{\frac {a - \abs \beta} 2}
    \end{align}
    holds for every $t \in (0, 1]$.
\end{lemma}
\begin{proof}
    Using \cite[Theorem VIII.2.7]{VaropoulosSaloffCosteCoulhon92},
    we know that the \emph{heat kernel} satisfies the following estimate
    \begin{align}
        \abs {\LeftDifferentialOperator \beta h_t(x)} \leq C \sqrt t^{-\abs \beta} \Volume {\sqrt t}^{-1} \e^{-\frac{\norm [\Group] x^2} {C_\beta t}}
        \label{eq:estimate_on_the_derivatives_of_the_heat_kernel}
    \end{align}

    Let us write
    \begin{align*}
        I_A \defeq \int_{\norm [\Group] h \in A} \abs {q(x) \LeftDifferentialOperator \beta h_t(x)},
    \end{align*}
    where $A$ is an interval in $\R$.
    For each $j \in \N$,
    we let
    \begin{align*}
        I_j \defeq
        \begin{cases}
            I_{[0, \sqrt t]} & \text{if } j = 0\\
            I_{[2^{j - 1} \sqrt t, 2^j \sqrt t]} & \text{otherwise}.
        \end{cases}
    \end{align*}

    Using the fact that $\abs {q(x)} \leq C_q \norm [\Group] x^a$
    together with~\eqref{eq:estimate_on_the_derivatives_of_the_heat_kernel},
    \begin{align*}
        I_0 \leq C_q \sqrt t^{a -\abs \beta}
    \end{align*}

    Using~\eqref{eq:estimate_on_the_derivatives_of_the_heat_kernel},
    it follows that
    \begin{align*}
        I_j &\leq C_q (2^j \sqrt t)^a \sqrt t^{-\abs \beta} \frac {\Volume {2^j \sqrt t}} {\Volume {\sqrt t}} \e^{-\frac{4^{j - 1}} {C_\beta}}\\
        &\leq C_q \sqrt t^{a - \abs \beta}
        \left(2^{j a} \frac {\Volume {2^j \sqrt t}} {\Volume {\sqrt t}} \e^{-\frac{4^{j - 1}} {C_\beta}}\right)
    \end{align*}

    Since the volume grows at most polynomially,
    the expression in brackets above decreases exponentially in $j$,
    hence
    \begin{align*}
        \sum_{j = 1}^\infty I_j \leq C_{q, \beta} \sqrt t^{a - \abs \beta}
    \end{align*}

    We have thus shown that
    \begin{align*}
        \int_\Group \abs {q(x) \LeftDifferentialOperator \beta h_t(x)} \dd x
        = \sum_{j = 0}^{+\infty} I_j \leq C_q t^{\frac{a - \abs \beta} 2},
    \end{align*}
    concluding the proof.
\end{proof}

\begin{proposition}
    Let $m \in \R$ and suppose that $q \in \SmoothFunctions \Group$ is a smooth function vanising up to order $a$ at the origin $e_\Group$.
    The map
    \begin{align*}
        \phi \in \Schwartz \Group \mapsto \conv \phi {(q h_t)}
    \end{align*}
    extends to a bounded map
    \begin{align*}
        T_{q h_t} : \Lebesgue 2 \Group \to \Sobolev {m - a}
    \end{align*}
    which satisfies
    \begin{align*}
        \norm [\Lin{\Lebesgue 2 \Group, \Sobolev {m - a}}] {T_{q h_t}}
        \leq t^{\frac m 2}
    \end{align*}
\end{proposition}
\begin{proof}
    Let $N \in \N$.
    By Lemma~\ref{lemma:estimate_on_heat_kernel},
    it is clear that after applying the Leibniz rule,
    we have
    \begin{align*}
        \int_\Group &\abs {(1 - \Laplacian)^N \{q h_t\}(x)} \dd x\\
        &\leq \sum_{\abs{\beta_1 + \beta_2} \leq 2N} \int_\Group \abs {\LeftDifferentialOperator {\beta_1} q(x) \LeftDifferentialOperator {\beta_2} h_t(x)} \dd x\\
        &\leq \sum_{\abs{\beta_1 + \beta_2} \leq 2N} t^{\frac {a - \abs {\beta_1} - \abs{\beta_2}} 2},
    \end{align*}
    since $\LeftDifferentialOperator {\beta_1}$ vanishes up to order $a - \abs {\beta_1}$.
    Continuing the above calculation,
    one gets
    \begin{align*}
        \int_\Group \abs {(1 - \Laplacian)^N \{q h_t\}(x)} \dd x
        \leq C_{q, N} \sum_{\abs{\beta_1 + \beta_2} \leq 2N} t^{\frac {a - \abs {\beta_1} - \abs{\beta_2}} 2}
        \leq C_{q, N} t^{\frac a 2 - N}.
    \end{align*}

    By the properties of interpolation and duality of Sobolev spaces,
    we may replace $2N$ by any real number,
    in particular by $m - a$.
    With this substitution,
    we obtain that
    \begin{align*}
        \int_\Group \abs {\BesselPotential {m - a} \{q h_t\}(x)} \dd x
        \leq C_{q, N} t^{\frac m 2},
    \end{align*}
    from which it follows that
    \begin{align*}
        \norm [\Lin {\Lebesgue 2 \Group, \Sobolev {m - a}}] {T_{q h_t}}
        \leq \int_\Group \abs {\BesselPotential {m - a} \{q h_t\}(x)} \dd x
        \leq t^{\frac m 2}
    \end{align*}
\end{proof}

\section{Kernel estimates}

First,
let us lower the requirement that we need a Littlewood-Paley decomposition
to prove the kernel estimates.

\begin{theorem}[Kernel estimates at the origin]
    Assume that $1 \geq \rho \geq \delta \geq 0$ with $\rho \neq 0$.

    Suppose there exists a family $(h_t)_{t \in (0, 1]} \subset \Schwartz \Group$
    satisfying the following properties.
    \begin{itemize}
        \item As $t$ tend to $0$, $h_t$ converges to $\delta_{e_\Group}$ in $\TemperedDistributions$.
        \item If $q \in \SmoothFunctions {\Group}$ vanishes up to order $a \in \N$,
            then for each $s \in \R$ there exists $C \geq 0$ such that
            \begin{align}
                \norm [\KernelsSobolev s {s - m + \rho a}] {q h_t}
                \leq C t^{\frac m 2}
                \label{eq:kernel_estimates:heat_kernel_condition}
            \end{align}
            holds for every $0 < t \leq 1$.
    \end{itemize}

    If $\kappa \in \TemperedDistributions \Group$ satisfies
    \begin{align}
        q \kappa \in \Sobolev \gamma,
        \quad \gamma < -\frac n 2 -m + \rho a
        \label{eq:kernel_estimates:assumption_on_the_kernel}
    \end{align}
    whenever $q$ vanishes up to order $a$ and $m > -n$,
    then
    \begin{align*}
        \abs{\kappa(g)} \leq C \norm [\Group] h^{-\frac{m + n} \rho}.
    \end{align*}
\end{theorem}
\begin{proof}
    Fix $g \in \Group \setminus \{e_\Group\}$,
    and let $t \defeq \norm [\Group] g^2$.

    Let us write $\kappa_t \defeq \conv \kappa {h_t}$.
    By the Sobolev inequality,
    we know that for each $s < -n/2$
    \begin{align*}
        \norm [\ContinuousFunctions \Group] {\kappa_t}
        \leq \norm [\Sobolev {-s}] {\conv \kappa {h_t}}
        \leq \norm [\KernelsSobolev {s - m} {-s}] {h_t}
        \norm [\Sobolev {s - m}] \kappa,
    \end{align*}
    where the right-hand side is finite by~\eqref{eq:kernel_estimates:assumption_on_the_kernel}.
    Using~\eqref{eq:kernel_estimates:heat_kernel_condition},
    we then obtain
    \begin{align*}
        \norm [\ContinuousFunctions \Group] {\kappa_t}
        \leq C t^{s - \frac m 2} \norm [\Sobolev {s - m}] \kappa
    \end{align*}

    Observing we may choose $s < -n/2$ so that
    \begin{align*}
        s - \frac m 2 = -\frac {m + n} {2 \rho},
    \end{align*}
    we obtain
    \begin{align}
        \abs {\kappa_t(g)}
        \leq C \norm [\Sobolev {\frac m 2 -\frac {m + n} {2 \rho}}] \kappa \norm [\Group] g^{- \frac {m + n} \rho}.
        \label{eq:kernel_estimates:estimate_on_kappa_t}
    \end{align}

    Now, let us treat the remainder $\kappa - \kappa_t$.
    Fix $\alpha \in \N^{\dimDifferenceOperators}$.
    Choosing $s < -n/2$,
    we can again apply the Sobolev inequality to obtain
    \begin{align*}
        &\abs {q^\alpha (\kappa - \kappa_t)(g)}\\
        &\leq \norm [\Sobolev {-s}] {q^\alpha(\conv \kappa {(\delta_{e_\Group} - \kappa_t)})}\\
        &\leq \sum_{\abs \alpha \leq \abs {\alpha_1} + \abs {\alpha_2} \leq C \abs \alpha}
        \norm [\Sobolev {-s}] {\conv {q^{\alpha_1} \kappa} {q^{\alpha_2} (\delta_{e_\Group} - \kappa_t)}}\\
        &\leq \sum_{\abs \alpha \leq \abs {\alpha_1} + \abs {\alpha_2} \leq C \abs \alpha}
        \norm [\KernelsSobolev {s - m + \rho \abs {\alpha_1}} {-s}] {q^{\alpha_2}(\delta_{e_\Group} - \kappa_t)}
        \norm [\Sobolev {s - m + \rho \abs {\alpha_1}}] {q^{\alpha_1} \kappa}
    \end{align*}

    Observe that by~\eqref{eq:kernel_estimates:heat_kernel_condition}
    and keeping in mind that $\abs {\alpha_1} + \abs {\alpha_2} \geq \abs \alpha$,
    \begin{align*}
        \norm [\KernelsSobolev {s - m + \rho \abs {\alpha_1}} {-s}] {q^{\alpha_2} (h_{\epsilon} - h_t)}
        \leq C (\epsilon^{\frac{2s - m + \rho \abs \alpha} 2} + t^{\frac{2s - m + \rho \abs \alpha} 2}),
    \end{align*}
    and thus letting $\epsilon \to 0$ yields
    \begin{align*}
        \norm [\KernelsSobolev {s - m + \rho \abs {\alpha_1}} {-s}] {q^{\alpha_2} (\delta_{e_\Group} - h_t)}
        \leq C t^{\frac {2s - m + \rho \abs \alpha} 2}
        = C \norm [\Group] g^{2s - m + \rho \abs \alpha},
    \end{align*}
    provided $2s - m + \rho \abs {\alpha} > 0$
    (which we shall assume henceforth).

    Using the fact that
    \begin{align*}
        \norm [\Group] g^{p r}
        \leq \sum_{\abs \alpha = p r} q^\alpha(g),
    \end{align*}
    we then obtain
    \begin{align}
        \abs {\kappa - \kappa_t(g)}
        \leq C \norm [\Group] g^{2s - m + p r(\rho - 1)}
        \quad
        \text{if }
        m - \rho p r < 2s < -n.
        \label{eq:kernel_estimate_with_choice_of_s}
    \end{align}

    Observing that the above estimate is optimal when $r$ is as small as possible,
    we are led to define
    \begin{align*}
        r \defeq \min \{q \in \N : m - \rho p q < -n \Longleftrightarrow q > \frac{m + n} {\rho p}\}
    \end{align*}

    Now, choose $s \in \R$ so that
    \begin{align}
        2s - m + p r (\rho - 1) = - \frac {m + n} \rho
        \label{eq:kernel_estimates:choice_of_s}
    \end{align}
    We check that
    \begin{align*}
        m - 2s &= p r (\rho - 1) + \frac {m + n} \rho,\\
        &> m + n
    \end{align*}
    which implies that $2 s < -n$.
    Similarly,
    we check that
    \begin{align*}
        m - \rho p r = 2s + \frac {m + n} \rho - p r
        < 2 s.
    \end{align*}

    We have shown that our choice of $s$ in~\eqref{eq:kernel_estimates:choice_of_s} satisfies~\eqref{eq:kernel_estimate_with_choice_of_s},
    so that we have
    \begin{align*}
        \abs {\kappa - \kappa_t(g)}
        \leq C \norm [\Group] g^{-\frac{m + n} \rho},
    \end{align*}
    concluding the proof.
\end{proof}

\section{Calder\'on-Zygmund property}


\printindex

\emergencystretch=2em
\printbibliography

\nocite{*}

\end{document}
