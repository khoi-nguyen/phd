\chapter{Conclusion}

Building upon the work of M.~Ruzhansky, V.~Turunen, V.~Fischer and others,
we managed to define a global pseudo-differential calculus on the motion group.
The main contribution of this document is arguably a shorter, less technical, and purely symbolic proof of the kernel estimates,
which is the key result in the strategy devised in~\cite{FischerRuzhansky16}.

Our analysis uses the specific structure of the motion group
only when we prove the existence of a finite strongly admissible family of polynomials satisying a \emph{Leibniz-like} rule,
and with adequate growth at infinity to control the decay of the kernel away from the origin.
Going forward,
it would be interesting to know which other Lie groups possess such a family of polynomials,
or whether the aforementioned conditions can be relaxed in any way.

Unfortunately,
the present document is rather academic
in the sense that applications of the developed calculus are left untreated.
As an example,
we expect a sharp Gårding inequality to hold on the motion group.
Moreover,
the study of the spectral and the propagation of singularities of our pseudo-differential operators is also of interest.
This, among many other things,
could be the object of future work.
