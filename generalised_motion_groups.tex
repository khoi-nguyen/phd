\chapter{Generalised Motion Groups}

\section{Definitions}

\begin{definition}[Semi-direct product]
\label{definition:semi-direct_products}
\index{semi-direct product}
    Let $N$ be a group and $H$ be a subgroup of $\Aut(N)$.
    The \emph{semi-direct product} $N \rtimes H$ is the group whose elements are that of $N \times H$ with the group law:
    \begin{align}
        (n_1, h_1) (n_2, h_2) = (n_1 h_1(n_2), h_1 h_2), \quad (n_1, h_1), (n_2, h_2) \in N \times H.
    \end{align}

    Given $(n, h)$ in $N \rtimes H$, its \emph{inverse} is given by $(h_1^{-1}(n_1^{-1}), h_1^{-1})$,
    while $(e_N, e_H)$ is the \emph{identity element} of the group.
\end{definition}

\begin{definition}[Generalised motion group]
\label{definition:generalised_motion_group}
\index{generalised motion group}
    Let $\Group$ be a group.
    We shall say that $\Group$ is a \emph{generalised motion group}
    if there exists a (finite dimensional) vector space $\VectorSpace$ and a compact group $\CompactGroup \subset \OrthogonalGroup{\VectorSpace}$ such that $\Group = \VectorSpace \rtimes \CompactGroup$.
\end{definition}

\begin{example}[Euclidean Motion Groups]
\label{example:Euclidean_motion_groups}
\index{Euclidean motion group}
    For each $n \in \N$, let
    \begin{align*}
        \MotionGroup{n} = \{g \in \AffineTransformations{\R^n} : \det g = 1\}.
    \end{align*}
    The elements of $\MotionGroup{n}$ are called \emph{rigid motions},
    while $\MotionGroup{n}$ is called the \emph{Euclidean motion group}.

    It is easily shown that associating $(x, k) \in \R^n \rtimes \SpecialOrthogonalGroup{n}$ to the motion
    \begin{align*}
        g_{(x, k)} : \R^n \to \R^n : y \mapsto x + ky
    \end{align*}
    defines a group isomorphism between $\R^n \rtimes \SpecialOrthogonalGroup{n}$ and $\MotionGroup{n}$.
    We shall therefore identify $\MotionGroup{n}$ with $\R^n \rtimes \SpecialOrthogonalGroup{n}$ from now on.
\end{example}

From now on, unless stated otherwise,
$\Group$ will denote a generalised motion group,
with $\VectorSpace$ its underlying vector space and $\CompactGroup$ its associated compact group.

The requirement that $K$ should be a subgroup of $\OrthogonalGroup{\VectorSpace}$ is motivated by the following result:

\begin{lemma}[Haar measure]
\label{lemma:Haar_measure}
    If $\dd x$ is a Lebesgue measure on $\VectorSpace$ and $\dd k$ is the normalised Haar measure on $\CompactGroup$,
    then the the product measure $\dd x \dd k$ is a Haar measure on $\Group = \VectorSpace \rtimes \CompactGroup$,
    which is both left and right-invariant.
\end{lemma}

\section{Lie algebra structure}

\section{Unitary representations}

\begin{definition}
\label{definition:reducible_representation}
    Let $\lambda \in \dualGroup{\VectorSpace}$.
    We define a unitary representation $\Rep{\lambda} \in \Hom(\Group, \End(\Lebesgue{2}{\CompactGroup}))$ of $\Group$ via:
    \begin{align}
        \Rep{\lambda} (x, k) F(u) = \lambda(u^{-1} x) F(k^{-1} u),
    \end{align}
    where $(x, k) \in \CompactGroup$, $F \in \Lebesgue{2}{\CompactGroup}$ and $u \in \CompactGroup$.
\end{definition}

The above representation is unfortunately reducible.
However, as we shall see later, the Fourier Transform on $\Group$ can be written exclusively with those representations.

\subsection{Unitary dual}

Throughout this section, fix $\lambda \in \dualGroup{\VectorSpace}$
and denote by $\IsotropySubgroup{\CompactGroup}{\lambda}$ its isotropy subgroup.

Let $\tau \in \dualGroup{\IsotropySubgroup{\CompactGroup}{\lambda}}$ and denote by $\dimRep{\tau}$ its dimension.
For $q = 1, \dots, \dimRep{\tau}$, let:
\begin{align}
    P^\tau_q F(u) = \dimRep{\tau} \int_\IsotropySubgroup{\CompactGroup}{\lambda} \conj{\tau_{qq}(m)} F(u m) \dd m,
    \quad F \in \Lebesgue{2}{\CompactGroup}.
\end{align}

By the Inverse Fourier Transform at $e \in \IsotropySubgroup{\CompactGroup}{\lambda}$,
if $F \in \Lebesgue{2}{\CompactGroup}$ and $u \in \CompactGroup$, then:
\begin{align}
    F(u)
    = \sum_{\tau \in \dualGroup{\IsotropySubgroup{\CompactGroup}{\lambda}}} \dimRep{\tau} \sum_{q = 1}^{\dimRep{\tau}} {\Fourier[\IsotropySubgroup{\CompactGroup}{\lambda}]{F(u \dummy)}}_{qq}
    = \sum_{\tau \in \dualGroup{\IsotropySubgroup{\CompactGroup}{\lambda}}} P^\tau_q F(u)
\end{align}

Now, write $\Hilbert{\tau}{q} = P^\tau_q \Lebesgue{2}{\CompactGroup}$.

\begin{lemma}
    The Hilbert spaces
    \begin{align*}
        \{\Hilbert{\tau}{q} : \tau \in \dualGroup{\IsotropySubgroup{\CompactGroup}{\lambda}}, q = 1, \dots, \dimRep{\tau} \}
    \end{align*}
    are mutually orthogonal.
    Therefore, we have
    \begin{align*}
        \Lebesgue{2}{K} = \bigoplus_{\tau \in \dualGroup{\IsotropySubgroup{\CompactGroup}{\lambda}}} \bigoplus_{q = 1}^{\dimRep{\tau}} \Hilbert{\tau}{q},
    \end{align*}
    where the above decomposition is orthogonal.
\end{lemma}

\begin{proposition}[Unitary dual]
\label{proposition:unitary_dual}
    Let $\lambda, \lambda' \in \dualGroup{\VectorSpace}$
    and $\tau \in \dualGroup{\IsotropySubgroup{\CompactGroup}{\lambda}}$,
    $\tau' \in \dualGroup{\IsotropySubgroup{\CompactGroup}{\lambda'}}$.
    The following properties hold:
    \begin{enumerate}
        \item $\Rep{\lambda}$ restricts to an irreducible unitary representation on each $\Hilbert{\tau}{q}$;
        \item $(\Hilbert{\tau}{q}, \Rep{\lambda})$ and $(\Hilbert{\tau'}{q'}, \Rep{\lambda'})$ are equivalent if and only if
            \begin{align*}
                \lambda' = k \lambda \quad \text{and} \quad \EquivalenceClass{\dualGroup{\IsotropySubgroup{\CompactGroup}{\lambda}}}{\tau} = \EquivalenceClass{\dualGroup{\IsotropySubgroup{\CompactGroup}{\lambda}}}{\tau'(k \dummy k^{-1})}
            \end{align*}
            for some $k \in \CompactGroup$.
    \end{enumerate}
\end{proposition}

\begin{definition}[Unitary dual]
\label{definition:unitary_dual}
\index{unitary dual}
    Fix $\lambda_0 \in \dualGroup{\VectorSpace}$.
    We define the \emph{unitary dual} of $\Group$, denoted by $\dualGroup{\Group}$, via:
    \begin{align*}
        \dualGroup{\Group} = \{ (\Hilbert{\tau}{1}, \Rep{\lambda}) : \lambda \in \LeftQuotient{\IsotropySubgroup{K}{\lambda_0}}{\dualGroup{\VectorSpace}}, \tau \in \dualGroup{\IsotropySubgroup{\CompactGroup}{\lambda}} \}.
    \end{align*}
\end{definition}

\subsection{Infinitesimal representations}

\begin{definition}[Infinitesimal Representation]
\label{definition:infinitesimal_representation}
\index{infinitesimal representation}
    Let $X \in \g$.
    We define the infinitesimal representation of $X$ as the operator
    \begin{align*}
        \Rep{\lambda}(X) : \SmoothFunctions{\CompactGroup} \to \SmoothFunctions{\CompactGroup}
    \end{align*}
    defined via
    \begin{align*}
        \Rep{\lambda}(X) F(u) = \left. \D{}{t} \right|_{t = 0} \Rep{\lambda}(\exp(t X)) F(u),
    \end{align*}
    where $F \in \SmoothFunctions{K}$.
\end{definition}

\begin{lemma}[Infinitesimal representation of $\Laplacian$]
\label{lemma:infinitesimal_representation_of_the_Laplacian}
    Let $\lambda \in \dualGroup{\VectorSpace}$.
    The infinitesimal representation of $\Laplacian$ is given by
    \begin{align*}
        \Rep{\lambda}(\Laplacian) = - \norm[\dualGroup{\VectorSpace}]{\lambda}^2 \Id{\Lebesgue{2}{\CompactGroup}} + \RightLaplacian[\CompactGroup].
    \end{align*}
\end{lemma}

\section{Fourier Transform}

\subsection{Definition and elementary properties}

\begin{definition}[Fourier transform]
\label{definition:Fourier_Transform}
\index{Fourier transform}
    Let $f \in \Lebesgue{1}{\Group}$ and $\lambda \in \dualGroup{\VectorSpace}$.
    We define its \emph{Fourier coefficient} at $\lambda$ via:
    \begin{align*}
        \Fourier{f}(\lambda) = \int_\Group f(g) \adj{\Rep{\lambda}(g)} \dd g.
    \end{align*}

    Moreover, the map
    \begin{align*}
        \Fourier{f} : \dualGroup{V} \to \End(\Lebesgue{2}{\CompactGroup}) :
        \lambda \mapsto \Fourier{f}(\lambda)
    \end{align*}
    is called the \emph{Fourier Transform} of $f$.
\end{definition}

\subsection{Plancherel formula}

\begin{proposition}[Plancherel formula]
\label{proposition:Plancherel_formula}
\index{Plancherel formula}
    Let $f \in \Lebesgue{1}{\Group} \cap \Lebesgue{2}{\Group}$.
    The following formula holds:
    \begin{align}
        \int_G \abs{f}^2 \dd g = \int_\dualGroup{\VectorSpace} \norm[\HilbertSchmidt{\Lebesgue{2}{\CompactGroup}}]{\Fourier{f}(\lambda)}^2 \dd \Plancherel{\VectorSpace}(\lambda).
        \label{proposition:Plancherel_formula:formula}
    \end{align}
\end{proposition}
\begin{proof}
    Let $\lambda \in \dualGroup{\VectorSpace}$ and $F \in \Lebesgue{2}{\CompactGroup}$.
    If $u \in \CompactGroup$, we can check that
    \begin{align*}
        \Fourier{f}(\lambda) F(u)
        &= \int_\VectorSpace \int_\CompactGroup f(x, k) \lambda(-u^{-1} k^{-1} x) F(k u) \dd k \dd x\\
        &= \int_\VectorSpace \int_\CompactGroup f(x, k) \conj{k u \lambda(x)} F(k u) \dd k \dd x,
    \end{align*}
    where we used $\lambda(-{(k u)}^{-1} x) = \conj{\lambda({(k u)}^{-1} x)} = \conj{(k u \lambda)(x)}$.

    Substituing $k$ for $k u^{-1}$, we get:
    \begin{align*}
        \Fourier{f}(\lambda) F(u)
        &= \int_\CompactGroup \Fourier[\VectorSpace]{f}(k \lambda, k u^{-1}) F(k) \dd k.
    \end{align*}

    Therefore, it follows by REFERENCE that:
    \begin{align*}
        \norm[\HilbertSchmidt{\Lebesgue{2}{\CompactGroup}}]{\Fourier{f}(\lambda)}^2
        &= \int_\CompactGroup \int_\CompactGroup \abs{\Fourier[\VectorSpace]{f}(k \lambda, k u^{-1})}^2 \dd u \dd k.
    \end{align*}

    Now, integrating with respect to $\lambda$, we obtain:
    \begin{align*}
        \int_\dualGroup{\VectorSpace} \norm[\HilbertSchmidt{\Lebesgue{2}{\CompactGroup}}]{\Fourier{f}(\lambda)}^2 \dd \Plancherel{\VectorSpace}(\lambda)
        &= \int_\dualGroup{\VectorSpace} \int_\CompactGroup \abs{\Fourier[\VectorSpace]{f}(\lambda, k)}^2 \dd k \dd \Plancherel{\VectorSpace}(\lambda)\\
        &= \int_\VectorSpace \int_\CompactGroup \abs{f(x, k)}^2 \dd u \dd k,
    \end{align*}
    where the last line was obtained by applying the Plancherel formula on $\VectorSpace$.
\end{proof}

\begin{proposition}[Inverse Fourier Transform]
\label{proposition:inverse_Fourier_Transform}
\index{inverse Fourier Transform}
    Let $\phi \in \Schwartz{\Group}$.
    For each $g \in \Group$,
    we have
    \begin{align*}
        \phi(g)
        = \int_\dualGroup{\VectorSpace}
        \tr \left( \Rep{\lambda}(g) \Fourier \phi(\lambda) \right) \dd \Plancherel{\VectorSpace}(\lambda).
    \end{align*}
\end{proposition}

\subsection{Sobolev spaces}

\subsection{Fourier Transform of distributions}
