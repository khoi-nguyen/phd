\chapter{Generalised Motion Groups}
\label{chapter:generalised_motion_groups}

\section{Definitions}
\label{section:definitions}

\begin{definition}[semi-direct product]
\label{definition:semi-direct_products}
\index{semi-direct product}
    Let $N$ be a group and $H$ be a subgroup of $\Aut(N)$.
    The \emph{semi-direct product} $N \rtimes H$ is the group whose elements are that of $N \times H$ with the group law:
    \begin{align}
        (n_1, h_1) (n_2, h_2) = (n_1 h_1(n_2), h_1 h_2), \quad (n_1, h_1), (n_2, h_2) \in N \times H.
    \end{align}

    Given $(n, h)$ in $N \rtimes H$, its \emph{inverse} is given by $({h_1}^{-1}({n_1}^{-1}), {h_1}^{-1})$,
    while $(e_N, e_H)$ is the \emph{identity element} of the group.
\end{definition}

\begin{definition}[generalised motion group]
\label{definition:generalised_motion_group}
\index{generalised motion group}
    Let $\Group$ be a group.
    We shall say that $\Group$ is a \emph{generalised motion group}
    if there exists a vector space $\VectorSpace$ and a compact group $\CompactGroup \subset \Aut(\VectorSpace)$ such that $\Group = \VectorSpace \rtimes \CompactGroup$.
\end{definition}

From now on, unless stated otherwise,
$\Group$ will denote a generalised motion group,
with $\VectorSpace$ its underlying vector space and $\CompactGroup$ its associated compact group.

\section{Unitary representations}
\label{section:unitary_representations}

\begin{definition}
\label{definition:reducible_representation}
    Let $\rho \in \dualGroup{V}$.
    We define a unitary representation $\Rep{\rho} \in \Hom(\Group, \End(\Lebesgue{2}{\CompactGroup}))$ of $\Group$ via:
    \begin{align}
        \Rep{\rho} (x, k) F(u) = \rho(u^{-1} x) F(k^{-1} u),
    \end{align}
    where $(x, k) \in \CompactGroup$, $F \in \Lebesgue{2}{\CompactGroup}$ and $u \in \CompactGroup$.
\end{definition}

The above representation is unfortunately reducible.
However, as we shall see later, the Fourier Transform on $G$ can be written exclusively with those representations.

\section{Fourier Transform}
\label{section:Fourier_transform}

\begin{definition}[Fourier transform]
\label{definition:Fourier_Transform}
\index{Fourier transform}
    Let $f \in \Lebesgue{1}{\Group}$ and $\rho \in \dualGroup{V}$.
    We define its \emph{Fourier coefficient} at $\rho$ via:
    \begin{align*}
        \Fourier{f}(\rho) = \int_\Group f(g) \adj{\Rep{\rho}(g)} \dd g.
    \end{align*}

    Moreover, the map
    \begin{align*}
        \Fourier{f} : \dualGroup{V} \to \End(\Lebesgue{2}{\CompactGroup}) :
        \rho \mapsto \Fourier{f}(\rho)
    \end{align*}
    is called the \emph{Fourier Transform} of $f$.
\end{definition}
