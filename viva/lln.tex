\documentclass{beamer}
%\documentclass[handout]{beamer}

\usepackage{amsmath, amsthm, amssymb, amsfonts}
\usepackage{mathtools}
\usepackage{xparse}
\usepackage{color}

\usepackage{../macros}
\DeclareMathOperator{\order}{order}

\usepackage[style=alphabetic]{biblatex}
\bibliography{../Bibliography.bib}

\title{Pseudo-Differential Calculus on Generalized Motion Groups}
\author{Binh-Khoi Nguyen}

\usetheme{Warsaw}

\begin{document}

\maketitle

\section{Differential operators}

\renewcommand \Group {\R^n}

\begin{frame}
    {Differential operators}

    Suppose that $T$ is a \emph{differential operator}
    \begin{align*}
        T =
        \sum_{\abs \alpha \leq N}
        c_\alpha(x)
        \iD {x^\alpha},
        \quad
        c_\alpha \in \SmoothFunctions {\R^n}
        \text{ bdd },
    \end{align*}
    where $c_\alpha \in \SmoothFunctions {\R^n}$ is \emph{bounded} and \emph{smooth}.

    \pause

    Denote by
    \begin{align*}
        \order(T)
    \end{align*}
    the \emph{order} of the differential operator.
\end{frame}

\begin{frame}
    {Order of differential operators}

    Suppose that $T_i, T$ are differential operators
    of order $m_i, m$
    with smooth bounded coefficients.

    \begin{enumerate}
        \item $T_1 \circ T_2$ has order $m_1 + m_2$;
            \pause
        \item For each $n \in \N$, $T^n$ has order $n m$.
            \pause
        \item $\adj T$ has order m;
            \pause
        \item $T$ is bounded between
            \begin{align*}
                \Sobolev k \to \Sobolev {k - m},
                \quad k = m, m + 1, \dots
            \end{align*}
    \end{enumerate}
\end{frame}

\begin{frame}
    {Motivation}

    If the operator
    \begin{align*}
        {\Laplacian [\R^n]}^\frac s 2
    \end{align*}
    "has order" $s \in \R$,
    \pause
    then the equation
    \begin{align*}
        {\Laplacian [\R^n]}^\frac s 2 u = f
    \end{align*}
    satisfies the \emph{subelliptic estimate}
    \begin{align*}
        \norm [\Sobolev {k + s}] u
        \leq C \norm [\Sobolev k] f.
    \end{align*}
    \pause
    \emph{Proof:} $u = {\Laplacian [\R^n]}^\frac {-s} 2 f$.
\end{frame}

\begin{frame}
    {Motivation (continued)}

    An adequate generalisation the notion of order to a larger algebra of operators allows us to study the regularity of solutions.
\end{frame}

\begin{frame}
    {Characteristic polynomials}

    \begin{definition}[Characteristic polynomials]
        Suppose that
        \begin{align*}
            T =
            \sum_{\abs \alpha \leq N}
            c_\alpha(x)
            \iD{x^\alpha}.
        \end{align*}

        We define the \emph{characteristic polynomial} 
        associated with $T$ via
        \begin{align*}
            \sigma_T(x, \xi) \defeq
            \sum_{\abs \alpha \leq N}
            c_\alpha(x)
            (\i \turn \xi)^\alpha
        \end{align*}
    \end{definition}

    \pause
    Essentially, we just replaced $\partial$ by $\i \turn \xi$.
\end{frame}

\end{document}
