\documentclass{beamer}
%\documentclass[handout]{beamer}

\usepackage{amsmath, amsthm, amssymb, amsfonts}
\usepackage{mathtools}
\usepackage{xparse}
\usepackage{color}

\usepackage{../macros}
\DeclareMathOperator{\order}{order}

\usepackage[style=alphabetic]{biblatex}
\bibliography{../Bibliography.bib}

\title{Pseudo-Differential Calculus on Generalized Motion Groups}
\author{Binh-Khoi Nguyen}

\usetheme{Warsaw}

\begin{document}

\maketitle

\section{Differential operators}

\renewcommand \Group {\R^n}

\begin{frame}
    {Differential operators}

    Suppose that $T$ is a \emph{differential operator}
    \begin{align*}
        T =
        \sum_{\abs \alpha \leq N}
        c_\alpha(x)
        \iD {x^\alpha},
        \quad
        c_\alpha \in \SmoothFunctions {\R^n}
        \text{ bdd },
    \end{align*}
    where $c_\alpha \in \SmoothFunctions {\R^n}$ is \emph{bounded} and \emph{smooth}.

    \pause

    Denote by
    \begin{align*}
        \order(T)
    \end{align*}
    the \emph{order} of the differential operator.
\end{frame}

\begin{frame}
    {Order of differential operators}

    Suppose that $T_i, T$ are differential operators
    of order $m_i, m$
    with smooth bounded coefficients.

    \begin{enumerate}
        \item $T_1 \circ T_2$ has order $m_1 + m_2$;
            \pause
        \item For each $n \in \N$, $T^n$ has order $n m$.
            \pause
        \item $\adj T$ has order m;
            \pause
        \item $T$ is bounded between
            \begin{align*}
                \Sobolev k \to \Sobolev {k - m},
                \quad k = m, m + 1, \dots
            \end{align*}
    \end{enumerate}
\end{frame}

\begin{frame}
    {Motivation}

    If the operator
    \begin{align*}
        {\Laplacian [\R^n]}^\frac s 2
    \end{align*}
    "has order" $s \in \R$,
    \pause
    then the equation
    \begin{align*}
        {\Laplacian [\R^n]}^\frac s 2 u = f
    \end{align*}
    satisfies the \emph{subelliptic estimate}
    \begin{align*}
        \norm [\Sobolev {k + s}] u
        \leq C \norm [\Sobolev k] f.
    \end{align*}
    \pause
    \emph{Proof:} $u = {\Laplacian [\R^n]}^\frac {-s} 2 f$.
\end{frame}

\begin{frame}
    {Motivation (continued)}

    An adequate generalisation the notion of order to a larger algebra of operators allows us to study the regularity of solutions.
\end{frame}

\begin{frame}
    {Characteristic polynomials}

    \begin{definition}[Characteristic polynomials]
        Suppose that
        \begin{align*}
            T =
            \sum_{\abs \alpha \leq N}
            c_\alpha(x)
            \iD{x^\alpha}.
        \end{align*}

        We define the \emph{characteristic polynomial} 
        associated with $T$ via
        \begin{align*}
            \sigma_T(x, \xi) \defeq
            \sum_{\abs \alpha \leq N}
            c_\alpha(x)
            (\i \turn \xi)^\alpha
        \end{align*}
    \end{definition}

    \pause
    Essentially, we just replaced $\partial$ by $\i \turn \xi$.
\end{frame}

\begin{frame}
    {Characteristic polynomials (continued)}

    \begin{enumerate}
        \item
            The order of $T$ can be obtained by studying the \emph{growth} of $\sigma_T$.
            \pause
        \item
            The composition of two differential operators arise from the $\chi$-polynomial
            \begin{align}
                \sigma(x, \xi)
                \defeq
                \sum_{\alpha \in \N}
                \frac 1 {\alpha!}
                \iD {\xi^\alpha} \sigma_1(x, \xi)
                \iD {x^\alpha} \sigma_2(x, \xi).
                \label{eq:Euclidean_composition_formula}
            \end{align}
            \pause
        \item
            The adjoint of a differential operator arises from the $\chi$-polynomial
            \begin{align}
                \sigma'(x, \xi)
                \defeq
                \sum_{\alpha \in \N}
                \frac 1 {\alpha!}
                \iD {\xi^\alpha}
                \iD {x^\alpha} \sigma(x, \xi).
                \label{eq:Euclidean_adjunction_formula}
            \end{align}
    \end{enumerate}
\end{frame}

\begin{frame}
    {Symbol}

    Each differential operator $T$ has a \emph{Schwartz kernel} $\kappa_T$, i.e.\
    \begin{align*}
        T \phi(x) = (\conv {\phi} {\kappa_T(x, \dummy)})(x).
    \end{align*}

    \pause
    Note that we have
    \begin{align*}
        \sigma_T(x, \xi)
        = \Fourier \{\kappa_T(x, \dummy)\}(\xi),
    \end{align*}
    \pause
    from which it follows that
    \begin{align*}
        T \phi(x) =
        \int_{\Group} \e^{i \turn \ip x \xi} \sigma(x, \xi) \Fourier f(\xi) \dd \xi.
    \end{align*}
\end{frame}

\begin{frame}
    {Symbol classes}

    \begin{definition}
        [Symbol classes]

        Let $1 \geq \rho > \delta \geq 0$.
        We shall say that $\sigma \in \SmoothFunctions {\R^n \times \R^n}$ is a \emph{symbol of order $m \in \R$ and type $(\rho, \delta)$}
        if there exists $C \geq 0$ such that
        \begin{align*}
            \abs {%
                \iD {\xi^\alpha}
                \iD {x^\beta}
                \sigma(x, \xi)
            } \leq
            C (1 + \norm [\R^n] \xi)^{m - \rho \abs \alpha + \delta \abs \beta}.
        \end{align*}

        \pause
        Notation: $\SymbolClass m {\rho, \delta}$.
    \end{definition}
\end{frame}

\begin{frame}
    {Symbols}

    The condition on $(\rho, \delta)$ is justified by the fact that we want the terms in
    \begin{align}
        \sigma(x, \xi)
        &\defeq
        \sum_{\alpha \in \N}
        \frac 1 {\alpha!}
        \iD {\xi^\alpha} \sigma_1(x, \xi)
        \iD {x^\alpha} \sigma_2(x, \xi),\\
        \sigma'(x, \xi)
        &\defeq
        \sum_{\alpha \in \N}
        \frac 1 {\alpha!}
        \iD {\xi^\alpha}
        \iD {x^\alpha} \sigma(x, \xi).
    \end{align}
    to have ''lower order`` as $\abs \alpha$ increases.
\end{frame}

\begin{frame}
    {Pseudo-differential operator}

    \begin{definition}
        [Pseudo-differential operator]

        Let $1 \geq \rho > \delta \geq 0$.
        A linear operator $T$ is a \emph{pseudo-differential operator of order $m$ and type $(\rho, \delta)$}
        if there exists $\sigma \in \SymbolClass m {\rho, \delta}$ such that
        \begin{align*}
            T \phi(x)
            = \int_{\R^n} \e^{\i \turn \ip x \xi} \sigma(x, \xi) \Fourier \phi(\xi) \dd \xi
        \end{align*}
        for all $\phi \in \Schwartz \Group$ and $x \in \R^n$.

        \pause
        Notation: $\OperatorClass m {\rho, \delta}$.
    \end{definition}
\end{frame}

\end{document}
