\documentclass{beamer}
%\documentclass[handout]{beamer}

\usepackage{amsmath, amsthm, amssymb, amsfonts}
\usepackage{mathtools}
\usepackage{xparse}
\usepackage{color}

\usepackage{../macros}

\usepackage[style=alphabetic]{biblatex}
\bibliography{../Bibliography.bib}

\title{Pseudo-Differential Calculus on Generalized Motion Groups}
\author{Binh-Khoi Nguyen}

\usetheme{Warsaw}

\begin{document}

\maketitle

\section{Introduction}

\renewcommand \Group {\R^n}

\begin{frame}
    {Differential operators}

    Suppose that $T$ is a \emph{differential operator}
    \begin{align*}
        T =
        \sum_{\abs \alpha \leq N}
        c_\alpha(x)
        \iD {x^\alpha},
        \quad
        c_\alpha \in \SmoothFunctions {\R^n},
    \end{align*}
    where $c_\alpha \in \SmoothFunctions {\R^n}$ is \emph{smooth} with \emph{bounded derivatives}.

    \pause

    Denote by
    \begin{align*}
        \mathrm{order}(T)
    \end{align*}
    the \emph{order} of the differential operator.
\end{frame}

\begin{frame}
    {Order of differential operators}

    Suppose that $T_i, T$ are differential operators
    of order $m_i, m$
    with smooth bounded coefficients.

    \begin{enumerate}
        \item $T_1 \circ T_2$ has order $m_1 + m_2$;
            \pause
        \item For each $n \in \N$, $T^n$ has order $n m$.
            \pause
        \item $\adj T$ has order m;
            \pause
        \item $T$ is bounded between
            \begin{align*}
                \Sobolev k \to \Sobolev {k - m},
                \quad k = m, m + 1, \dots
            \end{align*}
    \end{enumerate}
\end{frame}

\begin{frame}
    {Motivation}

    If the operator
    \begin{align*}
        {\Laplacian [\R^n]}^\frac s 2
    \end{align*}
    "has order" $s \in \R$,
    \pause
    then the equation
    \begin{align*}
        {\Laplacian [\R^n]}^\frac s 2 u = f
    \end{align*}
    satisfies the \emph{subelliptic estimate}
    \begin{align*}
        \norm [\Sobolev {k + s}] u
        \leq C \norm [\Sobolev k] f.
    \end{align*}
    \pause
    \emph{Proof:} $u = {\Laplacian [\R^n]}^\frac {-s} 2 f$.
\end{frame}

\begin{frame}
    {Motivation (continued)}

    An adequate generalization of the notion of order to a larger algebra of operators allows us to study the regularity of solutions.
\end{frame}

\begin{frame}
    {Characteristic polynomials}

    \begin{definition}[Characteristic polynomials]
        Suppose that
        \begin{align*}
            T =
            \sum_{\abs \alpha \leq N}
            c_\alpha(x)
            \iD{x^\alpha}.
        \end{align*}

        We define the \emph{characteristic polynomial}
        associated with $T$ via
        \begin{align*}
            \sigma_T(x, \xi) \defeq
            \sum_{\abs \alpha \leq N}
            c_\alpha(x)
            (\i \turn \xi)^\alpha
        \end{align*}
    \end{definition}

    \pause
    Essentially, we just replaced $\partial$ by $\i \turn \xi$.
\end{frame}

\begin{frame}
    {Characteristic polynomials (continued)}

    \begin{enumerate}
        \item
            The order of $T$ can be obtained by studying the \emph{growth} of $\sigma_T$.
            \pause
        \item
            The composition of two differential operators arises from the $\chi$-polynomial
            \begin{align}
                \sigma(x, \xi)
                \defeq
                \sum_{\alpha \in \N}
                \frac {(i \turn)^{-\abs \alpha}} {\alpha!}
                \iD {\xi^\alpha} \sigma_1(x, \xi)
                \iD {x^\alpha} \sigma_2(x, \xi).
                \label{eq:Euclidean_composition_formula}
            \end{align}
            \pause
        \item
            The adjoint of a differential operator arises from the $\chi$-polynomial
            \begin{align}
                \sigma'(x, \xi)
                \defeq
                \sum_{\alpha \in \N}
                \frac {(i \turn)^{-\abs \alpha}} {\alpha!}
                \iD {\xi^\alpha}
                \iD {x^\alpha} \adj \sigma(x, \xi).
                \label{eq:Euclidean_adjunction_formula}
            \end{align}
    \end{enumerate}
\end{frame}

\begin{frame}
    {Symbol}

    Each differential operator $T$ has a \emph{Schwartz kernel} $\kappa_T$, i.e.\
    \begin{align*}
        T \phi(x) = (\conv {\phi} {\kappa_T(x, \dummy)})(x).
    \end{align*}

    \pause
    Note that we have
    \begin{align*}
        \sigma_T(x, \xi)
        = \Fourier \{\kappa_T(x, \dummy)\}(\xi),
    \end{align*}
    \pause
    from which it follows that
    \begin{align*}
        T \phi(x) =
        \int_{\Group} \e^{i \turn \ip x \xi} \sigma(x, \xi) \Fourier f(\xi) \dd \xi.
    \end{align*}
\end{frame}

\section{Symbol classes}

\begin{frame}
    {Symbol classes}

    \begin{definition}
        [Symbol classes]

        Let $1 \geq \rho > \delta \geq 0$.
        We shall say that $\sigma \in \SmoothFunctions {\R^n \times \R^n}$ is a \emph{symbol of order $m \in \R$ and type $(\rho, \delta)$}
        if for every $\alpha, \beta \in \N^n$
        there exists $C \geq 0$ such that
        \begin{align*}
            \abs {%
                \iD {\xi^\alpha}
                \iD {x^\beta}
                \sigma(x, \xi)
            } \leq
            C (1 + \norm [\R^n] \xi)^{m - \rho \abs \alpha + \delta \abs \beta}
        \end{align*}
        holds for every $x, \xi \in \R^n$.

        \pause
        Notation: $\SymbolClass m {\rho, \delta}$.
    \end{definition}

    \pause
    Smoothing symbols: $\SmoothingSymbols \defeq \bigcap_{m \in \R} \SymbolClass m {\rho, \delta}$.
\end{frame}

\begin{frame}
    {Symbols}

    The condition on $(\rho, \delta)$ is justified by the fact that we want the terms in
    \begin{align*}
        &\sum_{\alpha \in \N}
        \frac {(i \turn)^{-\abs \alpha}} {\alpha!}
        \iD {\xi^\alpha} \sigma_1(x, \xi)
        \iD {x^\alpha} \sigma_2(x, \xi)\\
        &\sum_{\alpha \in \N}
        \frac {(i \turn)^{-\abs \alpha}} {\alpha!}
        \iD {\xi^\alpha}
        \iD {x^\alpha} \sigma(x, \xi)
    \end{align*}
    to have ''lower order`` as $\abs \alpha$ increases.
\end{frame}

\begin{frame}
    {Pseudo-differential operators}

    \begin{definition}
        [Pseudo-differential operators]

        Let $1 \geq \rho > \delta \geq 0$.
        A linear operator $T$ is a \emph{pseudo-differential operator of order $m$ and type $(\rho, \delta)$}
        if there exists $\sigma \in \SymbolClass m {\rho, \delta}$ such that
        \begin{align*}
            T \phi(x)
            = \int_{\R^n} \e^{\i \turn \ip x \xi} \sigma(x, \xi) \Fourier \phi(\xi) \dd \xi
        \end{align*}
        for all $\phi \in \Schwartz \Group$ and $x \in \R^n$.

        \pause
        Notation: $\OperatorClass m {\rho, \delta}$.
    \end{definition}
    \pause
    Smoothing operators: $\SmoothingOperators$.
\end{frame}

\begin{frame}
    {Examples of pseudo-differential operators}

    \begin{itemize}
        \item Differential operators with smooth coefficients and bounded derivatives.
        \item $T = \BesselPotential s$, $\sigma(\xi) = (1 + 4 \pi^2 \norm [\R^n] {\turn \xi}^2)^{\frac s 2}$.
    \end{itemize}
\end{frame}

\begin{frame}
    {Asymptotic convergence}

    Asymptotic convergence = convergence modulo smoothing symbols/operators.
    \pause
    \begin{theorem}[Asymptotic convergence]
        Let $m_j$ be a strictly decreasing sequence of real numbers.

        Suppose that for each $j \in \N$, $\sigma_j \in \SymbolClass {m_j} {\rho, \delta}$.
        There exists $\sigma \in \SymbolClass {m_0} {\rho, \delta}$ such that \begin{align*}
            \sigma - \sum_{j = 0}^{N} \sigma_j \in \SymbolClass {m_{N + 1}} {\rho, \delta}
        \end{align*}
        for each $N \in \N$,
        and $\sigma$ is unique modulo $\SmoothingSymbols$.
    \end{theorem}

    \pause
    Notation: $\sigma \sim \sum_{j \in \N} \sigma_j$
\end{frame}

\begin{frame}
    {Sobolev spaces}

    \begin{definition}[Sobolev spaces]
        Let $s \in \R$.
        The set $\Sobolev s$ is the set of $f \in \TemperedDistributions \Group$ such that
        \begin{align*}
            \norm [\Sobolev s] f^2
            \defeq \int_{\R^n} \abs{(1 + \norm [\R^n] \xi^2)^\frac s 2 \Fourier f(\xi)}^2 \dd \xi.
        \end{align*}
    \end{definition}
\end{frame}

\section{Main results}

\begin{frame}
    {Composition formula}

    The composition of pseudo-differential operators can be written at the symbolic level in terms of an asymptotic sum.

    \begin{theorem}
        [Composition formula]

        Let $\sigma_i \in \SymbolClass {m_i} {\rho, \delta}$, $i = 1, 2$,
        be a symbol associated with $T_i \in \OperatorClass {m_i} {\rho, \delta}$.
        Then $T_1 \circ T_2 \in \OperatorClass {m_1 + m_2} {\rho, \delta}$ and its symbol $\sigma$ is an asymptotic limit of
        \begin{align*}
            \sum_{\alpha \in \N} \frac {(\i \turn)^{-\abs \alpha}} {\alpha!}
            \iD {\xi^\alpha} \sigma_1(x, \xi)
            \iD {x^\alpha} \sigma_2(x, \xi)
        \end{align*}
    \end{theorem}
\end{frame}

\begin{frame}
    {Adjoint formula}

    \begin{theorem}
        [Adjoint formula]

        Let $\sigma \in \SymbolClass {m} {\rho, \delta}$
        be a symbol associated with $T \in \OperatorClass {m} {\rho, \delta}$.
        Then $\adj T \in \OperatorClass {m} {\rho, \delta}$ and its symbol $\sigma^{(*)}$ is an asymptotic limit of
        \begin{align*}
            \sum_{\alpha \in \N} \frac {(\i \turn)^{-\abs \alpha}} {\alpha!}
            \iD {\xi^\alpha} \iD {x^\alpha} \adj \sigma(x, \xi)
        \end{align*}
    \end{theorem}
\end{frame}

\begin{frame}
    {L2 boundedness}

    \begin{theorem}
        If $T \in \OperatorClass m {\rho, \delta}$,
        then for each $s \in \R$,
        $T$ extends to a bounded operator
        \begin{align*}
            T : \Sobolev s \to \Sobolev {s - m}.
        \end{align*}
    \end{theorem}
\end{frame}

\section{Application to PDEs}

\begin{frame}
    {Ellipticity}

    \begin{definition}
        [Ellipticity]

        Let $\sigma \in \SymbolClass m {\rho, \delta}$.
        We shall say that $\sigma$ is \emph{elliptic} if there exists $c, C \geq 0$ such that
        \begin{align*}
            \abs {\sigma(x, \xi)}
            \geq C \norm [\R^n] \xi^m
        \end{align*}
        for each $x, \xi \in \R^n$ satisfying $\norm [\R^n] \xi \geq c$.
    \end{definition}
\end{frame}

\begin{frame}
    {Parametrices}

    \begin{theorem}
        [Existence of a parametrix]

        Let $T \in \OperatorClass m {\rho, \delta}$.
        There exists $S \in \OperatorClass {-m} {\rho, \delta}$ such that
        \begin{align*}
            S \circ T = \Id {\Schwartz \Group}
            \quad
            \text{mod } \
            \SmoothingOperators
        \end{align*}
        if and only if the symbol associated with $T$ is elliptic.
    \end{theorem}
\end{frame}

\begin{frame}
    {Existence of a parametrix: sketch proof}

    \pause
    Denote by $\sigma$ the symbol of T.
    \begin{enumerate}
        \item Take $S_1 \defeq \Op(1 / \sigma) \in \OperatorClass {-m} {\rho, \delta}$ so that
            \begin{align*}
                S_1 \circ T = \Id {\Schwartz \Group} - E_1,
                \quad E_1 \in \OperatorClass {-(\rho - \delta)} {\rho, \delta}
            \end{align*}
            \pause
        \item Take $S_2 \defeq \Op(e_1/\sigma)$ so that
            \begin{align*}
                (S_1 + S_2) \circ T = \Id {\Schwartz \Group} - E_2,
                \quad E_2 \in \OperatorClass {-2(\rho - \delta)} {\rho, \delta}
            \end{align*}
            \pause
        \item Define $S \sim \sum_{j \in \N} S_j$ and check that
            \begin{align*}
                S \circ T = \Id {\Schwartz \Group} - E,
                \quad E \in \SmoothingOperators.
            \end{align*}
    \end{enumerate}
\end{frame}

\begin{frame}
    {Subelliptic estimates}

    \begin{theorem}
        Let $T \in \OperatorClass m {\rho, \delta}$ be elliptic.
        If $T \phi = f$,
        where $f \in \Sobolev s$ for some $s \in \R$,
        then $\phi \in \Sobolev {s + m}$.

        If $(\rho, \delta) = (1, 0)$,
        we may replace $2$ by any $1 < p < + \infty$.
    \end{theorem}
\end{frame}

\section{On Lie groups}

\renewcommand{\Group}{G}

\begin{frame}
    {Motivations}

    \pause
    \begin{enumerate}
        \item
            Pseudo-differential operators can be defined on manifolds (via local coordinates)
            provided
            \begin{align*}
                1 - \rho < \delta < \rho.
            \end{align*}
            \pause
        \item
            On manifolds, symbols are only defined modulo l.o.t (principal symbols).
    \end{enumerate}

    On Lie groups,
    can we use the group Fourier transform to define pseudo-differential operators?
\end{frame}

\begin{frame}
    {Timeline}

    \begin{itemize}
        \pause
        \item 2010: Compact Lie groups (Ruzhansky, Turunen)
        \pause
        \item 2015: Graded Lie groups (Ruzhansky, Fischer)
        \pause
        \item My PhD: Euclidean motion groups
    \end{itemize}
\end{frame}

\begin{frame}
    {Fourier transform on Lie groups}

    \begin{enumerate}
        \item Replace Lebesgue measure by Haar measure.
            \pause
        \item Replace complex exponentials by a family $\dualGroup \Group$ of group morphisms
            \begin{align*}
                \xi : \Group \to \End(\Hil),
            \end{align*}
            where $\Hil$ is a Hilbert space.
    \end{enumerate}
    \pause
    If $\phi \in \Schwartz \Group$,
    then
    \begin{align*}
        \Fourier \phi(\xi) \defeq \int_\Group \phi(g) \adj \xi(g) \dd g
    \end{align*}
\end{frame}

\begin{frame}
    {Plancherel theory}

    \begin{theorem}
        [Plancherel theorem]

        Let $f \in \Lebesgue 1 \Group \cap \Lebesgue 2 \Group$.
        There exists a measure $\mu$ on $\dualGroup \Group$ such that
        \begin{align*}
            \int_\Group \abs f^2 \dd g
            = \int_{\dualGroup \Group} \tr(\Fourier f(\xi) \adj{\Fourier f(\xi)}) \dd \mu(\xi).
        \end{align*}
    \end{theorem}

    \pause
    \begin{theorem}
        [Inverse formula]

        Let $f \in C^\infty_c(\Group)$.
        We have
        \begin{align*}
            f(g)
            = \int_{\dualGroup \Group} \tr(\xi(g) \Fourier f(\xi)) \dd \mu(\xi).
        \end{align*}
        for each $g \in \Group$.
    \end{theorem}
\end{frame}

\begin{frame}
    {Infinitesimal representations}

    Let $\LeftDifferentialOperatorFirstOrder {X_1}, \dots \LeftDifferentialOperatorFirstOrder {X_{\dim \Group}}$ be a basis of left-invariant vector fields on $\Group$.

    \begin{align*}
        \xi(X_i) \defeq \eval {\D*{1}{t} \xi(\exp(t X_i))}{t = 0}
    \end{align*}

    \pause
    As in the Euclidean case,
    \begin{align*}
        \Fourier \{X_i f\} = \xi(X_i) \Fourier f(\xi).
    \end{align*}

    \pause
    Example: $\Group = \R^n$
    \begin{align*}
        \Fourier [\R^n] \{\BesselPotentialSquared [\R^n] {} f\}(\xi)
        = (1 + 4 \pi^2 \norm [\R^n] \xi^2)
        \Fourier [\R^n] f(\xi)
    \end{align*}
\end{frame}

\begin{frame}
    {Symbol classes revisited}

    The symbol classes definition could be rewritten as:

    \begin{align*}
        \abs {%
            \xi
            \BesselPotential {-m + \rho \abs \alpha - \delta \abs \beta}
            \iD {\xi^\alpha}
            \iD {x^\beta}
            \sigma(x, \xi)
        } \leq
        C_{\alpha, \beta}
        < \infty.
    \end{align*}

    \pause
    \begin{itemize}
        \item $\sigma$ is a Fourier transform,
            so will be operator valued in the general case.
            \pause
        \item $\iD {x^\beta}$ can be replaced by $\LeftDifferentialOperator \beta$ for the non-Euclidean case.
            \pause
        \item What about the derivatives on $\xi \in \dualGroup \Group$?
    \end{itemize}
\end{frame}

\begin{frame}
    {Derivatives in $\xi$}

    Observe that
    \begin{align*}
        {\color{red} \iD {\xi^\alpha}} \Fourier [\R^n] \kappa(\xi)
        = C_\alpha \Fourier [\R^n] {\{{\color{red} x^\alpha} \kappa\}}(\xi)
    \end{align*}

    \pause
    We choose a finite family $\Delta \subset \SmoothFunctions \Group$ such that
    \begin{itemize}
        \pause
        \item
            We can do Taylor developments
            \pause
        \item The functions vanish simultaneously only at the identity
            \pause
        \item
            As close to a Leibniz rule as possible
    \end{itemize}
\end{frame}

\begin{frame}
    {Difference operators}

    \begin{definition}
        [Difference operators]

        If $\Delta = \{q_1, \dots, q_{\dimDifferenceOperators} \}$ and $\alpha \in \N^{\dimDifferenceOperators}$,
        we let
        \begin{align*}
            \DifferenceOperator \alpha \Fourier f(\xi)
            = \Fourier \{q^\alpha(\dummy^{-1}) f\}(\xi)
        \end{align*}
        for each $f \in C^\infty_c(\Group)$,
        $\xi \in \dualGroup \Group$.
    \end{definition}
\end{frame}

\begin{frame}{Symbol classes}
    \begin{definition}[Symbol classes]
        A map $\sigma$ is in $\SymbolClass m {\rho, \delta}$ if
        \begin{enumerate}
            \item
                $\sigma$ is smooth.
            \item
                For each $\alpha \in \N^M$ and each $\beta \in \N^{\dim \Group}$,
                \begin{align*}
                    \norm [op] {
                        \xi \BesselPotential {-m + \rho \abs \alpha - \delta \abs \beta}
                        \LeftDifferentialOperator \beta \DifferenceOperatorOrder \alpha \sigma(g, \lambda)
                    } < \infty
                \end{align*}
                uniformly in $g \in \Group$, $\xi \in \dualGroup \Group$.
        \end{enumerate}
    \end{definition}
\end{frame}

\begin{frame}{Pseudo-differential operators}
    If $\sigma \in \SymbolClass m {\rho, \delta}$,
    we let
    \begin{align*}
        \Op(\sigma) \phi(g) = \InverseFourier \left\{ \sigma(g, \dummy) \Fourier \phi \right\}(g).
    \end{align*}

    $\Op(\sigma)$ is the \emph{pseudo-differential operator} associated with $\sigma$.

    We write
    \begin{align*}
        \OperatorClass m {\rho, \delta} \defeq \Op \left(\SymbolClass m {\rho, \delta}\right).
    \end{align*}

    \pause

    The map
    \begin{align*}
        g \in \Group \mapsto \kappa_g \defeq \InverseFourier \{\sigma(g, \dummy)\}
    \end{align*}
    is called the \emph{kernel} associated with $\sigma$.
\end{frame}

\begin{frame}{Quantization}
    \begin{block}{Proposition}
        Let $\sigma \in \SymbolClass m {\rho, \delta}$.
        For every $\phi \in \Schwartz \Group$,
        we have
        \begin{align*}
            \Op(\sigma) \phi(g) = (\conv \phi {\kappa_g})(g)
        \end{align*}
        for every $g \in \Group$.
    \end{block}
\end{frame}

\section{Main results}

\begin{frame}{Composition formula}
    \begin{theorem}[Composition formula]
        Let $\sigma_i \in \SymbolClass {m_i} {\rho, \delta}$, $i = 1, 2$ with $\rho > \delta$.
        There exists $\sigma \in \SymbolClass {m_1 + m_2} {\rho, \delta}$ such that
        \begin{align*}
            \Op(\sigma) = \Op(\sigma_1) \circ \Op(\sigma_2).
        \end{align*}

        Moreover,
        for each $N \in \N$,
        \begin{align*}
            \sigma - \sum_{\abs \alpha \leq N} \frac 1 {\alpha!} \DifferenceOperatorOrder {\alpha} \sigma_1 \ \LeftDifferentialOperator {\alpha} \sigma_2 \in \SymbolClass {m - (\rho - \delta) (N + 1)} {\rho, \delta}
        \end{align*}
        with an adapted basis of the Lie algebra.
    \end{theorem}
\end{frame}

\begin{frame}{Adjunction formula}
    \begin{theorem}[Adjunction formula]
        Let $\sigma \in \SymbolClass {m} {\rho, \delta}$ with $\rho > \delta$.
        There exists $\sigma^{(*)} \in \SymbolClass {m} {\rho, \delta}$ such that
        \begin{align*}
            \Op(\sigma^{(*)}) = \adj {\Op(\sigma)}.
        \end{align*}

        Moreover,
        for each $N \in \N$,
        \begin{align*}
            \sigma^{(*)} - \sum_{\abs \alpha \leq N} \frac 1 {\alpha!} \DifferenceOperatorOrder {\alpha} \LeftDifferentialOperator {\alpha} \adj \sigma \in \SymbolClass {m - (\rho - \delta) (N + 1)} {\rho, \delta}
        \end{align*}
        with an adapted basis of the Lie algebra.
    \end{theorem}
\end{frame}

\begin{frame}{$L^2$ boundedness}
    \begin{theorem}[$L^2$ boundedness]
        Let $T \in \OperatorClass m {\rho, \delta}$ for $\rho > \delta$.
        The operator $T$ has a continuous extension
        \begin{align*}
            T : \Sobolev s \to \Sobolev {s - m}
        \end{align*}
        for each $s \in \R$.
    \end{theorem}
\end{frame}

\end{document}
