\documentclass[handout]{beamer}

\usepackage{amsmath, amsthm, amssymb, amsfonts}
\usepackage{mathtools}
\usepackage{xparse}

\usepackage{../macros}

\usepackage[style=alphabetic]{biblatex}
\bibliography{../Bibliography.bib}

\title{Pseudo-Differential Calculus on Generalized Motion Groups}
\author{Binh-Khoi Nguyen}

\usetheme{Warsaw}

\begin{document}

\maketitle

\section{Introduction}

\begin{frame}{Global pseudo-differential calculus on Lie groups}
    Is there a collection of operators $\bigcup_{m \in \R} \OperatorClass m {\rho, \delta}$ such that
    \begin{enumerate}
        \item
            For each $k \in \N$,
            $\mathrm{Diff}^k(\Group) \subset \OperatorClass k {\rho, \delta}$.
        \item
            If $m \leq m'$,
            then $\OperatorClass m {\rho, \delta} \subset \OperatorClass {m'} {\rho, \delta}$.
        \item
            If $T \in \OperatorClass m {\rho, \delta}$, then $\adj T \in \OperatorClass m {\rho, \delta}$.
        \item
            If $T_i \in \OperatorClass {m_i} {\rho, \delta}$, $i = 1, 2$,
            then $T_1 \circ T_2 \in \OperatorClass {m_1 + m_2} {\rho, \delta}$.
        \item
            If $T \in \OperatorClass m {\rho, \delta}$,
            then $T$ extends to a bounded operator
            \begin{align*}
                T : \Sobolev s \to \Sobolev {s - m}.
            \end{align*}
        \item There exists symbol classes and a map
            \begin{align*}
                \Op : \SymbolClass m {\rho, \delta} \to \OperatorClass m {\rho, \delta}
            \end{align*}
            such that
            $\Op (\sigma_1) \circ \Op(\sigma_2) = \Op (\sigma_1 \sigma_2)$
            mod l.o.t.
    \end{enumerate}
\end{frame}

\begin{frame}{State of the art}
    The answer is affirmative on the following settings.

    \begin{itemize}
        \item Euclidean spaces
        \item Compact groups \cite{RuzhanskyTurunen10}
        \item Graded Lie groups \cite{FischerRuzhansky16}
    \end{itemize}

    What about semi-direct product of such groups?
\end{frame}

\begin{frame}{Outline}
    \begin{enumerate}
        \item Define \emph{symbols} as the group Fourier Transform of right convolution kernels of operators.
        \item Define \emph{symbol classes} with Ruzhansky-Turunen \emph{difference operators}.
        \item Prove \emph{composition} and \emph{adjunction} formulas.
    \end{enumerate}
\end{frame}

\begin{frame}{Motion group}
    \begin{definition}[Motion group]
        We shall call
        \begin{align*}
            \Group \defeq \R^n \rtimes \SpecialOrthogonalGroup n.
        \end{align*}
        the \emph{(Euclidean) motion group}
    \end{definition}

    \pause

    In other words, $G = (\R^n \times \SpecialOrthogonalGroup n, \circ)$,
    with
    \begin{align*}
        (x, k) \circ (y, l) \defeq (x + k y, k l)
    \end{align*}
\end{frame}

\begin{frame}[Fischer-Ruzhansky]
    In~\cite{Fischer2015,FischerRuzhansky16},
    the key argument is an approximation of the \emph{identity symbol} via a \emph{Littlewood-Paley decomposition}.
    \begin{align*}
        \Id {\Lebesgue 2 \CompactGroup} = \sum_{j = 0}^{+ \infty} \eta_l
    \end{align*}

    This allows us to write
    \begin{align*}
        \sigma = \sum_{j = 0}^{+ \infty} \sigma_j,
        \quad
        \sigma_j \in \SmoothingSymbols
    \end{align*}
\end{frame}

\begin{frame}{Key argument}
    For each $\epsilon \in (0, 1]$,
    we let
    \begin{align*}
        \eta_\epsilon(\lambda)
        \defeq
        \frac 1 {\i \turn}
        \int_\Gamma
        \e^{-\epsilon z}
        (\Rep \lambda \BesselPotentialSquared {} - z \Id {\Lebesgue 2 \CompactGroup})^{-1}
        \dd z
    \end{align*}

    \pause

    \begin{theorem}[Approximation of $\Id {\Lebesgue 2 \CompactGroup}$]
        For each $\epsilon \in (0, 1]$, each $m \in \R$ and each $N \in \N$,
        there exists $C \geq 0$ such that
        \begin{align*}
                \SymbolSemiNorm m {} N {\eta_\epsilon}
                &\leq C \epsilon^\frac m 2
                & \text{if } m < 0\\
                \SymbolSemiNorm m {} N {\Id {\Lebesgue 2 \CompactGroup} - \eta_\epsilon}
                &\leq C \epsilon^\frac m 2
                & \text{if } m \geq 0
        \end{align*}
    \end{theorem}
\end{frame}

\begin{frame}{Key argument (continued)}
    \begin{enumerate}
        \item
            We replace all the Littlewood-Paley decompositions in~\cite{Fischer2015,FischerRuzhansky16} by
            \begin{align*}
                \Id {\Lebesgue 2 \CompactGroup}
                =
                \eta_\epsilon +
                (\Id {\Lebesgue 2 \CompactGroup} - \eta_\epsilon)
            \end{align*}
        \item
            In~\cite{FischerRuzhansky16,Fischer2015},
            the corresponding result is proved \emph{on the kernel side}.
    \end{enumerate}
\end{frame}

\end{document}
