\documentclass{beamer}
%\documentclass[handout]{beamer}

\usepackage{amsmath, amsthm, amssymb, amsfonts}
\usepackage{mathtools}
\usepackage{xparse}
\usepackage{color}

\usepackage{../macros}

\usepackage[style=alphabetic]{biblatex}
\bibliography{../Bibliography.bib}

\title{Pseudo-Differential Calculus on Generalized Motion Groups}
\author{Binh-Khoi Nguyen}

\usetheme{Warsaw}

\begin{document}

\maketitle

\section{Introduction}

\begin{frame}{Global pseudo-differential calculus on Lie groups}
    Is there a collection of operators $\bigcup_{m \in \R} \OperatorClass m {\rho, \delta}$ such that
    \begin{enumerate}
        \item
            For each $k \in \N$,
            $\mathrm{Diff}^k(\Group) \subset \OperatorClass k {\rho, \delta}$.
            \pause
        \item
            If $m \leq m'$,
            then $\OperatorClass m {\rho, \delta} \subset \OperatorClass {m'} {\rho, \delta}$.
            \pause
        \item
            If $T \in \OperatorClass m {\rho, \delta}$, then $\adj T \in \OperatorClass m {\rho, \delta}$.
            \pause
        \item
            If $T_i \in \OperatorClass {m_i} {\rho, \delta}$, $i = 1, 2$,
            then $T_1 \circ T_2 \in \OperatorClass {m_1 + m_2} {\rho, \delta}$.
            \pause
        \item
            If $T \in \OperatorClass m {\rho, \delta}$,
            then $T$ extends to a bounded operator
            \begin{align*}
                T : \Sobolev s \to \Sobolev {s - m}.
            \end{align*}
            \pause
        \item There exists symbol classes and a map
            \begin{align*}
                \Op : \SymbolClass m {\rho, \delta} \to \OperatorClass m {\rho, \delta}
            \end{align*}
            such that
            $\Op (\sigma_1) \circ \Op(\sigma_2) = \Op (\sigma_1 \sigma_2)$
            mod l.o.t.
    \end{enumerate}
\end{frame}

\begin{frame}{State of the art}
    The answer is affirmative on the following settings.

    \begin{itemize}
        \item Euclidean spaces
            \pause
        \item Compact groups \cite{RuzhanskyTurunen10}
            \pause
        \item Graded Lie groups \cite{FischerRuzhansky16}
            \pause
    \end{itemize}

    What about semi-direct product of such groups?
\end{frame}

\begin{frame}{Outline}
    \begin{enumerate}
        \item Define \emph{symbols} as the group Fourier Transform of right convolution kernels of operators.
            \pause
        \item Define \emph{symbol classes} with Ruzhansky-Turunen \emph{difference operators}.
            \pause
        \item Prove \emph{composition} and \emph{adjunction} formulas.
    \end{enumerate}
\end{frame}

\section{The motion group}

\begin{frame}{Motion group}
    \begin{definition}[Motion group]
        We shall call
        \begin{align*}
            \Group \defeq \R^n \rtimes \SpecialOrthogonalGroup n.
        \end{align*}
        the \emph{(Euclidean) motion group}
    \end{definition}

    \pause

    In other words, $G = (\R^n \times \SpecialOrthogonalGroup n, \circ)$,
    with
    \begin{align*}
        (x, k) \circ (y, l) \defeq (x + k y, k l)
    \end{align*}

    \pause

    We integrate with respect to the product measure $\dd x \dd k$.
\end{frame}

\begin{frame}{Laplacian and semi-direct structure}
    The operator
    \begin{align*}
        \Laplacian \defeq \Laplacian [\R^n] + \Laplacian [\SpecialOrthogonalGroup n]
    \end{align*}
    is called the \emph{left-invariant Laplacian}.

    \pause

    It is \emph{right-invariant} on $\R^n \times \SpecialOrthogonalGroup n$
    but \emph{not} on $\Group = \R^n \rtimes \SpecialOrthogonalGroup n$.
\end{frame}

\begin{frame}{Fourier transform}
    \begin{definition}[Representations]
        Let $\lambda \in \R^n$ and $(x, k) \in \Group$.
        We define
        \begin{align*}
            \Rep \lambda(x, k) F(u)
            = \e^{\i \turn \ip [\R^n] \lambda {u x}} F(u k)
        \end{align*}
        for each $F \in \Lebesgue 2 {\SpecialOrthogonalGroup n}$ and each $u \in \SpecialOrthogonalGroup n$.
    \end{definition}

    Note that the above representations are reducible.
\end{frame}

\begin{frame}{Plancherel formula}
    \begin{align*}
        \Fourier f(\lambda) \defeq \int_\Group f(g) \adj{\Rep \lambda(g)} \dd g
    \end{align*}

    \pause

    \begin{theorem}[Plancherel and inverse formulas]
        If $f \in \Schwartz \Group$,
        then we have
        \begin{align*}
            \int_\Group \abs f^2 \dd g
            = \int_{\R^n} \tr \left(\Fourier f(\lambda) \adj {\Fourier f(\lambda)}\right) \dd \lambda.
        \end{align*}

        Moreover,
        for each $g \in \Group$, we have
        \begin{align*}
            f(g)
            = \int_{\R^n} \tr \left(\Rep \lambda(g) \Fourier f(\lambda)\right) \dd \lambda.
        \end{align*}
    \end{theorem}
\end{frame}

\section{Difference operators}

\begin{frame}{Derivatives in frequency variables}
    On $\R^n$,
    \begin{align*}
        \abs {\iD {x^\beta} {\color{red} \iD {\xi^\alpha}} \sigma(x, \xi)}
        \leq C_{\alpha, \beta} (1 + \abs \xi^2)^{\frac {m - \rho \abs \alpha + \delta \abs \beta} 2}
    \end{align*}

    How do we generalize $\iD {\xi^\alpha}$?

    \pause

    \begin{align*}
        {\color{red} \iD {\xi^\alpha}} \Fourier [\R^n] \kappa(\xi)
        = C_\alpha \Fourier [\R^n] {\{{\color{red} x^\alpha} \kappa\}}(\xi)
    \end{align*}
\end{frame}

\begin{frame}{Ruzhansky-Turunen difference operators}
    \begin{definition}[Difference operators \cite{RuzhanskyTurunen10}]
        Let $q \in \SmoothFunctions \Group$.

        We define $\DifferenceOperator q$ via
        \begin{align*}
            \DifferenceOperator q \Fourier f(\lambda)
            \defeq \Fourier \{q f\}(\lambda)
        \end{align*}
    \end{definition}

    \pause

    Selecting a family $\Delta = \{q_1, \dots, q_M \in \SmoothFunctions \Group\}$,
    we write
    \begin{align*}
        \DifferenceOperatorOrder \alpha \defeq \DifferenceOperator {q^\alpha(\dummy^{-1})},
        \quad q^\alpha \defeq q^{\alpha_1}_1 \dots q^{\alpha_M}_M.
    \end{align*}
\end{frame}

\begin{frame}{Difference operators (continued)}
    We choose $\Delta \defeq \{q_1, \dots, q_M \in \SmoothFunctions \Group\}$ so that
    \begin{enumerate}
        \item
            $\Span \{\dd q_j(e) : j = 1, \dots, M\} = T_e G$.
            \pause
        \item
            $\bigcap_{q \in \Delta} \{q = 0\} = \{e\}$.
            \pause
        \item
            Leibniz-like rule:
    \end{enumerate}
    \begin{align*}
        \DifferenceOperator {q_i} \{\sigma_1 \sigma_2\}
        =
        \left(\DifferenceOperator {q_i} \sigma_1 \right) \sigma_2
        + \sigma_1 \left(\DifferenceOperator {q_i} \sigma_2\right)
        + \sum_{j, k = 1} c^i_{j k} \left(\DifferenceOperator {q_j} \sigma_1\right) \left(\DifferenceOperator {q_k} \sigma_2\right)
    \end{align*}

    \pause

    From now on,
    $\Delta$ will denote such a family.
\end{frame}

\begin{frame}{Symbol classes}
    \begin{definition}[Symbol classes]
        A map $\sigma: \Group \times \R^n \to \End {\SmoothFunctions \Group}$ is in $\SymbolClass m {\rho, \delta}$ if
        \begin{enumerate}
            \item
                $\sigma$ is smooth.
            \item
                For each $\alpha \in \N^M$ and each $\beta \in \N^{\dim \Group}$,
                \begin{align*}
                    \norm [\Lin {\Lebesgue 2 {\SpecialOrthogonalGroup n}}] {
                        \Rep \lambda \BesselPotential {-m + \rho \abs \alpha - \delta \abs \beta}
                        \LeftDifferentialOperator \beta \DifferenceOperatorOrder \alpha \sigma(g, \lambda)
                    } < \infty
                \end{align*}
                uniformly in $g \in \Group$, $\lambda \in \R^n$.
        \end{enumerate}
    \end{definition}
\end{frame}

\begin{frame}{Pseudo-differential operators}
    If $\sigma \in \SymbolClass m {\rho, \delta}$,
    we let
    \begin{align*}
        \Op(\sigma) \phi(g) = \InverseFourier \left\{ \sigma(g, \dummy) \Fourier \phi \right\}(g).
    \end{align*}

    $\Op(\sigma)$ is the \emph{pseudo-differential operator} associated with $\sigma$.

    We write
    \begin{align*}
        \OperatorClass m {\rho, \delta} \defeq \Op \left(\SymbolClass m {\rho, \delta}\right).
    \end{align*}

    \pause

    The map
    \begin{align*}
        g \in \Group \mapsto \kappa_g \defeq \InverseFourier \{\sigma(g, \dummy)\}
    \end{align*}
    is called the \emph{kernel} associated with $\sigma$.
\end{frame}

\begin{frame}{Quantization}
    \begin{block}{Proposition}
        Let $\sigma \in \SymbolClass m {\rho, \delta}$.
        For every $\phi \in \Schwartz \Group$,
        we have
        \begin{align*}
            \Op(\sigma) \phi(g) = (\conv \phi {\kappa_g})(g)
        \end{align*}
        for every $g \in \Group$.
    \end{block}
\end{frame}

\begin{frame}{Sketch proof of adjunction formula}
    Let $\sigma \in \SymbolClass m {\rho, \delta}$.
    \begin{enumerate}
        \item Compute $\adj T \phi(g) = (\conv \phi {\conj \kappa_{g \dummy^{-1}}}(\dummy^{-1}))(g)$.
            \pause
        \item Define $\sigma^{(*)}(g, \lambda) \defeq \Fourier \{ \conj \kappa_{g \dummy^{-1}}(\dummy^{-1}) \}(\lambda)$.
            \pause
        \item By a Taylor development,
            \begin{align*}
                \sigma^{(*)}(g, \lambda)
                &= \int_\Group \left(\sum_{\abs \alpha \leq N} \frac 1 {\alpha !} q^\alpha(h^{-1}) \LeftDifferentialOperator [g] \alpha \conj \kappa_g(h^{-1}) + R_N(h)\right) \adj {\Rep \lambda(h)} \dd h\\
                &= \sum_{\abs \alpha \leq N} \frac 1 {\alpha !} \DifferenceOperatorOrder \alpha \LeftDifferentialOperator [g] \alpha \adj \sigma(g, \lambda) + \int_\Group R_N(h) \adj {\Rep \lambda(h)} \dd h.
            \end{align*}
            \pause
        \item
            Taylor's Theorem: $\abs {R_N(h)} \leq C \norm [\Group] h^{N + 1} \sup_{\abs \alpha = N + 1} \abs {\LeftDifferentialOperator \alpha \conj \kappa_g}$.
    \end{enumerate}
\end{frame}

\begin{frame}{Kernel estimates}
    \begin{theorem}[Kernel estimates]
        Let $\sigma \in \SymbolClass m {\rho, \delta}$,
        and let $\kappa$ denote the associated kernel.

        \begin{enumerate}
            \item
                Away from the origin,
                $\kappa_g$ is smooth and has \emph{Schwartz decay}.
            \item
                Around the origin,
                if $m > -\dim \Group$
                \begin{align*}
                    \abs {\kappa_g(h)} \leq C \norm [\Group] h^{\gamma}
                \end{align*}
                for each $\gamma < -(m + \dim \Group) / \rho$.
        \end{enumerate}
    \end{theorem}
\end{frame}

\begin{frame}[Fischer-Ruzhansky]
    In~\cite{Fischer2015,FischerRuzhansky16},
    the key argument is an approximation of the \emph{identity symbol} via a \emph{Littlewood-Paley decomposition}.
    \begin{align*}
        \Id {\Lebesgue 2 {\SpecialOrthogonalGroup n}} = \sum_{j = 0}^{+ \infty} \Psi_j(\Rep \lambda (\Laplacian)),
        \quad \sum_j \Psi_j = 1.
    \end{align*}

    \pause

    This allows us to write
    \begin{align*}
        \sigma = \sum_{j = 0}^{+ \infty} \sigma_j,
        \quad
        \sigma_j \in \SmoothingSymbols.
    \end{align*}

    \pause

    Informally, we will choose
    \begin{align*}
        \Psi_0(\lambda) = \e^{\lambda}, \quad \Psi_1 = 1 - \e^{\lambda},
        \quad \Psi_j = 0, \ j > 1
    \end{align*}
\end{frame}

\begin{frame}{Key argument}
    For each $\epsilon \in (0, 1]$,
    we let
    \begin{align*}
        \eta_\epsilon(\lambda)
        \defeq
        \frac 1 {\i \turn}
        \int_\Gamma
        \e^{-\epsilon z}
        (\Rep \lambda \BesselPotentialSquared {} - z \Id {\Lebesgue 2 {\SpecialOrthogonalGroup n}})^{-1}
        \dd z
    \end{align*}

    \pause

    \begin{theorem}[Approximation of $\Id {\Lebesgue 2 {\SpecialOrthogonalGroup n}}$]
        For each $\epsilon \in (0, 1]$, each $m \in \R$ and each $N \in \N$,
        there exists $C \geq 0$ such that
        \begin{align*}
                \SymbolSemiNorm m {} N {\eta_\epsilon}
                &\leq C \epsilon^\frac m 2
                & \text{if } m < 0\\
                \SymbolSemiNorm m {} N {\Id {\Lebesgue 2 {\SpecialOrthogonalGroup n}} - \eta_\epsilon}
                &\leq C \epsilon^\frac m 2
                & \text{if } m \geq 0
        \end{align*}
    \end{theorem}
\end{frame}

\begin{frame}{Key argument (continued)}
    \begin{enumerate}
        \item
            We replace all the Littlewood-Paley decompositions in~\cite{Fischer2015,FischerRuzhansky16} by
            \begin{align*}
                \Id {\Lebesgue 2 {\SpecialOrthogonalGroup n}}
                =
                \eta_\epsilon +
                (\Id {\Lebesgue 2 {\SpecialOrthogonalGroup n}} - \eta_\epsilon)
            \end{align*}
            \pause
        \item
            In~\cite{FischerRuzhansky16,Fischer2015},
            the corresponding result is proved \emph{on the kernel side} with \emph{heat kernel} estimates.
    \end{enumerate}
\end{frame}

\begin{frame}{Comments on \cite{FischerRuzhansky16}}
    \begin{enumerate}
        \item
            It is difficult to establish symbolic estimates directly
            \begin{align*}
                \DifferenceOperator q \sigma = \Fourier \{q \kappa\}
            \end{align*}

            \pause
        \item
            We are able to obtain a simple proof because
            \pause
            \begin{enumerate}
                \item We use \emph{holomorphic functional calculus}
                    $\eta_\epsilon = \e^{-\epsilon \Rep \lambda \BesselPotentialSquared {}}$
                    \pause
                \item Symbolic estimates are obtained by applying difference operators to
                    \begin{align*}
                        \Id {\Lebesgue 2 {\SpecialOrthogonalGroup n}} =
                        \left(\Rep \lambda \BesselPotentialSquared {} - z \Id {\Lebesgue 2 {\SpecialOrthogonalGroup n}}\right)
                        \left(\Rep \lambda \BesselPotentialSquared {} - z \Id {\Lebesgue 2 {\SpecialOrthogonalGroup n}}\right)^{-1}
                    \end{align*}
            \end{enumerate}
    \end{enumerate}
\end{frame}

\begin{frame}{Conclusion}
    \begin{enumerate}
        \item Our approximation of the identity does not rely on explicit expressions specific to the motion group.
            \pause
        \item In particular,
            we can shorten the proofs in the compact and graded cases.
            \pause
        \item Strategy works for any Lie group with polynomial growth that has a good family of difference operators.
    \end{enumerate}
\end{frame}

\end{document}
