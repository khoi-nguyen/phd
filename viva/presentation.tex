\documentclass{beamer}

\usepackage{amsmath, amsthm, amssymb, amsfonts}
\usepackage{mathtools}
\usepackage{xparse}

\usepackage{../macros}

\usepackage[style=alphabetic]{biblatex}
\bibliography{../Bibliography.bib}

\title{Pseudo-Differential Calculus on Generalized Motion Groups}
\author{Binh-Khoi Nguyen}

\usetheme{Warsaw}

\begin{document}

\maketitle

\begin{frame}{Motion group}
    \begin{definition}[Motion group]
        We shall call
        \begin{align*}
            \Group \defeq \R^n \rtimes \SpecialOrthogonalGroup n.
        \end{align*}
        the \emph{(Euclidean) motion group}
    \end{definition}

    \pause

    In other words, $G = (\R^n \times \SpecialOrthogonalGroup n, \circ)$,
    with
    \begin{align*}
        (x, k) \circ (y, l) \defeq (x + k y, k l)
    \end{align*}
\end{frame}

\begin{frame}{Key argument}
    For each $\epsilon \in (0, 1]$,
    we let
    \begin{align*}
        \eta_\epsilon(\lambda)
        \defeq
        \frac 1 {\i \turn}
        \int_\Gamma
        \e^{-\epsilon z}
        (\Rep \lambda \BesselPotentialSquared {} - z \Id {\Lebesgue 2 \CompactGroup})^{-1}
        \dd z
    \end{align*}

    \pause

    \begin{theorem}[Approximation of $\Id {\Lebesgue 2 \CompactGroup}$]
        For each $\epsilon \in (0, 1]$, each $m \in \R$ and each $N \in \N$,
        there exists $C \geq 0$ such that
        \begin{align*}
                \SymbolSemiNorm m {} N {\eta_\epsilon}
                &\leq C \epsilon^\frac m 2
                & \text{if } m < 0\\
                \SymbolSemiNorm m {} N {\Id {\Lebesgue 2 \CompactGroup} - \eta_\epsilon}
                &\leq C \epsilon^\frac m 2
                & \text{if } m \geq 0
        \end{align*}
    \end{theorem}
\end{frame}

\end{document}
