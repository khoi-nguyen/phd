\documentclass[dottedtoc, headinclude, footinclude=true]{scrbook}
\usepackage[pdfspacing]{classicthesis}
\usepackage[backend=bibtex,style=alphabetic]{biblatex}

% Index
\usepackage{makeidx}
\usepackage[utf8]{inputenc}

\usepackage{amsmath, amsthm, amssymb, amsfonts}
\usepackage{mathtools}

\usepackage{xparse}

% Environments
\newtheorem{corollary}{Corollary}[section]
\newtheorem{definition}{Definition}[section]
\newtheorem{example}{Example}[section]
\newtheorem{lemma}{Lemma}[section]
\newtheorem{proposition}{Proposition}[section]
\newtheorem{remark}{Remark}[section]
\newtheorem{theorem}{Theorem}[section]

\newcounter{claimcounter}
\numberwithin{claimcounter}{theorem}
\newenvironment{claim}{\stepcounter{claimcounter}{\emph{Claim \theclaimcounter:}}}{}

% Notation
\DeclarePairedDelimiter{\Ceiling}{\lceil}{\rceil}
\newcommand{\abs}[1]{\left|{#1}\right|}
\newcommand{\adj}[1]{{#1}^\star}
\newcommand{\conj}[1]{\overline{#1}}
\newcommand{\conv}[2]{#1 \star #2}
\newcommand{\dualGroup}[1]{{\widehat{#1}}}
\newcommand{\defeq}{\mathrel{\overset{\makebox[0pt]{\mbox{\normalfont\tiny\sffamily def}}}{=}}}
\newcommand{\grad}[1][\VectorSpace]{\nabla_{#1}}
\newcommand{\directionalDerivative}[1]{\partial_{#1}}
\newcommand{\dd}{\,\mathrm{d}}
\newcommand{\dimRep}[1]{{d_{#1}}}
\newcommand{\dualBracket}[3][\Group]{{\langle #2, #3 \rangle}_{#1}}
\newcommand{\dummy}{\cdot}
\newcommand{\eval}[2]{\left. #1 \right|_{#2}}
\newcommand{\g}{\mathfrak{g}}
\newcommand{\norm}[2][\VectorSpace]{{\left\| #2 \right\|}_{#1}}
\newcommand{\seminorm}[3][\VectorSpace]{\norm[#1, #2]{#3}}
\newcommand{\ip}[3][\VectorSpace]{{\left(#2, #3\right)_{#1}}}
\newcommand{\AbelianGroup}{A}
\newcommand{\AffineTransformations}[1]{{\mathrm{Affine} (#1)}}
\newcommand{\Ball}[3][\VectorSpace]{{B_{#1} (#2, #3)}}
\newcommand{\BesselPotential}[2][\Group]{{(I - \Laplacian[#1])^\frac{#2}{2}}}
\newcommand{\BesselPotentialKernel}[2][\Group]{\mathfrak B^{#1}_{#2}}
\newcommand{\BigO}{\mathcal{O}}
\newcommand{\Character}[1]{{\chi_{#1}}}
\newcommand{\ContinuousFunctions}[1]{{C(#1)}}
\newcommand{\DifferenceOperator}[2][\Group]{{\Delta^{#1}_{#2}}}
\newcommand{\DifferenceOperatorOrder}[2][\Group]{{\Delta_{#1}^{#2}}}
\newcommand{\DiracDelta}[1]{\delta_{#1}}
\newcommand{\EquivalenceClass}[2]{{{[#2]}_{#1}}}
\newcommand{\Fourier}[1][\Group]{\mathcal{F}_{#1}}
\newcommand{\Group}{G}
\newcommand{\GroupDirect}{\VectorSpace \times \CompactGroup}
\newcommand{\GeneralLinear}[1]{\mathrm{GL}(#1)}
\newcommand{\Hil}{\mathcal{H}}
\newcommand{\Hilbert}[2]{\mathfrak{H}_{#1, #2}}
\newcommand{\HilbertRep}[1]{{\mathfrak{H}_{#1}}}
\newcommand{\HilbertCompactGroupColumn}[2]{\mathfrak{H}_{#1, #2}}
\newcommand{\HilbertCompactGroup}[1]{\mathfrak{H}_{#1}}
\newcommand{\InverseFourier}[1][\Group]{\mathcal{F}^{-1}_{#1}}
\newcommand{\CompactGroup}{K}
\newcommand{\HilbertSchmidt}[1]{{\mathcal{HS} \left(#1\right)}}
\newcommand{\Id}[1]{{I_{#1}}}
\newcommand{\IsotropySubgroup}[2]{{{#1}_{#2}}}
\newcommand{\JapaneseBracket}[2]{{\langle #2 \rangle}_{\dualGroup{#1}}}
\newcommand{\Kronecker}[2]{\delta_{#1,#2}}
\newcommand{\Kernels}[1]{\mathcal{K}(#1)}
\newcommand{\KernelsSobolev}[3][\Group]{\mathcal{K}_{#2, #3}(#1)}
\newcommand{\Lebesgue}[2]{{L^{#1} (#2)}}
\newcommand{\LebesgueDual}[3][]{{L^{#2}_{#1} (\dualGroup{#3})}}
\newcommand{\LeftDifferentialOperatorFirstOrder}[1]{{#1}}
\newcommand{\LeftDifferentialOperator}[1]{X^{#1}_L}
\newcommand{\LeftDifferentialOperatorOnCompactGroup}[2][]{Y^{#2}_{#1}}
\newcommand{\LeftRegularRepresentation}[1][\CompactGroup]{\pi^L_{#1}}
\newcommand{\RightRegularRepresentation}[1][\CompactGroup]{\pi^R_{#1}}
\newcommand{\Lie}{\mathfrak{Lie}}
\newcommand{\LieAlgebra}{\mathfrak{g}}
\newcommand{\LieAlgebraCompactGroup}{\mathfrak{k}}
\newcommand{\LieAlgebraVectorSpace}{\mathfrak{v}}
\newcommand{\LieBracket}[3][\LieAlgebra]{{[#2, #3]}_{#1}}
\newcommand{\Lin}[1]{{\mathcal{L} (#1)}}
\newcommand{\Laplacian}[1][\Group]{{\mathcal{L}_{#1}}}
\newcommand \SquareMatrices [2][\R] {{#1}^{#2 \times #2}}
\newcommand{\MotionGroup}[1]{{\mathrm{SE} (#1)}}
\newcommand{\OrthogonalGroup}[1]{{\mathrm{O} (#1)}}
\newcommand{\Op}[1][\Group]{\mathrm{Op}_{#1}}
\newcommand{\Plancherel}[1]{\mu_{\dualGroup{#1}}}
\newcommand{\Polynomials}[1]{{\mathrm{Pol}_{#1}}}
\newcommand{\Projection}[1]{\mathrm{Proj}_{#1}}
\newcommand{\LeftQuotient}[2]{{{#1} \backslash{} {#2}}}
\newcommand{\RightQuotient}[2]{{{#1} \slash{} {#2}}}
\newcommand{\Rep}[2][\Group]{\xi^{#2}_{#1}}
\newcommand{\RightDifferentialOperatorFirstOrder}[1]{\tilde{#1}}
\newcommand{\RightLaplacian}[1][\Group]{{\tilde{\mathcal{L}}_{#1}}}
\newcommand{\Rotation}[1]{\tilde R \left(#1\right)}
\newcommand{\InverseRotation}[1]{R\left(#1\right)}
\newcommand{\SmoothFunctions}[1]{{C^\infty(#1)}}
\newcommand{\SmoothVectors}[1]{#1^\infty}
\newcommand{\ScalarImageSchwartz}[1]{\tilde{\mathcal{S}}(#1)}
\newcommand{\SchattenClasses}[2]{S_{#1}(#2)}
\newcommand{\Schwartz}[1]{{\mathcal{S} (#1)}}
\newcommand{\SkewSymmetric}[1]{\mathrm{Skew}(#1)}
\newcommand{\Sobolev}[2][\Group]{L^2_{#2}(#1)}
\newcommand{\SpecialOrthogonalGroup}[1]{{\mathrm{SO} (#1)}}
\newcommand{\SpecialUnitaryGroup}[1]{{\mathrm{SU} (#1)}}
\newcommand{\TangentSpace}[2]{T_{#2} #1}
\newcommand{\TaylorLeftDifferentialOperator}[1]{X^{(#1)}_L}
\newcommand{\TemperedDistributions}[1]{{\mathcal{S}' (#1)}}
\newcommand{\UnitaryGroup}[1]{{\mathrm{U} (#1)}}
\newcommand{\LeftInvariantVectorFields}[1][\Group]{\mathfrak{X}_L(#1)}
\newcommand{\VectorFields}[1][\Group]{\mathfrak{X}(#1)}
\newcommand{\VectorSpace}{V}

\DeclareMathOperator{\Aut}{Aut}
\DeclareMathOperator{\End}{End}
\DeclareMathOperator{\Hessian}{Hess}
\DeclareMathOperator{\Hom}{Hom}
\DeclareMathOperator{\Span}{span}
\DeclareMathOperator{\tr}{tr}
\DeclareMathOperator{\order}{order}
\DeclareMathOperator{\rank}{rank}
\DeclareMathOperator{\supp}{supp}
\DeclareMathOperator*{\esssup}{ess\,sup}

\ExplSyntaxOn
\newcommand\der{}
\DeclareDocumentCommand \D{s m O{} m}{%
    % Choice of right d
    \IfBooleanTF{#1}{\renewcommand\der{\dd}}{\renewcommand\der{\partial}}
    % Write the derivative
    \frac{
        \der\ifnum\pdfstrcmp{#2}{1}=0\else^{#2}\fi {#3}
    }{%
        \clist_map_inline:nn {#4} {\der ##1}
    }
}
\ExplSyntaxOff

% Sets
\newcommand{\C}{\mathbb{C}}
\newcommand{\N}{\mathbb{N}}
\newcommand{\R}{\mathbb{R}}
\newcommand{\T}{\mathbb{T}}
\newcommand{\Z}{\mathbb{Z}}

% Constants
\newcommand{\e}{e}
\newcommand{\turn}{2 \pi}
\renewcommand{\i}{i}

% Pseudo-differential calculus
\newcommand{\SmoothingSymbols}[1][\Group]{{S^{-\infty} (#1)}}
\newcommand{\SmoothingOperators}[1][\Group]{{\Psi^{-\infty} (#1)}}
\newcommand{\SymbolClass}[3][\Group]{S^{#2}_{#3}(#1)}
\newcommand{\Symbols}[1][\Group]{S(#1)}
\newcommand{\OperatorClass}[3][\Group]{\Psi^{#2}_{#3}(#1)}
\newcommand{\SymbolSemiNorm}[4][\Group]{\norm[S^{#2}_{#3}{#1}]{#4}}

\title{Pseudo-differential calculus on generalised Motion Groups}
\author{Binh-Khoi Nguyen}
