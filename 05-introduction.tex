\chapter{Introduction}

On $\R^n$,
the success of \emph{pseudo-differential calculus} resides in the construction of
an algebra of operators satisfying the following properties:
\begin{enumerate}
    \item If $k \in \N$, then $\OperatorClass [\R^n] k {\rho, \delta}$ contains derivatives with constant coefficients.
    \item For each $m \in \R$, $\BesselPotential [\R^n] m \in \OperatorClass m {\rho, \delta}$
    \item The composition of $T_1 \in \OperatorClass [\R^n] {m_1} {\rho, \delta}$ and $T_2 \in \OperatorClass [\R^n] {m_2} {\rho, \delta}$ is in $\OperatorClass [\R^n] {m_1 + m_2} {\rho, \delta}$;
    \item $\OperatorClass [\R^n] m {\rho, \delta}$ is stable under taking the adjoint.
    \item If $T \in \OperatorClass [\R^n] m {\rho, \delta}$,
        then $T$ extends to a bounded operator
        \begin{align*}
            T : \SobolevOrder [\R^n] p s \to \SobolevOrder [\R^n] p {s - m}.
        \end{align*}
        for each $s \in \R$.
\end{enumerate}

More importantly,
each pseudo-differential arises from a smooth function called \emph{symbol}.
In fact,
the process of associating an appropriate smooth function to a pseudo-differential operator,
called \emph{quantisation},
generalises the association of a characteristic polynomial to its corresponding differential operator.
Moreover,
the composition and adjunction of pseudo-differential operators can be written approximately on the symbolic side,
with formulas that are formally virtually identical to the rules we obtain with characteristic polynomial for the composition of differential operators.
As a result,
we are able to characterise the set of pseudo-differential operators which admit an approximate inverse.

This leads us to wonder whether pseudo-differential calculus could be generalised onto other settings.
A first direction would be to try to define pseudo-differential operators on manifolds as the operators which are Euclidean pseudo-differential operators in every local coordinate chart.
Unfortunately,
this requires a condition such as
(see e.g.\ \cite[Section 4]{Shubin01})
\begin{align*}
    1 - \rho \leq \delta < \rho,
\end{align*}
which is too restrictive (TODO add example).
Moreover, we cannot assign a global symbol to a pseudo-differential operator (TODO explain what is problematic).

Another direction would be to restrict ourselves to an appropriate class of Lie groups,
and develop a \emph{global} pseudo-differential calculus using the group's Fourier analysis.
In~\cite{RuzhanskyTurunen10},
\citeauthor{RuzhanskyTurunen10} define a \emph{global, matrix-valued} symbol of an operator on compact Lie groups.
They also introduce the concept of \emph{difference operator}
to generalise the derivatives in the frequency variable which appear in the Euclidean case,
paving the way for further generalisations.
As a result,
they were able to define more general operator classes which coincide with the local calculus when the latter is defined.

\section{The Euclidean case}

It is known and elementary fact that
the composition of differential operators with smooth coefficients on $\R^n$ can be computed on the Fourier Transform side.
More precisely,
if $\sigma_1$ and $\sigma_2$ are two characteristic polynomials,
the composition of the underlying operators arise from the polynomial
\begin{align}
    \sum_{\alpha \in \N^n} \frac {(\i \turn)^{-\abs \alpha}} {\alpha!} \iD{\xi^\alpha} \sigma_1(x, \xi) \iD{x^\alpha} \sigma_2(x, \xi)
    \label{eq:composition_formula_for_characteristic_polynomials}.
\end{align}
In particular,
this formula can be used to construct approximate inverses to differential operators with smooth coefficients.

Laurent Schwartz's works allow us to extend the notion of \emph{characteristic polynomial} far beyond the class of differential operators.
The resulting concept,
called the \emph{symbol} of an operator,
is formally defined in the following way:
if $\kappa$ is the convolution kernel of $T$, i.e.\
\begin{align*}
    T \phi(x) = (\conv \phi {\kappa(x, \dummy)})(x),
\end{align*}
then the symbol of $T$ is defined via
\begin{align*}
    \sigma(x, \xi) = \Fourier [\R^n] \{\kappa(x, \dummy)\}(\xi).
\end{align*}

More importantly,
the analogue of~\eqref{eq:composition_formula_for_characteristic_polynomials} for smooth symbols is
\begin{align}
    \sigma(x, \xi)
    &= \int_{\R^n} \e^{\i \turn \ip [\R^n] x \eta} \sigma_1(x, \xi + \eta) \Fourier [\R^n] \sigma_2(\eta, \xi) \dd \eta \notag\\
    &\sim \sum_{\abs \alpha \leq N} \frac {(\i \turn)^{-\abs \alpha}} {\alpha!} \iD{\xi^\alpha} \sigma_1(x, \xi) \iD{x^\alpha} \sigma_2(x, \xi),
    \label{eq:asymptotic_sum_for_composition}
\end{align}
where the second line was obtained by a Taylor development of $\sigma_1$.
Unfortunately,
as we cannot guarantee that~\eqref{eq:asymptotic_sum_for_composition} will converge as $N \to \infty$,
the above also means that we shall have to resort to defining \emph{approximate limits}.

\begin{definition}[Symbol classes]
    Assume $0 \leq \delta < \rho \leq 1$.
    We shall say that $\sigma \in \SmoothFunctions {\R^n \times \R^n}$ is a \emph{symbol of order $m$}
    if for each $\alpha, \beta \in \N^n$,
    there exists $C \geq 0$ such that
    \begin{align*}
        \sup_{(x, \xi) \in \R^n \times \R^n} \abs {\iD{\xi^\alpha} \iD{x^\beta}\sigma(x, \xi)} \leq C (1 + \abs \xi)^{m - \rho \abs \alpha + \delta \abs \beta}.
    \end{align*}

    The class of such symbols will be denoted by $\SymbolClass [\R^n] m {\rho, \delta}$.
    Moreover, we let
    \begin{align*}
        \SmoothingSymbols [\R^n] \defeq \bigcap_{m \in \R} \SymbolClass [\R^n] m {\rho, \delta}
    \end{align*}
    and call its elements \emph{smoothing symbols}.
\end{definition}

\begin{theorem}[Symbolic calculus]
    Let three symbols
    $\sigma \in \SymbolClass m {\rho, \delta}$,
    $\sigma_1 \in \SymbolClass {m_1} {\rho, \delta}$,
    $\sigma_2 \in \SymbolClass {m_2} {\rho, \delta}$.
    We let
    \begin{align*}
        \sigma^{(*)}(x, \xi) &\sim \sum_{\alpha \in \N^n} \frac {(\i \turn)^{-\abs \alpha}} {\alpha!} \iD{\xi^\alpha} \iD{x^\alpha} \adj{\sigma(x, \xi)}\\
        (\sigma_1 \circ \sigma_2)(x, \xi) &\sim \sum_{\alpha \in \N^n} \frac {(\i \turn)^{-\abs \alpha}} {\alpha!} \iD{\xi^\alpha} \sigma_1(x, \xi) \iD{x^\alpha} \sigma_2(x, \xi),
    \end{align*}
    where the left-hand sides are thus defined modulo $\SmoothingSymbols [\R^n]$.

    The following properties hold.
    \begin{itemize}
        \item The equivalence class $\sigma_1 \circ \sigma_2$ belongs to $\SymbolClass [\R^n] {m_1 + m_2} {\rho, \delta}$ modulo $\SmoothingSymbols [\R^n]$.
        \item The equivalence class $\sigma^{(*)}$ belongs to $\SymbolClass [\R^n] m {\rho, \delta}$ modulo $\SmoothingSymbols [\R^n]$.
        \item
            There exists $\sigma^{(-1)} \in \SymbolClass [\R^n] {-m} {\rho, \delta}$ such that
            \begin{align*}
                \sigma \circ \sigma^{(-1)} = 1 \quad \text{mod } \SmoothingSymbols [R^n]
            \end{align*}
            if and only if $\sigma$ is \emph{elliptic},
            i.e. there exists $c \geq 0$ and $d \geq 0$ such that
            \begin{align*}
                \abs {\sigma(x, \xi)} \geq c \abs \xi^m
            \end{align*}
            for each $x \in \R^n$ and each $\xi \in \R^n$ satisfying $\norm [\R^n] \xi \geq d$.
    \end{itemize}
\end{theorem}

\section{Generalisation to other settings}

Pseudo-differential operators of type $(\rho, \delta)$ can be defined locally on manifolds provided that
% TODO cite Shubin, section 4
\begin{align*}
    1 - \rho \leq \delta < \rho,
\end{align*}
implying in particular that $\rho > \frac 1 2$.

\subsection{The compact case}

\subsection{The graded case}

\subsection{The motion group}

%\section{The Euclidean case}
%
%Given a partial differential operator
%\begin{align*}
%    T = \sum_{\abs \alpha \leq N} c_\alpha(x) \iD{x^\alpha}
%\end{align*}
%with smooth coefficients,
%let us recall that its \emph{characteristic polynomial} is given by the smooth map
%\begin{align*}
%    \sigma(x, \xi) \defeq \sum_{\abs \alpha \leq N} c_\alpha(x) (\i \turn \xi)^\alpha.
%\end{align*}
%
%Substituing $\iD{{x_i} }$ for $\i \turn \xi_i$ might, formally speaking, seem like a trivial and insignificant change,
%it is however good to keep in mind
%that smooth functions have their own operations and structures
%which could provide us with a different perspective.
%
%Note that the composition and adjunction of such operators can be written in terms of their characteristic polynomials.
%As an example,
%if $T_1$ and $T_2$ are two operators arising from the polynomials $\sigma_1(x, \xi)$ and $\sigma_2(x, \xi)$ via
%\begin{align*}
%    T_1 \defeq \sum_{\abs \alpha \leq N} \frac {(\i \turn)^{-\abs \alpha}} {\alpha!} \iD{\xi^\alpha} \sigma_1(x, 0) \iD{x^\alpha}
%    \quad
%    T_2 \defeq \sum_{\abs \alpha \leq N} \frac {(\i \turn)^{-\abs \alpha}} {\alpha!} \iD{\xi^\alpha} \sigma_2(x, 0) \iD{x^\alpha},
%\end{align*}
%then it follows from an elementary computation
%that the characteristic polynomial of $T_1 \circ T_2$ is given by the finite sum
%\begin{align}
%    (\sigma_1 \circ \sigma_2)(x, \xi) = \sum_{\alpha \in \N^n} \frac {(\i \turn)^{-\abs \alpha}} {\alpha!} \iD{\xi^\alpha} \sigma_1(x, \xi) \iD{x^\alpha} \sigma_2(x, \xi)
%    \label{eq:composition_formula_for_characteristic_polynomials},
%\end{align}
%while the adjoint of $T \defeq T_1$ arises from
%\begin{align}
%    \sigma^{(*)}(x, \xi) = \sum_{\alpha \in \N^n} \frac {(\i \turn)^{-\abs \alpha}} {\alpha!} \iD{\xi^\alpha} \iD{x^\alpha} \adj{\sigma(x, \xi)},
%    \label{eq:adjunction_formula_for_characteristic_polynomials}
%\end{align}
%where in the above $\sigma \defeq \sigma_1$.
%
%\subsection{Symbolic point of view}
%
%We observe that
%the terms appearing in the sums~\eqref{eq:composition_formula_for_characteristic_polynomials} or~\eqref{eq:adjunction_formula_for_characteristic_polynomials} make sense for smooth functions $\sigma \in \SmoothFunctions {\R^n \times \R^n}$ that are not necessarily characteristic polynomials.
%A natural question is thus the following:
%can we attach each smooth function $\sigma \in \SmoothFunctions {\R^n \times \R^n}$ to a different, more general linear operator,
%in such a way that both~\eqref{eq:composition_formula_for_characteristic_polynomials} and~\eqref{eq:adjunction_formula_for_characteristic_polynomials} hold?
%If the answer is affirmative,
%we shall see that
%\eqref{eq:composition_formula_for_characteristic_polynomials} is a convenient tool to construct approximate inverse to a suitable class of operators,
%justifying our approach.
%
%Moreover,
%the \emph{order} of a differential operator is more easily generalised on the ``characteristic polynomial'' side.
%Naively, we could say that $\sigma \in \SmoothFunctions {\R^n \times \R^n}$ is of order $m \in \R$ if
%\begin{align*}
%    \sup_{(x, \xi) \in \R^n \times \R^n} \abs {\sigma(x, \xi)} \leq C (1 + \abs \xi)^m.
%\end{align*}
%However,
%the following problems would arise.
%\begin{enumerate}
%    \item How do we define the convergence of the sums~\eqref{eq:composition_formula_for_characteristic_polynomials} and~\eqref{eq:adjunction_formula_for_characteristic_polynomials}?
%    \item Is it true that adjunction, as in~\eqref{eq:adjunction_formula_for_characteristic_polynomials}, preserves the order?
%    \item Does the composition of two functions $\sigma_1$, $\sigma_2$ of order $m_1$ and $m_2$ according to~\eqref{eq:composition_formula_for_characteristic_polynomials} yield a $\sigma$ of order $m_1 + m_2$?
%\end{enumerate}
%
%It appears that all three issues can be solved by requiring
%that the ``order'' of the terms in~\eqref{eq:adjunction_formula_for_characteristic_polynomials} and~\eqref{eq:composition_formula_for_characteristic_polynomials} strictly decrease to $-\infty$ as $\abs \alpha$ increases.
%We are therefore led to the following definition.
%\begin{definition}[Symbol classes]
%    Assume $0 \leq \delta < \rho \leq 1$.
%    We shall say that $\sigma \in \SmoothFunctions {\R^n \times \R^n}$ is a \emph{symbol of order $m$}
%    if for each $\alpha, \beta \in \N^n$,
%    there exists $C \geq 0$ such that
%    \begin{align*}
%        \sup_{(x, \xi) \in \R^n \times \R^n} \abs {\iD{\xi^\alpha} \iD{x^\beta}\sigma(x, \xi)} \leq C (1 + \abs \xi)^{m - \rho \abs \alpha + \delta \abs \beta}.
%    \end{align*}
%
%    The class of such symbols will be denoted by $\SymbolClass [\R^n] m {\rho, \delta}$.
%    Moreover, we let
%    \begin{align*}
%        \SmoothingSymbols [\R^n] \defeq \bigcap_{m \in \R} \SymbolClass [\R^n] m {\rho, \delta}
%    \end{align*}
%    and call its elements \emph{smoothing symbols}.
%\end{definition}
%
%In order to make sense of infinite sums of symbols,
%we shall use the following result,
%which is proved in~\cite[Proposition~2.5.33]{RuzhanskyTurunen10}.
%
%\begin{lemma}[Asymptotic sum of symbols]
%    Given a sequence $\sigma_j \in \SymbolClass [\R^n] {m_j} {\rho, \delta}$, $j \in \N$ such that $(m_j)_{j \in \N}$ is a \emph{strictly decreasing} sequence of real numbers,
%    there exists a unique symbol $\sigma \in \SymbolClass [\R^n] {m_0} {\rho, \delta}$ modulo $\SmoothingSymbols [\R^n]$ such that
%    \begin{align*}
%        \sigma - \sum_{j = 0}^N \sigma_j \in \SymbolClass {m_{N + 1}} {\rho, \delta},
%    \end{align*}
%    in which case we write
%    \begin{align*}
%        \sigma \sim \sum_{j = 0}^{+\infty} \sigma_j.
%    \end{align*}
%\end{lemma}
%
%We are now ready to state the result
%that shows the power of symbolic calculus.
%
%\begin{theorem}[Symbolic calculus]
%    Let three symbols
%    $\sigma \in \SymbolClass m {\rho, \delta}$,
%    $\sigma_1 \in \SymbolClass {m_1} {\rho, \delta}$,
%    $\sigma_2 \in \SymbolClass {m_2} {\rho, \delta}$.
%    We let
%    \begin{align*}
%        \sigma^{(*)}(x, \xi) &\sim \sum_{\alpha \in \N^n} \frac {(\i \turn)^{-\abs \alpha}} {\alpha!} \iD{\xi^\alpha} \iD{x^\alpha} \adj{\sigma(x, \xi)}\\
%        (\sigma_1 \circ \sigma_2)(x, \xi) &\sim \sum_{\alpha \in \N^n} \frac {(\i \turn)^{-\abs \alpha}} {\alpha!} \iD{\xi^\alpha} \sigma_1(x, \xi) \iD{x^\alpha} \sigma_2(x, \xi),
%    \end{align*}
%    where the left-hand sides are thus defined modulo $\SmoothingSymbols [\R^n]$.
%
%    The following properties hold.
%    \begin{itemize}
%        \item The equivalence class $\sigma_1 \circ \sigma_2$ belongs to $\SymbolClass [\R^n] {m_1 + m_2} {\rho, \delta}$ modulo $\SmoothingSymbols [\R^n]$.
%        \item The equivalence class $\sigma^{(*)}$ belongs to $\SymbolClass [\R^n] m {\rho, \delta}$ modulo $\SmoothingSymbols [\R^n]$.
%        \item
%            There exists $\sigma^{(-1)} \in \SymbolClass [\R^n] {-m} {\rho, \delta}$ such that
%            \begin{align*}
%                \sigma \circ \sigma^{(-1)} = 1 \quad \text{mod } \SmoothingSymbols [R^n]
%            \end{align*}
%            if and only if $\sigma$ is \emph{elliptic},
%            i.e. there exists $c \geq 0$ and $d \geq 0$ such that
%            \begin{align*}
%                \abs {\sigma(x, \xi)} \geq c \abs \xi^m
%            \end{align*}
%            for each $x \in \R^n$ and each $\xi \in \R^n$ satisfying $\norm [\R^n] \xi \geq d$.
%    \end{itemize}
%\end{theorem}
%
%\subsection{Quantisation and pseudo-differential calculus}
%
%While the previous subsection offers a formally attractive framework,
%our efforts would only be fruitful
%provided we could associate our symbols with an operator,
%in a way that generalises the association of a characteristic polynomial with its corresponding differential operator.
%Moreover, we will want $\sigma^{(*)}$ and $\sigma_1 \circ \sigma_2$ to correspond to the adjunction and composition of operators respectively.
%
%As our functions $\sigma$ are now much more general,
%the substitution of $\xi$ for $\iD x$ will of course not work.
%However, taking advantage of the \emph{global} group structure of $\R^n$,
%we easily check that the following map $\Op [\R^n]$
%\begin{align}
%    \Op [\R^n] (\sigma) \phi(x)
%    &\defeq \int_{\R^n} \e^{\i \turn \ip [\R^n] x \xi} \sigma(x, \xi) \Fourier [\R^n] \phi(\xi) \dd \xi
%    \label{eq:intro:Euclidean:quantisation}
%\end{align}
%not only \emph{does} send a characteristic polynomial to its corresponding partial differential operator
%but also is far more general in scope.
%The process of associating an operator with a symbol is known as \emph{quantisation}.
%
%We are thus naturally led to the following definition.
%\begin{definition}[Operator classes]
%    We shall say that a map $T : \Schwartz {\R^n} \to \Schwartz {\R^n}$ is of \emph{order $m$ and type $(\rho, \delta)$} if
%    \begin{align*}
%        T \in \OperatorClass m {\rho, \delta}
%        \defeq \Op [\R^n] \left(\SymbolClass [\R^n] m {\rho, \delta}\right).
%    \end{align*}
%
%    Moreover,
%    if $T$ belongs to
%    \begin{align*}
%        \SmoothingOperators [\R^n] \defeq \bigcap_{m \in \R} \OperatorClass [\R^n] m {\rho, \delta},
%    \end{align*}
%    we shall say that $T$ is a \emph{smoothing operator}.
%\end{definition}
%
%We need to check the above define an appropriate notion of order.
%Ideally, we would like to answer the following questions in the affirmative.
%
%\begin{itemize}
%    \item Does the composition of two operators of order $m_1$ and $m_2$ yield an operator of order $m_1 + m_2$?
%    \item Does taking the adjoint preserve the order?
%    \item Does the order control the loss of derivatives?
%        By this, we mean that if $T \in \OperatorClass m {\rho, \delta}$,
%        is it true that $T$ extends to a continuous map
%        \begin{align*}
%            T : \Sobolev [\R^n] s \to \Sobolev [\R^n] {s - m}?
%        \end{align*}
%\end{itemize}
%
%The above questions are the object of the following result.
%\begin{theorem}
%    Assume that $1 \geq \rho > \delta \geq 0$.
%    \begin{enumerate}
%        \item If $T \in \SmoothingOperators [\R^n]$,
%            then it extends continuously to a map
%            \begin{align*}
%                T : \TemperedDistributions {\R^n} \to \SmoothFunctions {\R^n}.
%            \end{align*}
%        \item If $T_1 = \Op [\R^n] (\sigma_1) \in \OperatorClass {m_1} {\rho, \delta}$ and $T_2 = \Op [\R^n] (\sigma_2) \in \OperatorClass {m_2} {\rho, \delta}$,
%            then we have
%            \begin{align*}
%                T_1 \circ T_2
%                = \Op [\R^n] (\sigma_1 \circ \sigma_2)
%                \quad \text{ mod }\ \SmoothingOperators [\R^n].
%            \end{align*}
%
%            In particular,
%            the composition of an operator in $\OperatorClass [\R^n] {m_1} {\rho, \delta}$
%            with an operator in $\OperatorClass [\R^n] {m_2} {\rho, \delta}$
%            yields an operator in $\OperatorClass [\R^n] {m_1 + m_2} {\rho, \delta}$.
%        \item If $T = \Op(\sigma) \in \OperatorClass [\R^n] m {\rho, \delta}$,
%            then its adjoint satisfies
%            \begin{align*}
%                \adj T
%                = \Op [\R^n] (\sigma^{(*)})
%                \quad \text{ mod }\ \SmoothingOperators [\R^n].
%            \end{align*}
%
%            In particular,
%            $\OperatorClass [\R^n] m {\rho, \delta}$ is stable under taking the adjoint.
%        \item $T = \Op [\R^n] (\sigma) \in \OperatorClass [\R^n] m {\rho, \delta}$
%            has an inverse modulo $\SmoothingOperators [\R^n]$
%            if and only if $\sigma$ is \emph{elliptic}.
%    \end{enumerate}
%\end{theorem}
%
%\section{Generalisation to other Lie groups}
%
%It would be desirable to have a symbolic calculus on other settings as well.
%Naturally, that setting should have a \emph{differential structure} and a \emph{Fourier transform}.
%% TODO Add requirements
%
%It follows that we should aim to establish a pseudo-differential calculus on \emph{locally compact Lie groups with polynomial growth}.
%
%\subsection{The abelian case}
%
%Naturally, the first Lie groups for which we seek to establish a pseudo-differential calculus is that of \emph{abelian group}
%As their irreducible representations are \emph{one-dimensional} (see e.g.~\cite[Corollary 6.3.26]{RuzhanskyTurunen10}),
%the Fourier analysis becomes considerably simpler.
%Since all abelian Lie groups are isomorphic to $\R^m \times \T^n$,
%we need only treat the case of $\T^n$.
%
%This time,
%the Fourier transform is the classical Fourier series,
%and differential operators act like their \emph{discrete} characteristic polynomial (i.e. restricted to $\Z^n$).
%As a result,
%a differential operator $T$ arising from a discrete polynomial $\sigma(x, \xi)$ is written
%\begin{align*}
%    T \defeq \sum_{\abs \alpha \leq N} \frac 1 {\alpha!} \DifferenceOperatorOrder [\T^n] \alpha \sigma_1(x, 0) \iD{x^\alpha},
%\end{align*}
%so that~\eqref{eq:composition_formula_for_characteristic_polynomials} becomes
%\begin{align}
%    \sigma_3(x, \xi) = \sum_{\alpha \in \N^n} \frac 1 {\alpha!} \DifferenceOperatorOrder [\T^n] \alpha \sigma_1(x, \xi) \iD{x^\alpha} \sigma_2(x, \xi)
%    \label{eq:composition_formula_for_discrete_characteristic_polynomials}
%\end{align}
%
%A symbolic calculus is obtained in a similar fashion as $\R^n$,
%with discrete differences instead of derivatives in $\xi$.
%In fact, the proofs are either identical to the classcial Euclidean case,
%or considerably simplified by the compactness of $\T^n$.
%
%\subsection{Compact groups}
%
%As the first non-specific case,
%a difficulty that arises is how to replace $\DifferenceOperatorOrder [\T^n] \alpha$ and $\iD{\xi^\alpha}$
%in~\eqref{eq:composition_formula_for_characteristic_polynomials} and~\eqref{eq:composition_formula_for_discrete_characteristic_polynomials} respectively.
%
%It was observed for the first time in~\cite{RuzhanskyTurunen10}
%that we could generalise the above operators by considering
%\begin{align*}
%    \DifferenceOperator q \Fourier f
%    \defeq \Fourier \{q f\},
%\end{align*}
%where $q \in \SmoothFunctions \Group$ is a smooth function on our group vanishing at the identify $e_\Group$.
%It is clear that
%\begin{align*}
%    \DifferenceOperator [\R^n] {x_j} &= \iD{\xi^\alpha}\\
%    \DifferenceOperator [\T^n] {\e^{\i \turn x_j} - 1} &= \DifferenceOperatorOrder [\T^n] {e_j}
%\end{align*}
%
%\subsection{Graded Lie groups}
%
%\section{The motion group}
%
%Good examples of locally compact Lie group that are neither \emph{abelian} nor \emph{compact} are the \emph{motion groups}, i.e.
%\begin{align*}
%    \Group \defeq \R^n \ltimes \SpecialOrthogonalGroup n,
%\end{align*}
%and they will be the object of the present thesis.
%Note that $\Group$ can equivalently defined as the \emph{smallest group of affine transformations that contains both rotations and translations}.
%
%In addition to the fact that it is one of the simplest groups on which symbolic calculus had not been developed,
%it also appears naturally in applications in fields such as biology and robotics.
%In fact, those fields have started using the Fourier Transform on $\Group$ as a mean to solve convolution and partial differential equations,
%like the paper \cite{ChirikjianKyatkin00} shows.
%
%Our aim is to show the following.
%
%\begin{theorem}[Pseudo-differential calculus on the motion group]
%    Suppose that $G$ is the smallest subgroup of affine transformations of the Euclidean space
%    which contains both translations and rotations.
%
%    We can define a collection $(\OperatorClass m {})_{m \in \R}$
%    of sets of operators
%    which satisfies the following properties.
%    \begin{enumerate}
%        \item If $X$ is a left-invariant vector field on $\Group$,
%            then $X \in \OperatorClass 1 {}$.
%        \item For every $m \in \R$, $\OperatorClass m {}$ is stable under taking the \emph{adjoint}.
%        \item If $T_1 \in \OperatorClass {m_1} {}$ and $T_2 \in \OperatorClass {m_2} {}$, then the \emph{composition} $T_1 \circ T_2$ belongs to $\OperatorClass {m_1 + m_1} {}$.
%    \end{enumerate}
%\end{theorem}
