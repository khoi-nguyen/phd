\chapter{Introduction}

Since their inception in the 1960s,
pseudo-differential operators have become standard tools in the study of partial differential equations.
On $\R^n$,
these operators are defined \emph{globally} as arising from smooth maps called \emph{symbols} via the Euclidean Fourier Transform.

While pseudo-differential operators can be defined locally on any connected manifold,
two issues arise.
Firstly,
our operators can only be defined provided that a condition such as
\begin{align*}
    1 - \rho \leq \delta < \rho
\end{align*}
holds (see e.g.~\cite[Section 4]{Shubin01}),
while their Euclidean counterparts make sense without restriction on $\rho$ and $\delta$.
Secondly,
a global notion of \emph{symbol} cannot be invariantly defined.

The first successful attempt at defining a global pseudo-differential calculus
on a non-Euclidean manifold is due to \citeauthor{RuzhanskyTurunen10},
and their treatment of compact Lie groups.
They defined the global symbol of an operator via the group Fourier transform of the right-convolution kernel,
which in particular is  \emph{matrix}-valued
due to the non-commutativity of the group law.
Particularly instrumental to their success was their definition of \emph{difference operators},
which generalises derivatives in Fourier variables from the Euclidean case.
It is worth remembering that,
while good properties of left-invariant pseudo-differential operators
are ensured by the Plancherel theory,
it is the estimates on the derivatives of the symbol and thus in the Fourier variables,
that extend the desired properties onto more general pseudo-differential operators.
More specifically,
their concepts of difference operators allowed them to firstly ensure pseudo-differential operators
have Calder\'on-Zygmund kernels,
and secondly control the order of the terms that appear in asymptotic sums
arising from the composition and adjunction of operators.

It is natural to wonder whether a global pseudo-differential calculus can be defined on other,
more complicated, settings.
To this end,
we observe that the Ruzhansky-Turunen theory relies on a Fourier-Plancherel theory,
while good properties of the operators rely on an understanding of the theory of singular integrals.
Therefore,
the natural settings to investigate are Lie groups with polynomial volume growth.

In 2013,
\citeauthor{FischerRuzhansky12} treated the case of graded nilpotent Lie groups.
In particular,
they dealt with the technical issues which arise from working with infinite-dimensional representations,
and more crucially,
non-central sub-Laplacians or Rockland operators.
They also present a strategy to obtain the composition and adjunction formulae,
which consists of proving symbolic estimates on the functional calculus for the sub-Laplacian or the Rockland operator,
and subsequently proving kernel estimates for general symbols.

The aim of this thesis is to study the case of the Euclidean motion group,
which is the smallest group of affine transformations containing both translations and rotations.
Our analysis will encounter difficulties due to the absence of properties such as compactness and commutativity of the group law,
and the lack of a dilation structure.
Furthermore, the infinite dimensional nature of the representations
and the fact that the Laplacian is not central also complicates matters.
Nevertheless, our interest in this group is justified for two reasons:
firstly, it is one of the more elementary non-compact groups not yet studied;
secondly, by its very definition,
it naturally describes rigid motions such as the position and orientation of a robot arm or polymer chains like DNA.
As a result,
many partial differential equations from engineering, biology and physics
are naturally expressed on the motion group.
More detail on the different applications can be found in \cite{ChirikjianWang04,ChirikjianKyatkin00,Chirikjian13}.

\section{Survey}

In this section,
we informally discuss an overview of the theory on compact groups and graded Lie groups.
For more detail,
the reader is invited to consult~\cite{RuzhanskyTurunen10} for the compact case
(with the understanding that the composition and adjunction formulae are only fully proved in~\cite{Fischer2015}),
and~\cite{FischerRuzhansky16} for the graded case.
To avoid repetition,
we shall discuss both cases simultaneously,
and contrast them to the case of the motion group when appropriate.

Essentially,
\cite{Fischer2015,FischerRuzhansky16} and the present document
all follow the following steps
to develop a global pseudo-differential calculus.

\begin{description}
    \item[Step 1.] Define a sufficiently general Fourier Transform.
    \item[Step 2.] Study the (sub-)Laplacian or Rockland operator,
        and the corresponding Sobolev spaces.
    \item[Step 3.] Find a family of smooth functions which have a Leibniz-like rule and can be used for a Taylor development.
    \item[Step 4.] Define the symbol classes.
    \item[Step 5.] Functional calculus of the (sub-)Laplacian or Rockland operator.
    \item[Step 6.] Obtain kernel estimates.
    \item[Step 7.] Prove the composition and adjunction formulae.
\end{description}
