\chapter{Introduction}

In the Euclidean setting,
it is a known fact that constant-coefficient differential operators act like their characteristic polynomials on the Fourier Transform side.
Moreover, composing, taking the adjoint, or inverting such operators can be done by multiplication, adjunction or inversion of their associated polynomials.

Although the above properties do not hold for general differential operators with smooth coefficients,
the composition and adjunction of such operators can still be written solely in terms of their characteristic polynomials.
As an example,
if $T_a$ and $T_b$ are two operators arising from the polynomials $a(x, \xi)$ and $b(x, \xi)$ via
\begin{align*}
    T_a \defeq \sum_{\abs \alpha \leq N} \frac 1 {\alpha!} \iD{\xi^\alpha} a(x, 0) \iD{x^\alpha}
    \quad
    T_b \defeq \sum_{\abs \alpha \leq N} \frac 1 {\alpha!} \iD{\xi^\alpha} b(x, 0) \iD{x^\alpha},
\end{align*}
then it follows from an elementary computation
that the characteristic polynomial of $T_c \defeq T_a \circ T_b$ is given by the finite sum
\begin{align}
    c(x, \xi) = \sum_{\alpha \in \N^n} \frac 1 {\alpha!} \iD{\xi^\alpha} a(x, \xi) \iD{x^\alpha} b(x, \xi)
    \label{eq:composition_formula_for_characteristic_polynomials}.
\end{align}
Importantly,
\eqref{eq:composition_formula_for_characteristic_polynomials} means that
composition is reduced to summing and multiplying smooth functions.

Laurent Schwartz's \emph{Kernel Theorem} and his extension of the \emph{Fourier transform} allows us to extend the concept of characteristic polynomial much beyond the class of ordinary differential operators.
The resulting concept, called the \emph{symbol} of an operator,
is formally defined by taking the Fourier transform of the convolution kernel,
i.e.\ if $T$ is formally given by
\begin{align*}
    T \phi(x) \defeq \conv \phi {\kappa_x}(x),
\end{align*}
then we define its symbol via
\begin{align*}
    \sigma(x, \xi) \defeq \Fourier \kappa_x(\xi).
\end{align*}

Can we define a generalised notion of differential operator
with an approprate notion of order,
stable under composition and adjunction,
and for which both these operations can be expressed in terms of their symbols?
