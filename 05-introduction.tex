\chapter{Introduction}

Since their inception in the 1960s,
pseudo-differential operators have become standard and natural tools in the study of partial differential equations,
particularly when elliptic or hypoelliptic operators are concerned.

On $\R^n$,
these operators are defined \emph{globally} as arising from smooth maps called \emph{symbols} via the Euclidean Fourier Transform,
the latter generalizing characteristic polynomials.
Their usefulness arises from the fact that they enjoy similar properties as their differential counterparts,
such as a notion of \emph{order} that behaves well under composition and adjunction,
as well as expected boundedness properties between the adequate Sobolev spaces.
Crucially,
the \emph{adjunction} and \emph{composition} formulae claim that
both these operations applied to pseudo-differential operators can be approximately expressed on the symbolic side via a formula that is formally virtually identical to the corresponding rule for characteristic polynomials.
In particular,
this means that elliptic pseudo-differential operators admit parametrices which are contained in the pseudo-differential calculus.
The literature on the subject is vast and
overviews can be found, for example, in~\cite{Shubin01,Hormander07}.

While pseudo-differential operators can be defined locally via local charts on any connected manifold,
two issues arise.
Firstly,
operators are usually defined provided that a condition on the \emph{type} such as
\begin{align*}
    1 - \rho \leq \delta < \rho
\end{align*}
holds (see e.g.~\cite[Section 4]{Shubin01}),
while their Euclidean counterparts make sense without restrictions on $\rho$ and $\delta$.
Secondly,
a global notion of \emph{symbol} cannot be invariantly defined,
making results which rely on a full symbol (such as Gårding's inequality) difficult to establish.

The first successful attempt at defining a global pseudo-differential calculus
on a non-Euclidean manifold is due to \citeauthor{RuzhanskyTurunen10},
and their treatment of compact Lie groups.
They defined the global symbol of an operator via the group Fourier transform of the right-convolution kernel,
which in particular is  \emph{matrix}-valued
due to the non-commutativity of the group law.
Particularly instrumental to their success was their definition of \emph{difference operators},
which generalize derivatives in Fourier variables from the Euclidean case.
It is worth remembering that,
while good properties of left-invariant pseudo-differential operators
are ensured by the Plancherel theory,
it is the estimates on the derivatives of the symbol and thus also in the Fourier variables,
that extend the desired properties onto more general pseudo-differential operators.
More specifically,
their concepts of difference operators allowed them to firstly ensure pseudo-differential operators
have Calder\'on-Zygmund kernels,
and secondly control both the first terms and the remainders of asymptotic sums
which arise from the composition and adjunction of operators.

Since then,
many results and applications have been obtained on compact Lie groups
by several authors including M.\ Ruzhansky, J.\ Wirth, V.\ Turunen and J.\ Delgado.
To cite a few,
these applications include
criteria for ellipticity and global hypoellipticity in terms of the global symbols~\cite{RuzhanskyTurunen10},
a proof of a sharp Gårding inequality~\cite{RuzhanskyTurunen11},
a global functional calculus \cite{RuzhanskyWirth14},
a study of $\Lebesgue p \Group$ Fourier multipliers~\cite{RuzhanskyWirth15},
and many others.

It is natural to wonder whether a global pseudo-differential calculus can be defined on other,
more complicated, settings.
To this end,
we observe that the Ruzhansky-Turunen theory relies on a Fourier-Plancherel theory,
while good properties of the operators rely on an understanding of the theory of singular integrals.
Therefore,
the natural settings to investigate are Lie groups with polynomial growth of the volume.

In 2013,
\citeauthor{FischerRuzhansky12} treated the case of graded nilpotent Lie groups.
In particular,
they dealt with the technical issues which arise from working with infinite-dimensional representations,
and more crucially,
non-central sub-Laplacians or positive Rockland operators.
They also present a strategy to obtain the composition and adjunction formulae,
which consists of proving symbolic estimates on the functional calculus for the sub-Laplacian or the Rockland operator,
and subsequently proving kernel estimates for general symbols.

The aim of this thesis is to study the case of the Euclidean motion group,
which is the smallest group of affine transformations on a Euclidean space containing both translations and rotations.
Our analysis will encounter difficulties due to the absence of properties such as compactness and commutativity of the group law,
and the lack of a dilation structure.
Furthermore, the infinite-dimensional nature of the representations,
and the fact that the Laplacian is not central, also complicates matters.
Nevertheless, our interest in this group is justified for two reasons:
firstly, it is one of the more elementary non-compact groups not yet studied in the context of global pseudo-differential calculus;
secondly, by its very definition,
it naturally describes rigid motions such as the position and orientation of a robot arm or polymer chains like DNA.
As a result,
many partial differential equations from engineering, biology and physics
are naturally expressed on the motion group.
More detail on the different applications can be found in \cite{ChirikjianWang04,ChirikjianKyatkin00,Chirikjian13}.

\section{Survey of the compact and graded cases}

In this section,
we informally discuss an overview of the theory on compact and graded Lie groups.
For more detail,
the reader is invited to consult~\cite{RuzhanskyTurunen10} for the compact case\footnote{%
    The adjunction and composition formulae are stated therein,
    although their correspending proofs contain a non-trivial gap which has since been filled but not published.
}
and~\cite{FischerRuzhansky16} for graded Lie groups.
To avoid repetition,
we shall discuss both settings simultaneously.

Essentially,
\cite{Fischer2015,FischerRuzhansky16} and the present document
all follow the following steps
to develop a global pseudo-differential calculus.

\begin{description}
    \item[Step 1.] Study the (sub-)Laplacian or positive Rockland operator,
        and the corresponding Sobolev spaces.
    \item[Step 2.] Study and generalize the group Fourier Transform.
    \item[Step 3.] Find a family of smooth functions which have a Leibniz-like rule and can be used for a Taylor development.
    \item[Step 4.] Define the symbol classes.
    \item[Step 5.] Study the functional calculus of the (sub-)Laplacian or positive Rockland operator.
    \item[Step 6.] Obtain kernel estimates.
    \item[Step 7.] Prove the composition and adjunction formulae.
\end{description}

We now present the above steps in more detail.
Unless stated otherwise,
$\Group$ will denote a compact or a graded Lie group until the end of this outline.

\subsection*{Step 1 - Sobolev spaces}

We start by fixing a (sub-)Laplacian or, in the graded case, a positive Rockland operator.
This allows us to define \emph{Sobolev spaces}.
We then need to show their basic properties,
the most important of which is the \emph{Sobolev embedding theorem}.

As our kernels and symbols will be linked via the Fourier transform,
the \emph{Plancherel theory} will allow to get $\Lebesgue 2 \Group$ and more generally $\Sobolev s$ estimates on the kernel from symbolic estimates.
We shall therefore rely on the Sobolev embedding to obtain further regularity results (e.g.~continuity).
The role of the Sobolev embedding theorem is thus pivotal in the study of the regularity and the singularities of the kernel.

Moreover,
as we require our pseudo-differential operators to be bounded as maps between suitable Sobolev spaces,
we expect these spaces to be somehow part of our symbol classes definition.

In the graded case,
Sobolev spaces are studied in~\cite[Chapter 4]{FischerRuzhansky16},
following the proofs of the stratified case (see~\cite{Folland75}).

In the case of the compact group,
the \emph{Laplace-Beltrami} operator associated with a normalized bi-invariant metric appears the natural candidate to define Sobolev spaces,
since its eigenspaces arise naturally from the group representations.

From this point onwards,
$\Laplacian$ will denote the operator with respect to which we have defined our Sobolev spaces.

\subsection*{Step 2 - Fourier analysis}

The main idea is to define an analogue of the Euclidean Kohn-Nirenberg quantization,
whereby the \emph{symbol} of a pseudo-differential operator is the \emph{Fourier transform of its convolution kernel}.
Our first step is thus to extend the group Fourier transform to a suitable subset of tempered distributions.

On our settings,
the Fourier transform is defined on $\Lebesgue 1 {\Group, \dd g}$ via
\begin{align*}
    \Fourier f(\xi)
    \defeq \int_\Group f(g) \adj{\xi(g)} \dd g,
\end{align*}
where $\dd g$ denotes a \emph{Haar measure},
$\xi \in \dualGroup \Group$,
and $\dualGroup \Group$ is the \emph{unitary dual} of $\Group$,
i.e.\ the set of all equivalence classes of irreducible, strongly continuous and unitary representations of $\Group$.

If $f \in \Lebesgue 1 \Group \cap \Lebesgue 2 \Group$,
we have the following \emph{Plancherel formula}
\begin{align*}
    \int_\Group \abs f^2 \dd g
    = \int_{\dualGroup \Group}
    \tr(\Fourier f(\xi) \adj{\Fourier f(\xi)})
    \dd \mu_{\dualGroup \Group}(\xi),
\end{align*}
where $\dd \mu_{\dualGroup \Group}$ denotes the so-called \emph{Plancherel measure},
while any $f \in \Schwartz \Group$ may be recovered from its Fourier Transform via
the following \emph{inverse formula}
\begin{align*}
    f(g) \defeq
    \int_{\dualGroup \Group}
    \tr(\xi(g) \Fourier f(\xi))
    \dd \mu_{\dualGroup \Group}(\xi),
    \quad g \in \Group.
\end{align*}

The problem of defining an extension of the Fourier Transform to make sense of a \emph{Kohn-Nirenberg quantization} procedure was solved in~\cite[Subsection 5.1.1]{FischerRuzhansky16} for a very large class of groups,
since their solution relies exclusively on the \emph{Plancherel} theorem and the definition of Sobolev spaces.

\subsection*{Step 3 - Difference operators}

In~\cite{RuzhanskyTurunen10},
\citeauthor{RuzhanskyTurunen10} introduce the following notion of difference operator
to generalize the derivatives in Fourier variables which appear in the Euclidean definition of symbol classes.
If $q \in \SmoothFunctions \Group$,
they define
\begin{align*}
    \DifferenceOperator [\Group] q \Fourier f \defeq \Fourier \{q f\}.
\end{align*}

When $\Group = \R^n$,
we recover the classical derivatives in frequency with the collection
$\Delta \defeq \{x_1, \dots, x_n \in \SmoothFunctions {\R^n}\}$,
where $x_i$ is the function that maps $x$ onto its $i$-th coordinate.

We now need to find the properties that a family of functions
\begin{align*}
    \Delta \defeq \{q_1, \dots, q_{\dimDifferenceOperators} \in \SmoothFunctions \Group\}
\end{align*}
must satisfy in order to develop a global pseudo-differential calculus.
We then let
\begin{align*}
    q^\alpha \defeq q^{\alpha_1}_1 \dots q^{\alpha_{\dimDifferenceOperators}}_{\dimDifferenceOperators},   
    \quad \DifferenceOperatorOrder \alpha \defeq \DifferenceOperator {q^\alpha(\dummy^{-1})}.
\end{align*}

Upon inspection of the different proofs in the Euclidean~\cite{Stein93},
compact~\cite{RuzhanskyTurunen10,Fischer2015} and graded~\cite{FischerRuzhansky16} cases,
the main requirements appear to be the following.
\begin{itemize}
    \item We can use a \emph{Taylor development} at the origin.
        This is necessary because the adjunction and the composition are difficult to express on the symbolic side without using a Taylor development.
    \item We have a reasonable \emph{Leibniz rule} for the associated difference operators.
        This ensures for example that the order of the composition of two left-invariant pseudo-differential operators is the sum of the orders.
    \item Ensure our pseudo-differentials of order $0$ are \emph{Calder\'on-Zygmund}.
        While we can reasonably expect the Plancherel theory to ensure $\Lebesgue 2 \Group$ boundedness,
        we rely on the theory of singular integrals to have the boundedness on $\Lebesgue p \Group$.
\end{itemize}

The first couple of conditions naturally require that each $q_j$ vanish at $e_\Group$ and that $\{dq_j(e) : 1 \leq j \leq \dimDifferenceOperators\}$ span the tangent space $T_{e_\Group} G$.
Such a family is called \emph{admissible} in~\cite{RuzhanskyTurunenWirth10} and subsequent literature.

Intuitively,
our definition of symbol classes will ensure that the regularity of the kernel will increase every time we multiply it by one of the $q_j \in \Delta$,
exactly like in the Euclidean case.
In particular,
this means that our kernels will be smooth outside of
\begin{align}
    \bigcap_{q \in \Delta} \{g \in \Group : q(g) = 0\},
    \label{eq:common_zeros_of_difference_operators}
\end{align}
while the above set may contain singularities.
However,
the Calder\'on-Zygmund theory requires that the singularity be confined to the identity.
Therefore,
\cite{RuzhanskyTurunenWirth10} defines $\Delta$ to be \emph{strongly admissible}
if~\eqref{eq:common_zeros_of_difference_operators} equals $\{e_\Group\}$.

\subsubsection*{The compact case}

In~\cite{RuzhanskyTurunenWirth10},
the authors show that a well-chosen finite subfamily $\Delta$ of
\begin{align*}
    \{(\tau - \tau(e_\Group))_{m n} : \tau \in \dualGroup \Group, 1 \leq m, n \leq \dimRep \tau\}
\end{align*}
can be strongly admissible.

Importantly,
this family yields the following \emph{Leibniz-like} rule
\begin{align*}
    \DifferenceOperatorOrder \alpha (\Fourier f_1 \Fourier f_2)
    = \sum_{\abs \alpha \leq \abs {\alpha_1} + \abs {\alpha_2} \leq 2 \abs \alpha}
    c^\alpha_{\alpha_1, \alpha_2}
    (\DifferenceOperatorOrder {\alpha_1} \Fourier f_1)
    (\DifferenceOperatorOrder {\alpha_2} \Fourier f_2),
\end{align*}
which is good enough to establish the calculus,
but slightly complicates some proofs related to the construction of parametrices.

The compactness of the group allows us to show that the calculus does \emph{not} depend on the choice of strongly admissible difference operators~\cite{RuzhanskyTurunenWirth10,Fischer2015}.
Such a proof relies strongly on the \emph{finite volume} of our group,
and it is important to note at this stage that the result does not hold even on $\R^n$.

\subsubsection*{The graded case}

The behaviour of difference operators in the graded case is somehow closer to the Euclidean setting.
In~\cite{FischerRuzhansky16},
the authors consider the difference operators associated with \emph{homogeneous polynomials}.
The dilation structure
shows that we have a Leibniz rule if we consider a sum on homogeneous degrees
(see~\cite[Subsection 5.2.1]{FischerRuzhansky16}),
while the strong admissibility is easily obtained by choosing a suitable basis.

\subsection*{Step 4 - Symbol and operator classes}

Very informally,
we say that $\sigma(g, \xi)$ is a \emph{symbol} if there exists $\kappa \in \SmoothFunctions {\Group, \TemperedDistributions \Group}$ such that
\begin{align*}
    \sigma(g, \xi) \defeq \Fourier \{\kappa(g, \dummy)\}(\xi)
\end{align*}
is defined.
The distribution $\kappa$ is called the \emph{kernel associated} wih $\sigma$.

Moreover,
given $m \in \R$
we shall say $\sigma \in \SymbolClass m {\rho, \delta}$ if
\begin{align*}
    \norm [\Lin {\mathcal H_\xi}] {%
        \xi \BesselPotential {-m + \rho \abs \alpha - \delta \abs \beta + \gamma}
        \LeftDifferentialOperator \beta \DifferenceOperatorOrder \alpha \sigma(g, \xi)
        \xi \BesselPotential {-\gamma}
    }
    < \infty
\end{align*}
uniformly in $g \in \Group$
and essentially uniformly in $\xi \in \dualGroup \Group$ (with respect to the Plancherel measure).
Note that if $\Laplacian$ is central,
we can assume $\gamma = 0$.
It should be clear that when $\Group = \R^n$,
we obtain the classical definition of symbol classes.

We also let
\begin{align*}
    \SmoothingSymbols \defeq \bigcap_{m \in \R} \SymbolClass m {\rho, \delta}
\end{align*}
denote the set of \emph{smoothing symbols}.

We define a \emph{pseudo-differential operator} to be an operator arising from a symbol via an analogue of the \emph{Kohn-Nirenberg quantization}.
More precisely,
the operator
\begin{align*}
    \Op(\sigma) \phi(g)
    \defeq \InverseFourier \left\{ \xi \in \dualGroup \Group \mapsto \xi(g) \sigma(g, \xi) \Fourier \phi(\xi)\right\}(g)
\end{align*}
is the \emph{pseudo-differential operator} associated with the symbol $\sigma$.
We easily check that $\Op [\R^n]$ is the usual Kohn-Nirenberg quantization.

Naturally,
the quantization allows us to define \emph{operator classes} via
\begin{align*}
    \OperatorClass m {\rho, \delta} &\defeq \Op(\SymbolClass m {\rho, \delta}),\\
    \SmoothingOperators &\defeq \Op(\SmoothingSymbols).
\end{align*}

For more information,
see~\cite[Section 5.1]{FischerRuzhansky16} for the graded case;
and~\cite[Section 3.1]{Fischer2015} for the compact case.

\subsection*{Step 5 - Functional calculus}

On $\R^n$,
the composition and adjunction formulae can be proved by carrying out our computations on the frequency space $\dualGroup {\R^n} = \R^n$,
exploiting the fact that the Euclidean space is \emph{abelian} and we have explicit expressions of our difference operators.
Since a group Fourier transform is \emph{operator-valued} in more complicated settings,
it makes sense to attempt to do our calculations on the kernel side,
especially since this is how difference operators were defined.
Unfortunately,
kernels may have a singularity at the origin,
unlike symbols, which are smooth.
Therefore, two questions naturally arise.

\begin{itemize}
    \item What is the \emph{strength} of the singularity of the kernel at the origin?
    \item Can we \emph{approximate} a symbol $\sigma$ by a sequence of smoothing symbols $(\sigma_n)_{n \in \N}$ so that the kernel can be approximated by the associated sequence of smooth kernels?
\end{itemize}

Looking at pseudo-differential operators such as $\BesselPotential m$
suggests that \emph{functional calculus} is the natural framework to answer such questions.

A crucial step of our analysis is to construct a family of smoothing symbols approximating the identity symbol.
Using the algebra structure of our symbol classes,
this will yield an important density result,
as we will be able to approximate any symbol with smoothing symbols,
and by extension any kernel with smooth kernels.

Ideally,
we would like to create a family $\{\eta_\epsilon = \Fourier p_\epsilon\}_{\epsilon \in (0, 1]} \subset \SmoothingSymbols$
such that
\begin{align}
    \SymbolSemiNorm m {\rho, \delta} N {\eta_\epsilon}
    &\leq \epsilon^\frac m \nu,
    \quad m < 0,
    \label{eq:survey:density_estimate:1}
    \\
    \SymbolSemiNorm m {\rho, \delta} N {\Id {} - \eta_\epsilon}
    &\leq \epsilon^\frac m \nu,
    \quad m \geq 0,
    \label{eq:survey:density_estimate:2}
\end{align}
where $\nu$ is the order of the operator $\Laplacian$.
Moreover,
to prove the Calder\'on-Zygmund property,
we need very good estimates on the kernels $p_\epsilon$.
Therefore, the natural candidate is the Fourier transform of the \emph{heat kernel}.

The family of symbols is used in the following proofs.
\begin{description}
    \item[Kernel estimates]

        Given a symbol $\sigma \in \SymbolClass m {\rho, \delta}$,
        we estimate the singularity of its kernel at the origin via
        \begin{align*}
            \abs {\kappa_g(h)}
            \leq
            \abs {\conv {\kappa_g} {p_\epsilon}(h)}
            +
            \abs {\conv {\kappa_g} {(\delta_{e_\Group} - p_\epsilon)(h)}},
        \end{align*}
        where we choose $\epsilon$ so that $\epsilon^\nu$ equals the distance between $h$ and $e_\Group$.

        The second term of the right-hand side is the one containing the singularity,
        and we need~\eqref{eq:survey:density_estimate:2} to control it.

    \item[Construction of asymptotic limits]

        One of the inherent characteristics of symbolic calculus is that compositions and adjunctions can only be expressed via infinite sums from which we cannot reasonably expect convergence.
        As certain applications such as the construction of parametrices rely on making sense of these sums in a suitable sense,
        it is important that we define a notion of \emph{asymptotic convergence}.

        The idea is the same as in the Euclidean case.
        We know that cutting off ``low frequencies'' will not affect the result modulo $\SmoothingSymbols$.
        Therefore, given a sequence of symbols $\sigma_j$ with strictly decreasing order,
        we thus define
        \begin{align*}
            \sum_{j = 0}^{+\infty} \sigma_j \defsim
            \sum_{j = 0}^{+\infty} \sigma_j (\Id {} - \eta_{\epsilon_j})
            \quad \text{modulo } \SmoothingSymbols,
        \end{align*}
        for a well chosen sequence $(\epsilon_j) \subset (0, 1]$ decreasing to $0$.

    \item[Density of smoothing symbols]

        Letting
        \begin{align*}
            \sigma_\epsilon \defeq \sigma \eta_\epsilon
        \end{align*}
        allows us to approximate $\sigma$ by smoothing symbols,
        while the error term
        \begin{align*}
            \sigma - \sigma_\epsilon = (\Id {} - \eta_\epsilon) \sigma
        \end{align*}
        can be controlled via our symbolic estimates on $\Id {} - \eta_\epsilon$.

        In particular,
        this allows us to show the composition and adjunction formulae only for
        for smoothing symbols and conclude by density.
\end{description}

\subsubsection{The compact case}

Following the classical heat kernel methods (see e.g.~\cite{FurioliMelziVeneruso06,VaropoulosSaloffCosteCoulhon92}),
\citeauthor{Fischer2015} shows symbolic estimates for the functional calculus of the Laplace-Beltrami operator in~\cite{Fischer2015}.
However, she proceeds slightly differently by letting
\begin{align*}
    \eta_\epsilon
    \defeq \chi(\epsilon \xi(\Laplacian [\Group])),
\end{align*}
where $\chi : \R^+ \to \R$ equals $1$ on $[0, 1]$ and has compact support.
To estimate the remainder,
she uses a \emph{Littlewood-Paley} decomposition
\begin{align*}
    \Id {} - \eta_\epsilon = \sum_{j = 1}^{+\infty} \eta_{\epsilon, j},
\end{align*}
where $\eta_{\epsilon, j} \defeq \eta_{2^{-j} \epsilon} - \eta_{2^{-j + 1} \epsilon}$
and the convergence is understood in the strong operator topology.

\subsubsection{The graded case}

The existence of such a family $\eta_\epsilon$ is ensured directly by the \emph{Hulanicki Theorem} (see \cite[Theorem 4.5.1]{FischerRuzhansky16}).
We can define
\begin{align*}
    \eta_\epsilon(\xi) \defeq \Fourier \{\e^{-\epsilon \Laplacian} \delta_{e_\Group}\}(\xi),
    \quad \xi \in \dualGroup \Group,
\end{align*}
and the aforementioned theorem ensures that this defines a smoothing symbol satisfying the appropriate symbolic estimates.

\subsection*{Steps 6 and 7 - Kernel estimates and calculus proofs}

As soon as we have obtained a good approximation of the identity symbol,
whether it is via a family $\eta_\epsilon, \epsilon \in (0, 1]$
or a Littlewood-Paley decomposition,
the proofs of the kernel estimates and subsequently the composition and adjunction formulae do not depend so much on the setting.
Note that only the case $\rho > \delta$ will be treated herein.

In order to better understand the importance of the kernel estimates,
let us sketch the proof of the \emph{adjunction formula}
while keeping in mind that the arguments for the composition formula are very similar.

Suppose that $T \defeq \Op(\sigma) \in \OperatorClass m {1, 0}$.

\begin{enumerate}
    \item As we intend to use a \emph{density} argument,
        we assume first that $\sigma$ is smoothing.
    \item We perform an exact computation of the kernel $\tilde \kappa$ of $\adj T$ in terms of the kernel $\kappa$ of $T$.

        More precisely, we show that $\adj T$ has right-convolution kernel
        \begin{align*}
            \tilde \kappa_g(h) \defeq \conj \kappa_{g h^{-1}}(h^{-1}).
        \end{align*}
    \item
        In particular, its \emph{symbol} is given by taking the Fourier transform of the above
        \begin{align*}
            \tilde \sigma(g, \xi) \defeq \Fourier \tilde \kappa_g(\xi)
            = \int_\Group \conj \kappa_{g h^{-1}}(h^{-1}) \adj{\xi(h)} \dd h.
        \end{align*}
    \item
        In the integral above,
        we take the \emph{Taylor development} with respect to the variable $h$ appearing in the subscript of the kernel
        \begin{align*}
            \tilde \sigma(g, \xi)
            &= \int_\Group (\sum_{\abs \alpha \leq N} q^{\alpha}(h^{-1}) \LeftDifferentialOperator [g] \alpha \conj \kappa_{g}(h^{-1}) + R_N(h)) \adj{\xi(h)} \dd h\\
            &= \sum_{\abs \alpha \leq N} \frac 1 {\alpha!} \DifferenceOperatorOrder \alpha \LeftDifferentialOperator \alpha \adj \sigma(g, \xi) + \int_\Group R_N(h) \adj{\xi(h)} \dd h.
        \end{align*}
    \item
        Using the Taylor remainder formula,
        the integrand on the right can be estimated by an expression such as
        \begin{align*}
            \abs {R_N(h)}
            \leq C d(h, e_\Group)^{N + 1} \sup_{\beta \in B} \sup_{\tilde h \in \Gamma_h} \abs {\LeftDifferentialOperator \beta \kappa_g(\tilde h)}
        \end{align*}
        so that the integral can be controlled via good estimates on $\kappa_g$.
    \item
        We now conclude the proof by density.
\end{enumerate}

For the compact case,
see~\cite[Sections 6.3 and 7.3]{Fischer2015};
for the graded case,
see~\cite[Sections 5.4 and 5.5]{FischerRuzhansky16}.

\section{Overview}

We essentially follow the strategy mentioned in the previous section,
and apply it for the motion group.

In Chapter~\ref{chapter:preliminaries},
we introduce the necessary material to follow the arguments developed in subsequent chapters.
In particular, we recall the definition of Fourier Transform on Lie groups of type I.

In Chapter~\ref{chapter:motion_groups},
we introduce the Euclidean motion groups,
study their representation theory and associated Fourier transform.
We also comment on the formal but deceptive resemblance between the Fourier theory of the motion group $\R^n \ltimes \SpecialOrthogonalGroup n$
and the one of the direct product $\R^n \times \SpecialOrthogonalGroup n$.

Chapter~\ref{chapter:symbolic_calculus} will contain the main results of our analysis.
We first define a good family of Ruzhansky-Turunen difference operators,
and use them to define our symbol and operator classes.

In Section~\ref{section:functional_calculus},
we define our symbol family which will allow the crucial density argument mentioned in the previous section.
Following the ideas of~\cite{Shubin01, RuzhanskyWirth14},
we define
\begin{align*}
    \eta_\epsilon(\xi) \defeq
    \frac 1 {\i \turn} \int_\Gamma \e^{-t z} \left(\xi(\Id {} - \Laplacian) - z \Id {}\right)^{-1} \dd z,
    \quad \xi \in \dualGroup \Group
\end{align*}
with an appropriate contour $\Gamma \subset \C$
and we show the symbolic estimates on $\eta_\epsilon$ and $\Id {} - \eta_\epsilon$
via the exponential decay on the contour and elementary estimates on the resolvent,
in particular \emph{without using the classical estimates on the heat kernel}.

This strategy generalizes verbatim to more general settings
(provided that we have good properties difference operators),
but also works on the compact group.

This approach has however at least two drawbacks.
\begin{itemize}
    \item
        We only show symbolic estimates for the functional calculus associated with the functions $\e^{t z}$,
        $1 - \e^{t z}$, and $z^\gamma$,
        while~\cite{FischerRuzhansky16} and~\cite{Fischer2015} prove a much more general result in their respective cases.
    \item
        The kernel estimates when $\rho = 1$,
        essential to show the Calder\'on-Zygmund property,
        are trickier to obtain
        and will require better estimates on the kernel.
        As a result,
        we will have to follow~\cite{Fischer2015}
        to prove the $\Lebesgue p \Group$ boundedness.
\end{itemize}

It is nevertheless interesting to observe that this approach provides a shorter and more elementary way
to establish a global pseudo-differential calculus
if compared with~\cite{Fischer2015,FischerRuzhansky16}.

The rest of Chapter~\ref{chapter:symbolic_calculus} is dedicated to showing the main result of the thesis,
i.e.\ the composition and adjunction formulae,
but also the $\Lebesgue 2 \Group$ boundedness.
We then conclude the proof with the classical application of pseudo-differential calculus to the construction of parametrices.
