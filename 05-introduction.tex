\chapter{Introduction}

\section{The Euclidean case}

In the Euclidean setting,
it is a known fact that constant-coefficient differential operators act like their characteristic polynomials on the Fourier Transform side.
Moreover, composing or inverting such operators can be done by multiplication or inversion of their associated polynomials.
Similarly, taking the adjoint can equally be expressed in terms of polynomials.

Although the above properties do not hold for differential operators with smooth, non-constant, coefficients,
the composition, inversion and adjunction of such operators can still be written solely in terms of their characteristic polynomials.
As an example,
if $T_{\sigma_1}$ and $T_{\sigma_2}$ are two operators arising from the polynomials $\sigma_1(x, \xi)$ and $\sigma_2(x, \xi)$ via
\begin{align*}
    T_1 \defeq \sum_{\abs \alpha \leq N} \frac 1 {\alpha!} \iD{\xi^\alpha} \sigma_1(x, 0) \iD{x^\alpha}
    \quad
    T_2 \defeq \sum_{\abs \alpha \leq N} \frac 1 {\alpha!} \iD{\xi^\alpha} \sigma_2(x, 0) \iD{x^\alpha},
\end{align*}
then it follows from an elementary computation
that the characteristic polynomial of $T_3 \defeq T_1 \circ T_2$ is given by the finite sum
\begin{align}
    \sigma_3(x, \xi) = \sum_{\alpha \in \N^n} \frac 1 {\alpha!} \iD{\xi^\alpha} \sigma_1(x, \xi) \iD{x^\alpha} \sigma_2(x, \xi)
    \label{eq:composition_formula_for_characteristic_polynomials}.
\end{align}
Importantly,
\eqref{eq:composition_formula_for_characteristic_polynomials} means that
composition is reduced to summing and multiplying smooth functions,
and motivates the introduction of the concept of \emph{ellipticity}
as this formula can be used to construct approximate inverses to a suitable class of differential operators.

Laurent Schwartz's \emph{Kernel Theorem} and his extension of the \emph{Fourier transform} allows us to extend the concept of characteristic polynomials far beyond the class of ordinary differential operators.
The resulting concept, called the \emph{symbol} of an operator,
is formally defined by taking the Fourier transform of the convolution kernel,
i.e.\ if $T$ is formally given by
\begin{align*}
    T \phi(x) \defeq \conv \phi {\kappa_x}(x),
\end{align*}
then we define its symbol via
\begin{align*}
    \sigma(x, \xi) \defeq \Fourier \kappa_x(\xi).
\end{align*}

Interestingly, the formula~\eqref{eq:composition_formula_for_characteristic_polynomials} still holds in a certain sense for smooth symbols.
The \emph{order} of a characteristic polynomial can therefore be extended onto smooth symbols as follows:
we shall say that $\sigma \in \SmoothFunctions {\R^n \times \R^n}$ is of order $m \in \R$
if the following holds
\begin{align*}
    \abs {\iD{\xi^\alpha, \xi^\beta} \sigma(x, \xi)} \leq C_{\alpha, \beta} (1 + \abs \xi)^{m - \abs \alpha}.
\end{align*}
Naturally, the condition on the derivatives is motivated by~\eqref{eq:composition_formula_for_characteristic_polynomials} and the fact that
if $\sigma_1$ and $\sigma_2$ are two symbols of order $m_1$ and $m_2 \in \R$ respectively,
the composition of their associated operators should yield a symbol of order $m_1 + m_2$.
We shall say that an operator associated with a symbol of order $m$ is also of order $m$.

Denoting the set of operators of order $m \in \R$ by $\OperatorClass [\R^n] m {1, 0}$,
we obtain the following important collection of results:
\begin{enumerate}
    \item
        If $T_1 \in \OperatorClass [\R^n] {m_1} {1, 0}$
        and $T_2 \in \OperatorClass [\R^n] {m_2} {1, 0}$,
        then $T_1 \circ T_2$
    \item
        The set $\OperatorClass [\R^n] m {1, 0}$ is stable under taking the adjoint.
    \item
        If $T \in \OperatorClass m {\rho, \delta}$ is uniformly elliptic,
        its inverse belongs to $T^{-1} \in \OperatorClass {-m} {\rho, \delta}$
    \item
        If $T \in \OperatorClass [\R^n] m {1, 0}$,
        then $T$ extends continuously to a bounded operator
        \begin{align*}
            T : \Sobolev [\R^n] s \to \Sobolev [\R^n] {s - m},
        \end{align*}
        where in the above $\Sobolev [\R^n] s$ denotes the Sobolev space of order $s \in \R$.
\end{enumerate}

\section{Generalisation to other Lie groups}

\subsection{The torus}

\subsection{Compact groups}

\subsection{Graded Lie groups}

\section{The motion group}
