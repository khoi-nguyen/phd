\chapter*{Acknowledgements}

Firstly, I would like to thank my parents, my sister and her partner for everything they have done for me.
They have always been there for me,
in good and bad times alike.
They have encouraged me throughout my life,
and have always done their best to make sure that I did not need to worry about anything else when I had to work or study.

I would also like to take the opportunity that these pages provide to thank Karen, Paul, Paula, Mike, Emma and Richard.
When being far away from my own family was difficult,
they were always there for me, going out of their way to make me feel part of the family.
From inviting me to spend Christmas or Easter with them,
to smaller everyday life gestures,
I am truly grateful for everything they have done for Chloe and me.

Of course,
these lines would not be complete without mentioning Chloe herself, who encouraged and supported me through difficult times,
and did so much to ensure I could fully concentrate on my studies.
Last but not least,
I would like to thank her for helping me type this document
when my hands were painful.

I would also like to thank Urbain for his friendship and for providing me with a place to stay whenever I needed to come to London.
Following the example of the current version of his future thesis,
I will also thank:
Bram Moolenaar for Vim,
Richard Stallman for starting the GNU project (minus Emacs) and signing my laptop\footnote{After carefully voiding my warranty and removing all Windows stickers.},
and Linus Torvalds for Linux and Git.
Although their tools certainly distracted me a great deal from my research
and exacerbated my tendinitis,
I like to try to convince myself that they have helped me to be more productive in the end.

During my eight years of studying mathematics,
I feel very fortunate to have spent the first three at \emph{Universit\'e Catholique de Louvain}.
What I appreciated the most was the effort that the lecturers made to look at
whether known results and concepts could be introduced or proved more easily,
as opposed to present it as they had learnt it themselves.
I will always remember Jean Van Schaftingen's and Augusto Ponce's elegant modification of McShane's definition of the Lebesgue integral,
Michel Willem's way of deriving all the convexity and duality results of Lebesgue spaces in \emph{one} corollary,
and the countless examples that made my education different than it would have been at any other institution.
They taught me to appreciate mathematical results,
but also to see the beauty in \emph{how} you obtain them.
In addition to their teaching duties,
they have often taken the time to help and advise me.
In particular,
I would like to thank Dr.\ Laurent Moonens,
Prof.\ Augusto Ponce, Prof.\ Jean Van Schaftingen and Prof.\ Pierre Bieliavsky for writing the recommendation letters which have allowed me to study here in the United Kingdom.

We are now getting to the two people who directly helped me with the contents of this thesis.
First, I wish to thank V\'eronique Fischer for the numerous hours she spent
explaining to me things that I should have known already,
but also for her advice about the writing-up period and beyond.
I owe her my (limited) understanding of the field.

And last but not least,
I would like to express my gratitude to my supervisor,
Prof. Michael Ruzhansky,
for giving me the opportunity to work with him and for his continuous support, patience, motivation and insight.
