\chapter*{Acknowledgements}

Firstly, I would like to thank my parents, my sister and her partner for everything they have done for me.
They have always been there for me,
in good and bad times alike.
They have encouraged me throughout my life,
and have always done their best to make sure that I did not need to worry about anything else when I had to study.

I would also like to take the opportunity that these pages provide to thank Karen, Paul, Paula, Mike, Emma and Richard.
When being far away from my own family was difficult,
they were always there for me, going out of their way to make me feel part of the family.
From inviting me to spend Christmas or Easter with them,
to smaller everyday gestures,
I am truly grateful for everything they have done for Chloe and me.

Of course,
these lines would not be complete without mentioning Chloe herself, who encouraged and supported me through difficult times,
and did so much to ensure I could fully concentrate on my studies.
I also owe her my thanks for helping me type this document
when my hands were painful.

I would also like to thank Urbain for his friendship and for providing me with a place to stay whenever I needed to go to London.
Following the example of the current version of his future thesis,
I would also like to thank:
Bram Moolenaar for writing Vim,
Richard Stallman for starting the GNU project (minus Emacs) and signing my laptop\footnote{After carefully voiding my warranty and removing all Windows stickers.},
and Linus Torvalds for developing Linux and Git.
Although their tools certainly distracted me a great deal from my research
and perhaps exacerbated my tendinitis,
I like convincing myself that they did eventually help my productivity.

During my eight years of studying mathematics,
I feel very fortunate to have spent the first three at \emph{Universit\'e Catholique de Louvain}.
What I appreciated the most was the effort that the lecturers made to look at
whether known results and concepts could be introduced or proved more easily,
as opposed to present it as they had learnt it themselves.
I will always remember Professor Jean Van Schaftingen's and Professor Augusto Ponce's elegant modification of McShane's definition of the Lebesgue integral,
Professor Michel Willem's way of deriving all the convexity and duality results of Lebesgue spaces in \emph{one} corollary,
and the countless examples that made my education different than it would have been at any other institution.
They taught me to appreciate mathematical results,
but also to see the beauty in \emph{how} you obtain them.
In addition to their teaching duties,
they have often taken the time to help and advise me.
In particular,
I would like to thank Dr.\ Laurent Moonens,
Professor Augusto Ponce, Professor Jean Van Schaftingen and Professor Pierre Bieliavsky for writing the recommendation letters which allowed me to study here in the United Kingdom.

Finally, I offer my sincerest gratitude to the two people who directly helped with the contents of this thesis.
Firstly, to V\'eronique Fischer for the numerous hours she spent
explaining material to me that I should have already known,
but also for her advice about the writing-up period and beyond.
I owe her my (limited) understanding of the field.
Lastly, but certainly not least,
to my supervisor,
Professor Michael Ruzhansky,
for giving me the opportunity to work with him and for his continuous support, patience, motivation and insight.
