\chapter{Generalised Motion Groups}

\section{Definitions}

\begin{definition}[Semi-direct product]
\label{definition:semi-direct_products}
\index{semi-direct product}
\index{motion group|see {generalised motion group}}
    Let $N$ be a group and $H$ be a subgroup of $\Aut(N)$.
    The \emph{semi-direct product} $N \rtimes H$ is the group whose elements are that of $N \times H$ with the group law
    \begin{align}
        (n_1, h_1) (n_2, h_2) \defeq (n_1 h_1(n_2), h_1 h_2), \quad (n_1, h_1), (n_2, h_2) \in N \times H.
    \end{align}

    Given $(n, h)$ in $N \rtimes H$, its \emph{inverse} is given by $(h_1^{-1}(n_1^{-1}), h_1^{-1})$,
    while $(e_N, e_H)$ is the \emph{identity element} of the group.
\end{definition}

\begin{definition}[Generalised motion group]
\label{definition:generalised_motion_group}
\index{generalised motion group}
    Let $\Group$ be a group.
    We shall say that $\Group$ is a \emph{generalised motion group}
    if there exists a (finite dimensional) vector space $\VectorSpace$ and a compact group $\CompactGroup \subset \OrthogonalGroup{\VectorSpace}$ such that $\Group = \VectorSpace \rtimes \CompactGroup$.
\end{definition}

\begin{example}[Euclidean Motion Groups]
\label{example:Euclidean_motion_groups}
\index{Euclidean motion group}
    For each $n \in \N$, let
    \begin{align*}
        \MotionGroup{n} \defeq \{g \in \AffineTransformations{\R^n} : \det g = 1\}.
    \end{align*}
    The elements of $\MotionGroup{n}$ are called \emph{rigid motions},
    while $\MotionGroup{n}$ is called the \emph{Euclidean motion group}.

    It is easily shown that associating $(x, k) \in \R^n \rtimes \SpecialOrthogonalGroup{n}$ to the motion
    \begin{align*}
        g_{(x, k)} : \R^n \to \R^n : y \mapsto x + ky
    \end{align*}
    defines a group isomorphism between $\R^n \rtimes \SpecialOrthogonalGroup{n}$ and $\MotionGroup{n}$.
    We shall therefore identify $\MotionGroup{n}$ with $\R^n \rtimes \SpecialOrthogonalGroup{n}$ from now on.
\end{example}

\begin{example}[$2$-dimensional Euclidean motion group]
\label{example:Euclidean_motion_groups:dimension_2}
\index{Euclidean motion group!dimenion 2}
    The $2$-dimensional case is worth mentioning because
    in that case, $\SpecialOrthogonalGroup{2}$ is \emph{abelian}.

    Since $\SpecialOrthogonalGroup{2}$ is isomorphic to $\T$,
    we shall often identify $\MotionGroup{2}$ with $\R^2 \times \T$ and the group law
    \begin{align*}
        (x, t) (y, s) \defeq (x + \e^{\i \turn t} y, t + s), \quad (x, t), (y, s) \in \R^2 \times \T.
    \end{align*}
\end{example}

From now on, unless stated otherwise,
$\Group$ will denote a generalised motion group,
with $\VectorSpace$ its underlying vector space and $\CompactGroup$ its associated compact group.

The requirement that $\CompactGroup$ should be a subgroup of $\OrthogonalGroup{\VectorSpace}$ is motivated by the following result

\begin{lemma}[Haar measure]
\label{lemma:Haar_measure}
    If $\dd x$ is a Lebesgue measure on $\VectorSpace$ and $\dd k$ is the normalised Haar measure on $\CompactGroup$,
    then the the product measure $\dd x \dd k$ is a Haar measure on $\Group = \VectorSpace \rtimes \CompactGroup$,
    which is both left and right-invariant.
\end{lemma}
\begin{proof}
    Let $(x, k) \in \Group$.
    \begin{align*}
        \int_\Group f((x, k) (y, l)) \dd (y, l)
        = \int_\VectorSpace \int_\CompactGroup f(x + ky, k l) \dd l \dd y
    \end{align*}

    Now, let us substitute $y$ for $k^{-1}(y - x)$ and $l$ for $k^{-1} l$ in the above.
    As the Lebesgue measure is invariant under $\OrthogonalGroup{\VectorSpace}$ and under translations,
    and because the Haar measure $\dd l$ is left-invariant,
    we obtain
    \begin{align*}
        \int_\Group f((x, k) (y, l)) \dd (y, l)
        &= \int_\VectorSpace \int_\CompactGroup f(y, l) \dd l \dd y\\
        &= \int_\Group f(y, l) \dd (y, l),
    \end{align*}
    showing that $\dd y \dd l$ is indeed a Haar measure on $\Group$.

    Since by Proposition~\ref{proposition:sufficient_conditions_to_be_unimodular} $\dd l$ is also right-invariant,
    arguing similarly shows that $\dd y \dd l$ is also right-invariant.
\end{proof}

\section{Lie algebra structure}

Using an isomorphism
\begin{align}
    \phi : \VectorSpace \to \R^n
\end{align}
where $n = \dim \VectorSpace$,
we can represent the generalised motion group $\Group$
by real matrices via the map
\begin{align}
    \Phi : \Group \to \R^{(n + 1) \times (n + 1)} :
        (x, k) \mapsto
            \begin{pmatrix}
                \phi \circ k \circ \phi^{-1} & \phi(x)\\
                0 & 1,
            \end{pmatrix}
    \label{eq:motion_group_realised_as_matrices}
\end{align}
where by $\phi \circ k \circ \phi^{-1}$ we actually mean its corresponding matrix in the canonical basis.

\begin{definition}[Lie Algebra]
\label{definition:Lie_Algebra}
\index{generalised motion group!Lie algebra}
    Suppose $\VectorSpace$ has dimension $n$.
    The real vector space
    \begin{align*}
        \LieAlgebra \defeq
            \{
                X \in \R^{(n + 1) \times (n + 1)} :
                \text{for each}\
                t \in \R,\
                \exp(t X) \in \Phi(\Group),
            \}
    \end{align*}
    where $\Phi$ is the map defined in~\eqref{eq:motion_group_realised_as_matrices},
    is called the \emph{Lie algebra} of $\Group$.
\end{definition}

\begin{definition}[Exponential map]
\label{definition:exponential_map}
\index{generalised motion group!exponential map}
    The \emph{exponential map} on $\Group$ is the map
    \begin{align*}
        \exp_\Group : \LieAlgebra \to \Group
    \end{align*}
    defined by $\exp_\Group \defeq \Phi^{-1} \circ \exp$,
    where $\Phi$ is the map defined in~\eqref{eq:motion_group_realised_as_matrices}.
\end{definition}

\begin{definition}
\label{definition:left-invariant_differential_operator}
    Let $X \in \LieAlgebra$.
    We define $\LeftDifferentialOperatorFirstOrder{X}$, the \emph{left-invariant differential operator associated to $X$}, via
    \begin{align*}
        \LeftDifferentialOperatorFirstOrder{X} f(g)
            = \D{}{t} f(g \exp_\Group(t X)),
    \end{align*}
    where $f \in \SmoothFunctions{\Group}$.
\end{definition}

\begin{example}[$2$-dimensional Euclidean motion group]
    Assume $\Group = \R^2 \rtimes \T$.
    The Lie Algebra is the vector space generated by the matrices
    \begin{align*}
        X_1 =
            \begin{pmatrix}
                0 & 0 & 1\\
                0 & 0 & 0\\
                0 & 0 & 0
            \end{pmatrix},\quad
        X_2 =
            \begin{pmatrix}
                0 & 0 & 0\\
                0 & 0 & 1\\
                0 & 0 & 0
            \end{pmatrix},\quad
        X_3 =
            \begin{pmatrix}
                0 & -1 & 0\\
                1 &  0 & 0\\
                0 &  0 & 0
            \end{pmatrix},
    \end{align*}
    which satisfy the commutation relations
    \begin{align*}
        \LieBracket{X_1}{X_2} = 0,\quad
        \LieBracket{X_2}{X_3} = X_1,\quad
        \LieBracket{X_3}{X_1} = X_2
    \end{align*}

    Moreover, if $f \in \SmoothFunctions{\Group}$,
    then the associated left-invariant operators act via
    \begin{align*}
        \LeftDifferentialOperatorFirstOrder{X_1} f(x, t)
            &= \cos(\turn t) \D{f}{x_1}(x, t) + \sin(\turn t) \D{f}{x_2}(x, t)\\
        \LeftDifferentialOperatorFirstOrder{X_2} f(x, t)
            &= -\sin(\turn t) \D{f}{x_1}(x, t) + \cos(\turn t) \D{f}{x_2}(x, t)\\
        \LeftDifferentialOperatorFirstOrder{X_3} f(x, t)
            &= \D{f}{t}(x, t),
    \end{align*}
    where $(x, t) \in \R^2 \rtimes \T$.
\end{example}

\section{Unitary representations}

\begin{definition}
\label{definition:reducible_representation}
    Let $\lambda \in \dualGroup{\VectorSpace}$.
    We define a unitary representation $\Rep{\lambda} \in \Hom(\Group, \End(\Lebesgue{2}{\CompactGroup}))$ of $\Group$ via
    \begin{align}
        \Rep{\lambda} (x, k) F(u) \defeq (u \lambda)(x) F(k^{-1} u),
    \end{align}
    where $(x, k) \in \CompactGroup$, $F \in \Lebesgue{2}{\CompactGroup}$ and $u \in \CompactGroup$.
\end{definition}

Unfortunately, the above representation is often reducible.
However, as we shall see later, the Fourier Transform on $\Group$ can be written exclusively with those representations.

\subsection{Unitary dual}

Throughout this section, fix $\lambda \in \dualGroup{\VectorSpace}$
and denote by $\IsotropySubgroup{\CompactGroup}{\lambda}$ its isotropy subgroup.

Let $\tau \in \dualGroup{\IsotropySubgroup{\CompactGroup}{\lambda}}$ and denote by $\dimRep{\tau}$ its dimension.
For $q = 1, \dots, \dimRep{\tau}$, let
\begin{align}
    P^\tau_q F(u) \defeq \dimRep{\tau} \int_\IsotropySubgroup{\CompactGroup}{\lambda} \conj{\tau_{qq}(m)} F(u m) \dd m,
    \quad F \in \Lebesgue{2}{\CompactGroup}.
    \label{eq:projection_on_L2_of_the_compact_group}
\end{align}

By the Inverse Fourier Transform at $e \in \IsotropySubgroup{\CompactGroup}{\lambda}$,
if $F \in \Lebesgue{2}{\CompactGroup}$ and $u \in \CompactGroup$, then
\begin{align}
    F(u)
    = \sum_{\tau \in \dualGroup{\IsotropySubgroup{\CompactGroup}{\lambda}}} \dimRep{\tau} \sum_{q = 1}^{\dimRep{\tau}} {\Fourier[\IsotropySubgroup{\CompactGroup}{\lambda}]{F(u \dummy)}}_{qq}
    = \sum_{\tau \in \dualGroup{\IsotropySubgroup{\CompactGroup}{\lambda}}} P^\tau_q F(u)
\end{align}

Now, write $\Hilbert{\tau}{q} \defeq P^\tau_q \Lebesgue{2}{\CompactGroup}$.

\begin{lemma}
    The Hilbert spaces
    \begin{align*}
        \{\Hilbert{\tau}{q} : \tau \in \dualGroup{\IsotropySubgroup{\CompactGroup}{\lambda}}, q = 1, \dots, \dimRep{\tau} \}
    \end{align*}
    are mutually orthogonal.
    Therefore, we have
    \begin{align*}
        \Lebesgue{2}{\CompactGroup} = \bigoplus_{\tau \in \dualGroup{\IsotropySubgroup{\CompactGroup}{\lambda}}} \bigoplus_{q = 1}^{\dimRep{\tau}} \Hilbert{\tau}{q},
    \end{align*}
    where the above decomposition is orthogonal.
\end{lemma}
\begin{proof}
    Let $\mu, \nu \in \dualGroup{\IsotropySubgroup{\CompactGroup}{\lambda}}$,
    $p, \in \{1, \dots, \dimRep{\mu}\}$,
    and $q, \in \{1, \dots, \dimRep{\nu}\}$.
    Suppose also that $F \in \Lebesgue{2}{\CompactGroup}$.

    \begin{align*}
        &\ip[\Lebesgue{2}{\CompactGroup}]{P^{\mu}_{p} F}{P^{\nu}_{q} H}\\
        &= \dimRep{\mu} \dimRep{\nu}
            \int_\CompactGroup
                \int_\IsotropySubgroup{\CompactGroup}{\lambda}
                    \int_\IsotropySubgroup{\CompactGroup}{\lambda}
                        \conj{{\mu}_{p p}(m_1)}
                        {\nu}_{q q}(m_2)
                        F(u m_1)
                        \conj{H(u m_2)}
                    \dd m_2
                \dd m_1
            \dd u
    \end{align*}

    Decomposing $u = u' m$ with
    \begin{align*}
    \int_\CompactGroup \dd u = \int_\RightQuotient{\CompactGroup}{\IsotropySubgroup{\CompactGroup}{\lambda}} \int_\IsotropySubgroup{\CompactGroup}{\lambda} \dd m \dd u'
    \end{align*}
    and substituing $m_i$ for $m^{-1} m_i$, $i = 1, 2$, we get
    \begin{align*}
        \ip[\Lebesgue{2}{\CompactGroup}]{P^{\mu}_{p} F}{P^{\nu}_{q} H}
        = \dimRep{\mu} \dimRep{\nu}
            \int_\RightQuotient{\CompactGroup}{\IsotropySubgroup{\CompactGroup}{\lambda}}
                \int_\IsotropySubgroup{\CompactGroup}{\lambda}
                    \int_\IsotropySubgroup{\CompactGroup}{\lambda}
                        \int_\IsotropySubgroup{\CompactGroup}{\lambda}
                            &\conj{{\mu}_{p p}(m^{-1} m_1)}
                            {\nu}_{q q}(m^{-1} m_2)\\
                            &F(u' m_1)
                            \conj{H(u' m_2)}
                        \dd m_2
                    \dd m_1
                \dd m
            \dd u'.
    \end{align*}

    Now, using the identities
    \begin{align*}
        \conj{\mu_{pp}(m^{-1} m_1)}
        = \sum_{i = 1}^\dimRep{\mu}
            \conj{\mu_{p i}(m^{-1})}
            \conj{\mu_{i p}(m_1)}
        = \sum_{i = 1}^\dimRep{\mu}
            \mu_{i p}(m)
            \conj{\mu_{i p}(m_1)}\\
        \nu_{qq}(m^{-1} m_2)
        = \sum_{j = 1}^\dimRep{\nu}
            \nu_{qj}(m^{-1})
            \nu_{jq}(m_2)
        = \sum_{j = 1}^\dimRep{\nu}
            \conj{\nu_{jq}(m)}
            \nu_{jq}(m_2),
    \end{align*}
    it follows that
    \begin{align*}
        &\ip[\Lebesgue{2}{\CompactGroup}]{P^{\mu}_{p} F}{P^{\nu}_{q} H}\\
        &= \dimRep{\mu} \dimRep{\nu}
            \int_\RightQuotient{\CompactGroup}{\IsotropySubgroup{\CompactGroup}{\lambda}}
                \int_\IsotropySubgroup{\CompactGroup}{\lambda}
                    \int_\IsotropySubgroup{\CompactGroup}{\lambda}
                        \int_\IsotropySubgroup{\CompactGroup}{\lambda}
                            \sum_{i = 1}^\dimRep{\mu}
                                \sum_{j = 1}^\dimRep{\nu}
                                    \mu_{i p}(m)
                                    \conj{\mu_{i p}(m_1)}
                                    \conj{\nu_{jq}(m)}
                                    \nu_{jq}(m_2)
                                    F(u' m_1)
                                    \conj{H(u' m_2)}
                        \dd m_2
                    \dd m_1
                \dd m
            \dd u'.
    \end{align*}

    Integrating with respect to $m$ and keeping in mind the Peter-Weyl Theorem gives the orthogonality relation
    \begin{align*}
        \dimRep{\nu} \int_\IsotropySubgroup{\CompactGroup}{\lambda} \mu_{ip}(m) \conj{\nu_{jq}(m)} \dd m
        = \Kronecker{\mu}{\nu} \Kronecker{i}{j} \Kronecker{p}{q},
    \end{align*}
    we obtain
    \begin{align*}
        &\ip[\Lebesgue{2}{\CompactGroup}]{P^{\mu}_{p} F}{P^{\nu}_{q} H}\\
        &= \Kronecker{\mu}{\nu} \Kronecker{p}{q} \dimRep{\mu}
            \int_\RightQuotient{\CompactGroup}{\IsotropySubgroup{\CompactGroup}{\lambda}}
                \int_\IsotropySubgroup{\CompactGroup}{\lambda}
                    \int_\IsotropySubgroup{\CompactGroup}{\lambda}
                        \sum_{i = 1}^\dimRep{\mu}
                            \conj{\mu_{i p}(m_1)}
                            \mu_{ip}(m_2)
                            F(u' m_1)
                            \conj{H(u' m_2)}
                    \dd m_2
                \dd m_1
            \dd u'.
    \end{align*}

    Using $\conj{\mu_{ip}(m_1)} = \mu_{p i}(m_1^{-1})$,
    the above becomes
    \begin{align*}
        &\ip[\Lebesgue{2}{\CompactGroup}]{P^{\mu}_{p} F}{P^{\nu}_{q} H}\\
        &= \Kronecker{\mu}{\nu} \Kronecker{p}{q} \dimRep{\mu}
            \int_\RightQuotient{\CompactGroup}{\IsotropySubgroup{\CompactGroup}{\lambda}}
                \int_\IsotropySubgroup{\CompactGroup}{\lambda}
                    \int_\IsotropySubgroup{\CompactGroup}{\lambda}
                        \mu_{pp}(m_1^{-1} m_2)
                        F(u' m_1)
                        \conj{H(u' m_2)}
                    \dd m_2
                \dd m_1
            \dd u'\\
        &= \Kronecker{\mu}{\nu} \Kronecker{p}{q} \dimRep{\mu}
            \int_\RightQuotient{\CompactGroup}{\IsotropySubgroup{\CompactGroup}{\lambda}}
                \int_\IsotropySubgroup{\CompactGroup}{\lambda}
                    \int_\IsotropySubgroup{\CompactGroup}{\lambda}
                        \mu_{pp}(m_1^{-1})
                        F(u' m_2 m_1)
                        \conj{H(u' m_2)}
                    \dd m_2
                \dd m_1
            \dd u'\\
        &= \Kronecker{\mu}{\nu} \Kronecker{p}{q} \dimRep{\mu}
            \int_\CompactGroup
                \int_\IsotropySubgroup{\CompactGroup}{\lambda}
                    \mu_{pp}(m_1^{-1})
                    F(m m_1)
                    \conj{H(u)}
                \dd m_1
            \dd u,
    \end{align*}
    where in the last line, we let $u = u' m_2$.
    Using the fact that $\mu_{pp}(m_1^{-1}) = \conj{\mu_{pp}(m_1)}$ and~\eqref{eq:projection_on_L2_of_the_compact_group},
    we conclude that in fact
    \begin{align*}
        \ip[\Lebesgue{2}{\CompactGroup}]{P^{\mu}_{p} F}{P^{\nu}_{q} H}
        &= \Kronecker{\mu}{\nu} \Kronecker{p}{q} \ip[\Lebesgue{2}{\CompactGroup}]{P^\mu_p F}{H}.
    \end{align*}

    From there, we easily deduce that the spaces $\Hilbert{\mu}{p}$,
    where $\mu \in \dualGroup{\IsotropySubgroup{\CompactGroup}{\lambda}}$ and $p \in \{1, \dots, \dimRep{\tau}\}$ are mutually orthogonal.

    Using the fact that $P^\tau_q F(u) = {\Fourier[\IsotropySubgroup{\CompactGroup}{\lambda}] \{ F(u \dummy) \} (\tau)}_{qq}$,
    we can apply the inverse Fourier Formula to $F(u \dummy)$ at $e \in \IsotropySubgroup{\CompactGroup}{\lambda}$ to obtain
    \begin{align*}
        F(u) = F(u e) = \sum_{\tau \in \dualGroup{\IsotropySubgroup{\CompactGroup}{\lambda}}}
        \dimRep{\tau} \tr( \Fourier[\IsotropySubgroup{\CompactGroup}{\lambda}] \{ F(u \dummy) \}(\tau))
        = \sum_{\tau \in \dualGroup{\IsotropySubgroup{\CompactGroup}{\lambda}}}
            \sum_{q = 1}^\dimRep{\tau}
                P^\tau_q F(u).
    \end{align*}
\end{proof}

\begin{proposition}[Unitary dual]
\label{proposition:unitary_dual}
    Let $\lambda, \lambda' \in \dualGroup{\VectorSpace}$
    and $\tau \in \dualGroup{\IsotropySubgroup{\CompactGroup}{\lambda}}$,
    $\tau' \in \dualGroup{\IsotropySubgroup{\CompactGroup}{\lambda'}}$.
    The following properties hold
    \begin{enumerate}
        \item $\Rep{\lambda}$ restricts to an irreducible unitary representation on each $\Hilbert{\tau}{q}$;
        \item $(\Hilbert{\tau}{q}, \Rep{\lambda})$ and $(\Hilbert{\tau'}{q'}, \Rep{\lambda'})$ are equivalent if and only if
            \begin{align*}
                \lambda' = k \lambda \quad \text{and} \quad \EquivalenceClass{\dualGroup{\IsotropySubgroup{\CompactGroup}{\lambda}}}{\tau} = \EquivalenceClass{\dualGroup{\IsotropySubgroup{\CompactGroup}{\lambda}}}{\tau'(k \dummy k^{-1})}
            \end{align*}
            for some $k \in \CompactGroup$.
    \end{enumerate}
\end{proposition}
\begin{proof}
    \begin{enumerate}
        \item It is clear that $\Rep{\lambda}$ is unitary.
            If $F \in \Hilbert{\tau}{q}$, we check that
            \begin{align*}
                P^\tau_q \Rep{\lambda}(x, k) F(u)
                &=  \dimRep{\tau}
                    \int_\IsotropySubgroup{\CompactGroup}{\lambda}
                        \conj{\tau_{qq}(m)}
                        \lambda(u^{-1} k)
                        F(k^{-1} u m)
                    \dd m\\
                &= \Rep{\lambda}(x, k) P^\tau_q F(u),
            \end{align*}
            in other words $P^\tau_q$ commutes with $\Rep{\lambda}(x, k)$.
            As $P^\tau_q F = F$, this implies that
            \begin{align}
                P^\tau_q \Rep{\lambda}(x, k) F(u)
                = \Rep{\lambda}(x, k) F(u),
            \end{align}
            in other words $\Rep{\lambda}(x, k) F$ also belongs to $\Hilbert{\tau}{q}$.

            It remains to show that $\Rep{\lambda}$ is irreducible.
    \end{enumerate}
\end{proof}

\begin{definition}[Unitary dual]
\label{definition:unitary_dual}
\index{generalised motion group!unitary dual}
    Fix $\lambda_0 \in \dualGroup{\VectorSpace}$.
    We define the \emph{unitary dual} of $\Group$, denoted by $\dualGroup{\Group}$, via
    \begin{align*}
        \dualGroup{\Group} \defeq \{ (\Hilbert{\tau}{1}, \Rep{\lambda}) : \lambda \in \LeftQuotient{\IsotropySubgroup{\CompactGroup}{\lambda_0}}{\dualGroup{\VectorSpace}}, \tau \in \dualGroup{\IsotropySubgroup{\CompactGroup}{\lambda}} \}.
    \end{align*}
\end{definition}

\subsection{Infinitesimal representations}

\begin{definition}[Infinitesimal Representation]
\label{definition:infinitesimal_representation}
\index{generalised motion group!infinitesimal representation}
    Let $X \in \g$.
    We define the infinitesimal representation of $X$ as the operator
    \begin{align*}
        \Rep{\lambda}(X) : \SmoothFunctions{\CompactGroup} \to \SmoothFunctions{\CompactGroup}
    \end{align*}
    defined via
    \begin{align*}
        \Rep{\lambda}(X) F(u) \defeq \left. \D{}{t} \right|_{t = 0} \Rep{\lambda}(\exp(t X)) F(u),
    \end{align*}
    where $F \in \SmoothFunctions{\CompactGroup}$.
\end{definition}

\begin{lemma}[Infinitesimal representation of $\Laplacian$]
\label{lemma:infinitesimal_representation_of_the_Laplacian}
    Let $\lambda \in \dualGroup{\VectorSpace}$.
    The infinitesimal representation of $\Laplacian$ is given by
    \begin{align*}
        \Rep{\lambda}(\Laplacian) = - \norm[\dualGroup{\VectorSpace}]{\lambda}^2 \Id{\Lebesgue{2}{\CompactGroup}} + \RightLaplacian[\CompactGroup].
    \end{align*}
\end{lemma}

\section{Fourier Transform}

\subsection{Definition and elementary properties}

\begin{definition}[Fourier transform]
\label{definition:Fourier_Transform}
\index{generalised motion group!Fourier transform}
    Let $f \in \Lebesgue{1}{\Group}$ and $\lambda \in \dualGroup{\VectorSpace}$.
    We define its \emph{Fourier coefficient} at $\lambda$ via
    \begin{align*}
        \Fourier{f}(\lambda) \defeq \int_\Group f(g) \adj{\Rep{\lambda}(g)} \dd g.
    \end{align*}

    Moreover, the map
    \begin{align*}
        \Fourier{f} : \dualGroup{\VectorSpace} \to \End(\Lebesgue{2}{\CompactGroup}) :
        \lambda \mapsto \Fourier{f}(\lambda)
    \end{align*}
    is called the \emph{Fourier Transform} of $f$.
\end{definition}

\subsection{Plancherel formula}

\begin{proposition}[Plancherel formula]
\label{proposition:Plancherel_formula}
\index{generalised motion group!Fourier transform!Plancherel formula}
    Let $f \in \Lebesgue{1}{\Group} \cap \Lebesgue{2}{\Group}$.
    The following formula holds
    \begin{align}
        \int_G \abs{f}^2 \dd g = \int_\dualGroup{\VectorSpace} \norm[\HilbertSchmidt{\Lebesgue{2}{\CompactGroup}}]{\Fourier{f}(\lambda)}^2 \dd \Plancherel{\VectorSpace}(\lambda).
        \label{proposition:Plancherel_formula:formula}
    \end{align}
\end{proposition}
\begin{proof}
    Let $\lambda \in \dualGroup{\VectorSpace}$ and $F \in \Lebesgue{2}{\CompactGroup}$.
    If $u \in \CompactGroup$, we can check that
    \begin{align*}
        \Fourier{f}(\lambda) F(u)
        &= \int_\VectorSpace \int_\CompactGroup f(x, k) \lambda(-u^{-1} k^{-1} x) F(k u) \dd k \dd x\\
        &= \int_\VectorSpace \int_\CompactGroup f(x, k) \conj{k u \lambda(x)} F(k u) \dd k \dd x,
    \end{align*}
    where we used $\lambda(-{(k u)}^{-1} x) = \conj{\lambda({(k u)}^{-1} x)} = \conj{(k u \lambda)(x)}$.

    Substituing $k$ for $k u^{-1}$, we get
    \begin{align*}
        \Fourier{f}(\lambda) F(u)
        &= \int_\CompactGroup \Fourier[\VectorSpace]{f}(k \lambda, k u^{-1}) F(k) \dd k.
    \end{align*}

    Therefore, it follows by REFERENCE that
    \begin{align*}
        \norm[\HilbertSchmidt{\Lebesgue{2}{\CompactGroup}}]{\Fourier{f}(\lambda)}^2
        &= \int_\CompactGroup \int_\CompactGroup \abs{\Fourier[\VectorSpace]{f}(k \lambda, k u^{-1})}^2 \dd u \dd k.
    \end{align*}

    Now, integrating with respect to $\lambda$, we obtain
    \begin{align*}
        \int_\dualGroup{\VectorSpace} \norm[\HilbertSchmidt{\Lebesgue{2}{\CompactGroup}}]{\Fourier{f}(\lambda)}^2 \dd \Plancherel{\VectorSpace}(\lambda)
        &= \int_\dualGroup{\VectorSpace} \int_\CompactGroup \abs{\Fourier[\VectorSpace]{f}(\lambda, k)}^2 \dd k \dd \Plancherel{\VectorSpace}(\lambda)\\
        &= \int_\VectorSpace \int_\CompactGroup \abs{f(x, k)}^2 \dd u \dd k,
    \end{align*}
    where the last line was obtained by applying the Plancherel formula on $\VectorSpace$.
\end{proof}

\begin{proposition}[Inverse Fourier Transform]
\label{proposition:inverse_Fourier_Transform}
\index{generalised motion group!Fourier transform!inverse formula}
    Let $\phi \in \Schwartz{\Group}$.
    For each $g \in \Group$,
    we have
    \begin{align*}
        \phi(g)
        = \int_\dualGroup{\VectorSpace}
        \tr \left( \Rep{\lambda}(g) \Fourier \phi(\lambda) \right) \dd \Plancherel{\VectorSpace}(\lambda).
    \end{align*}
\end{proposition}

\subsection{Sobolev spaces}

\subsection{Fourier Transform of distributions}
