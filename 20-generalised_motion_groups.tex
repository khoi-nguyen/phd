\chapter{Generalised Motion Groups}

\section{Definitions}

\begin{definition}[Semi-direct product]
\label{definition:semi-direct_products}
\index{semi-direct product}
\index{motion group|see {generalised motion group}}
    Let $N$ be a group and $H$ be a subgroup of $\Aut(N)$.
    The \emph{semi-direct product} $N \rtimes H$ is the group whose elements are that of $N \times H$ with the group law
    \begin{align}
        (n_1, h_1) (n_2, h_2) \defeq (n_1 h_1(n_2), h_1 h_2), \quad (n_1, h_1), (n_2, h_2) \in N \times H.
    \end{align}

    Given $(n, h)$ in $N \rtimes H$, its \emph{inverse} is given by $(h_1^{-1}(n_1^{-1}), h_1^{-1})$,
    while $(e_N, e_H)$ is the \emph{identity element} of the group.
\end{definition}

\begin{definition}[Generalised motion group]
\label{definition:generalised_motion_group}
\index{generalised motion group}
    Let $\Group$ be a group.
    We shall say that $\Group$ is a \emph{generalised motion group}
    if there exists a (finite dimensional) vector space $\VectorSpace$ and a compact group $\CompactGroup \subset \OrthogonalGroup{\VectorSpace}$ such that $\Group = \VectorSpace \rtimes \CompactGroup$ and $\CompactGroup$ acts transitively on $\VectorSpace$.
\end{definition}

\begin{remark}
    Let $x \in \VectorSpace$ and $k \in \CompactGroup$.
    We will never identify $x$ with $(x, \Id{\VectorSpace}) \in \VectorSpace \rtimes \CompactGroup$,
    or $k$ with $(0, k) \in \VectorSpace \rtimes \CompactGroup$.

    Therefore, when we write $k x$, it will \emph{always} mean the vector obtained by rotating $x$ by $k$.
    If we see them as elements of $\VectorSpace \rtimes \CompactGroup$,
    we shall explicitely write
    \begin{align*}
        (0, k) (x, \Id{\VectorSpace}) = (k x, k).
    \end{align*}
\end{remark}

\begin{example}[Euclidean Motion Groups]
\label{example:Euclidean_motion_groups}
\index{Euclidean motion group}
    For each $n \in \N$, let
    \begin{align*}
        \MotionGroup{n} \defeq \{g \in \AffineTransformations{\R^n} : \det g = 1\}.
    \end{align*}
    The elements of $\MotionGroup{n}$ are called \emph{rigid motions},
    while $\MotionGroup{n}$ is called the \emph{Euclidean motion group}.

    It is easily shown that associating $(x, k) \in \R^n \rtimes \SpecialOrthogonalGroup{n}$ to the motion
    \begin{align*}
        g_{(x, k)} : \R^n \to \R^n : y \mapsto x + ky
    \end{align*}
    defines a group isomorphism between $\R^n \rtimes \SpecialOrthogonalGroup{n}$ and $\MotionGroup{n}$.
    We shall therefore identify $\MotionGroup{n}$ with $\R^n \rtimes \SpecialOrthogonalGroup{n}$ from now on.
\end{example}

\begin{example}[$2$-dimensional Euclidean motion group]
\label{example:Euclidean_motion_groups:dimension_2}
\index{Euclidean motion group!dimenion 2}
    The $2$-dimensional case is worth mentioning because
    in that case, $\SpecialOrthogonalGroup{2}$ is \emph{abelian}.

    Since $\SpecialOrthogonalGroup{2}$ is isomorphic to $\T$,
    we shall often identify $\MotionGroup{2}$ with $\R^2 \times \T$ and the group law
    \begin{align*}
        (x, t) (y, s) \defeq (x + \e^{\i \turn t} y, t + s), \quad (x, t), (y, s) \in \R^2 \times \T.
    \end{align*}
\end{example}

\begin{example}
    Let $n \in \N$.
    Consider the group
    \begin{align*}
        \{g \in \AffineTransformations{\C^n} : \det_{\C^n} g = 1\}
    \end{align*}
    where the law is the composition of functions.

    Arguing like in Example~\ref{example:Euclidean_motion_groups},
    the above group can be identified with $\C^n \rtimes \SpecialUnitaryGroup{n}$.
\end{example}

From now on, unless stated otherwise,
$\Group$ will denote a generalised motion group,
with $\VectorSpace$ its underlying vector space and $\CompactGroup$ its associated compact group.

The requirement that $\CompactGroup$ should be a subgroup of $\OrthogonalGroup{\VectorSpace}$ is motivated by the following result

\begin{lemma}[Haar measure]
\label{lemma:Haar_measure}
    If $\dd x$ is a Lebesgue measure on $\VectorSpace$ and $\dd k$ is the normalised Haar measure on $\CompactGroup$,
    then the the product measure $\dd x \dd k$ is a Haar measure on $\Group = \VectorSpace \rtimes \CompactGroup$,
    which is both left and right-invariant.
\end{lemma}
\begin{proof}
    Let $(x, k) \in \Group$.
    \begin{align*}
        \int_\Group f((x, k) (y, l)) \dd (y, l)
        = \int_\VectorSpace \int_\CompactGroup f(x + ky, k l) \dd l \dd y
    \end{align*}

    Now, let us substitute $y$ for $k^{-1}(y - x)$ and $l$ for $k^{-1} l$ in the above.
    As the Lebesgue measure is invariant under $\OrthogonalGroup{\VectorSpace}$ and under translations,
    and because the Haar measure $\dd l$ is left-invariant,
    we obtain
    \begin{align*}
        \int_\Group f((x, k) (y, l)) \dd (y, l)
        &= \int_\VectorSpace \int_\CompactGroup f(y, l) \dd l \dd y\\
        &= \int_\Group f(y, l) \dd (y, l),
    \end{align*}
    showing that $\dd y \dd l$ is indeed a Haar measure on $\Group$.

    Since by Proposition~\ref{proposition:sufficient_conditions_to_be_unimodular} $\dd l$ is also right-invariant,
    arguing similarly shows that $\dd y \dd l$ is also right-invariant.
\end{proof}

\section{Lie algebra structure}

Using an isomorphism
\begin{align}
    \phi : \VectorSpace \to \R^n
\end{align}
where $n = \dim \VectorSpace$,
we can represent the generalised motion group $\Group$
by real matrices via the map
\begin{align}
    \Phi : \Group \to \R^{(n + 1) \times (n + 1)} :
        (x, k) \mapsto
            \begin{pmatrix}
                \phi \circ k \circ \phi^{-1} & \phi(x)\\
                0 & 1,
            \end{pmatrix}
    \label{eq:motion_group_realised_as_matrices}
\end{align}
where by $\phi \circ k \circ \phi^{-1}$ we actually mean its corresponding matrix in the canonical basis.

\begin{definition}[Lie Algebra]
\label{definition:Lie_Algebra}
\index{generalised motion group!Lie algebra}
    Suppose $\VectorSpace$ has dimension $n$.
    The real vector space
    \begin{align*}
        \LieAlgebra \defeq
            \{
                X \in \R^{(n + 1) \times (n + 1)} :
                \text{for each}\
                t \in \R,\
                \exp(t X) \in \Phi(\Group),
            \}
    \end{align*}
    where $\Phi$ is the map defined in~\eqref{eq:motion_group_realised_as_matrices},
    is called the \emph{Lie algebra} of $\Group$.

    Moreover, given $X, Y \in \LieAlgebra$, we define its \emph{Lie Bracket} via
    \begin{align*}
        \LieBracket{X}{Y} = X Y - Y X.
    \end{align*}
\end{definition}

\begin{definition}[Exponential map]
\label{definition:exponential_map}
\index{generalised motion group!exponential map}
    The \emph{exponential map} on $\Group$ is the map
    \begin{align*}
        \exp_\Group : \LieAlgebra \to \Group
    \end{align*}
    defined by $\exp_\Group \defeq \Phi^{-1} \circ \exp$,
    where $\Phi$ is the map defined in~\eqref{eq:motion_group_realised_as_matrices}.
\end{definition}

\begin{definition}
\label{definition:left-invariant_differential_operator}
    Let $X \in \LieAlgebra$.
    We define $\LeftDifferentialOperatorFirstOrder{X}$, the \emph{left-invariant differential operator associated to $X$}, via
    \begin{align*}
        \LeftDifferentialOperatorFirstOrder{X} f(g)
            = \D*{t}<t = 0> f(g \exp_\Group(t X)),
    \end{align*}
    where $f \in \SmoothFunctions{\Group}$.
\end{definition}

\begin{definition}
\label{definition:right-invariant_differential_operator}
    Let $X \in \LieAlgebra$.
    We define $\RightDifferentialOperatorFirstOrder{X}$, the \emph{right-invariant differential operator associated to $X$}, via
    \begin{align*}
        \RightDifferentialOperatorFirstOrder{X} f(g)
            = \D*{t}<t = 0> f(\exp_\Group(t X) g),
    \end{align*}
    where $f \in \SmoothFunctions{\Group}$.
\end{definition}

\begin{proposition}
    Let $X \in \LieAlgebra$.
    The differential operator $\LeftDifferentialOperatorFirstOrder{X}$ is the only differential operator satisfying the following properties:
    \begin{enumerate}
        \item $\LeftDifferentialOperatorFirstOrder{X}$ is \emph{left-invariant},
            i.e. for every $h \in \Group$, we have
            \begin{align*}
                (X f(h \dummy))(g) = (X f)(h g);
            \end{align*}
        \item The vector in $T_e \Group$ corresponding to the differentiation by $\LeftDifferentialOperatorFirstOrder{X}$ at $e$ is precisely $\dd \Phi^{-1}(X)$.
    \end{enumerate}
\end{proposition}

\begin{example}[$2$-dimensional Euclidean motion group]
\label{example:Lie_Algebra_of_2-dimensional_Euclidean_motion_group}
    Assume $\Group = \R^2 \rtimes \T$.
    The Lie Algebra is the vector space generated by the matrices
    \begin{align*}
        X_1 =
            \begin{pmatrix}
                0 & 0 & 1\\
                0 & 0 & 0\\
                0 & 0 & 0
            \end{pmatrix},\quad
        X_2 =
            \begin{pmatrix}
                0 & 0 & 0\\
                0 & 0 & 1\\
                0 & 0 & 0
            \end{pmatrix},\quad
        X_3 =
            \begin{pmatrix}
                0 & -1 & 0\\
                1 &  0 & 0\\
                0 &  0 & 0
            \end{pmatrix},
    \end{align*}
    which satisfy the commutation relations
    \begin{align*}
        \LieBracket{X_1}{X_2} = 0,\quad
        \LieBracket{X_2}{X_3} = X_1,\quad
        \LieBracket{X_3}{X_1} = X_2
    \end{align*}

    Moreover, if $f \in \SmoothFunctions{\Group}$,
    then the associated left-invariant operators act via
    \begin{align*}
        \LeftDifferentialOperatorFirstOrder{X_1} f(x, t)
            &= \cos(\turn t) \D[f]{x_1}(x, t) + \sin(\turn t) \D[f]{x_2}(x, t)\\
        \LeftDifferentialOperatorFirstOrder{X_2} f(x, t)
            &= -\sin(\turn t) \D[f]{x_1}(x, t) + \cos(\turn t) \D[f]{x_2}(x, t)\\
        \LeftDifferentialOperatorFirstOrder{X_3} f(x, t)
            &= \D[f]{t}(x, t),
    \end{align*}
    where $(x, t) \in \R^2 \rtimes \T$.
\end{example}

From now on, we fix a basis $X_1, \dots, X_{\dim \Group}$ of $\LieAlgebra$ such that:
\begin{enumerate}
    \item If $i = 1, \dots, \dim \VectorSpace$,
        \begin{align*}
            X_i =
                \begin{pmatrix}
                    \Id{\dim \VectorSpace} & e_i\\
                    0 & 1,
                \end{pmatrix}
        \end{align*}
        where $e_i$ is the $i$-th vector of the canonical basis of $\R^{\dim \VectorSpace}$;
    \item If $i = \dim \VectorSpace + 1, \dots, \dim \Group$, then
        \begin{align}
            X_i =
                \begin{pmatrix}
                    Y_i & 0\\
                    0 & 1,
                \end{pmatrix}
                \label{eq:Lie_algebra_vector_coming_from_compact_group}
        \end{align}
        where $\{Y_i : i = \dim \VectorSpace + 1, \dots, \dim \Group\}$ forms a basis of the Lie Algebra of $\CompactGroup$.
\end{enumerate}

\begin{definition}
    Let $\alpha \in \N^{\dim \Group}$.
    We define the left-invariant differential operator $\LeftDifferentialOperator{\alpha}$ via
    \begin{align*}
        \LeftDifferentialOperator{\alpha} =
        \LeftDifferentialOperatorFirstOrder{X_1}^{\alpha_1} \dots
        \LeftDifferentialOperatorFirstOrder{X_{\dim \Group}}^{\alpha_{\dim \Group}}
    \end{align*}
\end{definition}

\section{Unitary representations}

\begin{definition}
\label{definition:reducible_representation}
    Let $\lambda \in \dualGroup{\VectorSpace}$.
    We define a unitary representation $\Rep{\lambda} \in \Hom(\Group, \End(\Lebesgue{2}{\CompactGroup}))$ of $\Group$ via
    \begin{align}
        \Rep{\lambda} (x, k) F(u) \defeq (u \lambda)(x) F(k^{-1} u),
    \end{align}
    where $(x, k) \in \CompactGroup$, $F \in \Lebesgue{2}{\CompactGroup}$ and $u \in \CompactGroup$.
\end{definition}

Unfortunately, the above representation is often reducible.
However, as we shall see later, the Fourier Transform on $\Group$ can be written exclusively with those representations.

\begin{example}[$2$-dimensional Euclidean motion group]
    Let $\lambda \in \R^2$.
    Let $(x, t) \in \Group = \R^2 \rtimes \SpecialOrthogonalGroup{2}$.
    If $\lambda \neq 0$, then $\Rep{\lambda}$ is irreducible.
\end{example}

\subsection{Unitary dual}

Our description of the unitary dual comes from \cite{Kumahara73},
which itself is a minor adaptation of the one in \cite{Ito52}.
Although both articles only specifically mention the case of the Euclidean motion group,
the author of \cite{Ito52} mentions that the arguments generalise verbatim to generalised motion groups.

Throughout this section, fix $\lambda \in \dualGroup{\VectorSpace}$
and denote by $\IsotropySubgroup{\CompactGroup}{\lambda}$ its isotropy subgroup.

Let $\tau \in \dualGroup{\IsotropySubgroup{\CompactGroup}{\lambda}}$ and denote by $\dimRep{\tau}$ its dimension.
For $q = 1, \dots, \dimRep{\tau}$, let
\begin{align}
    P^\tau_q F(u) \defeq \dimRep{\tau} \int_\IsotropySubgroup{\CompactGroup}{\lambda} \conj{\tau_{qq}(m)} F(u m) \dd m,
    \quad F \in \Lebesgue{2}{\CompactGroup}.
    \label{eq:projection_on_L2_of_the_compact_group}
\end{align}

By the Inverse Fourier Transform at $e \in \IsotropySubgroup{\CompactGroup}{\lambda}$,
if $F \in \Lebesgue{2}{\CompactGroup}$ and $u \in \CompactGroup$, then
\begin{align}
    F(u)
    = \sum_{\tau \in \dualGroup{\IsotropySubgroup{\CompactGroup}{\lambda}}} \dimRep{\tau} \sum_{q = 1}^{\dimRep{\tau}} {\Fourier[\IsotropySubgroup{\CompactGroup}{\lambda}]{F(u \dummy)}}_{qq}
    = \sum_{\tau \in \dualGroup{\IsotropySubgroup{\CompactGroup}{\lambda}}} P^\tau_q F(u)
\end{align}

Now, write $\Hilbert{\tau}{q} \defeq P^\tau_q \Lebesgue{2}{\CompactGroup}$.

\begin{lemma}
    Let $\mu, \tau \in \dualGroup{\IsotropySubgroup{\CompactGroup}{\lambda}}$, $q \in \{1, \dots, \dimRep{\tau}\}$ and $m, n \in \{1, \dots, \dimRep{\tau} \}$.
    If $f \in \Lebesgue{2}{\RightQuotient{\CompactGroup}{\IsotropySubgroup{\CompactGroup}{\lambda}}}$, then
    \begin{align*}
        P^\mu_q (f \otimes \tau_{m n}) = \Kronecker{\mu}{\tau} \Kronecker{n}{q} (f \otimes \tau_{m n}).
    \end{align*}

    In particular, the following properties hold:
    \begin{enumerate}
        \item $P^\mu_q$ is an orthogonal projection onto
            \begin{align*}
                \Hilbert{\mu}{q} =
                    \Lebesgue{2}{\RightQuotient{\CompactGroup}{\IsotropySubgroup{\CompactGroup}{\lambda}}}
                    \otimes
                    \Span \{\mu_{p q} : p = 1, \dots, \dimRep{\mu}\};
            \end{align*}
        \item The Hilbert spaces
            \begin{align*}
                \{\Hilbert{\tau}{q} : \tau \in \dualGroup{\IsotropySubgroup{\CompactGroup}{\lambda}}, q = 1, \dots, \dimRep{\tau} \}
            \end{align*}
            are mutually orthogonal;
        \item We have the decomposition
            \begin{align*}
                \Lebesgue{2}{\CompactGroup} = \bigoplus_{\tau \in \dualGroup{\IsotropySubgroup{\CompactGroup}{\lambda}}} \bigoplus_{q = 1}^{\dimRep{\tau}} \Hilbert{\tau}{q}.
            \end{align*}
    \end{enumerate}
\end{lemma}
\begin{proof}
    Fix $u \in \CompactGroup$ and write $u = u' u''$,
    where $u' \in \RightQuotient{\CompactGroup}{\IsotropySubgroup{\CompactGroup}{\lambda}}$
    and $u'' \in \IsotropySubgroup{\CompactGroup}{\lambda}$.
    It follows that
    \begin{align*}
        P^\mu_q (f \otimes \tau_{m n}) (u)
        &= \dimRep{\mu}
            \int_\IsotropySubgroup{\CompactGroup}{\lambda}
                \conj{\mu_{q q}(m)}
                f(u')
                \tau_{m n}(u'' m)
            \dd m\\
        &= \sum_{p = 1}^\dimRep{\tau}
                f(u')
                \tau_{m p}(u'')
                \dimRep{\mu}
                \int_\IsotropySubgroup{\CompactGroup}{\lambda}
                    \conj{\mu_{q q}(m)}
                    \tau_{p n}(m)
                \dd m.
    \end{align*}

    Using the Peter-Weyl Theorem, we conclude that in fact
    \begin{align*}
        P^\mu_q (f \otimes \tau_{m n}) (u)
        &= \sum_{p = 1}^\dimRep{\tau}
            \Kronecker{\mu}{\tau}
            \Kronecker{q}{p}
            \Kronecker{q}{n}
            f(u')
            \tau_{m p}(u'')\\
        &= \Kronecker{\mu}{\tau}
            \Kronecker{q}{n}
            f(u')
            \tau_{m q}(u'')\\
        &= \Kronecker{\mu}{\tau}
            \Kronecker{q}{n}
            (f \otimes \tau_{m n})(u),
    \end{align*}
    which is what we wanted to show.
\end{proof}

The following result and its proof can be found in \cite[Theorem 1.1, 1.2, 1.3]{Ito52}.
Although the paper specifically treats the case of the Euclidean motion groups,
the author remarks (\cite[Remark p. 84]{Ito52}) that the argument works for generalised motion groups.

\begin{proposition}[Unitary dual]
\label{proposition:unitary_dual}
    Let $\lambda, \lambda' \in \dualGroup{\VectorSpace}$
    and $\tau \in \dualGroup{\IsotropySubgroup{\CompactGroup}{\lambda}}$,
    $\tau' \in \dualGroup{\IsotropySubgroup{\CompactGroup}{\lambda'}}$.
    The following properties hold
    \begin{enumerate}
        \item $\Rep{\lambda}$ restricts to an irreducible unitary representation on each $\Hilbert{\tau}{q}$;
        \item $(\Hilbert{\tau}{q}, \Rep{\lambda})$ and $(\Hilbert{\tau'}{q'}, \Rep{\lambda'})$ are equivalent if and only if
            \begin{align*}
                \lambda' = k \lambda \quad \text{and} \quad \EquivalenceClass{\dualGroup{\IsotropySubgroup{\CompactGroup}{\lambda}}}{\tau} = \EquivalenceClass{\dualGroup{\IsotropySubgroup{\CompactGroup}{\lambda}}}{\tau'(k \dummy k^{-1})}
            \end{align*}
            for some $k \in \CompactGroup$.
    \end{enumerate}
\end{proposition}

\begin{definition}[Unitary dual]
\label{definition:unitary_dual}
\index{generalised motion group!unitary dual}
    Fix $\lambda_0 \in \dualGroup{\VectorSpace}$.
    We define the \emph{unitary dual} of $\Group$, denoted by $\dualGroup{\Group}$, via
    \begin{align*}
        \dualGroup{\Group} \defeq \{ (\Hilbert{\tau}{1}, \Rep{\lambda}) : \lambda \in \LeftQuotient{\IsotropySubgroup{\CompactGroup}{\lambda_0}}{\dualGroup{\VectorSpace}}, \tau \in \dualGroup{\IsotropySubgroup{\CompactGroup}{\lambda}} \}.
    \end{align*}
\end{definition}

\subsection{Infinitesimal representations}

\begin{definition}[Infinitesimal Representation]
\label{definition:infinitesimal_representation}
\index{generalised motion group!infinitesimal representation}
    Let $X \in \g$.
    We define the infinitesimal representation of $X$ as the operator
    \begin{align*}
        \Rep{\lambda}(X) : \SmoothFunctions{\CompactGroup} \to \SmoothFunctions{\CompactGroup}
    \end{align*}
    defined via
    \begin{align*}
        \Rep{\lambda}(X) F(u) \defeq \D*{t}<t=0> \Rep{\lambda}(\exp(t X)) F(u),
    \end{align*}
    where $F \in \SmoothFunctions{\CompactGroup}$.
\end{definition}

\begin{proposition}[Infinitesimal representations]
    Let $\lambda \in \dualGroup{\VectorSpace}$ and let $j \in \{1, \dots, \dim \Group\}$.
    The infinitesimal representation of $X_j$ has the following expression:
    \begin{enumerate}
        \item if $j \leq \dim \VectorSpace$, then
            \begin{align*}
                \Rep{\lambda}(X_j) F(u) = \lambda(u^{-1} e_i) F(u)
            \end{align*}
        \item if $j > \dim \VectorSpace$, then
            \begin{align*}
                \Rep{\lambda}(X_j) F(u) = -\RightDifferentialOperatorFirstOrder{Y_j} F(u),
            \end{align*}
            where $Y_j$ is like in~\eqref{eq:Lie_algebra_vector_coming_from_compact_group} and
            $\RightDifferentialOperatorFirstOrder{Y_j}$ is the right-invariant differential operator on $\CompactGroup$ associated with $Y_j$.
    \end{enumerate}
    In the above, $F$ is an arbitrary function in $\SmoothFunctions{\CompactGroup}$.
\end{proposition}
\begin{proof}
    \begin{enumerate}
        \item Fix $j \in \{1, \dots, \dim \VectorSpace\}$.
            It follows that
            \begin{align*}
                \Rep{\lambda}(X) F(u) &= \D*{t}<t = 0> \lambda(t u^{-1} e_i) F(u)\\
                                      &= \lambda( u^{-1} e_i) F(u),
            \end{align*}
            where the last line was obtained because $\lambda$ is linear.
        \item Let $Y_j$ be like in~\eqref{eq:Lie_algebra_vector_coming_from_compact_group} so that
            \begin{align*}
                \exp_\Group(t X_j) = \Phi^{-1}
                    \begin{pmatrix}
                        \exp_\CompactGroup(t Y_j) & 0\\
                        0 & 1
                    \end{pmatrix}
                    = (0, \exp_\CompactGroup(t Y_j)).
            \end{align*}

            From there, it immediately follows that
            \begin{align*}
                \Rep{\lambda}(X_j) F(u)
                = \D*{t}<t = 0> F({(\exp_\CompactGroup(t Y_j))}^{-1} u)
                = -\RightDifferentialOperatorFirstOrder{Y_j} F(u).
            \end{align*}
    \end{enumerate}
\end{proof}

\begin{lemma}[Infinitesimal representation of $\Laplacian$]
\label{lemma:infinitesimal_representation_of_the_Laplacian}
    Let $\lambda \in \dualGroup{\VectorSpace}$.
    The infinitesimal representation of $\Laplacian$ is given by
    \begin{align*}
        \Rep{\lambda}(\Laplacian) = - \norm[\dualGroup{\VectorSpace}]{\lambda}^2 \Id{\Lebesgue{2}{\CompactGroup}} + \RightLaplacian[\CompactGroup].
    \end{align*}
\end{lemma}

\section{Fourier Transform}

\subsection{Definition and elementary properties}

\begin{definition}[Fourier transform]
\label{definition:Fourier_Transform}
\index{generalised motion group!Fourier transform}
    Let $f \in \Lebesgue{1}{\Group}$ and $\lambda \in \dualGroup{\VectorSpace}$.
    We define its \emph{Fourier coefficient} at $\lambda$ via
    \begin{align*}
        \Fourier{f}(\lambda) \defeq \int_\Group f(g) \adj{\Rep{\lambda}(g)} \dd g.
    \end{align*}

    Moreover, the map
    \begin{align*}
        \Fourier{f} : \dualGroup{\VectorSpace} \to \End(\Lebesgue{2}{\CompactGroup}) :
        \lambda \mapsto \Fourier{f}(\lambda)
    \end{align*}
    is called the \emph{Fourier Transform} of $f$.
\end{definition}

\subsection{Plancherel formula}

\begin{proposition}[Plancherel formula]
\label{proposition:Plancherel_formula}
\index{generalised motion group!Fourier transform!Plancherel formula}
    Let $f \in \Lebesgue{1}{\Group} \cap \Lebesgue{2}{\Group}$.
    The following formula holds
    \begin{align}
        \int_G \abs{f}^2 \dd g = \int_\dualGroup{\VectorSpace} \norm[\HilbertSchmidt{\Lebesgue{2}{\CompactGroup}}]{\Fourier{f}(\lambda)}^2 \dd \Plancherel{\VectorSpace}(\lambda).
        \label{proposition:Plancherel_formula:formula}
    \end{align}
\end{proposition}
\begin{proof}
    Let $\lambda \in \dualGroup{\VectorSpace}$ and $F \in \Lebesgue{2}{\CompactGroup}$.
    If $u \in \CompactGroup$, we can check that
    \begin{align*}
        \Fourier{f}(\lambda) F(u)
        &= \int_\VectorSpace \int_\CompactGroup f(x, k) \lambda(-u^{-1} k^{-1} x) F(k u) \dd k \dd x\\
        &= \int_\VectorSpace \int_\CompactGroup f(x, k) \conj{k u \lambda(x)} F(k u) \dd k \dd x,
    \end{align*}
    where we used $\lambda(-{(k u)}^{-1} x) = \conj{\lambda({(k u)}^{-1} x)} = \conj{(k u \lambda)(x)}$.

    Substituing $k$ for $k u^{-1}$, we get
    \begin{align*}
        \Fourier{f}(\lambda) F(u)
        &= \int_\CompactGroup \Fourier[\VectorSpace]{f}(k \lambda, k u^{-1}) F(k) \dd k.
    \end{align*}

    Therefore, it follows by REFERENCE that
    \begin{align*}
        \norm[\HilbertSchmidt{\Lebesgue{2}{\CompactGroup}}]{\Fourier{f}(\lambda)}^2
        &= \int_\CompactGroup \int_\CompactGroup \abs{\Fourier[\VectorSpace]{f}(k \lambda, k u^{-1})}^2 \dd u \dd k.
    \end{align*}

    Now, integrating with respect to $\lambda$,
    and using the fact that $\CompactGroup$ acts transitively on $\VectorSpace$,
    we obtain
    \begin{align*}
        \int_\dualGroup{\VectorSpace} \norm[\HilbertSchmidt{\Lebesgue{2}{\CompactGroup}}]{\Fourier{f}(\lambda)}^2 \dd \Plancherel{\VectorSpace}(\lambda)
        &= \int_\dualGroup{\VectorSpace} \int_\CompactGroup \abs{\Fourier[\VectorSpace]{f}(\lambda, k)}^2 \dd k \dd \Plancherel{\VectorSpace}(\lambda)\\
        &= \int_\VectorSpace \int_\CompactGroup \abs{f(x, k)}^2 \dd u \dd k,
    \end{align*}
    where the last line was obtained by applying the Plancherel formula on $\VectorSpace$.
\end{proof}

\begin{definition}[$\LebesgueDual{2}{\Group}$]
    We shall say that a map
    \begin{align*}
        \sigma : \dualGroup{\VectorSpace} \to \HilbertSchmidt{\Lebesgue{2}{\CompactGroup}}
    \end{align*}
    belongs to $\LebesgueDual{2}{\Group}$ if and only if $\sigma$ is measurable and the quantity
    \begin{align*}
        \norm[\LebesgueDual{2}{\Group}]{\sigma} \defeq
            \left(
                \int_\dualGroup{\VectorSpace}
                    \norm[\HilbertSchmidt{\Lebesgue{2}{\CompactGroup}}]{\sigma(\lambda)}^2
                \dd \Plancherel{\VectorSpace}(\lambda)
            \right)^{\frac{1}{2}}
    \end{align*}
    is finite.

    If $\sigma_1, \sigma_2 \in \LebesgueDual{2}{\Group}$, then we let
    \begin{align*}
        \ip[\LebesgueDual{2}{\Group}]{\sigma_1}{\sigma_2} \defeq
        \int_\dualGroup{\VectorSpace}
            \tr\left(
                \sigma_1(\lambda) \adj{\sigma_2(\lambda)}
            \right)
        \dd \Plancherel{\VectorSpace}(\lambda).
    \end{align*}
    If we quotient $\LebesgueDual{2}{\Group}$ by $\Plancherel{\VectorSpace}$-almost everywhere equality,
    which we shall do from now onwards,
    then the above gives $\LebesgueDual{2}{\Group}$ the structure of a Hilbert space.
\end{definition}

\begin{definition}[$\Kernels{\Group}$]
    We shall say that a tempered distribution $\kappa \in \TemperedDistributions{\Group}$ belongs to $\Kernels{\Group}$
    if and only if the map
    \begin{align}
        T_\kappa : \Schwartz{\Group} \to \TemperedDistributions{\Group} : f \mapsto \conv{f}{\kappa}
    \end{align}
    extends to a continuous map from $\Lebesgue{2}{\Group}$ into itself.
    In this case, we let
    \begin{align*}
        \norm[\Kernels{\Group}]{\kappa} \defeq \norm[\Lin{\Lebesgue{2}{\Group}}]{T_\kappa}.
    \end{align*}
\end{definition}

\begin{definition}
    We shall say that a map
    \begin{align*}
        \sigma : \dualGroup{\VectorSpace} \to \Lin{\Lebesgue{2}{\CompactGroup}}
    \end{align*}
    belongs to $\LebesgueDual{\infty}{\Group}$ if and only if the quantity
    \begin{align*}
        \norm[\LebesgueDual{\infty}{\Group}]{\sigma} \defeq
            \esssup_{\lambda \in \dualGroup{\VectorSpace}}
                \norm[\Lin{\Lebesgue{2}{\CompactGroup}}]{\sigma(\lambda)}
    \end{align*}
    is finite.
    The essential supremum is taken with respect to the Plancherel measure.
\end{definition}

\begin{theorem}[Abstract Plancherel formula]
    The Fourier Transform can be extended to a \emph{surjective} isometry
    \begin{align*}
        \Fourier : \Lebesgue{2}{\Group} \to \LebesgueDual{2}{\Group}.
    \end{align*}

    Moreover, for every left-invariant operator $T \in \Lin{\Lebesgue{2}{\Group}}$,
    there exists a unique element $\sigma \in \LebesgueDual{\infty}{\Group}$ such that
    \begin{align*}
        \Fourier\{T f\}(\lambda) = \sigma(\lambda) \Fourier f(\lambda)
    \end{align*}
    holds for $\Plancherel{\Group}$-almost every $\lambda \in \dualGroup{\VectorSpace}$.
\end{theorem}

\begin{proposition}[Inverse Fourier Transform]
\label{proposition:inverse_Fourier_Transform}
\index{generalised motion group!Fourier transform!inverse formula}
    Let $\phi \in \Schwartz{\Group}$.
    For each $g \in \Group$,
    we have
    \begin{align*}
        \phi(g)
        = \int_\dualGroup{\VectorSpace}
        \tr \left( \Rep{\lambda}(g) \Fourier \phi(\lambda) \right) \dd \Plancherel{\VectorSpace}(\lambda).
    \end{align*}
\end{proposition}

\subsection{Sobolev spaces}

\subsection{Fourier Transform of distributions}
