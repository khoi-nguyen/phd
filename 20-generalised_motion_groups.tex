\chapter{Generalised Motion Groups}

\section{Generalised motion groups}

\begin{definition}[Generalised motion group]
\label{definition:generalised_motion_group}
\index{generalised motion group}
    Let $\Group$ be a group.
    We shall say that $\Group$ is a \emph{generalised motion group}
    if there exists a (finite dimensional) real Euclidean space $\VectorSpace$
    and a compact connected Lie group $\CompactGroup \subset \GeneralLinear\VectorSpace$
    such that $\Group = \VectorSpace \rtimes \CompactGroup$
    and $\CompactGroup$ acts transitively on $\VectorSpace$.
\end{definition}

\begin{lemma}[$\CompactGroup$-invariant inner product]
    Let $\VectorSpace$ be a vector space,
    and $\CompactGroup$ be a compact Lie group acting acting on $\VectorSpace$.
    There exists an inner product $\ip{\dummy}{\dummy} : \VectorSpace \times \VectorSpace \to \R$ which is $\CompactGroup$-invariant,
    i.e.\ for each $k \in \CompactGroup$ and every $x, y \in \VectorSpace$, we have
    \begin{align*}
        \ip{x}{y} = \ip{k x}{k y}.
    \end{align*}
\end{lemma}
\begin{proof}
    Let $Q : \VectorSpace \times \VectorSpace \to \R$ be an arbitrary inner product.
    Given $x, y \in \VectorSpace$, we let
    \begin{align*}
        \ip{x}{y} \defeq \int_\CompactGroup Q(k x, k y) \dd k,
    \end{align*}
    where $\dd k$ is the Haar measure on $\CompactGroup$.

    The fact that $\ip{\dummy}{\dummy}$ is bilinear and nonnegative definite follows immediately from the fact that $Q$ has those properties.
    Now, if $x, y \in \VectorSpace$ are such that $\ip{x}{x} = 0$.
    It follows that $Q(k x, k x) = 0$ for almost every $k \in \CompactGroup$, hence for at least one such $k$.
    However, that means that $k x = 0$, hence $x = 0$ as $k$ is invertible.
\end{proof}

\begin{remark}
    Let $x \in \VectorSpace$ and $k \in \CompactGroup$.
    We will never identify $x$ with $(x, \Id{\VectorSpace}) \in \VectorSpace \rtimes \CompactGroup$,
    or $k$ with $(0, k) \in \VectorSpace \rtimes \CompactGroup$.

    Therefore, when we write $k x$, it will \emph{always} mean the vector obtained by rotating $x$ by $k$, i.e.\ $k(x)$.
    If we see them as elements of $\VectorSpace \rtimes \CompactGroup$,
    we shall explicitely write
    \begin{align*}
        (0, k) (x, \Id{\VectorSpace}) = (k x, k).
    \end{align*}
\end{remark}

\begin{example}[Vector spaces]
\label{example:trivial_case_of_generalised_motion_groups}
    Suppose that $\CompactGroup = \{\Id\VectorSpace\}$.
    It follows that $\Group \defeq \VectorSpace \rtimes \CompactGroup$ is isomorphic to $\VectorSpace$.
\end{example}

\begin{example}[Euclidean Motion Groups]
\label{example:Euclidean_motion_groups}
\index{Euclidean motion group}
    For each $n \in \N$, let
    \begin{align*}
        \MotionGroup{n} \defeq \{g \in \AffineTransformations{\R^n} : \det g = 1\}.
    \end{align*}
    The elements of $\MotionGroup{n}$ are called \emph{rigid motions},
    while $\MotionGroup{n}$ is called the \emph{Euclidean motion group}.

    It is easily shown that associating $(x, k) \in \R^n \rtimes \SpecialOrthogonalGroup{n}$ to the motion
    \begin{align*}
        g_{(x, k)} : \R^n \to \R^n : y \mapsto x + ky
    \end{align*}
    defines a group isomorphism between $\R^n \rtimes \SpecialOrthogonalGroup{n}$ and $\MotionGroup{n}$.
    We shall therefore identify $\MotionGroup{n}$ with $\R^n \rtimes \SpecialOrthogonalGroup{n}$ from now on.
\end{example}

\begin{example}[$2$-dimensional Euclidean motion group]
\label{example:Euclidean_motion_groups:dimension_2}
\index{Euclidean motion group!dimenion 2}
    The $2$-dimensional case is worth mentioning because
    in that case, $\SpecialOrthogonalGroup{2}$ is \emph{abelian}.

    Since $\SpecialOrthogonalGroup{2}$ is isomorphic to $\T$,
    we shall often identify $\MotionGroup{2}$ with $\R^2 \times \T$ and the group law
    \begin{align*}
        (x, t) (y, s) \defeq (x + \e^{\i \turn t} y, t + s), \quad (x, t), (y, s) \in \R^2 \times \T.
    \end{align*}
\end{example}

\begin{example}
    \label{example:complex_motion_groups}
    Let $n \in \N$.
    Consider the group
    \begin{align*}
        \{g \in \AffineTransformations{\C^n} : \det_{\C^n} g = 1\}
    \end{align*}
    where the law is the composition of functions.

    Arguing like in Example~\ref{example:Euclidean_motion_groups},
    the above group can be identified with $\C^n \rtimes \SpecialUnitaryGroup{n}$.
\end{example}

\begin{remark}
    Since in our examples (e.g. Example \ref{example:complex_motion_groups}) our vector space might be $\C^n$,
    we choose to use $\VectorSpace$ to denote the vector space instead of simply $\R^n$ to avoid any confusion.
\end{remark}

From now on, unless stated otherwise,
$\Group$ will denote a generalised motion group,
with $\VectorSpace$ its underlying Eucliean space and $\CompactGroup$ its associated compact group.

The requirement that $\CompactGroup$ should be a subgroup of $\OrthogonalGroup{\VectorSpace}$ is motivated by the following result

\begin{lemma}[Haar measure]
\label{lemma:Haar_measure}
    If $\dd x$ is a Lebesgue measure on $\VectorSpace$ and $\dd k$ is the normalised Haar measure on $\CompactGroup$,
    then the the product measure $\dd x \dd k$ is a Haar measure on $\Group = \VectorSpace \rtimes \CompactGroup$,
    which is both left and right-invariant.
\end{lemma}
\begin{proof}
    Let $(x, k) \in \Group$.
    \begin{align*}
        \int_\Group f((x, k) (y, l)) \dd (y, l)
        = \int_\VectorSpace \int_\CompactGroup f(x + ky, k l) \dd l \dd y
    \end{align*}

    Now, let us substitute $y$ for $k^{-1}(y - x)$ and $l$ for $k^{-1} l$ in the above.
    As the Lebesgue measure is invariant under $\OrthogonalGroup{\VectorSpace}$ and under translations,
    and because the Haar measure $\dd l$ is left-invariant,
    we obtain
    \begin{align*}
        \int_\Group f((x, k) (y, l)) \dd (y, l)
        &= \int_\VectorSpace \int_\CompactGroup f(y, l) \dd l \dd y\\
        &= \int_\Group f(y, l) \dd (y, l),
    \end{align*}
    showing that $\dd y \dd l$ is indeed a Haar measure on $\Group$.

    Since by Proposition~\ref{proposition:sufficient_conditions_to_be_unimodular} $\dd l$ is also right-invariant,
    arguing similarly shows that $\dd y \dd l$ is also right-invariant.
\end{proof}

\begin{proposition}
    Let $\VectorSpace$ be a finite dimensional vector space
    and $\CompactGroup$ be a subgroup of $\SpecialOrthogonalGroup\VectorSpace$.
    The following properties hold:
    \begin{enumerate}
        \item The Lebesgue measure on $\VectorSpace$ is invariant under $\CompactGroup$,
            i.e.\ for every $k \in \CompactGroup$ and each Borel set $A \subset \VectorSpace$, we have
            \begin{align*}
                \int_A 1 \dd x = \int_{kA} 1 \dd x;
            \end{align*}
        \item The Laplacian on $\VectorSpace$ is invariant under $\CompactGroup$,
            i.e.\ for every $k \in \CompactGroup$ and every $\phi \in \SmoothFunctions\VectorSpace$, we have
            \begin{align*}
                \Laplacian[\VectorSpace] (\phi \circ k)(x) = \Laplacian[\VectorSpace] \phi(k x);
            \end{align*}
        \item The action of $\CompactGroup$ on $\VectorSpace$ commutes with the dilation structure of $\VectorSpace$.
    \end{enumerate}
\end{proposition}
\begin{proof}
    \begin{enumerate}
        \item This follows easily from the change of variables formula,
            \begin{align*}
                \int_{k A} 1 \dd x
                = \int_A (1 \circ k) \det k \dd x
                = \int_A 1 \dd x,
            \end{align*}
            where we used the fact that $\det k = 1$ since $k \in \SpecialOrthogonalGroup\VectorSpace$.
        \item First, observe that
            \begin{align*}
                \dd (\phi \circ k)(x) = \dd \phi(k x) k
            \end{align*}
            which implies that $\grad (\phi \circ k)(x) = k^{-1} \grad \phi(k x)$.

            From there, using the fact that $\Hessian = \dd \grad$,
            \begin{align*}
                \Hessian (\phi \circ k)(x) = (\dd \grad(\phi \circ k))(x) = k^{-1} (\Hessian \phi(k x)) k.
            \end{align*}

            Therefore, we conclude by observing that
            \begin{align*}
                \Laplacian[\VectorSpace] (\phi \circ k)(x)
                = \tr (\Hessian (\phi \circ k)(x))
                = \tr (\Hessian \phi (k x))
                = \Laplacian[\VectorSpace] \phi(k x).
            \end{align*}
    \end{enumerate}
\end{proof}

\section{Lie algebra structure}

Using an isomorphism
\begin{align}
    \phi : \VectorSpace \to \R^n
\end{align}
where $n = \dim \VectorSpace$,
we can represent the generalised motion group $\Group$
by real matrices via the map
\begin{align}
    \Phi : \Group \to \R^{(n + 1) \times (n + 1)} :
        (x, k) \mapsto
            \begin{pmatrix}
                \phi \circ k \circ \phi^{-1} & \phi(x)\\
                0 & 1,
            \end{pmatrix}
    \label{eq:motion_group_realised_as_matrices}
\end{align}
where by $\phi \circ k \circ \phi^{-1}$ we actually mean its corresponding matrix in the canonical basis.

\begin{definition}[Lie Algebra]
\label{definition:Lie_Algebra}
\index{generalised motion group!Lie algebra}
    Suppose $\VectorSpace$ has dimension $n$.
    The real vector space
    \begin{align*}
        \LieAlgebra \defeq
            \{
                X \in \R^{(n + 1) \times (n + 1)} :
                \text{for each}\
                t \in \R,\
                \exp(t X) \in \Phi(\Group),
            \}
    \end{align*}
    where $\Phi$ is the map defined in~\eqref{eq:motion_group_realised_as_matrices},
    is called the \emph{Lie algebra} of $\Group$.

    Moreover, given $X, Y \in \LieAlgebra$, we define its \emph{Lie Bracket} via
    \begin{align*}
        \LieBracket{X}{Y} = X Y - Y X.
    \end{align*}
\end{definition}

\begin{definition}[Exponential map]
\label{definition:exponential_map}
\index{generalised motion group!exponential map}
    The \emph{exponential map} on $\Group$ is the map
    \begin{align*}
        \exp_\Group : \LieAlgebra \to \Group
    \end{align*}
    defined by $\exp_\Group \defeq \Phi^{-1} \circ \exp$,
    where $\Phi$ is the map defined in~\eqref{eq:motion_group_realised_as_matrices}.
\end{definition}

\begin{definition}
\label{definition:left-invariant_differential_operator}
    Let $X \in \LieAlgebra$.
    We define $\LeftDifferentialOperatorFirstOrder{X}$, the \emph{left-invariant differential operator associated to $X$}, via
    \begin{align*}
        \LeftDifferentialOperatorFirstOrder{X} f(g)
            \defeq \D*{t}<t = 0> f(g \exp_\Group(t X)),
    \end{align*}
    where $f \in \SmoothFunctions{\Group}$.
\end{definition}

\begin{definition}
\label{definition:right-invariant_differential_operator}
    Let $X \in \LieAlgebra$.
    We define $\RightDifferentialOperatorFirstOrder{X}$, the \emph{right-invariant differential operator associated to $X$}, via
    \begin{align*}
        \RightDifferentialOperatorFirstOrder{X} f(g)
            \defeq \D*{t}<t = 0> f(\exp_\Group(t X) g),
    \end{align*}
    where $f \in \SmoothFunctions{\Group}$.
\end{definition}

\begin{proposition}
    Let $X \in \LieAlgebra$.
    The differential operator $\LeftDifferentialOperatorFirstOrder{X}$ is the only differential operator satisfying the following properties:
    \begin{enumerate}
        \item $\LeftDifferentialOperatorFirstOrder{X}$ is \emph{left-invariant},
            i.e. for every $h \in \Group$, we have
            \begin{align*}
                (X f(h \dummy))(g) = (X f)(h g);
            \end{align*}
        \item The vector in $T_e \Group$ corresponding to the differentiation by $\LeftDifferentialOperatorFirstOrder{X}$ at $e$ is precisely $\dd \Phi^{-1}(X)$.
        \item Given $X, Y \in \LieAlgebra$, we have
            \begin{align*}
                \LeftDifferentialOperatorFirstOrder{X} \LeftDifferentialOperatorFirstOrder{Y} - \LeftDifferentialOperatorFirstOrder{Y} \LeftDifferentialOperatorFirstOrder{X}
                = \LeftDifferentialOperatorFirstOrder{\LieBracket{X}{Y}}.
            \end{align*}
    \end{enumerate}
\end{proposition}

\begin{example}[$2$-dimensional Euclidean motion group]
\label{example:Lie_Algebra_of_2-dimensional_Euclidean_motion_group}
    Assume $\Group = \R^2 \rtimes \T$.
    The Lie Algebra is the vector space generated by the matrices
    \begin{align*}
        X_1 =
            \begin{pmatrix}
                0 & 0 & 1\\
                0 & 0 & 0\\
                0 & 0 & 0
            \end{pmatrix},\quad
        X_2 =
            \begin{pmatrix}
                0 & 0 & 0\\
                0 & 0 & 1\\
                0 & 0 & 0
            \end{pmatrix},\quad
        X_3 =
            \begin{pmatrix}
                0 & -1 & 0\\
                1 &  0 & 0\\
                0 &  0 & 0
            \end{pmatrix},
    \end{align*}
    which satisfy the commutation relations
    \begin{align*}
        \LieBracket{X_1}{X_2} = 0,\quad
        \LieBracket{X_2}{X_3} = X_1,\quad
        \LieBracket{X_3}{X_1} = X_2
    \end{align*}

    Moreover, if $f \in \SmoothFunctions{\Group}$,
    then the associated left-invariant operators act via
    \begin{align*}
        \LeftDifferentialOperatorFirstOrder{X_1} f(x, t)
            &= \cos(\turn t) \D[f]{x_1}(x, t) + \sin(\turn t) \D[f]{x_2}(x, t)\\
        \LeftDifferentialOperatorFirstOrder{X_2} f(x, t)
            &= -\sin(\turn t) \D[f]{x_1}(x, t) + \cos(\turn t) \D[f]{x_2}(x, t)\\
        \LeftDifferentialOperatorFirstOrder{X_3} f(x, t)
            &= \D[f]{t}(x, t),
    \end{align*}
    where $(x, t) \in \R^2 \rtimes \T$.
\end{example}

From now on, we fix a basis $X_1, \dots, X_{\dim \Group}$ of $\LieAlgebra$ such that:
\begin{enumerate}
    \item If $i = 1, \dots, \dim \VectorSpace$,
        \begin{align*}
            X_i =
                \begin{pmatrix}
                    \Id{\dim \VectorSpace} & e_i\\
                    0 & 1,
                \end{pmatrix}
        \end{align*}
        where $e_i$ is the $i$-th vector of the canonical basis of $\R^{\dim \VectorSpace}$;
    \item If $i = \dim \VectorSpace + 1, \dots, \dim \Group$, then
        \begin{align}
            X_i =
                \begin{pmatrix}
                    Y_i & 0\\
                    0 & 1,
                \end{pmatrix}
                \label{eq:Lie_algebra_vector_coming_from_compact_group}
        \end{align}
        where $\{Y_i : i = \dim \VectorSpace + 1, \dots, \dim \Group\}$ forms a basis of the Lie Algebra of $\CompactGroup$.
\end{enumerate}

\begin{lemma}
    Let $j \in \{1, \dots, \dim \VectorSpace\}$.
    If $\phi \in \SmoothFunctions\Group$,
    \begin{align*}
        \LeftDifferentialOperatorFirstOrder{X_j} \phi(x, k)
        = \ip{k^{-1} \grad \phi(x, k)}{e_j}
    \end{align*}
\end{lemma}
\begin{proof}
    By a simple calculation,
    \begin{align*}
        \LeftDifferentialOperatorFirstOrder{X_j} \phi(x, k)
        =& \D*{t}<t = 0> \phi((x, k) (t e_j, \Id\VectorSpace))\\
        =& \D*{t}<t = 0> \phi(x + t k e_j, k)\\
        =& \ip{\grad \phi(x, k)}{k e_j}.
    \end{align*}

    From there, it follows that:
    \begin{align*}
        \LeftDifferentialOperatorFirstOrder{X_j} \phi(x, k)
        = \ip{k^{-1} \grad \phi(x, k)}{e_j}
    \end{align*}
\end{proof}

\begin{definition}
    Let $\alpha \in \N^{\dim \Group}$.
    We define the left-invariant differential operator $\LeftDifferentialOperator{\alpha}$ via
    \begin{align*}
        \LeftDifferentialOperator{\alpha} =
        \LeftDifferentialOperatorFirstOrder{X_1}^{\alpha_1} \dots
        \LeftDifferentialOperatorFirstOrder{X_{\dim \Group}}^{\alpha_{\dim \Group}}
    \end{align*}
\end{definition}

\begin{definition}[Left-invariant Laplacian]
\label{definition:left-invariant_Laplacian}
\index{Laplacian}
    The \emph{left-invariant Laplacian} $\Laplacian$ is the left-invariant differential operator
    \begin{align*}
        \Laplacian \defeq \sum_{j = 1}^{\dim \VectorSpace} \LeftDifferentialOperatorFirstOrder{X_j}^2
    \end{align*}
\end{definition}

\section{Unitary representations}

\begin{definition}
\label{definition:reducible_representation}
    Let $\lambda \in \dualGroup{\VectorSpace}$.
    We define a unitary representation $\Rep{\lambda} \in \Hom(\Group, \End(\Lebesgue{2}{\CompactGroup}))$ of $\Group$ via
    \begin{align}
        \Rep{\lambda} (x, k) F(u) \defeq (u \lambda)(x) F(k^{-1} u),
        \label{eq:reducible_representations_on_the_generalised_motion_groups}
    \end{align}
    where $(x, k) \in \CompactGroup$, $F \in \Lebesgue{2}{\CompactGroup}$ and $u \in \CompactGroup$.
\end{definition}

Unfortunately, the above representation is often reducible.
However, as we shall see later, the Fourier Transform on $\Group$ can be written exclusively with those representations.

\begin{example}[$2$-dimensional Euclidean motion group]
    Let $\lambda \in \R^2$.
    Let $(x, t) \in \Group = \R^2 \rtimes \SpecialOrthogonalGroup{2}$.
    If $\lambda \neq 0$, then $\Rep{\lambda}$ is irreducible.

    Using the isomorphism $\SpecialOrthogonalGroup{2} \sim \T$,
    \eqref{eq:reducible_representations_on_the_generalised_motion_groups} takes the form
    \begin{align*}
        \Rep\lambda(x, t) F(u)
        = \lambda(-\e^{\i \turn u}x) F(u - t),
    \end{align*}
    where $F \in \Lebesgue{2}{\T}$, $(x, t) \in \Group$, and $u \in \T$.

    Defining ${\Rep\lambda(x, t)}_{m n} \defeq \ip[\Lebesgue{2}\T]{\Rep\lambda(x, t) \e^{\i \turn n \dummy}}{\e^{\i \turn m \dummy}}$ for every $m, n \in \Z$,
    it follows that
    \begin{align*}
        {\Rep\lambda(x, t)}_{m n}
        = \e^{-\i \turn n t} \int_\T \lambda(-\e^{\i \turn u}x) \e^{\i \turn (n - m)u} \dd u.
    \end{align*}
\end{example}

\begin{lemma}
    Suppose $\Group = \R^2 \rtimes \SpecialOrthogonalGroup{2}$.
    If $(x, t) \in \Group$, then for each $m, n \in \Z$, we have
    \begin{align*}
        (m - n) {\Rep\lambda(x, t)}_{m n}
        = - \conj{x} \D{\conj{x}} {\Rep\lambda(x, t)}_{mn}
    \end{align*}
\end{lemma}
\begin{proof}
    Using integration by parts, we obtain:
    \begin{align}
        (m - n) {\Rep\lambda(x, t)}_{m n}
    = -\frac{\e^{-\i \turn n t}}{\i \turn} \int_\T \lambda(-\e^{\i \turn u}x) \D*{u} \e^{\i \turn (n - m)u} \dd u \notag\\
        = \frac{\e^{-\i \turn n t}}{\i \turn} \int_\T \D*{u} \lambda(-\e^{\i \turn u}x) \e^{\i \turn (n - m)u} \dd u.
        \label{eq:lemma:off_diagonal_decay_in_dimension_2}
    \end{align}

    By simple calculations, we can show that
    \begin{align*}
        \D*{u} \lambda(-\e^{\i \turn u} x) = {(-\i \turn)} \conj{x} \D{\conj{x}} \lambda(-\e^{\i \turn u} x).
    \end{align*}
    From there, it follows that~\eqref{eq:lemma:off_diagonal_decay_in_dimension_2} becomes
    \begin{align*}
        (m - n) {\Rep\lambda(x, t)}_{m n}
        = - \conj{x} \D{\conj{x}} {\Rep\lambda(x, t)}_{mn}
    \end{align*}
\end{proof}

\subsection{Unitary dual}

Our description of the unitary dual comes from \cite{Kumahara73},
which itself is a minor adaptation of the one in \cite{Ito52}.
Although both articles only specifically mention the case of the Euclidean motion group,
the author of \cite{Ito52} mentions that the arguments generalise verbatim to generalised motion groups.

Throughout this section, fix $\lambda \in \dualGroup{\VectorSpace}$
and denote by $\IsotropySubgroup{\CompactGroup}{\lambda}$ its isotropy subgroup.

Let $\tau \in \dualGroup{\IsotropySubgroup{\CompactGroup}{\lambda}}$ and denote by $\dimRep{\tau}$ its dimension.
For $q = 1, \dots, \dimRep{\tau}$, let
\begin{align}
    P^\tau_q F(u) \defeq \dimRep{\tau} \int_\IsotropySubgroup{\CompactGroup}{\lambda} \conj{\tau_{qq}(m)} F(u m) \dd m,
    \quad F \in \Lebesgue{2}{\CompactGroup}, u \in \CompactGroup.
    \label{eq:projection_on_L2_of_the_compact_group}
\end{align}

By the Inverse Fourier Transform at $e \in \IsotropySubgroup{\CompactGroup}{\lambda}$,
if $F \in \Lebesgue{2}{\CompactGroup}$ and $u \in \CompactGroup$, then
\begin{align*}
    F(u)
    &= F(u e) = \sum_{\tau \in \dualGroup{\IsotropySubgroup{\CompactGroup}{\lambda}}} \dimRep{\tau} \tr\left(\Rep[\IsotropySubgroup{\CompactGroup}{\lambda}]{\tau}(e) \Fourier[\IsotropySubgroup{\CompactGroup}{\lambda}]{F(u \dummy)}\right)\\
    &= \sum_{\tau \in \dualGroup{\IsotropySubgroup{\CompactGroup}{\lambda}}} \dimRep{\tau} \sum_{q = 1}^{\dimRep{\tau}} {\Fourier[\IsotropySubgroup{\CompactGroup}{\lambda}]{F(u \dummy)}}_{qq}
    = \sum_{\tau \in \dualGroup{\IsotropySubgroup{\CompactGroup}{\lambda}}} \sum_{q = 1}^\dimRep\tau P^\tau_q F(u).
\end{align*}
Now, writing $\Hilbert{\tau}{q} \defeq P^\tau_q \Lebesgue{2}{\CompactGroup}$,
it then follows that
\begin{align*}
    \Lebesgue{2}{\CompactGroup}
    = \sum_{\tau \in \dualGroup{\IsotropySubgroup{\CompactGroup}{\lambda}}} \sum_{q = 1}^\dimRep\tau \Hilbert{\tau}{q}.
\end{align*}

In fact, we shall see that the $P^\tau_q$ are \emph{orthogonal projections}.

\begin{lemma}
    Let $\mu, \tau \in \dualGroup{\IsotropySubgroup{\CompactGroup}{\lambda}}$, $q \in \{1, \dots, \dimRep{\tau}\}$ and $m, n \in \{1, \dots, \dimRep{\tau} \}$.
    If $f \in \Lebesgue{2}{\RightQuotient{\CompactGroup}{\IsotropySubgroup{\CompactGroup}{\lambda}}}$, then
    \begin{align*}
        P^\mu_q (f \otimes \tau_{m n}) = \Kronecker{\mu}{\tau} \Kronecker{n}{q} (f \otimes \tau_{m n}).
    \end{align*}

    In particular, the following properties hold:
    \begin{enumerate}
        \item $P^\mu_q$ is an orthogonal projection onto
            \begin{align*}
                \Hilbert{\mu}{q} =
                    \Lebesgue{2}{\RightQuotient{\CompactGroup}{\IsotropySubgroup{\CompactGroup}{\lambda}}}
                    \otimes
                    \Span \{\mu_{p q} : p = 1, \dots, \dimRep{\mu}\};
            \end{align*}
        \item The Hilbert spaces
            \begin{align*}
                \{\Hilbert{\tau}{q} : \tau \in \dualGroup{\IsotropySubgroup{\CompactGroup}{\lambda}}, q = 1, \dots, \dimRep{\tau} \}
            \end{align*}
            are mutually orthogonal;
        \item We have the decomposition
            \begin{align*}
                \Lebesgue{2}{\CompactGroup} = \bigoplus_{\tau \in \dualGroup{\IsotropySubgroup{\CompactGroup}{\lambda}}} \bigoplus_{q = 1}^{\dimRep{\tau}} \Hilbert{\tau}{q}.
            \end{align*}
    \end{enumerate}
\end{lemma}
\begin{proof}
    Fix $u \in \CompactGroup$ and write $u = u' u''$,
    where $u' \in \RightQuotient{\CompactGroup}{\IsotropySubgroup{\CompactGroup}{\lambda}}$
    and $u'' \in \IsotropySubgroup{\CompactGroup}{\lambda}$.
    It follows that
    \begin{align*}
        P^\mu_q (f \otimes \tau_{m n}) (u)
        &= \dimRep{\mu}
            \int_\IsotropySubgroup{\CompactGroup}{\lambda}
                \conj{\mu_{q q}(m)}
                f(u')
                \tau_{m n}(u'' m)
            \dd m\\
        &= \sum_{p = 1}^\dimRep{\tau}
                f(u')
                \tau_{m p}(u'')
                \dimRep{\mu}
                \int_\IsotropySubgroup{\CompactGroup}{\lambda}
                    \conj{\mu_{q q}(m)}
                    \tau_{p n}(m)
                \dd m.
    \end{align*}

    Using the Peter-Weyl Theorem, we conclude that in fact
    \begin{align*}
        P^\mu_q (f \otimes \tau_{m n}) (u)
        &= \sum_{p = 1}^\dimRep{\tau}
            \Kronecker{\mu}{\tau}
            \Kronecker{q}{p}
            \Kronecker{q}{n}
            f(u')
            \tau_{m p}(u'')\\
        &= \Kronecker{\mu}{\tau}
            \Kronecker{q}{n}
            f(u')
            \tau_{m q}(u'')\\
        &= \Kronecker{\mu}{\tau}
            \Kronecker{q}{n}
            (f \otimes \tau_{m n})(u),
    \end{align*}
    which is what we wanted to show.
\end{proof}

The following result and its proof can be found in \cite[Theorem 1.1, 1.2, 1.3]{Ito52}.
Although the paper specifically treats the case of the Euclidean motion groups,
the author remarks (\cite[Remark p. 84]{Ito52}) that the argument works for generalised motion groups.

\begin{proposition}[Unitary dual]
\label{proposition:unitary_dual}
    Let $\lambda, \lambda' \in \dualGroup{\VectorSpace} \setminus \{0\}$
    and $\tau \in \dualGroup{\IsotropySubgroup{\CompactGroup}{\lambda}}$,
    $\tau' \in \dualGroup{\IsotropySubgroup{\CompactGroup}{\lambda'}}$.
    The following properties hold
    \begin{enumerate}
        \item $\Rep{\lambda}$ restricts to an infinite-dimensional irreducible unitary representation on each $\Hilbert{\tau}{q}$;
        \item $(\Hilbert{\tau}{q}, \Rep{\lambda})$ and $(\Hilbert{\tau'}{q'}, \Rep{\lambda'})$ are equivalent if and only if
            \begin{align*}
                \lambda' = k \lambda \quad \text{and} \quad \EquivalenceClass{\dualGroup{\IsotropySubgroup{\CompactGroup}{\lambda}}}{\tau} = \EquivalenceClass{\dualGroup{\IsotropySubgroup{\CompactGroup}{\lambda}}}{\tau'(k \dummy k^{-1})}
            \end{align*}
            for some $k \in \CompactGroup$.
            In particular, if $q_1, q_2 \in \{1, \dots, \dimRep\tau\}$,
            then $(\Hilbert{\tau}{q_1}, \Rep{\lambda})$ and $(\Hilbert{\tau}{q_2}, \Rep{\lambda})$ are equivalent.
    \end{enumerate}
\end{proposition}

\begin{definition}[Unitary dual]
\label{definition:unitary_dual_of_generalised_motion_group}
\index{generalised motion group!unitary dual}
    Fix $\lambda_0 \in \dualGroup{\VectorSpace} \setminus \{0\}$.
    We define the \emph{unitary dual} of $\Group$, denoted by $\dualGroup{\Group}$, via
    \begin{align*}
        \dualGroup{\Group} \defeq \{ (\Hilbert{\tau}{1}, \Rep{\lambda \lambda_0}) : \lambda \in \R^+, \tau \in \dualGroup{\IsotropySubgroup{\CompactGroup}{\lambda}} \}.
    \end{align*}
\end{definition}

\begin{definition}
    A \emph{measurable field of operators on $\dualGroup\Group$} is a map
    \begin{align*}
        \sigma : \dualGroup\VectorSpace \to \End(\SmoothFunctions{\CompactGroup})
    \end{align*}
    satisfying the following properties.
    \begin{enumerate}
        \item For each $\lambda \in \dualGroup\VectorSpace$,
            each $\tau \in \dualGroup{\IsotropySubgroup{\CompactGroup}{\lambda}}$, and each $q \in \{1, \dots, \dimRep \tau\}$, we have
            \begin{align*}
                \sigma(\lambda) \SmoothVectors{\Hilbert{\tau}{q}} \subset \SmoothVectors{\Hilbert{\tau}{q}}
            \end{align*}
        \item If $\lambda_1, \lambda_2 \in \dualGroup\VectorSpace$ and if $H_1, H_2$ are two Hilbert subspaces of $\Lebesgue{2}{\CompactGroup}$ such that
            $(H_1, \Rep{\lambda_1})$ and $(H_2, \Rep{\lambda_2})$ are equivalent,
            then if $T : H_1 \to H_2$ is the intertwining operator,
            we have
            \begin{align*}
                \eval{\sigma(\lambda_1)}{H_1}  = T \sigma(\lambda_2) T^{-1}
            \end{align*}
    \end{enumerate}
\end{definition}

\subsection{Infinitesimal representations}

\begin{definition}[Infinitesimal Representation]
\label{definition:infinitesimal_representation}
\index{generalised motion group!infinitesimal representation}
    Let $X \in \g$.
    We define the infinitesimal representation of $X$ as the operator
    \begin{align*}
        \Rep{\lambda}(X) : \SmoothFunctions{\CompactGroup} \to \SmoothFunctions{\CompactGroup}
    \end{align*}
    defined via
    \begin{align*}
        \Rep{\lambda}(X) F(u) \defeq \D*{t}<t=0> \Rep{\lambda}(\exp(t X)) F(u),
    \end{align*}
    where $F \in \SmoothFunctions{\CompactGroup}$.
\end{definition}

\begin{proposition}[Infinitesimal representations]
\label{proposition:infinitesimal_representations_of_differential_operators}
    Let $\lambda \in \dualGroup{\VectorSpace}$ and let $j \in \{1, \dots, \dim \Group\}$.
    The infinitesimal representation of $X_j$ has the following expression:
    \begin{enumerate}
        \item if $j \leq \dim \VectorSpace$, then
            \begin{align*}
                \Rep{\lambda}(X_j) F(u) = \directionalDerivative{u^{-1} e_i} \lambda(0) F(u)
            \end{align*}
        \item if $j > \dim \VectorSpace$, then
            \begin{align*}
                \Rep{\lambda}(X_j) F(u) = -\RightDifferentialOperatorFirstOrder{Y_j} F(u),
            \end{align*}
            where $Y_j$ is like in~\eqref{eq:Lie_algebra_vector_coming_from_compact_group} and
            $\RightDifferentialOperatorFirstOrder{Y_j}$ is the right-invariant differential operator on $\CompactGroup$ associated with $Y_j$.
    \end{enumerate}
    In the above, $F$ is an arbitrary function in $\SmoothFunctions{\CompactGroup}$.
\end{proposition}
\begin{proof}
    \begin{enumerate}
        \item Fix $j \in \{1, \dots, \dim \VectorSpace\}$.
            Since $\exp_\Group(t X_j) = (t e_i, \Id{\VectorSpace})$.
            It follows that
            \begin{align*}
                \Rep{\lambda}(X) F(u) = \D*{t}<t = 0> \lambda(t u^{-1} e_i) F(u)
                = \directionalDerivative{u^{-1} e_i} \lambda(0) F(u)
            \end{align*}
            which is what we wanted to show.
        \item Let $Y_j$ be like in~\eqref{eq:Lie_algebra_vector_coming_from_compact_group} so that
            \begin{align*}
                \exp_\Group(t X_j) = \Phi^{-1}
                    \begin{pmatrix}
                        \exp_\CompactGroup(t Y_j) & 0\\
                        0 & 1
                    \end{pmatrix}
                    = (0, \exp_\CompactGroup(t Y_j)).
            \end{align*}

            From there, it immediately follows that
            \begin{align*}
                \Rep{\lambda}(X_j) F(u)
                = \D*{t}<t = 0> F({(\exp_\CompactGroup(t Y_j))}^{-1} u)
                = -\RightDifferentialOperatorFirstOrder{Y_j} F(u).
            \end{align*}
    \end{enumerate}
\end{proof}

\begin{corollary}[Infinitesimal representation of $\Laplacian$]
\label{corollary:infinitesimal_representation_of_the_Laplacian}
    Let $\lambda \in \dualGroup{\VectorSpace}$.
    The infinitesimal representation of $\Laplacian$ is given by
    \begin{align*}
        \Rep{\lambda}(\Laplacian) = - \norm[\dualGroup{\VectorSpace}]{\lambda}^2 \Id{\Lebesgue{2}{\CompactGroup}} + \RightLaplacian[\CompactGroup].
    \end{align*}
\end{corollary}
\begin{proof}
    By Proposition~\ref{proposition:infinitesimal_representations_of_differential_operators},
    we know that
    \begin{align*}
        \Rep{\lambda}(\Laplacian) F(u) =
        \left(
            \sum_{j = 1}^{\dim \VectorSpace}
                \directionalDerivative{u^{-1} e_i}^2 \lambda(0)
                + \RightLaplacian[\CompactGroup]
        \right)
        F(u).
    \end{align*}

    Since the Laplacian on $\VectorSpace$ is invariant under $\CompactGroup$,
    we have
    \begin{align*}
        \Laplacian[\VectorSpace] \lambda(0)
        = (\Laplacian[\VectorSpace] \lambda \circ u)(0)
        = \Laplacian (\lambda \circ u)(0).
    \end{align*}

    Since the right-hand side is exactly equal to
    \begin{align*}
        \sum_{j = 1}^{\dim \VectorSpace} \directionalDerivative{u^{-1} e_j}^2 \lambda(0),
    \end{align*}
    it follows that we have
    \begin{align*}
        \Rep{\lambda}(\Laplacian) F(u)
        = (\Laplacian[\VectorSpace] \lambda(0) + \RightLaplacian[\CompactGroup]) F(u),
    \end{align*}
    which concludes the proof.
\end{proof}

\section{Fourier Transform}

\subsection{Definition and elementary properties}

\begin{definition}[Fourier transform]
\label{definition:Fourier_Transform}
\index{generalised motion group!Fourier transform}
    Let $f \in \Lebesgue{1}{\Group}$ and $\lambda \in \dualGroup{\VectorSpace}$.
    We define its \emph{Fourier coefficient} at $\lambda$ via
    \begin{align*}
        \Fourier{f}(\lambda) \defeq \int_\Group f(g) \adj{\Rep{\lambda}(g)} \dd g.
    \end{align*}

    Moreover, the map
    \begin{align*}
        \Fourier{f} : \dualGroup{\VectorSpace} \to \End(\Lebesgue{2}{\CompactGroup}) :
        \lambda \mapsto \Fourier{f}(\lambda)
    \end{align*}
    is called the \emph{Fourier Transform} of $f$.
\end{definition}

\begin{lemma}
\label{lemma:kernels_of_Fourier_coefficients}
    Let $\lambda \in \VectorSpace$, and $f \in \Lebesgue{1}\Group$.
    The \emph{integral kernel} of the operator $\Fourier f(\lambda)$ is given by
    \begin{align}
        K_{f, \lambda}(u, k) \defeq \int_\CompactGroup \Fourier[\VectorSpace] f(k \lambda, k u^{-1}),
        \label{integral_kernel_of_Fourier_coefficient}
    \end{align}
    i.e. for every $F \in \Lebesgue{2}\CompactGroup$ and every $u \in \CompactGroup$, we have
    \begin{align*}
        \Fourier f(\lambda) F(u) = \int_\CompactGroup K_{f, \lambda}(u, k) F(k) \dd k.
    \end{align*}

    In particular, the following properties hold:
    \begin{enumerate}
        \item if $f \in \Schwartz\Group$, then $K_{f, \lambda}$ is smooth,
            $\Fourier f(\lambda)$ is trace class and
            \begin{align*}
                \tr(\Fourier f(\lambda)) = \int_\CompactGroup \Fourier[\VectorSpace] f(k \lambda, e) \dd k;
            \end{align*}
        \item if $f \in \Lebesgue{1}\Group \cap \Lebesgue{2}\Group$, then for almost every $\lambda \in \VectorSpace$,
            $K_{f, \lambda}$ is Hilbert-Schmidt and
            \begin{align*}
                \norm[\SchattenClasses{2}{\Lebesgue{2}\CompactGroup}]{\Fourier f(\lambda)}^2
                = \int_\CompactGroup \int_\CompactGroup \abs{\Fourier[\VectorSpace] f(k \lambda, u)}^2 \dd u \dd k
            \end{align*}
    \end{enumerate}
\end{lemma}
\begin{proof}
    Let $F \in \Lebesgue{2}\CompactGroup$ and $u \in \CompactGroup$.
    By definition of the Fourier Transform,
    \begin{align*}
        \Fourier f(\lambda) F(u) =
        \int_\VectorSpace
            \int_\CompactGroup
                f(x, k) \e^{-\i \turn \ip{k u \lambda}{x}} F(k u).
            \dd k
        \dd x
    \end{align*}

    Recognising the Fourier Transform on $\VectorSpace$ in the above, we obtain
    \begin{align*}
        \Fourier f(\lambda) F(u) =
        \int_\CompactGroup
            \Fourier[\VectorSpace] f(k u \lambda, k) F(k u)
        \dd k
        =
        \int_\CompactGroup
            \Fourier[\VectorSpace] f(k \lambda, k u^{-1}) F(k),
        \dd k
    \end{align*}
    where we substituted $k$ for $k u^{-1}$ to obtain the last line.

    From there, it follows that the kernel is indeed given by~\eqref{integral_kernel_of_Fourier_coefficient}.
    Let us now prove the two remaining claims.

    \begin{enumerate}
        \item If $f \in \Schwartz\Group$, it follows that the integral kernel is smooth.
            Using~\cite[Corollary 4.1]{DelgadoRuzhansky14}, it follows that $\Fourier f(\lambda)$ is trace-class, and
            \begin{align*}
            \tr(\Fourier f(\lambda))
            = \int_\CompactGroup K_{f, \lambda}(k, k) \dd k
            = \int_\CompactGroup \Fourier[\VectorSpace] f(k \lambda, e) \dd k.
        \end{align*}
    \item Now, if $f \in \Lebesgue{1}\Group \cap \Lebesgue{2}\Group$,
        then $K_{f, \lambda} \in \Lebesgue{2}{\CompactGroup \times \CompactGroup}$ for almost every $\lambda \in \VectorSpace$.
        For such $\lambda$, it follows by~\cite[Theorem VI.23]{Reed72} that $\Fourier f(\lambda)$ is Hilbert-Schmidt and
        \begin{align*}
            \norm[\SchattenClasses{2}{\Lebesgue{2}\CompactGroup}]{\Fourier f(\lambda)}^2
            &= \int_\CompactGroup \int_\CompactGroup \abs{K_{f, \lambda}(u, k)}^2 \dd k \dd u\\
            &= \int_\CompactGroup \int_\CompactGroup \abs{\Fourier[\VectorSpace] f(k \lambda, k u^{-1})}^2 \dd k \dd u.
        \end{align*}
        Substituing $u$ for $u^{-1} k$ in the above, we obtain
        \begin{align*}
            \norm[\SchattenClasses{2}{\Lebesgue{2}\CompactGroup}]{\Fourier f(\lambda)}^2
            &= \int_\CompactGroup \int_\CompactGroup \abs{\Fourier[\VectorSpace] f(k \lambda, u)}^2 \dd k \dd u,
        \end{align*}
        as required.
    \end{enumerate}
\end{proof}

\subsection{Plancherel formula}

\begin{proposition}[Plancherel formula]
\label{proposition:Plancherel_formula}
\index{generalised motion group!Fourier transform!Plancherel formula}
    Let $f \in \Lebesgue{1}{\Group} \cap \Lebesgue{2}{\Group}$.
    The following formula holds
    \begin{align}
        \int_G \abs{f}^2 \dd g = \int_\dualGroup{\VectorSpace} \norm[\HilbertSchmidt{\Lebesgue{2}{\CompactGroup}}]{\Fourier{f}(\lambda)}^2 \dd \Plancherel{\VectorSpace}(\lambda).
        \label{proposition:Plancherel_formula:formula}
    \end{align}
\end{proposition}
\begin{proof}
    It follows from Lemma~\ref{lemma:kernels_of_Fourier_coefficients} that for almost every $\lambda \in \VectorSpace$,
    $\Fourier f(\lambda)$ is trace class and
    \begin{align*}
        \norm[\SchattenClasses{2}{\Lebesgue{2}\CompactGroup}]{\Fourier f(\lambda)}^2
        = \int_\CompactGroup \int_\CompactGroup \abs{\Fourier[\VectorSpace] f(k \lambda, u)}^2 \dd u \dd k.
    \end{align*}

    Now, integrating with respect to $\lambda$,
    we obtain
    \begin{align*}
        \int_\dualGroup{\VectorSpace} \norm[\HilbertSchmidt{\Lebesgue{2}{\CompactGroup}}]{\Fourier{f}(\lambda)}^2 \dd \Plancherel{\VectorSpace}(\lambda)
        &= \int_\dualGroup{\VectorSpace} \int_\CompactGroup \abs{\Fourier[\VectorSpace]{f}(\lambda, k)}^2 \dd k \dd \Plancherel{\VectorSpace}(\lambda)\\
        &= \int_\VectorSpace \int_\CompactGroup \abs{f(x, k)}^2 \dd u \dd k,
    \end{align*}
    where the last line was obtained by applying the Plancherel formula on $\VectorSpace$.
\end{proof}

%\begin{lemma}
%    Let $\phi \in \Schwartz\Group$.
%    For each $\lambda \in \VectorSpace$,
%    the operator $\Fourier \phi(\lambda)$ is trace class.
%    Moreover, for each $N \in \N$, there exists $C \geq 0$ such that
%    \begin{align*}
%        \norm[\SchattenClasses{1}{\Lebesgue{2}{\CompactGroup}}]{\Fourier \phi(\lambda)}
%        &\leq C {(1 + \abs\lambda)}^{-N}.
%    \end{align*}
%    In particular, the map
%    \begin{align*}
%        \lambda \in \VectorSpace \mapsto \norm[\SchattenClasses{1}{\Lebesgue{2}{\CompactGroup}}]{\Fourier \phi(\lambda)}
%    \end{align*}
%    is integrable.
%\end{lemma}
%\begin{proof}
%    Let $\alpha > \dim \CompactGroup$.
%    It follows by \cite[Proposition 3.3]{DelgadoRuzhansky14} that
%    \begin{align}
%        \BesselPotential[\CompactGroup]{-\alpha} \in \SchattenClasses{1}{\Lebesgue{2}\CompactGroup}
%        \label{lemma:preparation_for_inverse_formula:Bessel_potential_in_trace_class}
%    \end{align}
%
%    Let $\lambda \in \VectorSpace$.
%    We check that for each $F \in \Lebesgue{2}\CompactGroup$,
%    \begin{align*}
%        \BesselPotential[\CompactGroup]{\alpha} \Fourier \phi(\lambda) F(u)
%        = \int_\CompactGroup \BesselPotential[\CompactGroup]{\alpha}_u K_{\phi, \lambda}(u, k) F(k) \dd k,
%    \end{align*}
%    where $K_{\phi, \lambda}(u, k)$ represents the kernel of $\Fourier \phi(\lambda)$.
%
%    Since $K_{\phi, \lambda}$ is smooth with respect to $(u, k)$ and is rapidly decaying in $\lambda$,
%    it follows that for each $N \in \N$, there exists $C \geq 0$ such that
%    \begin{align*}
%        \norm[\Lin{\Lebesgue{2}\CompactGroup}]{\BesselPotential[\CompactGroup]{\alpha} \Fourier \phi(\lambda)}
%        \leq C {(1 + \abs\lambda)}^{-N}.
%    \end{align*}
%
%    Combining the above with~\eqref{lemma:preparation_for_inverse_formula:Bessel_potential_in_trace_class},
%    we obtain
%    \begin{align*}
%        \norm[\SchattenClasses{1}{\Lebesgue{2}{\CompactGroup}}]{\Fourier \phi(\lambda)}
%        &\leq
%        \norm[\SchattenClasses{1}{\Lebesgue{2}\CompactGroup}]{\BesselPotential[\CompactGroup]{-\alpha}}
%            \norm[\Lin{\Lebesgue{2}\CompactGroup}]{\BesselPotential[\CompactGroup]{\alpha} \Fourier \phi(\lambda)}\\
%        &\leq C {(1 + \abs\lambda)}^{-N},
%    \end{align*}
%    which is the desired estimate.
%\end{proof}

\begin{proposition}[Inverse Fourier Transform]
\label{proposition:inverse_Fourier_Transform}
\index{generalised motion group!Fourier transform!inverse formula}
    Let $\phi \in \Schwartz{\Group}$.
    For each $g \in \Group$,
    we have
    \begin{align*}
        \phi(g)
        = \int_\dualGroup{\VectorSpace}
        \tr \left( \Rep{\lambda}(g) \Fourier \phi(\lambda) \right) \dd \Plancherel{\VectorSpace}(\lambda).
    \end{align*}
\end{proposition}
\begin{proof}
    Let us assume that $g = e$.
    By Lemma~\ref{lemma:kernels_of_Fourier_coefficients}, we know that $\Fourier \phi(\lambda)$ is trace class and
    \begin{align*}
        \tr(\Fourier \phi(\lambda))
        = \int_\CompactGroup \Fourier[\VectorSpace] \phi(k \lambda, e) \dd k.
    \end{align*}

    Integrating with respect to $\lambda$, we obtain
    \begin{align*}
        \int_\VectorSpace \tr(\Fourier \phi(\lambda)) \dd \lambda
        &= \int_\VectorSpace \int_\CompactGroup \Fourier[\VectorSpace] \phi(k \lambda, e) \dd k \dd \lambda\\
        &= \int_\VectorSpace \Fourier[\VectorSpace] \phi(\lambda, e) \dd \lambda,
    \end{align*}
    where the last line was obtained by a change of variables after permuting the integrals.

    Recognising an inverse Fourier Transform in the right-hand side of the above, we obtain
    \begin{align*}
        \int_\VectorSpace \tr(\Fourier \phi(\lambda)) \dd \lambda
        = \phi(0, e),
    \end{align*}
    concluding the case $g = e$.

    The general case follows immediately, since
    \begin{align*}
        \phi(g) = \phi(e g) = \int_\VectorSpace \tr(\Fourier \{\phi(\dummy g)\}(\lambda)) \dd \lambda
        = \int_\VectorSpace \tr(\Rep\lambda(g) \Fourier \phi(\lambda)) \dd \lambda,
    \end{align*}
    where the last equality was obained by Proposition~\ref{proposition:elementary_properties_of_the_Fourier_transform}.
\end{proof}

\begin{definition}[$\LebesgueDual{2}{\Group}$]
    We shall say that a map
    \begin{align*}
        \sigma : \dualGroup{\VectorSpace} \to \HilbertSchmidt{\Lebesgue{2}{\CompactGroup}}
    \end{align*}
    belongs to $\LebesgueDual{2}{\Group}$ if and only if the following conditions are met:
    \begin{enumerate}
        \item $\sigma$ is measurable;
        \item for each $k \in \CompactGroup$, we have
            \begin{align*}
                \sigma(k \lambda) = R_k \sigma(\lambda) R_k^{-1}
            \end{align*}
        \item the quantity
            \begin{align*}
                \norm[\LebesgueDual{2}{\Group}]{\sigma} \defeq
                    \left(
                        \int_\dualGroup{\VectorSpace}
                            \norm[\HilbertSchmidt{\Lebesgue{2}{\CompactGroup}}]{\sigma(\lambda)}^2
                        \dd \Plancherel{\VectorSpace}(\lambda)
                    \right)^{\frac{1}{2}}
            \end{align*}
            is finite.
    \end{enumerate}

    If $\sigma_1, \sigma_2 \in \LebesgueDual{2}{\Group}$, then we let
    \begin{align*}
        \ip[\LebesgueDual{2}{\Group}]{\sigma_1}{\sigma_2} \defeq
        \int_\dualGroup{\VectorSpace}
            \tr\left(
                \sigma_1(\lambda) \adj{\sigma_2(\lambda)}
            \right)
        \dd \Plancherel{\VectorSpace}(\lambda).
    \end{align*}
    If we quotient $\LebesgueDual{2}{\Group}$ by $\Plancherel{\VectorSpace}$-almost everywhere equality,
    which we shall do from now onwards,
    then the above gives $\LebesgueDual{2}{\Group}$ the structure of a Hilbert space.
\end{definition}

\begin{definition}[$\Kernels{\Group}$]
    We shall say that a tempered distribution $\kappa \in \TemperedDistributions{\Group}$ belongs to $\Kernels{\Group}$
    if and only if the map
    \begin{align}
        T_\kappa : \Schwartz{\Group} \to \TemperedDistributions{\Group} : f \mapsto \conv{f}{\kappa}
    \end{align}
    extends to a continuous map from $\Lebesgue{2}{\Group}$ into itself.
    In this case, we let
    \begin{align*}
        \norm[\Kernels{\Group}]{\kappa} \defeq \norm[\Lin{\Lebesgue{2}{\Group}}]{T_\kappa}.
    \end{align*}
\end{definition}

\begin{definition}
    We shall say that a map
    \begin{align*}
        \sigma : \dualGroup{\VectorSpace} \to \Lin{\Lebesgue{2}{\CompactGroup}}
    \end{align*}
    belongs to $\LebesgueDual{\infty}{\Group}$ if and only if it is invariant under $\CompactGroup$ and the quantity
    \begin{align*}
        \norm[\LebesgueDual{\infty}{\Group}]{\sigma} \defeq
            \esssup_{\lambda \in \dualGroup{\VectorSpace}}
                \norm[\Lin{\Lebesgue{2}{\CompactGroup}}]{\sigma(\lambda)}
    \end{align*}
    is finite.
    The essential supremum is taken with respect to the Plancherel measure.
\end{definition}

\begin{theorem}[Abstract Plancherel formula]
    The Fourier Transform can be extended to a \emph{surjective} isometry
    \begin{align*}
        \Fourier : \Lebesgue{2}{\Group} \to \LebesgueDual{2}{\Group}.
    \end{align*}

    Moreover, for every left-invariant operator $T \in \Lin{\Lebesgue{2}{\Group}}$,
    there exists a unique element $\sigma \in \LebesgueDual{\infty}{\Group}$ such that
    \begin{align*}
        \Fourier\{T f\}(\lambda) = \sigma(\lambda) \Fourier f(\lambda)
    \end{align*}
    holds for $\Plancherel{\Group}$-almost every $\lambda \in \dualGroup{\VectorSpace}$.
\end{theorem}

\subsection{Sobolev spaces}

\begin{definition}[Sobolev norm]
    Let $s \in \R$.
    If $\phi \in \Schwartz\Group$, we let
    \begin{align*}
        \norm[\Sobolev{s}]{\phi} \defeq
        \left(
            \int_\dualGroup\VectorSpace
                \norm[\HilbertSchmidt{\Lebesgue{2}{\CompactGroup}}]{%
                    \Rep\lambda \BesselPotential{s}
                    \Fourier \phi(\lambda)
                    }^2
            \dd \Plancherel\VectorSpace(\lambda)
        \right)^{1 / 2}.
    \end{align*}
\end{definition}

\begin{definition}[Sobolev spaces]
\label{definition:Sobolev_spaces}
    Let $s \in \R$.
    We define the \emph{Sobolev space of order $s$} to be the completion of $\Schwartz\Group$ with the norm $\norm[\Sobolev{s}]{\dummy}$.
\end{definition}

\subsubsection{Sobolev embeddings}

\begin{proposition}[Sobolev embedding]
\label{proposition:Sobolev_embedding}
    If $s > \dim \Group / 2$, then we have the following continuous inclusion
    \begin{align*}
        \Sobolev{s} \subset \ContinuousFunctions\Group \cap \Lebesgue{\infty}{\Group}.
    \end{align*}
    More precisely, there exists $C \geq 0$ such that the following property holds:
    for every $f \in \Sobolev{s}$,
    there exists a continuous function $\tilde{f} \in \ContinuousFunctions\Group$ such that $f = \tilde{f}$ almost everywhere and
    \begin{align*}
        \norm[\ContinuousFunctions\Group]{\tilde{f}} \leq C \norm[\Sobolev{s}]{f}.
    \end{align*}
\end{proposition}

\subsection{Fourier Transform of distributions}

For the sequel, we need to be able to take the Fourier Transform of certain distributions.
To this end, we follow the ideas of \cite{FischerRuzhansky15}.

\begin{definition}
    Let $a, b \in \R$.
    We shall say that $f \in \TemperedDistributions\Group$ belongs to $\KernelsSobolev{a}{b}$ if and only if the map
    \begin{align*}
        \Schwartz\Group \to \Schwartz\Group : \phi \to \conv{\phi}{f}
    \end{align*}
    is a continuous map in $\Lin{\Sobolev{a}, \Sobolev{b}}$.
\end{definition}

\begin{definition}
    Let $a, b \in \R$.
    We shall say that a map
    \begin{align*}
        \sigma : \dualGroup\VectorSpace \to \End(\SmoothFunctions{\CompactGroup})
    \end{align*}
    belongs to $\LebesgueDual[a, b]{\infty}{\Group}$ if and only if
    \begin{enumerate}
        \item TODO: Invariance condition
        \item The map
            \begin{align*}
                \lambda \in \dualGroup\VectorSpace \mapsto
                \Rep\lambda \BesselPotential{b} \sigma(\lambda) \Rep\lambda \BesselPotential{-a}
            \end{align*}
            belongs to $\LebesgueDual{\infty}{\Group}$.
    \end{enumerate}
\end{definition}

\begin{proposition}[Extension of the Fourier Transform]
    Define the sets
    \begin{align*}
        K \defeq \bigcup_{a, b \in \R} \KernelsSobolev{a}{b}, \quad
        L \defeq \bigcup_{a, b \in \R} \LebesgueDual[a, b]{\infty}{\Group}.
    \end{align*}

    The Fourier Transform can be extended as a bijective map
    \begin{align*}
        \Fourier : K \to L,
    \end{align*}
    and preserve the following properties.
    \begin{enumerate}
        \item If $f \in \Lebesgue{1}{\Group}$, then it coincides with Definition~\ref{definition:Fourier_Transform}.
        \item If $f_1, f_2 \in K$ are such that $\conv{f_1}{f_2} \in K$, then
            \begin{align*}
                \Fourier\{\conv{f_1}{f_2}\} = \Fourier f_2 \Fourier f_1.
            \end{align*}
        \item If $f \in \SmoothFunctions{\Group} \cap K$, $X \in \LieAlgebra$ and $\LeftDifferentialOperatorFirstOrder{X} f \in K$, then
            \begin{align*}
                \Fourier\{\LeftDifferentialOperatorFirstOrder{X} f\}(\lambda) = \Rep\lambda(X) \Fourier f(\lambda).
            \end{align*}
    \end{enumerate}
\end{proposition}

\section{Taylor formula}

\begin{proposition}[Taylor remainder]
    There exists an admissible collection of smooth functions $q_1, \dots, q_M \in \SmoothFunctions\Group$
    and a collection $\{\TaylorLeftDifferentialOperator{\alpha}\}_{\alpha \in \N^M}$ of left-invariant differential operators satisfying the following properties:
    \begin{enumerate}
        \item for each $\alpha \in \N^M$, $\TaylorLeftDifferentialOperator\alpha$'s order is less than $\abs\alpha$;
        \item if $f \in \SmoothFunctions\Group$ and $(x, k) \in \Group$,
            we have the following Taylor development
            \begin{align*}
                f(x, k) &= \sum_{\abs\alpha \leq N} \frac{1}{\alpha!} q^\alpha({(x, k)}^{-1}) \TaylorLeftDifferentialOperator\alpha f(0, e) + \BigO(h(x, k)^N),
            \end{align*}
            where $h : \Group \to \R^+$ denotes the geodesic distance to the identity.
    \end{enumerate}
\end{proposition}
\begin{proof}
    Let $f \in \SmoothFunctions\Group$.
    Choose an embedding
    \begin{align*}
        \iota : \CompactGroup \to \R^D.
    \end{align*}

    By REFERENCE, there exists an open neighbourhood $O \subset \R^D$ containing $\iota(\CompactGroup)$
    such that an orthogonal projection $p : O \to \iota(\CompactGroup)$ is defined and smooth.
    We can therefore extend $f$ onto $O$ via:
    \begin{align*}
        F : \VectorSpace \times O : (x, y) \mapsto f(x, (\iota^{-1} \circ p)(y)).
    \end{align*}

    By the Taylor Theorem on $\VectorSpace \times O$ at $(0, \iota(e))$, we get
    \begin{align*}
        F(x, y) =
        \sum_{\substack{\alpha = (\alpha_1, \alpha_2)\\ \abs\alpha \leq N}}
            \frac{1}{\alpha!}
            &x^{\alpha_1} {(y - \iota(e))}^{\alpha_2}
            \D[F]{x^{\alpha_1}, y^{\alpha_2}}(0, \iota(e))\\
            &+ \BigO\left((\norm{x} + \norm[\R^N]{y - \iota(e)})^N\right).
    \end{align*}
    In particular, denoting by $h(g)$ the geodesic distance between $g \in \Group$ and $e \in \Group$,
    we observe that
    \begin{align}
        f(x, k) =
        \sum_{\substack{\alpha = \alpha_1 + \alpha_2\\ \abs\alpha \leq N}}
            \frac{1}{\alpha!}
            x^{\alpha_1} {(\iota(k) - \iota(e))}^{\alpha_2}
            \D[F]{x^{\alpha_1}, y^{\alpha_2}}(0, \iota(e))
            + \BigO\left({h(x, k)}^N\right).
        \label{proposition:Taylor_remainder_theorem:Taylor_development}
    \end{align}

    Now, let
    \begin{align*}
        q_j(x, k) &= -\ip{x}{k e_i}, \quad &1 \leq &j \leq \dim \VectorSpace\\
        q_j(x, k) &= \ip[\R^N]{\iota(k^{-1}) - \iota(e)}{e_{j}}, \quad &\dim \VectorSpace < &j \leq \dim \VectorSpace + D
    \end{align*}
    where $e_1, \dots, e_{\dim V}$ is an orthonormal basis of $\VectorSpace$
    and $e_{\dim V + 1}, \dots, e_{\dim \VectorSpace + D}$ is an orthonormal basis of $\R^D$.
    Moreover, we know that for each $\alpha = (\alpha_1, \alpha_2)$,
    there exists a left-invariant differential operator $\TaylorLeftDifferentialOperator{\alpha}$ of order at most $\abs\alpha$ such that
    \begin{align*}
        \TaylorLeftDifferentialOperator{\alpha} f(e) = \D[F]{x^{\alpha_1}, y^{\alpha_2}}(0, \iota(e)),
    \end{align*}
    since $F$ doesn't locally vary in the directions perpendicular to the tangent plane of $\iota(\CompactGroup)$.

    Now, let us check that if $\alpha = (\alpha_1, \alpha_2)$, then
    \begin{align*}
        q^\alpha({(x, k)}^{-1})
        = q^\alpha(-k^{-1} x, k^{-1})
        = x^{\alpha_1} {(\iota(k) - \iota(e))}^{\alpha_2}.
    \end{align*}

    It follows that~\eqref{proposition:Taylor_remainder_theorem:Taylor_development} becomes
    \begin{align*}
        f(x, k) &= \sum_{\abs\alpha \leq N} \frac{1}{\alpha!} q^\alpha({(x, k)}^{-1}) \TaylorLeftDifferentialOperator\alpha f(0, e) + \BigO(h(x, k)^N),
    \end{align*}
    which concludes our proof.
\end{proof}
