\chapter*{Abstract}

In recent years,
effort has been put into following the ideas of \citeauthor{RuzhanskyTurunen10}
to construct a global pseudo-differential calculus on Lie groups.
By this, we mean a collection of operators containing the left-invariant differential calculus
with the additional requirement that it be stable under composition and adjunction.
Moreover,
we would like those operators to
have adequate boundedness properties between Sobolev spaces.
Our approach consists in using the group Fourier transform to define a global, operator-valued symbol,
yielding pseudo-differential operators via an analogue of the Euclidean Kohn-Nirenberg quantisation.

The present document treats the case of the Euclidean motion group,
which is the smallest subset of Euclidean affine transformations containing translations and rotations.
As our representations are infinite-dimensional,
the proofs of the calculus properties are more naturally carried out on the kernel side,
which means that particular care is required to treat the singularity at the origin.
The key argument is a density result which allows us to approximate singular kernels via smooth ones,
and is proved herein via purely spectral arguments without using classical estimates on the heat kernel.
