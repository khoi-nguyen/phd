\chapter{Symbols}
\label{chapter:symbols}

\section{Difference operators}

\begin{definition}
\label{definition:difference_operators}
    Let $q \in \SmoothFunctions{\Group}$.
    The \emph{difference operator} associated with $q$, $\DifferenceOperator{q}$ is defined via
    \begin{align*}
        \DifferenceOperator{q} \Fourier f \defeq \Fourier\{q f\},
    \end{align*}
    where $f \in \Schwartz{\Group}$.

    Moreover, if $q$ vanishes at order $k \in \N$,
    we shall say that $\DifferenceOperator{q}$ is a \emph{difference operator of order $k$}.
\end{definition}

\begin{definition}
\label{definition:admissibility_of_difference_operators}
\index{difference operators!admissibility}
    A finite collection $q_1$, \dots, $q_M \in \SmoothFunctions{\Group}$ of smooth functions is said to be \emph{admissible}
    if $\dd q_j(e) \neq 0$ for each $j \in \{1, \dots, M\}$
    and if
    \begin{align*}
        \rank(\dd q_1(e), \dots, \dd q_m(e)) = \dim \Group.
    \end{align*}

    Moreover, if
    \begin{align*}
        \bigcap_{j = 1}^M \{ q_j = 0 \} = \{e\},
    \end{align*}
    we shall say that the collection is \emph{strongly admissible}.

    A collection of \emph{difference operators} is called \emph{(strongly) admissible}
    if the associated smooth functions form a \emph{(strongly) admissible} collection.
\end{definition}

Using \cite[Lemma 4.4]{RuzhanskyTurunenWirth10}, we can show

\begin{lemma}
    There exists a strongly admissible family $q_1$, \dots, $q_M \in \SmoothFunctions{\Group}$ on $\Group$.
\end{lemma}

\begin{remark}
    From now on, we fix an admissible family $q_1, \dots, q_{\dimDifferenceOperators}$ on $\Group$.
    Given $\alpha \in \N^{\dimDifferenceOperators}$, we let
    \begin{align*}
        q^\alpha = \prod_{j = 1}^{\dimDifferenceOperators} q_j^{\alpha_j}.
    \end{align*}
    Moreover, we let $\DifferenceOperatorOrder{\alpha}$ be the difference operator associated with the smooth function
    \begin{align*}
        \Group \to \C : g \mapsto q^\alpha(g^{-1}).
    \end{align*}
\end{remark}

\section{Symbol and operators classes}

\begin{definition}[Symbol classes]
\label{definition:symbol_classes}
    Let $m \in \R$ and fix $\rho, \delta \in \R$ such that $0 \leq \rho \leq \delta \leq 1$.
    We shall say that a map
    \begin{align*}
        \sigma : \Group \times \VectorSpace \mapsto \End(\SmoothFunctions \CompactGroup)
    \end{align*}
    is a \emph{symbol of order $m$ and of type $(\rho, \delta)$} if the following conditions are satisfied.
    \begin{enumerate}
        \item
            For each $F \in \SmoothFunctions \CompactGroup$ and each $k \in \CompactGroup$,
            the map
            \begin{align*}
                (g, \lambda) \mapsto \sigma(g, \lambda) F(k)
            \end{align*}
            is smooth.
        \item
            For each $\beta \in \N^{\dim \Group}$,
            $\LeftDifferentialOperator \beta \sigma(g, \dummy) \in \TemperedDistributions {\dualGroup \Group}$
            for each $g \in \Group$ and the map
            \begin{align*}
                g \in \Group \mapsto \LeftDifferentialOperator \beta \sigma(g, \dummy) \in \TemperedDistributions {\dualGroup \Group}
            \end{align*}
            is continuous.
        \item \label{item:symbol_bound_condition}
            For each $\alpha \in \N^{\dim \Group}$, $\beta \in \N^m$, and each $\gamma \in \R$,
            the operator
            \begin{align*}
                \Rep{\lambda} \BesselPotential{\rho \abs\alpha - m - \delta \abs\beta + \gamma} \LeftDifferentialOperator{\beta} \DifferenceOperatorOrder{\alpha} \sigma(g, \lambda) \Rep{\lambda} \BesselPotential{-\gamma}
            \end{align*}
            is bounded in $\Lebesgue 2 \CompactGroup$ by a finite constant that does not depend on $g \in \Group$ or $\lambda \in \VectorSpace$.
    \end{enumerate}

    The set $\SymbolClass{m}{\rho, \delta}$ will be used to denote the set of all symbols of order $m$ and type $(\rho, \delta)$.
\end{definition}

\begin{definition}[Symbol semi-norms]
    Let $m \in \R$ and fix $\rho, \delta \in \R$ such that $0 \leq \rho \leq \delta \leq 1$.
    If $\sigma \in \SymbolClass m {\rho, \delta}$,
    then for each $N \in \N$,
    we let
    \begin{align*}
        &\SymbolSemiNorm m {\rho, \delta} N {\sigma(g; \lambda)}
        \defeq
        \sup_{\abs \alpha, \abs \beta, \abs \gamma \leq N}
        \\
        &\qquad
        \norm [\Lin{\Lebesgue 2 \CompactGroup}] {%
            \Rep{\lambda} \BesselPotential{\rho \abs\alpha - m - \delta \abs\beta + \gamma} \LeftDifferentialOperator{\beta} \DifferenceOperatorOrder{\alpha} \sigma(g, \lambda) \Rep{\lambda} \BesselPotential{-\gamma}
        }.
    \end{align*}

    Moreover, we also let
    \begin{align*}
        \SymbolSemiNorm m {\rho, \delta} N \sigma
        \defeq \sup_{g \in \Group} \sup_{\lambda \in \VectorSpace}
        \SymbolSemiNorm m {\rho, \delta} N {\sigma(g; \lambda)}
    \end{align*}
    for each $N \in \N$.
\end{definition}

\begin{lemma}
    Suppose that $m_1, m_2 \in \R$ and $\rho_1, \rho_2, \delta_1, \delta_2 \in [0, 1]$.
    If the following inequalities hold
    \begin{align*}
        m_1 \leq m_2, \quad \delta_1 \leq \delta_2, \quad \rho_1 \geq \rho_2,
    \end{align*}
    then $\SymbolClass {m_1} {\rho_1, \delta_1} \subset \SymbolClass {m_2} {\rho_2, \delta_2}$.
\end{lemma}

\begin{definition}[Smoothing symbols]
\label{definition:smoothing_symbols}
    We let
    \begin{align*}
        \SmoothingSymbols \defeq \bigcap_{m \in \R} \SymbolClass{m}{1, 0}.
    \end{align*}
    The elements of $\SmoothingSymbols$ will be called \emph{smoothing symbols}.
\end{definition}

\begin{definition}[Operator classes]
\label{definition:operator_classes}
    Let $\sigma \in \SymbolClass{m}{\rho, \delta}$.
    We define the operator $\Op(\sigma)$ via
    \begin{align*}
        \Op(\sigma) \phi(g) \defeq
        \int_\VectorSpace
            \tr\left(\Rep\lambda(g) \sigma(g, \lambda) \Fourier \phi(\lambda)\right)
        \dd \lambda,
    \end{align*}
    where $\phi \in \Schwartz\Group$, and $g \in \Group$.


    If $T = \Op(\sigma)$ for a certain $\sigma \in \SymbolClass{m}{\rho, \delta}$,
    we shall say that $T$ is an \emph{operator of order $m$ and of type $(\rho, \delta)$}.

    The set of all such operators will be denoted by
    \begin{align*}
        \OperatorClass{m}{\rho, \delta} \defeq \Op(\SymbolClass{m}{\rho, \delta}).
    \end{align*}
    Naturally, an operator in
    \begin{align*}
        \SmoothingOperators \defeq \Op(\SmoothingSymbols)
    \end{align*}
    is called \emph{smoothing}.
\end{definition}

\subsection{First properties of symbol classes}
% TODO: Move because kernels are a prerequisites.

\begin{proposition}
    Suppose that $\rho, \delta \in \R$ satisfy $1 \geq \rho \geq \delta \geq 0$.
    Given $m, m_1, m_2 \in \R$,
    choose three symbols $\sigma \in \SymbolClass m {\rho, \delta}$,
    $\sigma_1 \in \SymbolClass {m_1} {\rho, \delta}$,
    $\sigma_2 \in \SymbolClass {m_2} {\rho, \delta}$.
    The kernels of the aforementioned symbols will be denoted by
    $\kappa$, $\kappa_1$ and $\kappa_2$ respectively.

    The following properties hold.
    \begin{enumerate}
        \item For each $\alpha \in \N^{\dimDifferenceOperators}$ and each $\beta \in \N^{\dim \Group}$,
            \begin{align*}
                (g, \lambda) \in \Group \times \VectorSpace \mapsto
                \LeftDifferentialOperator \beta \DifferenceOperator \alpha \sigma(g, \lambda)
            \end{align*}
            belongs to $\SymbolClass {m - \rho \abs \alpha + \delta \abs \beta} {\rho, \delta}$,
            and its kernel is given by
            \begin{align*}
                g \in \lambda \mapsto q^\alpha \LeftDifferentialOperator \beta \kappa_g.
            \end{align*}
        \item The pointwise adjunction of $\sigma$,
            \begin{align*}
                \adj \sigma(g, \lambda) \defeq \adj {\sigma(g, \lambda)},
            \end{align*}
            defines a symbol in $\SymbolClass m {\rho, \delta}$ whose kernel is given by
            \begin{align*}
                \adj \kappa_g(h) \defeq \conj {\kappa_g(h^{-1})},
            \end{align*}
            where the above is interpreted in the sense of distributions.
        \item The pointwise composition
            \begin{align*}
                (g, \lambda) \mapsto \sigma_1(g, \lambda) \sigma_2(g, \lambda)
            \end{align*}
            defines a symbol in $\SymbolClass {m_1 + m_2} {\rho, \delta}$,
            whose symbol is given by
            \begin{align*}
                g \in \Group \mapsto \conv {\kappa_{1, g}} {\kappa_{2, g}}
            \end{align*}
    \end{enumerate}
\end{proposition}

\section{Kernels and quantisation}

\begin{definition}[Kernel of a symbol]
\label{definition:kernel_of_symbol}
    Let $\sigma \in \SymbolClass{m}{\rho, \delta}$.
    For each $g \in \Group$, we let
    \begin{align*}
        \kappa_g \defeq \InverseFourier\{\sigma(g, \dummy)\} \in \TemperedDistributions\Group.
    \end{align*}
    The map
    \begin{align*}
        \kappa : \Group \to \TemperedDistributions\Group : g \mapsto \kappa_g
    \end{align*}
    is called the \emph{kernel} of $\sigma$.
\end{definition}

\begin{lemma}
    Let $\sigma \in \SymbolClass m {\rho, \delta}$,
    and denote by $\kappa$ its associated kernel.
    For each $g \in \Group$,
    $\kappa_g$ is a tempered distribution and
    \begin{align*}
        g \in \Group \mapsto \kappa_g \in \TemperedDistributions \Group
    \end{align*}
    is smooth,
    i.e.\ $\kappa \in \SmoothFunctions {\Group, \TemperedDistributions \Group}$.
\end{lemma}
\begin{proof}
    Let $g \in \Group$ and $\beta \in \N^{\dim \Group}$.
    By ???, $\kappa_g = \InverseFourier \{\LeftDifferentialOperator \beta \sigma(g, \dummy)\}$ is a tempered distribution.
    % TODO: Give reference
    Since the inverse Fourier transform is also continuous,
    the continuity of $\sigma$ in $g$ implies that
    \begin{align*}
        g \mapsto \kappa_g = \InverseFourier \{\LeftDifferentialOperator \beta \sigma(g, \dummy)\}
    \end{align*}
    is continuous.
    This concludes the proof.
\end{proof}

\begin{proposition}[Quantisation]
    Let $\sigma \in \SymbolClass{m}{\rho, \delta}$,
    and denote by $\kappa$ its associated kernel.
    If $\phi \in \Schwartz\Group$, then for each $g \in \Group$, we have
    \begin{align*}
        \Op(\sigma) \phi(g) = \conv{\phi}{\kappa_g}.
    \end{align*}

    In other words, $\kappa$ is the right convolution kernel associated with $\Op(\sigma)$.
\end{proposition}
\begin{proof}
    Let $g \in \Group$ and fix $s < -\max \{\dim \Group/2, \abs m\}$.
    By definition, we know that $\conv \phi \kappa_g \in \Sobolev s$ if and only if
    \begin{align*}
        \Rep \dummy \BesselPotential s \Fourier \{\conv \phi {\kappa_g}\}
        = \Rep \dummy \BesselPotential s \sigma(g, \dummy) \Fourier \phi
    \end{align*}
    belongs to $\LebesgueDual 2 \Group$.
    To show this,
    observe that the above identity implies
    \begin{align*}
        \norm[\SchattenClasses 2 {\Lebesgue 2 \CompactGroup}] {\Rep \dummy \BesselPotential s \Fourier \{\conv \phi {\kappa_g}\}}^2
        \leq C_s \norm[\SchattenClasses 2 {\Lebesgue 2 \CompactGroup}] {\Fourier \phi}^2,
    \end{align*}
    which, since the right-hand side is integrable on $\VectorSpace$,
    means that $\conv \phi {\kappa_g}$ belongs to $\Sobolev s$.
    By Proposition~\ref{proposition:general_Fourier_inverse_formula},
    we get that
    \begin{align*}
        \conv \phi {\kappa_g}(g)
        &= \int_\VectorSpace \tr\left(\Rep \lambda(g) \Fourier \{\conv \phi {\kappa_g}\}\right) \dd \lambda\\
        &= \int_\VectorSpace \tr\left(\Rep \lambda(g) \sigma(g; \lambda) \Fourier \phi(\lambda)\right) \dd \lambda,
    \end{align*}
    which concludes the proof,
    as the right-hand side is exactly $\Op(\sigma) \phi(g)$.
\end{proof}

\section{Link with the H\"ormander classes}

\begin{definition}[Rotation of symbols]
    Let $\tilde \sigma \in \SymbolClass[\GroupDirect]{m}{\rho, \delta}$.
    We define the operator
    \begin{align*}
        \Rotation {\tilde \sigma} : \SmoothFunctions \CompactGroup \to \SmoothFunctions \CompactGroup
    \end{align*}
    via the formula
    \begin{align*}
        \Rotation {\tilde \sigma}(x, k; \lambda) F(u) \defeq \tilde \sigma(x, k; k u^{-1} \lambda) F(u),
    \end{align*}
    where $x, \lambda \in \VectorSpace$, $k, u \in \CompactGroup$, and $F \in \SmoothFunctions \CompactGroup$.

    Similarly, given $\sigma \in \SymbolClass m {\rho, \delta}$,
    we define the operator
    \begin{align*}
        \InverseRotation \sigma : \SmoothFunctions \CompactGroup \to \SmoothFunctions \CompactGroup
    \end{align*}
    via the formula
    \begin{align*}
        \InverseRotation \sigma (x, k; \lambda) F(u) \defeq \sigma(x, k; u k^{-1} \lambda) F(u),
    \end{align*}
    where again $x, \lambda \in \VectorSpace$, $k, u \in \CompactGroup$, and $F \in \SmoothFunctions \CompactGroup$.
\end{definition}

\begin{lemma}
\label{lemma:Y_derivative_on_lambda_variable_of_symbols}
    Let $Y \in \LieAlgebraCompactGroup$, $\sigma \in \SymbolClass m {\rho, \delta}$, $\tilde \sigma \in \SymbolClass[\GroupDirect] m {\rho, \delta}$.
    We have the following expression
    \begin{align*}
        \LeftDifferentialOperatorFirstOrder Y_l \sigma(x, k; l \lambda)
        &= \sum_{j = 1}^{\dim \VectorSpace} \DifferenceOperator{j} \sigma(x, k; l \lambda) \Rep {l Y \lambda} (X_j)\\
        \LeftDifferentialOperatorFirstOrder Y_l \tilde \sigma(x, k; l \lambda)
        &= \sum_{j = 1}^{\dim \VectorSpace} \DifferenceOperator[\VectorSpace]{j} \tilde \sigma(x, k; l \lambda) \Rep[\GroupDirect] {l Y \lambda} (\partial_j),
    \end{align*}
    where $(x, k) \in \GroupDirect$, $\lambda \in \VectorSpace$ and $l \in \CompactGroup$.
\end{lemma}
\begin{proof}
    By definition, we know that
    \begin{align*}
        \LeftDifferentialOperatorFirstOrder Y_l \sigma(x, k; l \lambda)
        &= \eval{\D*{1}{t}}{t = 0} \sigma(x, k; l \exp_\CompactGroup (t Y) \lambda),
    \end{align*}
    which after applying the chain rule, becomes
    \begin{align}
        &\LeftDifferentialOperatorFirstOrder Y_l \sigma(x, k; l \lambda) \notag\\
        &\quad = \sum_{j = 1}^{\dim \VectorSpace} \eval{\D*{1}{s}}{s = 0} \sigma(x, k; l \lambda + s u e_j) \eval{\D*{1}{t}}{t = 0} \ip {l \exp_\CompactGroup (t Y) \lambda} {u e_j}.
        \label{eq:k_differentiation_of_lambda_variable_in_symbol}
    \end{align}

    Now, we observe that
    \begin{align*}
        \eval{\D*{1}{t}}{t = 0} \ip {l \exp_\CompactGroup (t Y) \lambda} {u e_j}
        = \ip {l Y \lambda} {u e_j} = \frac{1}{\i \turn} \Rep {l Y \lambda} (X_j),
    \end{align*}
    while at the same time
    \begin{align*}
        \eval{\D*{1}{s}}{s = 0} \sigma(x, k; l \lambda + s u e_j)
        = \i \turn \DifferenceOperator{j} \sigma(x, k; l \lambda).
    \end{align*}

    Therefore, it follows that~\eqref{eq:k_differentiation_of_lambda_variable_in_symbol} becomes
    \begin{align*}
        \LeftDifferentialOperatorFirstOrder Y_l \sigma(x, k; l \lambda)
        = \sum_{j = 1}^{\dim \VectorSpace} \DifferenceOperator{j} \sigma(x, k; l \lambda) \Rep {l Y \lambda} (X_j).
    \end{align*}

    Now, we turn to the case of symbols on the direct product.
    Again, we start with
    \begin{align*}
        \LeftDifferentialOperatorFirstOrder Y_l \tilde \sigma(x, k; l \lambda)
        &= \eval{\D*{1}{t}}{t = 0} \tilde \sigma(x, k; l \exp_\CompactGroup (t Y) \lambda).
    \end{align*}
    Applying the chain rule, we obtain
    \begin{align}
        \LeftDifferentialOperatorFirstOrder Y_l \tilde \sigma(x, k; l \lambda)
        &= \sum_{j = 1}^{\dim \VectorSpace} \eval{\D*{1}{s}}{s = 0} \tilde \sigma(x, k; l \lambda + s e_j) \eval{\D*{1}{t}}{t = 0} \ip {l \exp_\CompactGroup (t Y) \lambda} {e_j}.
        \label{eq:k_differentiation_of_lambda_variable_in_symbol_2}
    \end{align}

    By definition, it is clear that
    \begin{align*}
        \DifferenceOperator[\VectorSpace]{j} \tilde \sigma(x, k; l \lambda) = \frac 1 {\i \turn} \D{1}[\tilde \sigma]{{\lambda_j} } (x, k, l \lambda),
    \end{align*}
    while
    \begin{align*}
        \eval{\D*{1}{t}}{t = 0} \ip {l \exp_\CompactGroup (t Y) \lambda} {e_j}
        &= \ip {l Y \lambda} {e_j}
        = \frac{1}{\i \turn} \Rep[\GroupDirect] {l Y \lambda} (\partial_j).
    \end{align*}

    Therefore, it follows that \eqref{eq:k_differentiation_of_lambda_variable_in_symbol_2} becomes
    \begin{align}
        \LeftDifferentialOperatorFirstOrder Y_l \tilde \sigma(x, k; l \lambda)
        &= \sum_{j = 1}^{\dim \VectorSpace} \DifferenceOperator[\VectorSpace]{j} \tilde \sigma(x, k; l \lambda) \Rep[\GroupDirect] {l Y \lambda} (\partial_j).
    \end{align}
\end{proof}

\begin{lemma}
\label{lemma:link_between_symbols}
    Let $\sigma$ and $\tilde \sigma$ be symbols on $\Group$ and $\GroupDirect$ respectively be such that
    \begin{align*}
        \Op[\Group] (\sigma) = \Op[\GroupDirect] (\tilde \sigma).
    \end{align*}
    \begin{enumerate}
        \item
            \label{item:action_of_difference_operators}
            If $q \in \SmoothFunctions \Group$,
            then defining $\tilde q(y, l) = q(l y, l)$ yields
            \begin{align*}
                \DifferenceOperator{q} \sigma = \Rotation {\DifferenceOperator[\GroupDirect]{\tilde q} \tilde \sigma}
                \quad \text{and} \quad
                \DifferenceOperator[\GroupDirect]{\tilde q} \tilde \sigma = \InverseRotation {\DifferenceOperator{q} \sigma}.
            \end{align*}
            In particular, $\sigma = \Rotation {\tilde \sigma}$ and $\tilde \sigma = \InverseRotation \sigma$.
        \item
            \label{item:action_of_Euclidean_derivative}
            If $X \in \LieAlgebra \cap \VectorSpace$, then
            \begin{align*}
                \LeftDifferentialOperatorFirstOrder{X} \sigma
                = \Rotation {\LeftDifferentialOperatorFirstOrder{X} \tilde \sigma}
                \quad \text{and} \quad
                \LeftDifferentialOperatorFirstOrder{X} \tilde \sigma
                = \InverseRotation {\LeftDifferentialOperatorFirstOrder{X} \sigma}.
            \end{align*}
        \item
            \label{item:action_of_K-derivative}
            If $Y \in \LieAlgebraCompactGroup$, we have
            \begin{align*}
                \LeftDifferentialOperatorFirstOrder{Y} \sigma(x, k; \lambda)
                &= \Rotation {
                    \LeftDifferentialOperatorFirstOrder{Y} \tilde \sigma
                    + \sum_{j = 1}^{\dim \VectorSpace} (\DifferenceOperator[\VectorSpace]{j} \tilde \sigma) \Rep[\GroupDirect] {k Y k^{-1} \lambda} (\partial_j)
                }(x, k; \lambda)\\
                \LeftDifferentialOperatorFirstOrder Y \tilde \sigma(x, k; \lambda)
                &= \InverseRotation {\LeftDifferentialOperatorFirstOrder Y \sigma
                - \sum_{j = 1}^{\dim \VectorSpace} \DifferenceOperator{j} \sigma \ \Rep \lambda (Y^t X_j)
                }(x, k; \lambda).
            \end{align*}
    \end{enumerate}
\end{lemma}
\begin{proof}
    Let us write $T = \Op (\sigma)$.
    It follows that by
    % TODO: Reference both quantisations
    \begin{align*}
        T \phi(x, k)
        &= \int_{\GroupDirect} \phi(y, l) {\tilde \kappa}_{x, k}(x - y, l^{-1} k) \dd (y, l)\\
        &= \int_{\GroupDirect} \phi(y, l) {\kappa}_{x, k}({(y, l)}^{-1} (x, k)) \dd (y, l)
    \end{align*}
    in the sense of distributions.
    Therefore, it follows that we have
    \begin{align}
        {\tilde \kappa}_{x, k}(x - y, l^{-1} k) =
        {\kappa}_{x, k}({(y, l)}^{-1} (x, k)),
        \label{eq:link_between_the_kernels}
    \end{align}
    again in the sense of distributions.

    \begin{enumerate}
        \item
            Suppose first that $\sigma \in \SymbolClass m {\rho, \delta}$ for $m < -\dim \Group$
            so that its kernel $\kappa_{x, k} \in \Schwartz \Group$ by Theorem~\ref{theorem:kernel_estimates}.

            By definition, $\DifferenceOperator{q} \sigma(x, k; \lambda)$ is equal to
            \begin{align*}
                &\quad \int_\Group q((y, l)^{-1}) \kappa_{x, k}((y, l)^{-1}) \e^{\i \turn \ip {u^{-1} \lambda} y} \RightRegularRepresentation(l) \dd (y, l)\\
                &= \int_\Group q((y, l)^{-1} (x, k)) \kappa_{x, k}((y, l)^{-1} (x, k))\\
                &\qquad \qquad \e^{\i \turn \ip {u^{-1} \lambda} {k^{-1} (y - x)}} \RightRegularRepresentation(k^{-1} l) \dd (y, l)
            \end{align*}
            where we substituted $(y, l)$ for $(x, k)^{-1} (y, l)$ to obtain the last line.

            Using
            \begin{align*}
                (y, l)^{-1} (x, k) = (l^{-1}(x - y), l^{-1} k)
            \end{align*}
            and~\eqref{eq:link_between_the_kernels},
            if follows that $\DifferenceOperator{q} \sigma(x, k; \lambda)$ becomes
            \begin{align*}
                \int_\Group q(l^{-1}(x - y), l^{-1} k) \tilde \kappa_{x, k}(x - y, l^{-1} k) \e^{\i \turn \ip {k u^{-1} \lambda} {(y - x)}} \RightRegularRepresentation(k^{-1} l) \dd (y, l).
            \end{align*}

            We can now substitute $y$ for $y + x$ and $l$ for $k l$ to obtain that
            \begin{align*}
                &\DifferenceOperator{q} \sigma(x, k; \lambda) F(u) =\\
                &\quad \int_\Group q(-l^{-1} y, l^{-1}) \tilde \kappa_{x, k}(-y, l^{-1}) \e^{\i \turn \ip {k u^{-1} \lambda} y} \RightRegularRepresentation(l) \dd (y, l) F(u)
            \end{align*}
            which we recognise to be exactly $\DifferenceOperator {\tilde q} \tilde \sigma(x, k, k u^{-1} \lambda) F(u)$.
            Therefore, we have shown that
            \begin{align*}
                \DifferenceOperator{q} \sigma(x, k; \lambda) F(u)
                = \DifferenceOperator {\tilde q} \tilde \sigma(x, k, k u^{-1} \lambda) F(u),
            \end{align*}
            or in other words $\DifferenceOperator q \sigma = \Rotation {\DifferenceOperator[\GroupDirect] {\tilde q} \tilde \sigma}$.

            Now, suppose that $m \geq -\dim \Group$.
            Setting
            \begin{align*}
                \gamma \defeq -\dim \Group - 1 - m.
            \end{align*}
            Since $\Rep \lambda \BesselPotential \gamma \sigma$ is a symbol of order $< -\dim \Group$,
            then by the adove, we have
            \begin{align*}
                \Rep \lambda \BesselPotential \gamma \sigma
                = \Rotation {\Rep \lambda \BesselPotential \gamma \tilde \sigma}
                = \Rep \lambda \BesselPotential \gamma \Rotation {\tilde \sigma},
            \end{align*}
            or in other words $\sigma = \Rotation {\tilde \sigma}$.
            % TODO Conclude for difference operators

            The other identity is proven by observing that $\InverseRotation \dummy$ is the inverse of $\Rotation \dummy$.
        \item
            This identity follows easily from the previous point.
        \item
            Applying $\LeftDifferentialOperatorFirstOrder Y$ on both sides of $\sigma = \Rotation {\tilde \sigma}$,
            we obtain
            \begin{align}
                \LeftDifferentialOperatorFirstOrder Y_k \sigma(x, k; \lambda)
                &= \LeftDifferentialOperatorFirstOrder Y_{k' = k} \tilde \sigma(x, k'; k u^{-1} \lambda) + \LeftDifferentialOperatorFirstOrder Y_{k' = k} \tilde \sigma(x, k; k' u^{-1} \lambda)\\
                &= \Rotation {\LeftDifferentialOperatorFirstOrder Y \tilde \sigma}(x, k; \lambda) + \LeftDifferentialOperatorFirstOrder Y_{k' = k} \tilde \sigma(x, k; k' u^{-1} \lambda).
                \label{eq:Y_derivative_of_rotated_symbol_on_direct_product}
            \end{align}

            The second term on the right-hand side can be computed via Lemma \ref{lemma:Y_derivative_on_lambda_variable_of_symbols} to be
            \begin{align*}
                \LeftDifferentialOperatorFirstOrder Y_{k' = k} \tilde \sigma(x, k; k' u^{-1} \lambda)
                &= \sum_{j = 1}^{\dim \VectorSpace} \DifferenceOperator[\VectorSpace]{j} \tilde \sigma(x, k; k u^{-1} \lambda) \Rep[\GroupDirect]{k Y u^{-1} \lambda} (\partial_j)\\
                &= \sum_{j = 1}^{\dim \VectorSpace} \DifferenceOperator[\VectorSpace]{j} \tilde \sigma(x, k; k u^{-1} \lambda) \Rep[\GroupDirect]{k Y k^{-1} (k u^{-1} \lambda)} (\partial_j)\\
                &= \sum_{j = 1}^{\dim \VectorSpace} \Rotation {\DifferenceOperator[\VectorSpace]{j} \tilde \sigma} (x, k; \lambda) \Rotation {\Rep[\GroupDirect]{k Y k^{-1} \lambda} (\partial_j)}.
            \end{align*}

            Plugging the above into \eqref{eq:Y_derivative_of_rotated_symbol_on_direct_product}, we obtain
            \begin{align*}
                \LeftDifferentialOperatorFirstOrder Y \sigma(x, k; \lambda)
                &= \Rotation {\LeftDifferentialOperatorFirstOrder Y \tilde \sigma}(x, k; \lambda)
                + \sum_{j = 1}^{\dim \VectorSpace} \Rotation {\DifferenceOperator[\VectorSpace]{j} \tilde \sigma \ \Rep[\GroupDirect]{k Y k^{-1} \lambda} (\partial_j)}(x, k; \lambda)\\
                &= \Rotation {\LeftDifferentialOperatorFirstOrder Y \tilde \sigma
                    + \sum_{j = 1}^{\dim \VectorSpace} \DifferenceOperator[\VectorSpace]{j} \tilde \sigma \ \Rep[\GroupDirect]{k Y k^{-1} \lambda} (\partial_j)
                }(x, k; \lambda),
            \end{align*}
            which is what we wanted to show.

            To show the second identity,
            we apply $\LeftDifferentialOperatorFirstOrder Y$ on both sides of $\tilde \sigma = \InverseRotation {\sigma}$ to obtain
            \begin{align*}
                \LeftDifferentialOperatorFirstOrder Y_k \tilde \sigma(x, k; \lambda)
                = \LeftDifferentialOperatorFirstOrder Y_{k' = k} \sigma(x, k'; u k^{-1} \lambda) + \LeftDifferentialOperatorFirstOrder Y_{k' = k} \sigma(x, k; u {k'}^{-1} \lambda) \notag\\
                = \InverseRotation {\LeftDifferentialOperatorFirstOrder Y \sigma}(x, k; \lambda) - \LeftDifferentialOperatorFirstOrder Y_{u} \tilde \sigma(x, k; u k^{-1} \lambda).
                \label{eq:Y_derivative_of_rotated_symbol_on_semi-direct_product}
            \end{align*}

            Using Lemma~\ref{lemma:Y_derivative_on_lambda_variable_of_symbols} again to compute the second term of the right-hand side,
            \begin{align*}
                \LeftDifferentialOperatorFirstOrder Y_u \tilde \sigma(x, k; u k^{-1} \lambda)
                &= \sum_{j = 1}^{\dim \VectorSpace} \DifferenceOperator{j} \sigma (x, k; u k^{-1} \lambda) \Rep {u Y k \lambda} (X_j)\\
                &= \sum_{j = 1}^{\dim \VectorSpace} \DifferenceOperator{j} \sigma (x, k; u k^{-1} \lambda) \Rep {u Y u^{-1} u k \lambda} (X_j)\\
                &= \sum_{j = 1}^{\dim \VectorSpace} \InverseRotation {\DifferenceOperator{j} \sigma \Rep {u Y u^{-1} \lambda} (X_j)}.
            \end{align*}

            Observing that in the above,
            \begin{align*}
                \Rep {u Y u^{-1} \lambda} (X_j)
                = \i \turn \ip {u Y u^{-1} \lambda} {u X_j}
                = \i \turn \ip {\lambda} {u Y^t X_j}
                = \Rep \lambda (Y^t X_j)
            \end{align*}
            so that \eqref{eq:Y_derivative_of_rotated_symbol_on_direct_product} becomes
            \begin{align*}
                \LeftDifferentialOperatorFirstOrder Y_k \tilde \sigma(x, k; \lambda)
                = \InverseRotation {\LeftDifferentialOperatorFirstOrder Y \sigma
                - \sum_{j = 1}^{\dim \VectorSpace} \DifferenceOperator{j} \sigma \ \Rep \lambda (Y^t X_j)
                }(x, k; \lambda),
            \end{align*}
            concluding the proof.
    \end{enumerate}
\end{proof}

\begin{lemma}
\label{lemma:inclusion_in_zero_class}
    Let $m \defeq \frac {-\dim \CompactGroup} 2 (1 - \rho)$.
    % TODO: Can we prove the result for $m = 0$ or improve?
    \begin{enumerate}
        \item
            If $\sigma \in \SymbolClass m {\rho, \delta}$,
            then $\tilde \sigma = \InverseRotation \sigma$ satisfies
            \begin{align*}
                \sup_{(x, k) \in \Group} \esssup_{\lambda \in \VectorSpace}
                \norm [\Lin {\Lebesgue 2 \CompactGroup}] {\tilde \sigma(x, k; \lambda)} < \infty.
            \end{align*}
        \item
            If $\tilde \sigma \in \SymbolClass [\GroupDirect] m {\rho, \delta}$,
            then $\sigma = \Rotation {\tilde \sigma}$ satisfies
            \begin{align*}
                \sup_{(x, k) \in \Group} \esssup_{\lambda \in \VectorSpace}
                \norm [\Lin {\Lebesgue 2 \CompactGroup}] {\sigma(x, k; \lambda)} < \infty.
            \end{align*}
    \end{enumerate}
\end{lemma}
\begin{proof}
    Let $F \in \Lebesgue 2 \CompactGroup$ and $u \in \CompactGroup$.
    By the Sobolev Inequality,
    we know that
    \begin{align*}
        \int_\CompactGroup \sup_{v \in \CompactGroup} \abs {\sigma(x, k; v \lambda) F(u)}^2 \dd u
        \leq C \sum_{\beta} \int_\CompactGroup \int_\CompactGroup \abs{Y^\beta_v \sigma(x, k; v \lambda) F(u)}^2 \dd v \dd u
    \end{align*}
    Using Lemma~\ref{lemma:Y_derivative_on_lambda_variable_of_symbols},
    we know that each $Y^\beta_v \sigma \in \SymbolClass {m + \abs \beta (1 - \rho)} {\rho, \delta} \subset \SymbolClass 0 {\rho, \delta}$ so that the above becomes
    \begin{align*}
        \sup_{(x, k) \in \Group} \esssup_{\lambda \in \VectorSpace}
        \int_\CompactGroup \sup_{v \in \CompactGroup} \abs {\sigma(x, k; v \lambda) F(u)}^2 \dd u
        \leq C \norm [\Lebesgue 2 \CompactGroup] {F}^2.
    \end{align*}

    From the above inequality, we easily derive that
    \begin{align*}
        \sup_{(x, k) \in \Group}& \esssup_{\lambda \in \VectorSpace} \norm [\Lebesgue 2 \CompactGroup] {\tilde \sigma(x, k; \lambda) F}^2\\
        &= \sup_{(x, k) \in \Group} \esssup_{\lambda \in \VectorSpace} \int_\CompactGroup \abs {\sigma(x, k; u k^{-1} \lambda) F(u)}^2 \dd u\\
        &\leq \sup_{(x, k) \in \Group} \esssup_{\lambda \in \VectorSpace} \int_\CompactGroup \sup_{v \in \CompactGroup} \abs {\sigma(x, k; v \lambda) F(u)}^2 \dd u\\
        &\leq C \norm [\Lebesgue 2 \CompactGroup] {F}^2,
    \end{align*}
    which concludes the proof.

    The second bound is obtained with an identical argument.
\end{proof}

\begin{theorem}
    Let $\delta' = \max \{\delta, 1 - \rho\}$ and $m' = m + \frac {\dim \CompactGroup} 2 (1 - \rho)$.

    \begin{enumerate}
        \item
            For each $\sigma \in \SymbolClass m {\rho, \delta}$,
            the symbol $\tilde \sigma = \InverseRotation \sigma$ belongs to $\SymbolClass [\GroupDirect] {m'} {\rho, \delta'}$
            and satisfies
            \begin{align*}
                \Op[\Group] (\sigma) = \Op[\GroupDirect] (\tilde \sigma).
            \end{align*}
        \item
            Reciprocally, for each $\tilde \sigma \in \SymbolClass [\GroupDirect] m {\rho, \delta}$,
            the symbol $\sigma = \Rotation {\tilde \sigma}$ belongs to $\SymbolClass {m'} {\rho, \delta'}$
            and satisfies
            \begin{align*}
                \Op[\GroupDirect] (\tilde \sigma) = \Op[\Group] (\sigma).
            \end{align*}
    \end{enumerate}

    In particular, if $\rho = 1$, the symbol classes coincide
    \begin{align*}
        \SymbolClass m {1, \delta} = \SymbolClass [\GroupDirect] m {1, \delta}.
    \end{align*}
\end{theorem}
\begin{proof}
    Let us prove the first claim,
    as the proof of the second uses an identical argument.

    Let $\sigma \in \SymbolClass m {\rho, \delta}$,
    and write $\tilde \sigma \defeq \InverseRotation \sigma$.

    \begin{claim}
        Given $\tilde{q} \in \CompactGroup$ and $\beta \in \N^{\dim \Group}$,
        there exists a symbol $\tau \in \SymbolClass {m - \rho \order(q) + \delta' \abs \beta} {\rho, \delta'}$ such that
        \begin{align*}
            \DifferenceOperator [\GroupDirect] {\tilde q} X^\beta \tilde \sigma = \InverseRotation \tau.
        \end{align*}
    \end{claim}
    \begin{proof}[Proof of the claim]
        Let us prove the claim by induction on $\abs \beta$.
        The initial case $\beta = 0$ follows easily from Lemma~\ref{lemma:link_between_symbols} (\ref{item:action_of_difference_operators})
        and the inclusion
        \begin{align*}
            \SymbolClass {m - \rho \order(q)} {\rho, \delta} \subset \SymbolClass {m - \rho \order(q)} {\rho, \delta'}.
        \end{align*}

        Now, let us assume the claim holds for a certain $\beta$,
        so that there exists $\tau \in \SymbolClass {m - \rho \order(q) + \delta' \abs \beta} {\rho, \delta'}$ such that
        \begin{align*}
            \DifferenceOperator [\GroupDirect] {\tilde q} X^\beta \tilde \sigma = \InverseRotation \tau.
        \end{align*}

        If $X \in \LieAlgebra \cap \VectorSpace$,
        then by (\ref{item:action_of_Euclidean_derivative}) of Lemma~\ref{lemma:link_between_symbols} we have
        \begin{align*}
            \DifferenceOperator [\GroupDirect] {\tilde q} X X^\beta \tilde \sigma = \InverseRotation {X \tau},
        \end{align*}
        where $X \tau \in \SymbolClass {m - \rho \order(q) + \delta' (\abs \beta + 1)} {\rho, \delta'}$.

        If $X \in \LieAlgebraCompactGroup$,
        then by (\ref{item:action_of_K-derivative}) of Lemma~\ref{lemma:link_between_symbols} we have
        \begin{align*}
            \DifferenceOperator [\GroupDirect] {\tilde q} X X^\beta \tilde \sigma = \InverseRotation {X \tau - \sum_{j = 1}^{\dim \VectorSpace} \DifferenceOperator{j} \tau \ \Rep {\lambda} (X^t X_j) }
        \end{align*}
        where $X \tau - \sum_{j = 1}^{\dim \VectorSpace} \DifferenceOperator{j} \tau \ \Rep {\lambda} (X^t X_j) \in \SymbolClass {m - \rho \order(q) + \delta' (\abs \beta + 1)} {\rho, \delta'}$.
        To see this,
        we observe the condition $\delta' \geq 1 - \rho$ implies that $X \tau$ is the term with higher order.

        Either way, this concludes the induction, and thus the proof of the claim.
    \end{proof}

    By our claim,
    there exists a symbol $\tau \in \SymbolClass {m - \rho \order(q) + \delta' \abs \beta} {\rho, \delta'}$ such that
    \begin{align*}
        \DifferenceOperator [\GroupDirect] {\tilde q} X^\beta \tilde \sigma = \InverseRotation \tau.
    \end{align*}

    Note that the above implies that
    \begin{align*}
        \Rep[\GroupDirect] \lambda \BesselPotential{-m' + \rho \order(q) - \delta' \abs \beta + \gamma} \DifferenceOperator [\GroupDirect] {\tilde q} X^\beta \tilde \sigma(x, k; \lambda) \Rep [\GroupDirect] \lambda \BesselPotential{-\gamma}\\
        = \InverseRotation {\Rep \lambda \BesselPotential{-m' + \rho \order(q) - \delta' \abs \beta + \gamma} \tau \Rep \lambda \BesselPotential{-\gamma}} (x, k; \lambda).
    \end{align*}

    Since the operator inside $\InverseRotation \dummy$ on the second line belongs to
    \begin{align*}
        \SymbolClass {-\frac {\dim \CompactGroup} 2 (1 - \rho) } {\rho, \delta'},
    \end{align*}
    then Lemma~\ref{lemma:inclusion_in_zero_class} implies that
    \begin{align*}
        &\sup_{g \in \GroupDirect} \esssup_{\lambda \in \VectorSpace}\\
        &\quad
        \norm [\Lin {\Lebesgue 2 \CompactGroup}] {%
            \Rep[\GroupDirect] \lambda \BesselPotential{-m' + \rho \abs \alpha  - \delta' \abs \beta} \DifferenceOperator [\GroupDirect] {\tilde q} X^\beta \tilde \sigma(x, k; \lambda) \Rep [\GroupDirect] \lambda
        }
        % TODO: show the direct product doesn't require the gamma trick
    \end{align*}
    is finite,
    which by definition means that $\tilde \sigma \in \SymbolClass [\GroupDirect] {m'} {\rho, \delta'}$.
\end{proof}

\section{Littlewood-Paley decomposition}

\begin{lemma}
\label{lemma:derivatives_of_radial_functions}
    Let $\alpha \in \N^n$,
    and fix a radial function $\chi \in \SmoothFunctions{\R^n}$.
    If $\alpha \in \N^n$, then
    \begin{align}
        \D{\abs \alpha}[\chi]{x^\alpha}(x)
        = \sum_{r = 1}^{C_\alpha} f_r(\norm[\R^n]{x}) P_r(x),
    \end{align}
    where $P_r$ is a polynomial depending only on $\alpha$.

    Moreover, if $\supp \chi$ is compact
    and if there exists $\delta > 0$ such that the radial derivative $\D{\abs \alpha}[\chi]{\lambda}$ vanishes on on $\Ball[\R^n]{0}{\delta}$,
    then we have
    \begin{align*}
        \sup_r \sup_{\lambda \in \R^+} \abs{f_r} < \infty
    \end{align*}
\end{lemma}
\begin{proof}
    Using the chain rule, we know that for a purely radial function $f$
    \begin{align}
        \D{1}[f]{{x_i} } = \D{1}[\lambda]{{x_i} } \D{1}[f]{\lambda} = \frac{\D{1}[f]{\lambda}}{\norm[\R^n]{x}} x_i.
    \end{align}

    We know proceed to show the claim by induction on $\alpha$.
    The result is clearly true when $\abs{\alpha} = 0$.
    If we assume it is true for some $\alpha \in \N^n$, then by the above,
    \begin{align}
        \D{\abs \alpha + 1}[\chi]{{x_i} , x^\alpha}(x)
        &= \D{1}{{x_i} } \sum_{r = 1}^{C_\alpha} f_r(\norm[\R^n]{x}) P_r(x)\\
        &= \sum_{r = 1}^{C_\alpha} \frac{\D{1}[f_r]{\lambda}}{\norm[\R^n]{x}}(\norm[\R^n]{x}) x_i P_r(x)
        + \sum_{r = 1}^{C_\alpha} f_r(\norm[\R^n]{x}) \D{1}[P_r]{{x_i} }(x),
    \end{align}
    which concludes the proof.
\end{proof}

\begin{lemma}
\label{lemma:left_regular_representation_of_polynomials}
    Let $P \in \SmoothFunctions{\dualGroup{\VectorSpace}}$ be a polynomial.
    We can find functions $q_i \in \Polynomials{\CompactGroup}$, $f_i \in \Lebesgue{2}{\dualGroup{\VectorSpace}}$, $i = 1, \dots, N$ such that
    \begin{align*}
        P(k \lambda) = \sum_{i = 1}^N q_i(k) f_i(\lambda)
    \end{align*}
    for each $k \in \CompactGroup$ and each $\lambda \in \dualGroup{\VectorSpace}$.

    Moreover, the $q_i$ satisfy the bound
    \begin{align*}
        \sup_i \sup_\CompactGroup \abs{q_i} < \infty.
    \end{align*}
\end{lemma}

\begin{theorem}[Littlewood-Paley decomposition]
\label{theorem:Littlewood-Paley_decomposition}
\index{Littlewood-Paley decomposition}
    Let $\AbelianGroup$ be a \emph{locally compact abelian Lie group},
    and suppose that $\CompactGroup \subset \Aut(\AbelianGroup)$ is \emph{compact}.

    We now consider the group $\Group = \AbelianGroup \ltimes \CompactGroup$.

    Suppose further that the following conditions are satisfied.
    \begin{enumerate}
        \item The Haar measure of $\AbelianGroup$ is invariant under $\CompactGroup$,
            i.e.\ for each $f \in \Lebesgue{1}{\Group}$ and each $k \in \CompactGroup$, we have
            \begin{align*}
                \int_\AbelianGroup f(a) \dd a = \int_\AbelianGroup f(k a) \dd a.
            \end{align*}
        \item The Laplacian on $\AbelianGroup$ is invariant under $\CompactGroup$,
            i.e.\ for every $\phi \in \Schwartz\AbelianGroup$ and every $k \in \CompactGroup$, we have
            \begin{align*}
                (\Laplacian[\AbelianGroup] \phi(k \dummy))(a)
                = (\Laplacian[\AbelianGroup] \phi)(k a).
            \end{align*}
        \item There exists a Littlewood-Paley decomposition on $\AbelianGroup$ invariant under $\CompactGroup$,
            i.e.\ a sequence $\chi_j \in \Fourier(\Schwartz\AbelianGroup), j \in \N$ such that:
            \begin{enumerate}
                \item they sum to 1, i.e.\ we have $\sum_{j = 0}^\infty \chi_j = 1$.
                \item the functions $\chi_j$, $j \in \N$, are invariant under $\CompactGroup$:
                    for each $\lambda \in \dualGroup\AbelianGroup$ and each $k \in \CompactGroup$, we have $\chi_j(k \lambda) = \chi_j(\lambda)$.
                \item There exists $C > 0$ such that for every $j \in \N$, we have
                    \begin{align*}
                        \chi_j(\lambda) = 0 \quad \text{if} \quad \JapaneseBracket{\AbelianGroup}{\lambda} \geq C 2^j.
                    \end{align*}
                \item for each $q \in \Polynomials\AbelianGroup$,
                    there exists a finite family $q_1, \dots, q_{C_q} \in \Polynomials\CompactGroup$ such that
                    \begin{align*}
                        q(k a) = \sum_{r = 1}^{C_q} f_r(a) q_j(k)
                    \end{align*}
                    for some bounded functions $f_r : A \to \C$, $r \in \{1, \dots, C_q\}$.
            \end{enumerate}
    \end{enumerate}

    If all the above conditions hold,
    there exists a sequence $\eta_l \in \SmoothingSymbols$, $l \in \N$ of smoothing symbols satisfying the following properties
    \begin{enumerate}
        \item the semi-norms decay in the following way
            \begin{align}
                \SymbolSemiNorm{m}{\rho, \delta}{\eta_l} \leq C 2^{-lm}
            \end{align}
        \item the associated kernels $\kappa_l$ satisfy
            \begin{align*}
                \sum_{l = 0}^\infty \kappa_l = \DiracDelta{e_\Group}
            \end{align*}
            in the sense of distributions.
    \end{enumerate}
\end{theorem}
\begin{proof}
    \begin{description}
        \item[Step 1] Constructing the dyadic decomposition.

            First, let us find a smooth function $\chi_0 \in \SmoothFunctions{\dualGroup{\VectorSpace}}$ invariant under $\CompactGroup$ such that
            \begin{align*}
                \chi_0(\lambda) = 1 \  \text{if}\  \norm[\dualGroup{\VectorSpace}]{\lambda} \leq 1, \quad \text{and} \quad
                \chi_0(\lambda) \equiv 0 \ \text{if}\  \norm[\dualGroup{\VectorSpace}]{\lambda} \geq 2.
            \end{align*}

            Then, for each $l \in \N$ satisfying $l \geq 1$, let
            \begin{align*}
                \chi_l = \chi_0(2^{-l} \dummy) - \chi_0(2^{-l + 1} \dummy).
            \end{align*}
            so that $\supp \chi_l \subset \Ball{0}{2^{l + 1}}$.

            In particular, it should be clear that
            \begin{align*}
                \sum_{l = 0}^N \chi_l = \chi_0(2^{-N} \dummy)
            \end{align*}
            so that in fact
            \begin{align}
                \sum_{l = 0}^\infty \chi_l = 1.
                \label{eq:theorem:Littlewood-Paley_decomposition:partition_of_unity}
            \end{align}

            Fix $l \in \N$.
            We define our symbol $\eta_l$ as follows.
            For each $\tau \in \dualGroup{\CompactGroup}$, we let
            \begin{align*}
                \eta_l(\lambda)
                = \sum_{\JapaneseBracket{\CompactGroup}{\tau} \leq 2^l}
                \chi_{l - \Ceiling{\log_2 \JapaneseBracket{\CompactGroup}{\tau}}}(\lambda) \Id{V_\tau},
            \end{align*}
            where $V_\tau = \Span \{ \tau_{ij} : i, j = 1, \dots, \dimRep{\tau} \}$.

            Note that since $\supp \chi_l \subset \Ball{0}{2^{l + 1}}$,
            we get that
            \begin{align}
                \eta_l(\lambda) = 0 \quad \text{if } \norm[\dualGroup{\CompactGroup}]{v} \geq 2^{l + 1}
                \label{eq:theorem:Littlewood-Paley_decomposition:cancellation_condition}
            \end{align}

            We check that
            \begin{align*}
                \sum_{l = 0}^\infty \eta_l
                &= \sum_{l = 0}^\infty
                    \sum_{\JapaneseBracket{\CompactGroup}{\tau} \leq 2^l}
                        \chi_{l - \Ceiling{\log_2 \JapaneseBracket{\CompactGroup}{\tau}}} \Id{V_\tau}\\
                &= \sum_{\tau \in \dualGroup{\CompactGroup}}
                    \sum_{l = \Ceiling{\log_2 \JapaneseBracket{\CompactGroup}{\tau}}}^\infty
                        \chi_{l - \Ceiling{\log_2 \JapaneseBracket{\CompactGroup}{\tau}}} \Id{V_\tau},
            \end{align*}
            where the last line was obtained by commuting the two sums.

            Substituing $l$ for $l + \Ceiling{\log_2 \JapaneseBracket{\CompactGroup}{\tau}}$ in the inner sum,
            the above becomes
            \begin{align*}
                \sum_{l = 0}^\infty \eta_l
                = \sum_{\tau \in \dualGroup{\CompactGroup}}
                    \sum_{l = 0}^\infty
                        \chi_l \Id{V_\tau}
                = \sum_{\tau \in \dualGroup{\CompactGroup}}
                    \Id{V_\tau}
                = \Id{\Lebesgue{2}{\CompactGroup}},
            \end{align*}
            where the second to last inequality was obtained from~\eqref{eq:theorem:Littlewood-Paley_decomposition:partition_of_unity}.

        \item[Step 2] Computing the associated kernels $\kappa_l$.

            By applying the inverse Fourier Transform (Proposition~\ref{proposition:inverse_Fourier_Transform})
            we obtain that the kernel is given by
            \begin{align}
                \kappa_l(x, k)
                = \sum_{\JapaneseBracket{\CompactGroup}{\tau} \leq 2^l}
                    \int_\dualGroup{\VectorSpace}
                        \chi_{l - \Ceiling{\log_2 \JapaneseBracket{\CompactGroup}{\tau}}}(\lambda) \tr( \left. \Rep{\lambda}(x, k) \right|_{V_\tau} )
                    \dd \Plancherel{\VectorSpace}(\lambda)
                \label{eq:theorem:Littlewood-Paley_decomposition:computing_kernel}
            \end{align}

            By the Peter-Weyl Theorem,
            $\{ \sqrt{\dimRep{\tau}} \tau_{pq} : p, q = 1, \dots, \dimRep{\tau} \}$
            is an orthonormal basis of $V_\tau$,
            allowing us to compute the trace as
            \begin{align*}
                \tr( \left. \Rep{\lambda}(x, k) \right|_{V_\tau})
                &= \sum_{p = 1}^\dimRep{\tau}
                    \dimRep{\tau}
                    \int_\Lebesgue{2}{\CompactGroup}
                    (u \lambda)(x) \tau_{pp}(k^{-1} u) \conj{\tau_{pp}(u)}
                    \dd u\\
                &= \sum_{p,q = 1}^\dimRep{\tau}
                    \dimRep{\tau}
                    \int_\Lebesgue{2}{\CompactGroup}
                        (u \lambda)(x) \tau_{pq}(k^{-1}) \tau_{q p}(u) \conj{\tau_{pp}(u)}
                    \dd u.
            \end{align*}

            Using the above in~\eqref{eq:theorem:Littlewood-Paley_decomposition:computing_kernel},
            and substituing $\lambda$ for $u^{-1} \lambda$,
            we obtain
            \begin{align}
                \kappa_l (x, k)
                = &\sum_{\JapaneseBracket{\CompactGroup}{\tau} \leq 2^l}
                        \sum_{p,q = 1}^\dimRep{\tau}
                        \dimRep{\tau}
                        \int_\dualGroup{\VectorSpace}
                                \int_\Lebesgue{2}{\CompactGroup} \notag\\
                                    &\chi_{l - \Ceiling{\log_2 \JapaneseBracket{\CompactGroup}{\tau}}}(u^{-1}\lambda) \lambda(x) \tau_{pq}(k^{-1}) \tau_{qp}(u) \conj{\tau_{pp}(u)}
                                \dd u
                            \dd \Plancherel{\VectorSpace}(\lambda)
                    \label{eq:theorem:Littlewood-Paley_decomposition:computing_kernel:2}
            \end{align}

            Using the invariance of $\chi_{k}$ under $\CompactGroup$ and
            \begin{align*}
                \dimRep{\tau} \int_\Lebesgue{2}{\CompactGroup} \tau_{qp}(u) \conj{\tau_{pp}(u)} \dd u = \Kronecker{p}{q},
            \end{align*}
            then~\eqref{eq:theorem:Littlewood-Paley_decomposition:computing_kernel:2} becomes
            \begin{align*}
                \kappa_l (x, k)
                = &\sum_{\JapaneseBracket{\CompactGroup}{\tau} \leq 2^l}
                    \int_\dualGroup{\VectorSpace}
                        \chi_{l - \Ceiling{\log_2 \JapaneseBracket{\CompactGroup}{\tau}}}(\lambda) \lambda(x)
                    \dd \Plancherel{\VectorSpace}(\lambda)
                    \conj{\Character{\tau}(k)}
            \end{align*}
            which, after recognising the inverse Fourier Transform on $\dualGroup{\VectorSpace}$,
            yields the following expression for the kernel
            \begin{align}
                \kappa_l (x, k)
                = &\sum_{\JapaneseBracket{\CompactGroup}{\tau} \leq 2^l}
                    \InverseFourier[\VectorSpace]{\chi_{l - \Ceiling{\log_2 \JapaneseBracket{\CompactGroup}{\tau}}}}(x) \conj{\Character{\tau}(k)}.
                \label{eq:theorem:Littlewood-Paley_decomposition:kernel}
            \end{align}

        \item[Step 3] We show that for every $q \in \Polynomials{\Group}$
            and every $\lambda \in \dualGroup{\VectorSpace}$,
            \begin{align}
                \sup_{l \in \N} \norm[\Lin{\Lebesgue{2}{\CompactGroup}}]{\DifferenceOperator{q} \eta_l(\lambda)} < \infty.
            \end{align}

            Fix $q_1 \in \Polynomials{\VectorSpace}$, $q_2 \in \Polynomials{\CompactGroup}$
            and write $q(x, k) = q_1(x) q_2(k)$.
            We also choose an arbitrary function $F \in \Lebesgue{2}{\CompactGroup}$,
            and an element $u \in \CompactGroup$.

            Informally, the idea behind the proof of this step is the following
            if we can write
            \begin{align*}
                \DifferenceOperator{q} \eta_l(\lambda) F(u)
                = \sum_{\tau \in \dualGroup{\CompactGroup}}
                    \dimRep{\tau}
                    \tr\left( \tau(u) \sigma_{l, \lambda, q, \lambda}(u, \tau) \Fourier[\CompactGroup] F(\tau) \right),
            \end{align*}
            then a bound on the operator norm of $\DifferenceOperator{q} \eta_l (\lambda)$ can be obtained by finding an appropriate bound on $\sigma_{l, q, \lambda}$.
            Looking at~\eqref{eq:theorem:Littlewood-Paley_decomposition:kernel},
            we can see the latter is the right approach
            as we have a sum on $\dualGroup{\CompactGroup}$ already.

            Multiplying $\kappa_l$ par $q$ and taking the Fourier Transform, we get
            \begin{align}
                \DifferenceOperator{q} \eta_l (\lambda) F(u)
                = &\sum_{\JapaneseBracket{\CompactGroup}{\tau} \leq 2^l}
                    \int_\VectorSpace
                        \int_\CompactGroup
                            q_1(x) \InverseFourier[\VectorSpace]{\chi_{l - \Ceiling{\log_2 \JapaneseBracket{\CompactGroup}{\tau}}}}(x) (k u \lambda)(-x) \notag\\
                            &\quad q_2(k) \conj{\Character{\tau}(k)} F(k u)
                        \dd k
                    \dd x\notag\\
                = &\sum_{\JapaneseBracket{\CompactGroup}{\tau} \leq 2^l}
                    \int_\CompactGroup
                        \DifferenceOperator[\VectorSpace]{q_1} \chi_{l - \Ceiling{\log_2 \JapaneseBracket{\CompactGroup}{\tau}}}(k u \lambda)
                        q_2(k) \conj{\Character{\tau}(k)} F(k u)
                    \dd k \label{eq:theorem:Littlewood-Paley_decomposition:rho_condition},
            \end{align}
            where the second line was obtained by integrating with respect to $x$.

            Substituing $k$ for $k u^{-1}$ in the above,
            and using the Leibniz rule for polynomials on $q_2$, we obtain
            \begin{align*}
                \DifferenceOperator{q} \eta_l (\lambda) F(u)
                = &\sum_{\JapaneseBracket{\CompactGroup}{\tau} \leq 2^l}
                    \sum_{p = 1}^{C_q}
                        \int_\CompactGroup
                            \DifferenceOperator[\VectorSpace]{q_1} \chi_{l - \Ceiling{\log_2 \JapaneseBracket{\CompactGroup}{\tau}}}(k \lambda)\\
                            &\quad q_{2, p}(k) {q'}_{2, p}(u^{-1}) \conj{\Character{\tau}(k u^{-1})} F(k)
                        \dd k,
            \end{align*}

            \begin{claim}
                We have the decomposition
                \begin{align*}
                    \DifferenceOperator[\VectorSpace]{q_1} \chi_{l - \Ceiling{\log_2 \JapaneseBracket{\CompactGroup}{\tau}}}(k \lambda) = \sum_{r = 1}^{C_q} f_{l, r}(\tau, \lambda) q_r(k),
                \end{align*}
                where $q_r$ and $f_{l, r}$ satisfies the following bound
                \begin{align}
                    \sup_{l \in \N} \sup_{\tau \in \dualGroup{\CompactGroup}} \sup_{\lambda \in \dualGroup{\VectorSpace}} \abs{f_{l, r}(\tau, \lambda)} < \infty,\quad
                    \sup_{r} \sup_{k \in \CompactGroup} \abs{q(k)} < \infty
                    \label{eq:theorem:Littlewood-Paley_decomposition:claim_bound}
                \end{align}
            \end{claim}
            \begin{proof}[Proof of the claim]
                Assume first that $l - \Ceiling{\log_2 \JapaneseBracket{\CompactGroup}{\tau}} \neq 0$.
                Since
                \begin{align*}
                    \chi_{l - \Ceiling{\log_2 \JapaneseBracket{\CompactGroup}{\tau}}}(\lambda)
                    = \chi_1(2^{-l + \Ceiling{\log_2 \JapaneseBracket{\CompactGroup}{\tau} + 1}} \lambda),
                \end{align*}
                it follows that
                \begin{align*}
                    \DifferenceOperator[\VectorSpace]{q_1} &\chi_{l - \Ceiling{\log_2 \JapaneseBracket{\CompactGroup}{\tau}}}(k \lambda)
                    =\\
                    &\qquad 2^{(-l + \Ceiling{\log_2 \JapaneseBracket{\CompactGroup}{\tau} + 1}) \order{q_1}}
                    \DifferenceOperator[\VectorSpace]{q_1} \chi_1(2^{-l + \Ceiling{\log_2 \JapaneseBracket{\CompactGroup}{\tau} + 1}} k \lambda).
                \end{align*}

                Using Lemma~\ref{lemma:derivatives_of_radial_functions} and Lemma~\ref{lemma:left_regular_representation_of_polynomials}
                we obtain
                \begin{align*}
                    \DifferenceOperator[\VectorSpace]{q_1} &\chi_{l - \Ceiling{\log_2 \JapaneseBracket{\CompactGroup}{\tau}}}(k \lambda) =
                    2^{(-l + \Ceiling{\log_2 \JapaneseBracket{\CompactGroup}{\tau} + 1}) \order{q_1}}\\
                    &\sum_{r = 1}^{C_q} f_r(2^{-l + \Ceiling{\log_2 \JapaneseBracket{\CompactGroup}{\tau} + 1}} \norm[\dualGroup{\VectorSpace}]{\lambda}) q_r(k) P_r(2^{-l + \Ceiling{\log_2 \JapaneseBracket{\CompactGroup}{\tau} + 1}} \lambda),
                \end{align*}
                where each $f_r$, $q_r$ and each $P_r$ is independent of $l$ and $\tau$.
                Now, writing
                \begin{align*}
                    f_{l, r}(\tau, \lambda) =
                    &2^{(-l + \Ceiling{\log_2 \JapaneseBracket{\CompactGroup}{\tau} + 1}) \order{q_1}}\\
                    &f_r(2^{-l + \Ceiling{\log_2 \JapaneseBracket{\CompactGroup}{\tau} + 1}} \norm[\dualGroup{\VectorSpace}]{\lambda}) P_r(2^{-l + \Ceiling{\log_2 \JapaneseBracket{\CompactGroup}{\tau} + 1}} \lambda),
                \end{align*}
                we obtain the desired formula.
                The bound comes from the bounds in Lemma~\ref{lemma:derivatives_of_radial_functions} and~\ref{lemma:left_regular_representation_of_polynomials}.

                The case $l - \Ceiling{\log_2 \JapaneseBracket{\CompactGroup}{\tau}} = 0$ can be treated similarly.
            \end{proof}

            Using the above claim,
            and the identity
            \begin{align*}
                \conj{\Character{\tau}(k u^{-1})} = \sum_{i, j = 1}^\dimRep{\tau} \tau_{ij}(u) \conj{{\tau_{ij}(k)}},
            \end{align*}
            we observe that
            \begin{align}
                \DifferenceOperator{q} \eta_l (\lambda) F(u)
                = &\sum_{p, r = 1}^{C_q}
                    \sum_{\JapaneseBracket{\CompactGroup}{\tau} \leq 2^l}
                        \sum_{i, j = 1}^\dimRep{\tau}\notag
                            \tau_{ij}(u) {q'}_{2, p}(u^{-1})\\
                            &\quad f_{l, r}(\tau, \lambda)
                            \int_\CompactGroup
                                q_{r}(k) q_{2, p}(k) F(k) \conj{\tau_{ij}(k)}
                            \dd k
                            \label{eq:theorem:Littlewood-Paley_decomposition:exact_expression_for_operator}\\
                = &\sum_{p, r = 1}^{C_q}
                    \sum_{\JapaneseBracket{\CompactGroup}{\tau} \leq 2^l}
                        \sum_{i, j = 1}^\dimRep{\tau}
                            \tau_{ij}(u) {q'}_{2, p}(u^{-1})\notag\\
                            &\quad f_{l, r}(\tau, \lambda)
                            \Fourier[\CompactGroup]{} {\left\{ q_{r} q_{2, p} F\right\}}_{j i}(\tau).\notag
            \end{align}

            For $p, r = 1, \dots, \dimRep{\tau}$, defining the symbols
            \begin{align}
                \sigma_{l, \lambda, p, r}(u, \tau) =
                \begin{cases}
                    \frac{1}{\dimRep{\tau}} {q'}_{2, p}(u^{-1}) f_{l, r}(\tau, \lambda) \Id{\dimRep{\tau}} & \text{if } \JapaneseBracket{\CompactGroup}{\tau} \leq 2^l\\
                    0 & \text{otherwise}
                \end{cases}
            \end{align}
            and denoting by $T_{l, \lambda, p, r}$ the corresponding operators,
            we see that in fact,
            \begin{align*}
                \DifferenceOperator{q} \eta_l (\lambda) F(u)
                = &\sum_{p, r = 1}^{C_q}
                    \sum_{\tau \in \dualGroup{\CompactGroup}}
                        \dimRep{\tau}
                        \tr \left(
                            \tau(u)
                            \sigma_{l, \lambda, p, r}(u, \tau)
                            \Fourier[\CompactGroup]{} \left\{ q_r q_{2, p} F\right\}(\tau)
                        \right)\\
                = &\sum_{p, r = 1}^{C_q}
                        T_{l, \lambda, p, r} (q_r q_{2, p} F)(u).
            \end{align*}

            By~\eqref{eq:theorem:Littlewood-Paley_decomposition:claim_bound},
            we obtain
            \begin{align*}
                \norm[\Lin{\Lebesgue{2}{\CompactGroup}}]{T_{l, \lambda, p, r}}
                &\leq C \sup_{\tau \in \dualGroup{\CompactGroup}} \sup_{u \in \CompactGroup} \norm[\Lin{\HilbertRep{\tau}}]{\sigma_{l, \lambda, p, r}(u, \tau)}\\
                &\leq C_q < \infty
            \end{align*}
            From there, it follows that
            \begin{align*}
                \norm[\Lebesgue{2}{\CompactGroup}]{\DifferenceOperator{q} \eta_l (\lambda) F}
                &\leq C_q \sum_{p, r = 1}^{C_q} \norm[\Lebesgue{2}{\CompactGroup}]{q_r q_{2, p} F}
            \end{align*}
            where $C_q$ does not depend on $l$.

            Now, using the fact that
            \begin{align*}
                \sup_\CompactGroup \abs{q_r q_{2, p}} \leq C_q < \infty
            \end{align*}
            in the above, this concludes the step.

        \item[Step 4] $\ip[\Lebesgue{2}{\CompactGroup}]{\DifferenceOperator{q} \eta_l(\lambda) \mu_{mn}}{\nu_{kl}}$ is non-zero
            only if $\norm[\dualGroup{\VectorSpace}]{\lambda}, \JapaneseBracket{\CompactGroup}{\mu}, \JapaneseBracket{\CompactGroup}{\nu} \leq C_q 2^l$.

            Choose $C_q \geq 2$ so that
            \begin{align*}
                q_r q_{2, p}, q'_{2, p}(\dummy^{-1})
            \end{align*}
            can be generated by the representations $\{ \tau \in \dualGroup{\CompactGroup} : \JapaneseBracket{\CompactGroup}{\tau} \leq \frac{C_q}{2} \}$.

            Suppose now that $\max\{\JapaneseBracket{\CompactGroup}{\mu}, \JapaneseBracket{\CompactGroup}{\nu}\} > C_q 2^l$.
            It follows from our choice of $C_q$ that if $\JapaneseBracket{\CompactGroup}{\tau} \leq 2^l$,
            either of the following equations hold
            \begin{align*}
                \int_\CompactGroup q_r(k) q_{2, p}(k) \mu_{mn}(k) \conj{\tau_{ij}(k)} \dd k &= 0\\
                \int_\CompactGroup \tau_{ij}(u) q'_{2, p}(u) \conj{\nu_{mn}(k)} \dd k &= 0.
            \end{align*}

            From~\eqref{eq:theorem:Littlewood-Paley_decomposition:exact_expression_for_operator} with $F = \mu_{mn}$,
            we can see that the above implies
            \begin{align*}
                \ip[\Lebesgue{2}{\CompactGroup}]{\DifferenceOperator{q} \eta_l(\lambda) \mu_{mn}}{\nu_{kl}} = 0.
            \end{align*}

            The condition on $\lambda$ is obvious by~\eqref{eq:theorem:Littlewood-Paley_decomposition:rho_condition}
            keeping in mind that $\supp \chi_k \subset \Ball[\dualGroup{\VectorSpace}]{0}{2^{k + 1}}$.

        \item[Step 5] Conclusion.

            Let
            \begin{align*}
                L_l(\lambda) =
                \begin{cases}
                    {\left. \Rep{\lambda} \BesselPotential{1} \right|}_{\oplus_{\JapaneseBracket{\CompactGroup}{\tau} \leq C_q 2^l} V_\tau}
                    & \text{if } \norm[\dualGroup{\VectorSpace}]{\lambda} \leq C_q 2^l\\
                    0 & \text{otherwise},
                \end{cases}
            \end{align*}
            where $C_q$ is given by Step 4.

            Observe that $L_l(\lambda)$ is a bounded operator in $\Lebesgue{2}{\CompactGroup}$,
            and the operator norm is bounded by $C_q 2^l$ uniformly in $\lambda$.

            By Step 4,
            \begin{align*}
                \Rep{\lambda} \BesselPotential{- m - \order(q) + \gamma}&
                \DifferenceOperator{q} \eta_l(\lambda)
                \Rep{\lambda} \BesselPotential{-\gamma}\\
                &= {L_l(\lambda)}^{-m - \order(q) + \gamma}
                \DifferenceOperator{q} \eta_l(\lambda)
                {L_l(\lambda)}^{-\gamma},
            \end{align*}
            from which it follows that
            \begin{align*}
                \norm[\Lin{\Lebesgue{2}{\CompactGroup}}]{\Rep{\lambda} \BesselPotential{- m - \order(q) + \gamma} \DifferenceOperator{q} \eta_l(\lambda) \Rep{\lambda} \BesselPotential{-\gamma}}
                \leq C_q 2^{-l m},
            \end{align*}
            which is what we wanted to show.
    \end{description}
\end{proof}

\section{Kernel estimates}

As we have seen so far,
a pseudo-differential operator is entirely determined by either its \emph{right-convolution kernel} or its \emph{symbol}.
Symbols have the distinctive advantage of being \emph{smooth},
which is why for \emph{abelian} Lie groups,
a pseudo-differential calculus can be developed by doing all the calculations on the symbolic side
(for this, see e.g.\ \cite{RuzhanskyTurunen10}).

For non-commutative groups such as the motion group,
the Fourier transform is \emph{operator valued} and therefore, so are symbols.
This unfortunately means that it will sometimes be impractical to work with symbols,
and that calculations will have to be carried out in terms of the kernels.
It is therefore crucial to study their properties,
such as their \emph{singularity} or their \emph{decay} away from the origin.

We shall see that the regularity of the kernel increases
as the order of the symbol decreases.
If $\rho > 1$, difference operators decrease the order of the symbol
or equivalently multiplying kernels by some suitable polynomial increases its regularity.
The reader should therefore expect smoothness and Schwartz decay away from the origin,
as the latter is the only point where we allow our polynomials to all vanish.

Before stating the main result of this section,
let us define the quantity
\begin{align*}
    \norm [\Group] {(x, k)} \defeq \norm {x} + d(k, \Id{\VectorSpace}), \quad (x, k) \in \Group,
\end{align*}
which represents the distance between $(x, k)$ and the origin.
The above is of course an abuse of notation, as $\norm [\Group] \dummy$ is not a norm.

\begin{theorem}[Kernel estimates]
\label{theorem:kernel_estimates}
    Let $\rho, \delta \in \R$ be such that $1 \geq \rho \geq \delta \geq 0$
    and assume further that $\rho > 0$.
    Suppose that $\sigma \in \SymbolClass{m}{\rho, \delta}$,
    and $\kappa \in \SmoothFunctions {\Group, \TemperedDistributions \Group}$ is its associated kernel.
    \begin{enumerate}
        \item \label{item:kernel_estimates:at_infinity}
            For every $d > 0$ and every $N \in \N$,
            there exists $C \geq 0$ such that for every $g, h = (y, l) \in \Group$ with $\norm {y} \geq d$, we have
            \begin{align*}
                \abs{\kappa_g(h)} \leq C \norm y^{-N}.
            \end{align*}
        \item \label{item:kernel_estimates:at_origin:positive}
            If $m > - \dim \Group$, there exists $C \geq 0$ such that for every $g, h \in \Group$ with $h \neq 0$, we have
            \begin{align*}
                \abs{\kappa_g(h)} \leq C \norm [\Group] h^{- \frac{\dim \Group + m}{\rho}}.
            \end{align*}
        \item \label{item:kernel_estimates:at_origin:zero}
            If $m = -\dim \Group$, there exists $C \geq 0$ such that for every $g, h \in \Group$ with $h \neq 0$, we have
            \begin{align*}
                \abs{\kappa_g(h)} \leq C \log \norm [\Group] h.
            \end{align*}
        \item \label{item:kernel_estimates:at_origin:negative}
            If $m < -\dim \Group$, $\kappa_g$ is continuous on $\Group$ and is bounded
            \begin{align*}
                \sup_{g, h \in \Group} \abs{\kappa_g(h)} \leq C < \infty.
            \end{align*}
    \end{enumerate}
\end{theorem}

\subsection{\texorpdfstring{$L^2$}{L2} estimates for the kernel}

The first step towards establishing estimates on $\sup_{g, h \in \Group} \abs {\kappa_g(h)}$
is to provide $\Lebesgue 2 \Group$ estimates for derivatives of $\kappa_g$ that are not dependent on $g \in \Group$.
This is, of course, a consequence of the Sobolev inequality (Proposition~\ref{proposition:Sobolev_embedding}).

\begin{proposition}
\label{proposition:L2_bound_on_the_kernel}
    Let $\sigma \in \SymbolClass m {\rho, \delta}$ with $1 \geq \rho \geq \delta \geq 0$,
    and denote by $\kappa \in \SmoothFunctions {\Group, \TemperedDistributions \Group}$ its associated kernel.
    If $m < -\dim \Group / 2$,
    then $\kappa_g \in \Lebesgue 2 \Group$ for every $g \in \Group$,
    \begin{align*}
        \sup_{g \in \Group} \norm [\Lebesgue 2 \Group] {\kappa_g} < \infty.
    \end{align*}
\end{proposition}
\begin{proof}
    Let $g \in \Group$ and $\lambda \in \VectorSpace$.
    Clearly,
    \begin{align*}
        &\norm [\SchattenClasses 2 {\Lebesgue 2 \CompactGroup}] {\sigma(g, \lambda)}^2\\
        &\quad= \norm [\SchattenClasses 2 {\Lebesgue 2 \CompactGroup}] {\Rep \lambda \BesselPotential m \Rep \lambda \BesselPotential {-m} \sigma(g, \lambda)}^2\\
        &\quad\leq
        \norm [\SchattenClasses 2 {\Lebesgue 2 \CompactGroup}] {\Rep \lambda \BesselPotential {-m}}^2
        \norm [\Lin {\Lebesgue 2 \CompactGroup}] {\Rep \lambda \BesselPotential {-m} \sigma(g, \lambda)}^2.
    \end{align*}

    Using $\sigma \in \SymbolClass m {\rho, \delta}$,
    the above becomes
    \begin{align*}
        \norm [\SchattenClasses 2 {\Lebesgue 2 \CompactGroup}] {\sigma(g, \lambda)}^2
        \leq C \norm [\SchattenClasses 2 {\Lebesgue 2 \CompactGroup}] {\Rep \lambda \BesselPotential {-m}}^2.
    \end{align*}

    Integrating both sides of the above with respect to $\lambda \in \VectorSpace$
    and using the Plancherel formula (Proposition~\ref{proposition:Plancherel_formula}) on either sides,
    we get
    \begin{align*}
        \norm [\Lebesgue 2 \CompactGroup] {\kappa_g}^2 \leq C \norm [\Lebesgue 2 \CompactGroup] {\BesselPotentialKernel {-m}}^2.
    \end{align*}

    We conclude the proof by observing that the right-hand side is finite by assumption on $m$ and Proposition~\ref{proposition:Sobolev_embedding},
    but also that the right-hand side does not depend on $g$.
\end{proof}

\begin{corollary}
\label{corollary:Sobolev_estimates_on_the_kernel}
    Let $\sigma \in \SymbolClass m {\rho,\delta}$ with $1 \geq \rho \geq \delta \geq 0$,
    and denote by $\kappa \in \SmoothFunctions {\Group, \TemperedDistributions \Group}$ its associated kernel.

    For every $s \in \R$ satisfying
    \begin{align*}
        s < -\frac{\dim \Group} 2 - m,
    \end{align*}
    the kernel $\kappa_g$ belongs to $\Sobolev s$ for every $g \in \Group$.
    Moreover, for such an $s \in \R$, we have the estimate:
    \begin{align*}
        \sup_{g \in \Group} \norm [\Sobolev s] {\kappa_g} < \infty.
    \end{align*}
\end{corollary}
\begin{proof}
    The symbol
    \begin{align*}
        \Rep \lambda \BesselPotential s \sigma(g, \lambda)
    \end{align*}
    belongs to $\SymbolClass {m + s} {\rho, \delta}$,
    with $m + s \leq -\dim \Group/2$ by our assumption.

    Therefore, applying Proposition~\ref{proposition:L2_bound_on_the_kernel},
    we know that for every $g \in \Group$,
    \begin{align*}
        \BesselPotential s \kappa_g \in \Lebesgue 2 \Group,
    \end{align*}
    or in other words, $\kappa_g \in \Sobolev s$, and
    \begin{align*}
        \sup_{g \in \Group} \norm [\Sobolev s] {\kappa_g}
        = \sup_{g \in \Group} \norm [\Lebesgue 2 \Group] {\BesselPotential s \kappa_g}
        < \infty,
    \end{align*}
    concluding the proof.
\end{proof}

We can now prove Part~\ref{item:kernel_estimates:at_origin:negative} of Theorem~\ref{theorem:kernel_estimates}.

\begin{corollary}
\label{corollary:kernel_estimates:at_origin:negative}
    Let $\sigma \in \SymbolClass m {\rho,\delta}$ with $1 \geq \rho \geq \delta \geq 0$,
    and denote by $\kappa \in \SmoothFunctions {\Group, \TemperedDistributions \Group}$ its associated kernel.

    If $m < -\dim \Group$,
    then $\kappa_g \in \ContinuousFunctions \Group$ is continuous for every $g \in \Group$ and
    \begin{align*}
        \sup_{g \in \Group} \norm [\ContinuousFunctions \Group] {\kappa_g} < \infty.
    \end{align*}
\end{corollary}
\begin{proof}
    Choose $s \in \R$ such that
    \begin{align*}
        \frac {\dim \Group} 2 < s < -\frac {\dim \Group} 2 - m,
    \end{align*}
    which is possible since $m < -\dim \Group$.

    Since $m < -\dim \Group/2 - s$,
    Corollary~\ref{corollary:Sobolev_estimates_on_the_kernel} implies that
    $\kappa_g \in \Sobolev s$ for every $g \in \Group$,
    \begin{align*}
        \sup_{g \in \Group} \norm [\Sobolev s] {\kappa_g} < \infty.
    \end{align*}

    However,
    since $s > \dim \Group / 2$,
    the Sobolev inequality (Proposition~\ref{proposition:Sobolev_embedding})
    implies that
    \begin{align*}
        \sup_{g \in \Group} \norm [\ContinuousFunctions \Group] {\kappa_g}
        \leq C \sup_{g \in \Group} \norm [\Sobolev s] {\kappa_g} < \infty.
    \end{align*}
    This concludes the proof.
\end{proof}

\subsection{Estimates at infinity}

We start by proving Part \ref{item:kernel_estimates:at_infinity} of Theorem~\ref{theorem:kernel_estimates}.

\begin{proof}
    Let $M \in \R$ and choose $p \in 2 \N$ sufficiently large so that
    \begin{align*}
        g \geq M \quad \text{and} \quad
        m - \rho p + \Ceiling {\dim \Group / 2} < -\frac {\dim \Group} 2
    \end{align*}

    Fix $d > 0$,
    and let $\chi \in \SmoothFunctions \Group$ be a smooth radial function satisfying
    \begin{align*}
        \chi(y, l) =
        \begin{cases}
            0 & \text{if } \norm y \leq d/2\\
            1 & \text{if } \norm y \geq d
        \end{cases}
    \end{align*}

    From now on, we shall write $h = (y, l)$.

    By the Sobolev inequality (Proposition~\ref{proposition:Sobolev_embedding}),
    we have
    \begin{align}
        \sup_{\norm y \geq d} &\abs{\norm y^M \kappa_g(h)} \notag\\
        &\leq \sup_{h \in \Group} \abs{\norm y^M \kappa_g(h) \chi(h)} \notag\\
        &\leq C \sum_{\abs \beta \leq \Ceiling {\dim \Group / 2}} \norm [\Lebesgue 2 {\Group, \dd h}] {\LeftDifferentialOperator [h] \beta \{\norm y^M \kappa_g(h) \chi(h)\}}.
        \label{eq:Schwartz_decay_of_kernel:first_estimate}
    \end{align}

    Let us do some calculations on the right-hand side of the above.
    By the Leibniz formula, we have
    \begin{align}
        &\abs {\LeftDifferentialOperator [h] \beta \{\kappa_g(h) \chi(h)\}} \notag\\
        &\leq \lcsum {\beta' \leq \beta} \abs{\LeftDifferentialOperator [h] {\beta - \beta'} \{\norm y^{M - p} \chi(h)\} \LeftDifferentialOperator [h] {\beta'} \{\norm y^p \kappa_g(h)\}} \notag\\
        &\leq
        \left(\lcsum {\beta' \leq \beta} \abs{\LeftDifferentialOperator [h] {\beta'} \{\norm y^{M - p} \chi(h)\}}^2\right)^{\frac 1 2}
        \left(\lcsum {\beta' \leq \beta} \abs{\LeftDifferentialOperator [h] {\beta'} \{\norm y^p \kappa_g(h)\}}^2\right)^{\frac 1 2},
        \label{eq:Schwartz_decay_of_kernel:Leibniz}
    \end{align}
    where the last line was obtained by applying the Cauchy-Schwartz inequality to the sum.
    Since $p \geq M$ and $\chi$ vanishes around $\{0_\VectorSpace\} \times \CompactGroup$,
    it follows that
    \begin{align*}
        \sup_{h \in \Group} \lcsum {\beta' \leq \beta} \abs{\LeftDifferentialOperator [h] {\beta'} \{\norm y^{M - p} \chi(h)\}}^2 < \infty,
    \end{align*}
    which means that~\eqref{eq:Schwartz_decay_of_kernel:Leibniz} becomes
    \begin{align*}
        \abs {\LeftDifferentialOperator [h] \beta \{\kappa_g(h) \chi(h)\}}
        \leq C \left(\lcsum {\beta' \leq \beta} \abs{\LeftDifferentialOperator [h] {\beta'} \{\norm y^p \kappa_g(h)\}}^2\right)^{\frac 1 2}.
    \end{align*}

    Therefore, if we square both sides of the above and integrate with respect to $h$, we obtain
    \begin{align*}
        &\int_\Group \abs {\LeftDifferentialOperator [h] \beta \{\kappa_g(h) \chi(h)\}}^2 \dd h\\
        &\qquad \leq C \int_\Group \lcsum {\beta' \leq \beta} \abs{\LeftDifferentialOperator [h] {\beta'} \{\norm y^p \kappa_g(h)\}}^2 \dd h\\
        &\qquad \leq C \lcsum {\beta' \leq \beta} \sum_{\abs \alpha = p} \int_\Group \abs{\LeftDifferentialOperator {\beta'} \{q^\alpha \kappa_g\}}^2 \dd h,
    \end{align*}
    where the last line was obtained by using ??. % TODO: reference!
    Continuing the above calculation,
    we get
    \begin{align*}
        &\int_\Group \abs {\LeftDifferentialOperator [h] \beta \{\kappa_g(h) \chi(h)\}}^2 \dd h\\
        &\quad \leq C \sum_{\abs \alpha = p} \int_\Group \abs{\BesselPotential {\Ceiling {\dim \Group / 2}} \{q^\alpha \kappa_g\}}^2 \dd h.
    \end{align*}

    Combinig the above with~\eqref{eq:Schwartz_decay_of_kernel:first_estimate},
    this yields
    \begin{align}
        \sup_{\norm y \geq d} &\abs{\norm y^M \kappa_g(h)}
        \leq C \sum_{\abs \alpha = p} \norm [\Lebesgue 2 \Group] {\BesselPotential {\Ceiling {\dim \Group / 2}} \{q^\alpha \kappa_g\}}.
        \label{eq:Schwartz_decay_of_kernel:final_step}
    \end{align}
    In the above,
    each symbol $\Rep \lambda \BesselPotential {\Ceiling {\dim \Group / 2}} \DifferenceOperatorOrder \alpha \sigma(g, \lambda)$ is of order
    \begin{align*}
        m - \rho p + \Ceiling {\dim \Group / 2} < -\frac {\dim \Group} 2
    \end{align*}
    so that its kernel $\BesselPotential {\Ceiling {\dim \Group / 2}} \{q^\alpha \kappa_g\}$ belongs to $\Lebesgue 2 \Group$ uniformly in $g \in \Group$ by Proposition~\ref{proposition:L2_bound_on_the_kernel}.
    This implies that~\eqref{eq:Schwartz_decay_of_kernel:final_step} becomes
    \begin{align*}
        \sup_{g \in \Group} \sup_{\norm y \geq d} &\abs{\norm y^M \kappa_g(h)} < \infty,
    \end{align*}
    concluding the proof.
\end{proof}

\subsection{Estimates at the origin}

We have shown so far that the regularity of the kernel is linked to the order of its corresponding symbol in the following way:
the lower the order, the more regular the kernel is.
As applying difference operator strictly \emph{reduces the order} when $\rho > 0$,
the distribution
\begin{align*}
    q^\alpha \kappa_g \defeq \InverseFourier \{ \DifferenceOperatorOrder \alpha \sigma(g, \dummy) \}
\end{align*}
becomes more regular as $\abs \alpha$ increases.
This means that $\kappa_g$ can only have singularities at the common zeros
\begin{align*}
    \bigcap_{j = 1}^{\dimDifferenceOperators} \{ q_j = 0 \} = (0_\VectorSpace, \Id \VectorSpace),
\end{align*}
which happens to simply be the origin
as the family $\{q_j : j = 1, \dots, \dimDifferenceOperators\}$ was chosen to be \emph{strongly admissible}.

The aim of this subsection is therefore to study the strength of the only singularity that a kernel can have,
which will lead to the proof of parts \ref{item:kernel_estimates:at_origin:positive} and \ref{item:kernel_estimates:at_origin:zero} of Theorem~\ref{theorem:kernel_estimates}.

\begin{lemma}
\label{lemma:prepare_kernel_estimates_at_the_origin}
    Let $\sigma \in \SymbolClass m {\rho, \delta}$ be a symbol
    with kernel $\kappa \in \SmoothFunctions {\Group, \TemperedDistributions \Group}$,
    where $1 \geq \rho \geq \delta \geq 0$.
    For each $g \in \Group$ and each $l \in \N$,
    we let
    \begin{align*}
        \kappa_{l, g} \defeq \InverseFourier \{ \sigma(g, \dummy) \eta_l \},
    \end{align*}
    where $\eta_l$ is the symbol defined in Theorem~\ref{theorem:Littlewood-Paley_decomposition}.

    The following properties hold.
    \begin{enumerate}
        \item For every $m_1 < -\dim \Group$,
            we can find $C_{m_1} \geq 0$ such that
            \begin{align*}
                \sup_{h \in \Group} \abs {\kappa_{l, g}(h)} \leq C_{m_1} 2^{l(m - m_1)}
            \end{align*}
            holds for every $l \in \N$.
        \item For any compact subset of $S \subset \Group \setminus \{e\}$ and any $g \in \Group$,
            the series
            \begin{align*}
                \sum_{l = 0}^\infty \sup_{h \in S} \abs {\kappa_{l, g}(h)} < \infty
            \end{align*}
            converges.
    \end{enumerate}
\end{lemma}
\begin{proof}
    \begin{enumerate}
        \item
        \item
    \end{enumerate}
\end{proof}

\begin{proof}
    By the second part of Lemma~\ref{lemma:prepare_kernel_estimates_at_the_origin},
    we know that
    \begin{align*}
        \abs {\kappa_g(h)} \leq \sum_{l = 0}^\infty \abs{\kappa_{l, g}(h)}
    \end{align*}
    for each $h \in \Group \setminus \{e_\Group\}$.
    Moreover,
    for every $r \in 2 \N$ and every $m_1 < -\dim \Group$,
    we have
    \begin{align}
        \sup_{h \in \Group} \norm [\Group] h^{r} \abs{\kappa_{l, g}(h)}
        &\leq \sup_{h \in \Group} \sum_{\abs \alpha = r} \abs{q^\alpha \kappa_{l, g}(h)} \notag\\
        &\leq C 2^{l(m - m_1 - \rho r)}
        \label{eq:singularity_of_frequency_localised_kernels}
    \end{align}
    or equivalently
    \begin{align}
        \sup_{h \in \Group} \abs{\kappa_{l, g}(h)}
        &\leq C 2^{l(m - m_1 - \rho r)} \norm [\Group] h^{-r}
        \label{eq:singularity_of_frequency_localised_kernels:2}
    \end{align}
    by applying the first part of Lemma~\ref{lemma:preparation_for_general_inverse_formula} to the symbol $\DifferenceOperatorOrder \alpha \sigma$.

    Now, fix $h \in \Group$ such that $\norm [\Group] h \leq 1$.
    We let $l_0 \in \N$ be the unique natural number such that
    \begin{align*}
        2^{-l_0 - 1} < \norm [\Group] h \leq 2^{-l_0}.
    \end{align*}

    The idea is to estimate the singularity on the kernel via
    \begin{align*}
        \abs {\kappa_g(h)} \leq \sum_{l = 0}^{l_0 - 1} \abs{\kappa_{l, g}(h)} + \sum_{l = l_0}^\infty \abs{\kappa_{l, g}(h)}
    \end{align*}
    and use~\eqref{eq:singularity_of_frequency_localised_kernels} with carefully chosen $m_1 \in \R$ and $r \in 2 \N$ to estimate each term on the right-hand side.

    Let us now treat the two cases separately.
    \begin{description}
        \item[Case 1: $m > -\dim \Group$.]
            Let us start with the sum over $l = 0, \dots, l_0 - 1$.
            We let $r = 0$ and choose $m_1 \in \R$ such that
            \begin{align*}
                m - m_1 = \frac {\dim \Group + m} \rho \iff m_1 = m\left(1 - \frac 1 \rho\right) - \frac {\dim \Group} \rho
            \end{align*}

            Since $m > -\dim \Group$, we know that
            \begin{align*}
                m_1 \leq -m(1 - \frac 1 \rho - \frac 1 \rho) \leq -m < -\dim \Group
            \end{align*}
            so that we may apply~\eqref{eq:singularity_of_frequency_localised_kernels},
            which yields
            \begin{align*}
                \sum_{l = 0}^{l_0 - 1} \abs{\kappa_{l, g}(h)} &\leq C \sum_{l = 0}^{l_0 - 1} 2^{l(m - m_1)} \leq C 2^{l_0(m - m_1)}\\
                &\leq C \norm [\Group] h^{-\frac {\dim \Group + m} \rho}
            \end{align*}
            by our choice of $m_1$.

            Now, let us deal with the sum over $l \geq l_0$.
            In order to be able to apply~\eqref{eq:singularity_of_frequency_localised_kernels} and have a convergent sum,
            we need
            \begin{align}
                \begin{cases}
                    m_1 < -\dim \Group,\\
                    m - m_1 - \rho r < 0,\\
                    r(1 - \rho) + m - m_1 = \frac{\dim \Group + m} \rho.
                \end{cases}
                \label{eq:conditions_to_apply_dyadic_sum_for_large_terms}
            \end{align}
            To this end,
            we let
            \begin{align*}
                r &\defeq \min_{\N} \{n \in 2 \N : n > {\frac {m + \dim \Group} \rho}\}\\
                m_1 &\defeq r(1 - \rho) + m - \frac{\dim \Group + m} \rho.
            \end{align*}

            The third identify in~\eqref{eq:conditions_to_apply_dyadic_sum_for_large_terms} holds by definition of $m_1$.
            For the second, we observe that
            \begin{align*}
                m - m_1 - \rho r
                &= - r + \frac{\dim \Group + m} \rho < 0
            \end{align*}
            by definition of $r$.
            Similarly,
            \begin{align*}
                m - m_1
                &= r(\rho - 1) + \frac{\dim \Group + m} \rho\\
                &> (\dim \Group + m) (1 - \frac 1 \rho + \frac 1 \rho)\\
                &> \dim \Group + m,
            \end{align*}
            which is equivalent to $m_1 < - \dim \Group$.
            We have thus shown that~\eqref{eq:conditions_to_apply_dyadic_sum_for_large_terms} holds.

            Since we may now use~\eqref{eq:singularity_of_frequency_localised_kernels:2},
            we obtain
            \begin{align*}
                \sum_{l = l_0}^\infty \abs{\kappa_{l, g}(h)}
                &\leq C \sum_{l = l_0}^\infty 2^{l(m - m_1 - \rho r)} \norm [\Group] h^{-r}\\
                &\leq C 2^{l_0(m - m_1 - \rho r)} \norm [\Group] h^{-r}\\
                &\leq C \norm [\Group] h^{m_1 - m + \rho r - r} = C \norm [\Group] h^{-\frac{\dim \Group + m} \rho},
            \end{align*}
            concluding the case when $m > -\dim \Group$.
        \item[Case 2: $m = -\dim \Group$.]
            % TODO: Ask Veronique
    \end{description}
\end{proof}

\section{Adjoint and composition formulas}

\subsection{Adjoint formula}

\subsection{Composition formula}

\begin{lemma}
    Let $\sigma_1, \sigma_2 \in \SmoothingSymbols$
    and denote by $\kappa_1$ and $\kappa_2$ their respective associated kernels.
    We also set
    \begin{align*}
        \kappa_g(h) \defeq \int_\Group \kappa_{2, g l^{-1}}(h l^{-1}) \kappa_{1, g}(l) \dd l
    \end{align*}
    for each $g, h \in \Group$,
    and subsequently define the map
    \begin{align*}
        \sigma(g, \lambda) \defeq \Fourier \kappa_g(\lambda),
        \quad g \in \Group, \lambda \in \VectorSpace.
    \end{align*}

    The map $\sigma$ defined above is a smoothing symbol for which
    \begin{align*}
        \sigma(g, \lambda) = \int_\Group \kappa_{1, g}(l) \adj {\Rep \lambda(l)} \sigma_2(g l^{-1}, \lambda) \dd l
    \end{align*}
    holds for all $g \in \Group$ and $\lambda \in \VectorSpace$.
\end{lemma}

\begin{lemma}
    Let $m_1, m_2 \in \R$ and suppose $\rho, \delta \in \R$ are chosen such that $1 \geq \rho > \delta \geq 0$.
    % TODO: Add conditions on constants

    There exists $C \geq 0$ and an $N \in \N$ such that
    for any $\sigma_1, \sigma_2 \in \SmoothingSymbols$ and any $(g, \lambda) \in \Group \times \VectorSpace$,
    we have
    \begin{align*}
        &\norm [\Lin {\Lebesgue 2 \CompactGroup}] {%
            \LeftDifferentialOperator [g]  \beta \left(
                \sigma_1 \circ \sigma_2(g, \lambda)
                - \sum_{\abs \alpha \leq M} \sigma_1(g, \lambda) \LeftDifferentialOperator [g] \alpha \sigma_2(x, \lambda)
            \right)
        }\\
        &\qquad \leq C \SymbolSemiNorm {m_1, R} {\rho, \delta} N {\sigma_1} \SymbolSemiNorm {m_2} {\rho, \delta} N {\sigma_2}
    \end{align*}
\end{lemma}

\section{\texorpdfstring{$L^2$}{L2} boundedness}

The aim of this section is to prove
that symbols of order $0$ induce bounded operators in $\Lebesgue 2 \Group$
under one of the following conditions

\begin{enumerate}
    \item $\rho > \delta$;
    \item $\rho = \delta = 0$.
\end{enumerate}

These results would naturally generalise to symbols of order $m$ as follows:
if $\sigma \in \SymbolClass m {\rho, \delta}$,
where $\rho, \delta$ satisfy either of the conditions above,
then $\Op(\sigma)$ extends continuously to a continuous operator
\begin{align*}
    \Op(\sigma) : \Sobolev s \to \Sobolev {s - m}
\end{align*}
for every $s \in \R$.

\subsection{The case \texorpdfstring{$\rho > \delta$}{rho bigger than delta}}

Suppose that $\sigma_T \in \SymbolClass 0 {\rho, \delta}$ for $0 \leq \delta < \rho \leq 1$.
One way of showing that $T \defeq \Op(\sigma_T)$ is bounded would be to find $S \in \OperatorClass 0 {\rho, \delta}$ such that
\begin{align}
    \adj T T = C \Id {\Schwartz \Group} - \adj S S + R,
    \label{eq:L2_boundedness_decomposition}
\end{align}
where $R \in \OperatorClass m {\rho, \delta}$ for some $m < 0$.

It would then follow that
\begin{align*}
    \norm [\Lebesgue 2 \Group] {T \phi}^2
    &= \ip [\Lebesgue 2 \Group] {\adj T T \phi} \phi\\
    &= C \norm [\Lebesgue 2 \Group] {\phi}^2 - \norm [\Lebesgue 2 \Group] {S \phi}^2 + \ip [\Lebesgue 2 \Group] {R \phi} \phi\\
    &\leq C \norm [\Lebesgue 2 \Group] {\phi}^2 + \norm [\Lebesgue 2 \Group] {R \phi} \norm [\Lebesgue 2 \Group] \phi,
\end{align*}
reducing the boundedness of $T$ to that of $R$,
which is much easier to prove.

A sufficient condition to have \eqref{eq:L2_boundedness_decomposition} would be to have
\begin{align*}
    \adj {\sigma_T} \sigma_T = C \Id {\Lebesgue 2 \CompactGroup} - \adj {\sigma_S} \sigma_S,
\end{align*}
on the symbol side,
with $\sigma_S \defeq \Op^{-1}(S)$.
A good candidate for $\sigma_S$ would therefore be
\begin{align}
    \sigma_S \defeq \sqrt{C \Id {\Lebesgue 2 \CompactGroup} - \adj {\sigma_T} \sigma_T}.
    \label{eq:symbol_to_prove_L2_boundedness}
\end{align}

The condition that $S \in \OperatorClass 0 {\rho, \delta}$ is equivalent to showing
that $\sigma_S \in \SymbolClass 0 {\rho, \delta}$.
The crucial part of the proof is therefore to show that~\eqref{eq:symbol_to_prove_L2_boundedness} is a symbol of order $0$.

\begin{lemma}
\label{lemma:inverse_of_square_root_of_a_symbol_of_order_zero}
    Let $\rho, \delta \in \R$ be such that $1 \geq \rho > \delta \geq 0$.
    If $\sigma \in \SymbolClass 0 {\rho, \delta}$ is \emph{positive definite} and \emph{elliptic},
    then the map
    \begin{align*}
        \tilde \sigma(x, k; \lambda)
        \defeq \int_{\Gamma} z^{-\frac 1 2} (\sigma(x, k; \lambda) - z \Id {\Lebesgue 2 \CompactGroup})^{-1} \dd z
        % TODO specify what \gamma is
    \end{align*}
    also belongs to $\SymbolClass 0 {\rho, \delta}$.
\end{lemma}
\begin{proof}
    To shorten the notation,
    we write
    \begin{align*}
        R(z, \sigma)(x, k; \lambda) \defeq (\sigma(x, k; \lambda) - z \Id {\Lebesgue 2 \CompactGroup})^{-1}
    \end{align*}

    Let us now prove the following claim.

    \begin{claim}
        For every $\beta \in \N^{\dim \Group}$, and each $\alpha$
        % TODO: specify set for \alpha
        \begin{align}
            &\norm [\Lin {\Lebesgue 2 \CompactGroup}] {\Rep \lambda \BesselPotential {\rho \abs \alpha - \delta \abs \beta} \LeftDifferentialOperator \beta \DifferenceOperatorOrder \alpha R(z, \sigma)(x, k; \lambda)} \notag\\
            &\qquad \leq C_{\alpha, \beta} (1 + \abs z)^{-1}.
            \label{eq:estimates_on_the_resolvent}
        \end{align}
    \end{claim}
    \begin{proof}[Proof of the claim]
        First let us prove the claim for $\abs \alpha = \abs \beta = 0$.
    \end{proof}

    Using~\eqref{eq:estimates_on_the_resolvent},
    we can show that
    \begin{align*}
        &\norm [\Lin {\Lebesgue 2 \Group}] {\Rep \lambda \BesselPotential {\rho \abs \alpha - \delta \abs \beta} \LeftDifferentialOperator \beta \DifferenceOperatorOrder \alpha \tilde \sigma(x, k; \lambda)}\\
        &\quad \leq C_{\alpha, \beta} \int_\Gamma \abs z^{-\frac 1 2} (1 + \abs z)^{-1} \abs{\dd z} < +\infty,
    \end{align*}
    showing that $\tilde \sigma$ belongs to $\SymbolClass 0 {\rho, \delta}$.
\end{proof}

\begin{corollary}
\label{corollary:square_root_of_a_symbol_of_order_zero}
    Let $\rho, \delta \in \R$ be such that $1 \geq \rho > \delta \geq 0$.
    If $\sigma \in \SymbolClass 0 {\rho, \delta}$ is \emph{positive definite} and \emph{elliptic},
    then its \emph{square root}
    \begin{align*}
        (x, k; \lambda) \in \Group \times \VectorSpace \mapsto \sqrt{\sigma(x, k; \lambda)}
    \end{align*}
    also belongs to $\SymbolClass 0 {\rho, \delta}$.
\end{corollary}
\begin{proof}
    Let $\tilde \sigma$ be the symbol defined in Lemma~\ref{lemma:inverse_of_square_root_of_a_symbol_of_order_zero}.
    By the Helffer-Sjostrand formula,
    % TODO Reference for Helffer-Sjostrand
    we get
    \begin{align*}
        \sqrt{\sigma(x, k; \lambda)} = \tilde \sigma(x, k; \lambda) \sigma(x, k; \lambda),
    \end{align*}
    which must be in $\SymbolClass 0 {\rho, \delta}$,
    since $\tilde \sigma, \sigma \in \SymbolClass 0 {\rho, \delta}$ by Lemma~\ref{lemma:inclusion_in_zero_class}.
\end{proof}

\begin{proposition}[$L^2$ boundedness]
    Let $\rho, \delta \in \R$ be such that $1 \geq \rho > \delta \geq 0$.
    If $\sigma \in \SymbolClass 0 {\rho, \delta}$,
    then its associated operator $T \defeq \Op(\sigma)$ is bounded in $\Lebesgue 2 \Group$,
    i.e.\ we have
    \begin{align*}
        \norm [\Lebesgue 2 \Group] {T \phi} \leq C \norm [\Lebesgue 2 \Group] \phi.
    \end{align*}
    for every $\phi \in \Schwartz \Group$.
\end{proposition}
\begin{proof}
    \begin{description}
        \item[Step 1.] If $\sigma \in \SymbolClass m {\rho, \delta}$ for $m < -1$, then
            \begin{align*}
                \norm [\Lin{\Lebesgue 2 \Group}] {\Op(\sigma)} < \infty.
            \end{align*}
        \item[Step 2.] If $\sigma \in \SymbolClass m {\rho, \delta}$ for $m < 0$, then
            \begin{align*}
                \norm [\Lin{\Lebesgue 2 \Group}] {\Op(\sigma)} < \infty.
            \end{align*}

            Suppose by contradiction that the claim does not hold.
            Therefore, there exists $m < 0$ and $\sigma \in \SymbolClass m {\rho, \delta}$ such that
            $\Op(\sigma)$ is not bounded.
            In particular, for each $n \in \N$,
            \begin{align*}
                (\Op(\sigma) \adj {\Op(\sigma)})^n
            \end{align*}
            is in $\OperatorClass {2 m n} {\rho, \delta}$ and unbounded.
            When $n$ is such that $-2 m n < -1$, this contradicts Step 1.
        \item[Step 3.] Conclusion

            Let $C \geq 0$ be such that
            \begin{align*}
                \sup_{(x, k) \in \Group} \esssup_{\lambda \in \VectorSpace} \norm [\Lebesgue 2 \CompactGroup] {\sigma(x, k; \lambda)} \leq C.
            \end{align*}

            It follows that
            \begin{align*}
                4 C^2 \Id {\Lebesgue 2 \CompactGroup} - \adj \sigma \sigma
            \end{align*}
            is an operator of order $0$,
            and so by Corollary~\ref{corollary:square_root_of_a_symbol_of_order_zero}, the symbol
            \begin{align*}
                \sigma_S \defeq \sqrt{4 C^2 \Id {\Lebesgue 2 \CompactGroup} - \adj \sigma \sigma}
            \end{align*}
            is also in $\SymbolClass 0 {\rho, \delta}$.
            By definition, we have
            \begin{align}
                \adj \sigma \sigma = 4 C^2 \Id {\Lebesgue 2 \CompactGroup} - \adj {\sigma_S} \sigma_S.
                \label{eq:decomposition_of_unity_in_terms_of_symbols_squared}
            \end{align}

            By the adjoint and composition formulas,
            we have
            \begin{align*}
                \Op(\adj \sigma \sigma) = \adj T T + R_1\\
                \Op(\adj {\sigma_S} \sigma_S) = \adj S S +  R_2,
            \end{align*}
            with $R_1, R_2 \in \OperatorClass {\rho - \delta} {\rho, \delta}$.
            Therefore, applying $\Op$ on both sides of~\eqref{eq:decomposition_of_unity_in_terms_of_symbols_squared} yields
            \begin{align*}
                \adj T T = 4 C^2 \Id {\Schwartz \Group} - \adj S S + R,
                \quad R \defeq (R_2 - R_1) \in \OperatorClass {\rho - \delta} {\rho, \delta}.
            \end{align*}

            Now, fix $\phi \in \Lebesgue 2 \Group$.
            Since Step 2 implies that $R$ is bounded,
            we get
            \begin{align*}
                \norm [\Lebesgue 2 \Group] {T \phi}^2
                &= \ip [\Lebesgue 2 \Group] {\adj T T \phi} \phi\\
                &= 4 C^2 \norm [\Lebesgue 2 \Group] \phi^2 - \norm [\Lebesgue 2 \Group] {S \phi}^2 + \ip [\Lebesgue 2 \Group] {R \phi} {\phi}\\
                &\leq (4 C^2 + \norm [\Lin {\Lebesgue 2 \Group}] R) \norm [\Lebesgue 2 \Group] \phi^2,
            \end{align*}
            concluding the proof.
    \end{description}
\end{proof}
