\chapter{Towards groups with polynomial growth}

\section{The setting}

In this chapter,
$\Group$ will denote a connected locally compact Lie group,
for which we fix a non-zero left-invariant Haar measure $\mu = \dd g$.
Moreover,
we shall assume that $\Group$ has \emph{polynomial growth},
in the sense of the following definition.

\begin{definition}[Group of polynomial growth]
    We shall say that $\Group$ has \emph{polynomial growth}
    if for every compact neighbourhood $U \subset \Group$
    containing the identity $e_\Group$,
    there exists $C \geq 0$ such that
    \begin{align*}
        \mu(U^n) \leq C n^C
    \end{align*}
    for every $n \in \N$.
\end{definition}

This implies that $\Group$ is unimodular.

\section{The heat kernel}

\section{Kernel estimates}

First,
let us lower the requirement that we need a Littlewood-Paley decomposition
to prove the kernel estimates.

\begin{theorem}[Kernel estimates at the origin]
    Assume that $1 \geq \rho \geq \delta \geq 0$ with $\rho \neq 0$.

    Suppose there exists a family $(h_t)_{t \in (0, 1]} \subset \Schwartz \Group$
    satisfying the following properties.
    \begin{itemize}
        \item As $t$ tend to $0$, $h_t$ converges to $\delta_{e_\Group}$.
        \item If $q \in \SmoothFunctions {\Group}$ vanishes up to order $a \in \N$,
            then for each $s \in \R$ there exists $C \geq 0$ such that
            \begin{align}
                \norm [\KernelsSobolev s {s - m}] {q h_t}
                \leq C t^{m - \rho a}
                \label{eq:kernel_estimates:heat_kernel_condition}
            \end{align}
            holds for every $0 < t \leq 1$.
    \end{itemize}

    If $\kappa \in \TemperedDistributions \Group$ satisfies
    \begin{align}
        q \kappa \in \Sobolev \gamma,
        \quad \gamma < -\frac n 2 -m + \rho a
        \label{eq:kernel_estimates:assumption_on_the_kernel}
    \end{align}
    whenever $q$ vanishes up to order $a$ and $m > -n$,
    then
    \begin{align*}
        \abs{\kappa(g)} \leq C \norm [\Group] h^{-\frac{m + n} \rho}.
    \end{align*}
\end{theorem}
\begin{proof}
    Fix $g \in \Group \setminus \{e_\Group\}$,
    and let $t \defeq \norm [\Group] g^2$.

    Let us write $\kappa_t \defeq \conv \kappa {h_t}$.
    By the Sobolev inequality,
    we know that for each $s < -n/2$
    \begin{align*}
        \norm [\ContinuousFunctions \Group] {\kappa_t}
        \leq \norm [\Sobolev {-s}] {\conv \kappa {h_t}}
        \leq \norm [\KernelsSobolev {s - m} {-s}] {h_t}
        \norm [\Sobolev {s - m}] \kappa,
    \end{align*}
    where the right-hand side is finite by~\eqref{eq:kernel_estimates:assumption_on_the_kernel}.
    Using~\eqref{eq:kernel_estimates:heat_kernel_condition},
    we then obtain
    \begin{align*}
        \norm [\ContinuousFunctions \Group] {\kappa_t}
        \leq C t^{s - \frac m 2} \norm [\Sobolev {s - m}] \kappa
    \end{align*}

    Observing we may choose $s < -n/2$ so that
    \begin{align*}
        s - \frac m 2 = -\frac {m + n} {2 \rho},
    \end{align*}
    we obtain
    \begin{align*}
        \abs {\kappa_t(g)}
        \leq C \norm [\Sobolev {\frac m 2 -\frac {m + n} {2 \rho}}] \kappa \norm [\Group] g^{- \frac {m + n} \rho}.
    \end{align*}
\end{proof}

\section{Calder\'on-Zygmund property}
