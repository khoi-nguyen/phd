\chapter{Symbols}
\label{chapter:symbols}

\section{Littlewood-Paley decomposition}
\label{section:littlewood-paley_decomposition}

\begin{theorem}[Littlewood-Paley decomposition]
\label{theorem:Littlewood-Paley_decomposition}
\index{Littlewood-Paley decomposition}
    There exists a sequence $\eta_l \in \SmoothSymbols$, $l \in \N$ of smoothing symbols satisfying the following properties:
    \begin{enumerate}
        \item the semi-norms decay in the following way:
            \begin{align}
                \SymbolSemiNorm{m}{\rho, \delta}{\eta_l} \leq C 2^{-lm}
            \end{align}
        \item the associated kernels $\kappa_l$ satisfy
            \begin{align*}
                \sum_{l = 0}^\infty \kappa_l = \DiracDelta{e_\Group}
            \end{align*}
            in the sense of distributions.
    \end{enumerate}
\end{theorem}
\begin{proof}
    \begin{description}
        \item[Step 1] Constructing the dyadic decomposition.

            First, let us find a smooth function $\chi_0 \in \SmoothFunctions{\dualGroup{\VectorSpace}}$ invariant under $K$ such that:
            \begin{align*}
                \chi_0(\rho) = 1 \  \text{if}\  \norm[\dualGroup{V}]{\rho} \leq 1, \quad \text{and} \quad
                \chi_0(\rho) \equiv 0 \ \text{if}\  \norm[\dualGroup{V}]{\rho} \geq 2.
            \end{align*}

            Then, for each $l \in \N$ satisfying $l \leq 1$, let
            \begin{align*}
                \chi_l = \chi_0(2^{-l} \dummy) - \chi_0(2^{-l + 1} \dummy).
            \end{align*}
            so that $\supp \chi_l \subset \Ball{0}{2^{l + 1}}$.

            In particular, it should be clear that
            \begin{align*}
                \sum_{l = 0}^N \chi_l = \chi_0(2^{-N} \dummy)
            \end{align*}
            so that in fact
            \begin{align}
                \sum_{l = 0}^\infty \chi_l = 1.
                \label{eq:theorem:Littlewood-Paley_decomposition:partition_of_unity}
            \end{align}

            Fix $l \in \N$.
            We define our symbol $\eta_l$ as follows.
            For each $\tau \in \dualGroup{\CompactGroup}$, we let:
            \begin{align*}
                \eta_l(\rho)
                = \sum_{\JapaneseBracket{\CompactGroup}{\tau} \leq 2^l}
                \chi_{l - \Ceiling{\log_2 \JapaneseBracket{\CompactGroup}{\tau}}}(\rho) \Id{V_\tau},
            \end{align*}
            where $V_\tau = \Span \{ \tau_{ij} : i, j = 1, \dots, \dimRep{\tau} \}$.

            Note that since $\supp \chi_l \subset \Ball{0}{2^{l + 1}}$,
            we get that
            \begin{align}
                \eta_l(\rho) = 0 \quad \text{if } \norm[\dualGroup{\CompactGroup}]{v} \geq 2^{l + 1}
                \label{eq:theorem:Littlewood-Paley_decomposition:cancellation_condition}
            \end{align}

            We check that
            \begin{align*}
                \sum_{l = 0}^\infty \eta_l
                &= \sum_{l = 0}^\infty
                    \sum_{\JapaneseBracket{\CompactGroup}{\tau} \leq 2^l}
                        \chi_{l - \Ceiling{\log_2 \JapaneseBracket{\CompactGroup}{\tau}}} \Id{V_\tau}\\
                &= \sum_{\tau \in \dualGroup{\CompactGroup}}
                    \sum_{l = \Ceiling{\log_2 \JapaneseBracket{\CompactGroup}{\tau}}}^\infty
                        \chi_{l - \Ceiling{\log_2 \JapaneseBracket{\CompactGroup}{\tau}}} \Id{V_\tau},
            \end{align*}
            where the last line was obtained by commuting the two sums.

            Substituing $l$ for $l + \Ceiling{\log_2 \JapaneseBracket{\CompactGroup}{\tau}}$ in the inner sum,
            the above becomes:
            \begin{align*}
                \sum_{l = 0}^\infty \eta_l
                = \sum_{\tau \in \dualGroup{\CompactGroup}}
                    \sum_{l = 0}^\infty
                        \chi_l \Id{V_\tau}
                = \sum_{\tau \in \dualGroup{\CompactGroup}}
                    \Id{V_\tau}
                = \Id{\Lebesgue{2}{\CompactGroup}},
            \end{align*}
            where the second to last inequality was obtained from~\eqref{eq:theorem:Littlewood-Paley_decomposition:partition_of_unity}.

        \item[Step 2] Computing the associated kernels.

            By applying the inverse Fourier Transform,
            we obtain that the kernel is given by:
            \begin{align}
                \kappa_l(x, k)
                = \sum_{\JapaneseBracket{\CompactGroup}{\tau} \leq 2^l} \int_\dualGroup{\VectorSpace}
                        \chi_{l - \Ceiling{\log_2 \JapaneseBracket{\CompactGroup}{\tau}}}(\rho) \tr( \left. \Rep{\rho}(x, k) \right|_{V_\tau} )
                    \dd \Plancherel{\VectorSpace}(\rho)
                    \label{eq:theorem:Littlewood-Paley_decomposition:computing_kernel}
            \end{align}

            Observing that
            \begin{align*}
                \tr( \left. \Rep{\rho}(x, k) \right|_{V_\tau})
                &= \sum_{p = 1}^\dimRep{\tau}
                    \dimRep{\tau}
                    \int_\Lebesgue{2}{\CompactGroup}
                    (u \rho)(x) \tau_{pp}(k^{-1} u) \conj{\tau_{pp}(u)}
                    \dd u\\
                &= \sum_{p,q = 1}^\dimRep{\tau}
                    \dimRep{\tau}
                    \int_\Lebesgue{2}{\CompactGroup}
                        (u \rho)(x) \tau_{pq}(k^{-1}) \tau_{q p}(u) \conj{\tau_{pp}(u)}
                    \dd u
            \end{align*}

            Using the above in~\eqref{eq:theorem:Littlewood-Paley_decomposition:computing_kernel},
            and substituing $\rho$ for $u^{-1} \rho$,
            we obtain:
            \begin{align}
                \kappa_l (x, k)
                = &\sum_{\JapaneseBracket{\CompactGroup}{\tau} \leq 2^j}
                        \sum_{p,q = 1}^\dimRep{\tau}
                        \dimRep{\tau}
                        \int_\dualGroup{\VectorSpace}
                                \int_\Lebesgue{2}{\CompactGroup} \notag\\
                                    &\chi_{l - \Ceiling{\log_2 \JapaneseBracket{\CompactGroup}{\tau}}}(u^{-1}\rho) \rho(x) \tau_{pq}(k^{-1}) \tau_{qp}(u) \conj{\tau_{pp}(u)}
                                \dd u
                            \dd \Plancherel{\VectorSpace}(\rho)
                    \label{eq:theorem:Littlewood-Paley_decomposition:computing_kernel:2}
            \end{align}

            Using the invariance of $\chi_{l - j}$ under $\CompactGroup$ and
            \begin{align*}
                \dimRep{\tau} \int_\Lebesgue{2}{\CompactGroup} \tau_{qp}(u) \conj{\tau_{pp}(u)} \dd u = \Kronecker{p}{q},
            \end{align*}
            then~\eqref{eq:theorem:Littlewood-Paley_decomposition:computing_kernel:2} becomes
            \begin{align*}
                \kappa_l (x, k)
                = &\sum_{\JapaneseBracket{\CompactGroup}{\tau} \leq 2^j}
                    \int_\dualGroup{\VectorSpace}
                        \chi_{l - \Ceiling{\log_2 \JapaneseBracket{\CompactGroup}{\tau}}}(\rho) \rho(x)
                    \dd \Plancherel{\VectorSpace}(\rho)
                    \conj{\Character{\tau}(k)}
            \end{align*}
            which, after recognising the inverse Fourier Transform on $\dualGroup{V}$,
            yields the following expression for the kernel:
            \begin{align}
                \kappa_l (x, k)
                = &\sum_{\JapaneseBracket{\CompactGroup}{\tau} \leq 2^j}
                    \InverseFourier[\VectorSpace]{\chi_{l - \Ceiling{\log_2 \JapaneseBracket{\CompactGroup}{\tau}}}}(x) \conj{\Character{\tau}(k)}.
                \label{eq:theorem:Littlewood-Paley_decomposition:kernel}
            \end{align}

        \item[Step 3] Uniform boundedness of $\DifferenceOperator{q} \eta_l$ in $\Lebesgue{2}{\CompactGroup}$

            Fix $q_1 \in \Polynomials{\VectorSpace}$ and $q_2 \in \Polynomials{\CompactGroup}$
            and write $q(x, k) = q_1(x) q_2(k)$.

            Multiplying $\kappa_l$ par $q$ and taking the Fourier Transform, we get:
            \begin{align*}
                \DifferenceOperator{q} \eta_l (\rho) F(u)
                = &\sum_{\JapaneseBracket{\CompactGroup}{\tau} \leq 2^j}
                    \int_\VectorSpace
                        \int_\CompactGroup
                            q_1(x) \InverseFourier[\VectorSpace]{\chi_{l - \Ceiling{\log_2 \JapaneseBracket{\CompactGroup}{\tau}}}}(x) (k u \rho)(-x)
                            q_2(k) \conj{\Character{\tau}(k)} F(k u)
                        \dd k
                    \dd x\\
                = &\sum_{\JapaneseBracket{\CompactGroup}{\tau} \leq 2^j}
                    \int_\CompactGroup
                        \DifferenceOperator[\VectorSpace]{q_1} \chi_{l - \Ceiling{\log_2 \JapaneseBracket{\CompactGroup}{\tau}}}(k u \rho)
                        q_2(k) \conj{\Character{\tau}(k)} F(k u)
                    \dd k,
            \end{align*}
            where the second line was obtained by integrating with respect to $x$.

            Substituing $k$ for $k u^{-1}$ in the above,
            and using the Leibniz rule for polynomials on $q_2$, we obtain
            \begin{align*}
                \DifferenceOperator{q} \eta_l (\rho) F(u)
                = &\sum_{\JapaneseBracket{\CompactGroup}{\tau} \leq 2^j}
                    \sum_{p = 1}^{C_q}
                        \int_\CompactGroup
                            \DifferenceOperator[\VectorSpace]{q_1} \chi_{l - \Ceiling{\log_2 \JapaneseBracket{\CompactGroup}{\tau}}}(k \rho)
                            q_{2, p}(k) {q'}_{2, p}(u^{-1}) \conj{\Character{\tau}(k u^{-1})} F(k)
                        \dd k,
            \end{align*}

            \begin{claim}
                We can find a decomposition
                \begin{align*}
                    \DifferenceOperator[\VectorSpace]{q_1} \chi_{l - \Ceiling{\log_2 \JapaneseBracket{\CompactGroup}{\tau}}}(\rho) = \sum_{r = 1}^{C_q} f_{l, r}(\tau) P_{l, r}(\rho),
                \end{align*}
                where both $P_{l, r}$ and $f_{l, r}$ are uniformly bounded in $l$ and $r$.
            \end{claim}

            Using the above claim,
            and the identity $\conj{\Character{\tau}(k u^{-1})} = \sum_{i, j = 1}^\dimRep{\tau} \tau_{ij}(u) \conj{{\tau_{ij}(k)}}$,
            we observe that:
            \begin{align*}
                \DifferenceOperator{q} \eta_l (\rho) F(u)
                = &\sum_{p, r = 1}^{C_q}
                    \sum_{\JapaneseBracket{\CompactGroup}{\tau} \leq 2^j}
                        \sum_{i, j = 1}^\dimRep{\tau}\\
                            &\tau_{ij}(u) {q'}_{2, p}(u^{-1})
                            f_{l, r}(\tau)
                            \int_\CompactGroup
                                P_{l, r}(k \rho) q_{2, p}(k) F(k) \conj{\tau_{ij}(k)}
                            \dd k\\
                = &\sum_{p, r = 1}^{C_q}
                    \sum_{\JapaneseBracket{\CompactGroup}{\tau} \leq 2^j}
                        \sum_{i, j = 1}^\dimRep{\tau}
                            \tau_{ij}(u) {q'}_{2, p}(u^{-1})
                            f_{l, r}(\tau)
                            \Fourier[\CompactGroup]{} {\left\{ P_{l, r}(\dummy \rho) q_{2, p} F\right\}}_{j i}.
            \end{align*}

            For $p, r = 1, \dots, \dimRep{\tau}$, defining the symbols
            \begin{align}
                \sigma_{l, p, r}(u, \tau) =
                \begin{cases}
                    \frac{1}{\dimRep{\tau}} {q'}_{2, p}(u^{-1}) f_{l, r}(\tau) \Id{\dimRep{\tau}} & \text{if } \JapaneseBracket{\CompactGroup}{\tau} \leq 2^l\\
                    0 & \text{otherwise}
                \end{cases}
            \end{align}
            and denoting by $T_{l, p, r}$ the corresponding operators,
            we see that in fact,
            \begin{align*}
                \DifferenceOperator{q} \eta_l (\rho) F(u)
                = &\sum_{p, r = 1}^{C_q}
                    \sum_{\tau \in \dualGroup{\CompactGroup}}
                        \dimRep{\tau}
                        \tr \left(
                            \tau(u)
                            \sigma_{l, p, r}(u, \tau)
                            \Fourier[\CompactGroup]{} \left\{ P_{l, r}(\dummy \rho) q_{2, p} F\right\}
                        \right)\\
                = &\sum_{p, r = 1}^{C_q}
                        T_{p, r} (P_r(\dummy \rho) q_{2, p} F)(u).
            \end{align*}

            Since the symbols $\sigma_{l, p, r}$ are uniformly bounded,
            it follows that the operators $T_{l, p, r}$ are uniformly bounded in $\Lebesgue{2}{\CompactGroup}$.
            From there, it follows that
            \begin{align}
                \norm[\Lebesgue{2}{\CompactGroup}]{\DifferenceOperator{q} \eta_l (\rho) F}
                \leq C_q \norm[\Lebesgue{2}{\CompactGroup}]{F},
            \end{align}
            where $C_q$ does not depend on $l$.

        \item[Step 4] $\DifferenceOperator{q} \eta_l(\rho)$ is non-zero on $V_\tau$
            only if $\norm[\dualGroup{V}]{\rho}, \JapaneseBracket{\CompactGroup}{\tau} \leq C_q 2^l$.

        \item[Step 5] Conclusion.

            Let
            \begin{align*}
                L_l(\rho) =
                \begin{cases}
                    \bigoplus_{\JapaneseBracket{\CompactGroup}{\tau}
                    \leq C_q 2^l} {\left(\norm[\dualGroup{\VectorSpace}]{\rho}^2 + {\JapaneseBracket{\CompactGroup}{\tau}}^2 \right)}^\frac{1}{2} \Id{V_\tau}
                    & \text{if } \norm[\dualGroup{V}]{\rho} \leq C_q 2^l\\
                    0 & \text{otherwise}.
                \end{cases}
            \end{align*}

            Observe that $L_l(\rho)$ is a bounded operator in $\Lebesgue{2}{K}$,
            and the operator norm is bounded by $C_q 2^l$ uniformly in $\rho$.

            Therefore, the operator
            \begin{align*}
                \Rep{\rho} \BesselPotential{- m - \order(q) + \gamma}&
                \DifferenceOperator{q} \eta_l(\rho)
                \Rep{\rho} \BesselPotential{-\gamma}\\
                &= {L_l(\rho)}^{-m - \order(q) + \gamma}
                \DifferenceOperator{q} \eta_l(\rho)
                {L_l(\rho)}^{-\gamma}
            \end{align*}
            is a bounded operator on $\Lebesgue{2}{\CompactGroup}$ with norm less than $C_q 2^{-l m}$.
    \end{description}
\end{proof}
