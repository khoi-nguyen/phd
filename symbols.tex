\chapter{Symbols}

\section{Difference operators}

\section{Symbols and Kohn-Nirenberg quantization}

\section{Symbol classes}

\section{Link with the Hormander classes}

\section{Littlewood-Paley decomposition}

\begin{lemma}
\label{lemma:derivatives_of_radial_functions}
    Let $\alpha \in \N^n$,
    and fix a radial function $\chi \in \SmoothFunctions{\R^n}$.
    If $\alpha \in \N^n$, then
    \begin{align}
        \D{\chi}{x^\alpha}(x)
        = \sum_{r = 1}^{C_\alpha} f_r(\norm[\R^n]{x}) P_r(x),
    \end{align}
    where $P_r$ is a polynomial depending only on $\alpha$.

    Moreover, if $\supp \chi$ is compact
    and if there exists $\delta > 0$ such that $\D{\chi}{\lambda} = 0$ on $\Ball[\R^n]{0}{\delta}$,
    then we have
    \begin{align*}
        \sup_r \sup_{\lambda \in \R^+} \abs{f_r} < \infty
    \end{align*}
\end{lemma}
\begin{proof}
    Using the chain rule, we know that for a purely radial function $f$
    \begin{align}
        \D{f}{x_i} = \D{\lambda}{x_i} \D{f}{\lambda} = \frac{\D{f}{\lambda}}{\norm[\R^n]{x}} x_i.
    \end{align}

    We know proceed to show the claim by induction on $\alpha$.
    The result is clearly true when $\abs{\alpha} = 0$.
    If we assume it is true for some $\alpha \in \N^n$, then by the above,
    \begin{align}
        \D{\chi}{x_i,x^\alpha}(x)
        &= \D{}{x_i} \sum_{r = 1}^{C_\alpha} f_r(\norm[\R^n]{x}) P_r(x)\\
        &= \sum_{r = 1}^{C_\alpha} \frac{\D{f_r}{\lambda}}{\norm[\R^n]{x}}(\norm[\R^n]{x}) x_i P_r(x)
        + \sum_{r = 1}^{C_\alpha} f_r(\norm[\R^n]{x}) \D{P_r}{x_i}(x),
    \end{align}
    which concludes the proof.
\end{proof}

\begin{lemma}
\label{lemma:left_regular_representation_of_polynomials}
    Let $P \in \SmoothFunctions{\dualGroup{\VectorSpace}}$ be a polynomial.
    We can find functions $q_i \in \Polynomials{\CompactGroup}$, $f_i \in \Lebesgue{2}{\dualGroup{\VectorSpace}}$, $i = 1, \dots, N$ such that
    \begin{align*}
        P(k \lambda) = \sum_{i = 1}^N q_i(k) f_i(\lambda)
    \end{align*}
    for each $k \in \CompactGroup$ and each $\lambda \in \dualGroup{\VectorSpace}$.

    Moreover, the $q_i$ satisfy the bound
    \begin{align*}
        \sup_i \sup_\CompactGroup \abs{q_i} < \infty.
    \end{align*}
\end{lemma}
\begin{proof}
    Let $\mathcal{H}$ be the set of polynomials of order at most $m$ on $\dualGroup{\VectorSpace}$,
    with the norm
    \begin{align}
        {\left( \int_\dualGroup{\VectorSpace} \abs{P}^2 \right)}^\frac{1}{2}, \quad P \in \mathcal{H}.
    \end{align}

    Consider $L_m$ the left-regular representaion of $\CompactGroup$ on the above Hilbert space.
    Decomposing $L_m$ into irreducible representations
    \begin{align}
        \mathcal{H} = \bigoplus_{i = 1}^N \mathcal{H}_i,
        \quad L_m = \bigoplus_{i = 1}^N \left. L_m \right|_{\mathcal{H}_i},
    \end{align}
    then $P$ can be written as
    \begin{align*}
        P = \sum_{i = 1}^N \sum_{q = 1}^{d_i} c_{i, q} e_{i, q},
    \end{align*}
    and if $\tau_{i, pq}(k) = \int_\dualGroup{\VectorSpace} L_m(k) e_{i, q} \conj{e_{i, p}}$, then
    \begin{align*}
        P(k \lambda) &= L_m(k^{-1}) P(\lambda)
        = \sum_{i = 1}^N \sum_{q = 1}^{d_i} c_{i, q} L_m(k^{-1}) e_{i, q}(\lambda)\\
        &= \sum_{i = 1}^N \sum_{q = 1}^{d_i} c_{i, q} \tau_{i, pq}(k^{-1}) e_{i, p}(\lambda),
    \end{align*}
    which concludes the proof.
\end{proof}

\begin{theorem}[Littlewood-Paley decomposition]
\label{theorem:Littlewood-Paley_decomposition}
\index{Littlewood-Paley decomposition}
    There exists a sequence $\eta_l \in \SmoothSymbols$, $l \in \N$ of smoothing symbols satisfying the following properties
    \begin{enumerate}
        \item the semi-norms decay in the following way
            \begin{align}
                \SymbolSemiNorm{m}{\rho, \delta}{\eta_l} \leq C 2^{-lm}
            \end{align}
        \item the associated kernels $\kappa_l$ satisfy
            \begin{align*}
                \sum_{l = 0}^\infty \kappa_l = \DiracDelta{e_\Group}
            \end{align*}
            in the sense of distributions.
    \end{enumerate}
\end{theorem}
\begin{proof}
    \begin{description}
        \item[Step 1] Constructing the dyadic decomposition.

            First, let us find a smooth function $\chi_0 \in \SmoothFunctions{\dualGroup{\VectorSpace}}$ invariant under $\CompactGroup$ such that
            \begin{align*}
                \chi_0(\lambda) = 1 \  \text{if}\  \norm[\dualGroup{\VectorSpace}]{\lambda} \leq 1, \quad \text{and} \quad
                \chi_0(\lambda) \equiv 0 \ \text{if}\  \norm[\dualGroup{\VectorSpace}]{\lambda} \geq 2.
            \end{align*}

            Then, for each $l \in \N$ satisfying $l \leq 1$, let
            \begin{align*}
                \chi_l = \chi_0(2^{-l} \dummy) - \chi_0(2^{-l + 1} \dummy).
            \end{align*}
            so that $\supp \chi_l \subset \Ball{0}{2^{l + 1}}$.

            In particular, it should be clear that
            \begin{align*}
                \sum_{l = 0}^N \chi_l = \chi_0(2^{-N} \dummy)
            \end{align*}
            so that in fact
            \begin{align}
                \sum_{l = 0}^\infty \chi_l = 1.
                \label{eq:theorem:Littlewood-Paley_decomposition:partition_of_unity}
            \end{align}

            Fix $l \in \N$.
            We define our symbol $\eta_l$ as follows.
            For each $\tau \in \dualGroup{\CompactGroup}$, we let
            \begin{align*}
                \eta_l(\lambda)
                = \sum_{\JapaneseBracket{\CompactGroup}{\tau} \leq 2^l}
                \chi_{l - \Ceiling{\log_2 \JapaneseBracket{\CompactGroup}{\tau}}}(\lambda) \Id{V_\tau},
            \end{align*}
            where $V_\tau = \Span \{ \tau_{ij} : i, j = 1, \dots, \dimRep{\tau} \}$.

            Note that since $\supp \chi_l \subset \Ball{0}{2^{l + 1}}$,
            we get that
            \begin{align}
                \eta_l(\lambda) = 0 \quad \text{if } \norm[\dualGroup{\CompactGroup}]{v} \geq 2^{l + 1}
                \label{eq:theorem:Littlewood-Paley_decomposition:cancellation_condition}
            \end{align}

            We check that
            \begin{align*}
                \sum_{l = 0}^\infty \eta_l
                &= \sum_{l = 0}^\infty
                    \sum_{\JapaneseBracket{\CompactGroup}{\tau} \leq 2^l}
                        \chi_{l - \Ceiling{\log_2 \JapaneseBracket{\CompactGroup}{\tau}}} \Id{V_\tau}\\
                &= \sum_{\tau \in \dualGroup{\CompactGroup}}
                    \sum_{l = \Ceiling{\log_2 \JapaneseBracket{\CompactGroup}{\tau}}}^\infty
                        \chi_{l - \Ceiling{\log_2 \JapaneseBracket{\CompactGroup}{\tau}}} \Id{V_\tau},
            \end{align*}
            where the last line was obtained by commuting the two sums.

            Substituing $l$ for $l + \Ceiling{\log_2 \JapaneseBracket{\CompactGroup}{\tau}}$ in the inner sum,
            the above becomes
            \begin{align*}
                \sum_{l = 0}^\infty \eta_l
                = \sum_{\tau \in \dualGroup{\CompactGroup}}
                    \sum_{l = 0}^\infty
                        \chi_l \Id{V_\tau}
                = \sum_{\tau \in \dualGroup{\CompactGroup}}
                    \Id{V_\tau}
                = \Id{\Lebesgue{2}{\CompactGroup}},
            \end{align*}
            where the second to last inequality was obtained from~\eqref{eq:theorem:Littlewood-Paley_decomposition:partition_of_unity}.

        \item[Step 2] Computing the associated kernels $\kappa_l$.

            By applying the inverse Fourier Transform (Proposition~\ref{proposition:inverse_Fourier_Transform})
            we obtain that the kernel is given by
            \begin{align}
                \kappa_l(x, k)
                = \sum_{\JapaneseBracket{\CompactGroup}{\tau} \leq 2^l}
                    \int_\dualGroup{\VectorSpace}
                        \chi_{l - \Ceiling{\log_2 \JapaneseBracket{\CompactGroup}{\tau}}}(\lambda) \tr( \left. \Rep{\lambda}(x, k) \right|_{V_\tau} )
                    \dd \Plancherel{\VectorSpace}(\lambda)
                \label{eq:theorem:Littlewood-Paley_decomposition:computing_kernel}
            \end{align}

            By the Peter-Weyl Theorem,
            $\{ \sqrt{\dimRep{\tau}} \tau_{pq} : p, q = 1, \dots, \dimRep{\tau} \}$
            is an orthonormal basis of $V_\tau$,
            allowing us to compute the trace as
            \begin{align*}
                \tr( \left. \Rep{\lambda}(x, k) \right|_{V_\tau})
                &= \sum_{p = 1}^\dimRep{\tau}
                    \dimRep{\tau}
                    \int_\Lebesgue{2}{\CompactGroup}
                    (u \lambda)(x) \tau_{pp}(k^{-1} u) \conj{\tau_{pp}(u)}
                    \dd u\\
                &= \sum_{p,q = 1}^\dimRep{\tau}
                    \dimRep{\tau}
                    \int_\Lebesgue{2}{\CompactGroup}
                        (u \lambda)(x) \tau_{pq}(k^{-1}) \tau_{q p}(u) \conj{\tau_{pp}(u)}
                    \dd u.
            \end{align*}

            Using the above in~\eqref{eq:theorem:Littlewood-Paley_decomposition:computing_kernel},
            and substituing $\lambda$ for $u^{-1} \lambda$,
            we obtain
            \begin{align}
                \kappa_l (x, k)
                = &\sum_{\JapaneseBracket{\CompactGroup}{\tau} \leq 2^l}
                        \sum_{p,q = 1}^\dimRep{\tau}
                        \dimRep{\tau}
                        \int_\dualGroup{\VectorSpace}
                                \int_\Lebesgue{2}{\CompactGroup} \notag\\
                                    &\chi_{l - \Ceiling{\log_2 \JapaneseBracket{\CompactGroup}{\tau}}}(u^{-1}\lambda) \lambda(x) \tau_{pq}(k^{-1}) \tau_{qp}(u) \conj{\tau_{pp}(u)}
                                \dd u
                            \dd \Plancherel{\VectorSpace}(\lambda)
                    \label{eq:theorem:Littlewood-Paley_decomposition:computing_kernel:2}
            \end{align}

            Using the invariance of $\chi_{k}$ under $\CompactGroup$ and
            \begin{align*}
                \dimRep{\tau} \int_\Lebesgue{2}{\CompactGroup} \tau_{qp}(u) \conj{\tau_{pp}(u)} \dd u = \Kronecker{p}{q},
            \end{align*}
            then~\eqref{eq:theorem:Littlewood-Paley_decomposition:computing_kernel:2} becomes
            \begin{align*}
                \kappa_l (x, k)
                = &\sum_{\JapaneseBracket{\CompactGroup}{\tau} \leq 2^l}
                    \int_\dualGroup{\VectorSpace}
                        \chi_{l - \Ceiling{\log_2 \JapaneseBracket{\CompactGroup}{\tau}}}(\lambda) \lambda(x)
                    \dd \Plancherel{\VectorSpace}(\lambda)
                    \conj{\Character{\tau}(k)}
            \end{align*}
            which, after recognising the inverse Fourier Transform on $\dualGroup{\VectorSpace}$,
            yields the following expression for the kernel
            \begin{align}
                \kappa_l (x, k)
                = &\sum_{\JapaneseBracket{\CompactGroup}{\tau} \leq 2^l}
                    \InverseFourier[\VectorSpace]{\chi_{l - \Ceiling{\log_2 \JapaneseBracket{\CompactGroup}{\tau}}}}(x) \conj{\Character{\tau}(k)}.
                \label{eq:theorem:Littlewood-Paley_decomposition:kernel}
            \end{align}

        \item[Step 3] We show that for every $q \in \Polynomials{\Group}$
            and every $\lambda \in \dualGroup{\VectorSpace}$,
            \begin{align}
                \sup_{l \in \N} \norm[\Lin{\Lebesgue{2}{\CompactGroup}}]{\DifferenceOperator{q} \eta_l(\lambda)} < \infty.
            \end{align}

            Fix $q_1 \in \Polynomials{\VectorSpace}$, $q_2 \in \Polynomials{\CompactGroup}$
            and write $q(x, k) = q_1(x) q_2(k)$.
            We also choose an arbitrary function $F \in \Lebesgue{2}{\CompactGroup}$,
            and an element $u \in \CompactGroup$.

            Informally, the idea behind the proof of this step is the following
            if we can write
            \begin{align*}
                \DifferenceOperator{q} \eta_l(\lambda) F(u)
                = \sum_{\tau \in \dualGroup{\CompactGroup}}
                    \dimRep{\tau}
                    \tr\left( \tau(u) \sigma_{l, \lambda, q, \lambda}(u, \tau) \Fourier[\CompactGroup] F(\tau) \right),
            \end{align*}
            then a bound on the operator norm of $\DifferenceOperator{q} \eta_l (\lambda)$ can be obtained by finding an appropriate bound on $\sigma_{l, q, \lambda}$.
            Looking at~\eqref{eq:theorem:Littlewood-Paley_decomposition:kernel},
            we can see the latter is the right approach
            as we have a sum on $\dualGroup{\CompactGroup}$ already.

            Multiplying $\kappa_l$ par $q$ and taking the Fourier Transform, we get
            \begin{align}
                \DifferenceOperator{q} \eta_l (\lambda) F(u)
                = &\sum_{\JapaneseBracket{\CompactGroup}{\tau} \leq 2^l}
                    \int_\VectorSpace
                        \int_\CompactGroup
                            q_1(x) \InverseFourier[\VectorSpace]{\chi_{l - \Ceiling{\log_2 \JapaneseBracket{\CompactGroup}{\tau}}}}(x) (k u \lambda)(-x)
                            q_2(k) \conj{\Character{\tau}(k)} F(k u)
                        \dd k
                    \dd x\notag\\
                = &\sum_{\JapaneseBracket{\CompactGroup}{\tau} \leq 2^l}
                    \int_\CompactGroup
                        \DifferenceOperator[\VectorSpace]{q_1} \chi_{l - \Ceiling{\log_2 \JapaneseBracket{\CompactGroup}{\tau}}}(k u \lambda)
                        q_2(k) \conj{\Character{\tau}(k)} F(k u)
                    \dd k \label{eq:theorem:Littlewood-Paley_decomposition:rho_condition},
            \end{align}
            where the second line was obtained by integrating with respect to $x$.

            Substituing $k$ for $k u^{-1}$ in the above,
            and using the Leibniz rule for polynomials on $q_2$, we obtain
            \begin{align*}
                \DifferenceOperator{q} \eta_l (\lambda) F(u)
                = &\sum_{\JapaneseBracket{\CompactGroup}{\tau} \leq 2^l}
                    \sum_{p = 1}^{C_q}
                        \int_\CompactGroup
                            \DifferenceOperator[\VectorSpace]{q_1} \chi_{l - \Ceiling{\log_2 \JapaneseBracket{\CompactGroup}{\tau}}}(k \lambda)
                            q_{2, p}(k) {q'}_{2, p}(u^{-1}) \conj{\Character{\tau}(k u^{-1})} F(k)
                        \dd k,
            \end{align*}

            \begin{claim}
                We have the decomposition
                \begin{align*}
                    \DifferenceOperator[\VectorSpace]{q_1} \chi_{l - \Ceiling{\log_2 \JapaneseBracket{\CompactGroup}{\tau}}}(k \lambda) = \sum_{r = 1}^{C_q} f_{l, r}(\tau, \lambda) q_r(k),
                \end{align*}
                where $q_r$ and $f_{l, r}$ satisfies the following bound
                \begin{align}
                    \sup_{l \in \N} \sup_{\tau \in \dualGroup{\CompactGroup}} \sup_{\lambda \in \dualGroup{\VectorSpace}} \abs{f_{l, r}(\tau, \lambda)} < \infty,\quad
                    \sup_{r} \sup_{k \in \CompactGroup} \abs{q(k)} < \infty
                    \label{eq:theorem:Littlewood-Paley_decomposition:claim_bound}
                \end{align}
            \end{claim}
            \begin{proof}[Proof of the claim]
                Assume first that $l - \Ceiling{\log_2 \JapaneseBracket{\CompactGroup}{\tau}} \neq 0$.
                Since
                \begin{align*}
                    \chi_{l - \Ceiling{\log_2 \JapaneseBracket{\CompactGroup}{\tau}}}(\lambda)
                    = \chi_1(2^{-l + \Ceiling{\log_2 \JapaneseBracket{\CompactGroup}{\tau} + 1}} \lambda),
                \end{align*}
                it follows that
                \begin{align*}
                    \DifferenceOperator[\VectorSpace]{q_1} \chi_{l - \Ceiling{\log_2 \JapaneseBracket{\CompactGroup}{\tau}}}(k \lambda)
                    =
                    2^{(-l + \Ceiling{\log_2 \JapaneseBracket{\CompactGroup}{\tau} + 1}) \order{q_1}}
                    \DifferenceOperator[\VectorSpace]{q_1} \chi_1(2^{-l + \Ceiling{\log_2 \JapaneseBracket{\CompactGroup}{\tau} + 1}} k \lambda).
                \end{align*}

                Using Lemma~\ref{lemma:derivatives_of_radial_functions} and Lemma~\ref{lemma:left_regular_representation_of_polynomials}
                we obtain
                \begin{align*}
                    \DifferenceOperator[\VectorSpace]{q_1} &\chi_{l - \Ceiling{\log_2 \JapaneseBracket{\CompactGroup}{\tau}}}(k \lambda) =
                    2^{(-l + \Ceiling{\log_2 \JapaneseBracket{\CompactGroup}{\tau} + 1}) \order{q_1}}\\
                    &\sum_{r = 1}^{C_q} f_r(2^{-l + \Ceiling{\log_2 \JapaneseBracket{\CompactGroup}{\tau} + 1}} \norm[\dualGroup{\VectorSpace}]{\lambda}) q_r(k) P_r(2^{-l + \Ceiling{\log_2 \JapaneseBracket{\CompactGroup}{\tau} + 1}} \lambda),
                \end{align*}
                where each $f_r$, $q_r$ and each $P_r$ is independent of $l$ and $\tau$.
                Now, writing
                \begin{align*}
                    f_{l, r}(\tau, \lambda) &=
                    2^{(-l + \Ceiling{\log_2 \JapaneseBracket{\CompactGroup}{\tau} + 1}) \order{q_1}}
                    f_r(2^{-l + \Ceiling{\log_2 \JapaneseBracket{\CompactGroup}{\tau} + 1}} \norm[\dualGroup{\VectorSpace}]{\lambda}) P_r(2^{-l + \Ceiling{\log_2 \JapaneseBracket{\CompactGroup}{\tau} + 1}} \lambda),
                \end{align*}
                we obtain the desired formula.
                The bound comes from the bounds in Lemma~\ref{lemma:derivatives_of_radial_functions} and~\ref{lemma:left_regular_representation_of_polynomials}.

                The case $l - \Ceiling{\log_2 \JapaneseBracket{\CompactGroup}{\tau}} = 0$ can be treated similarly.
            \end{proof}

            Using the above claim,
            and the identity $\conj{\Character{\tau}(k u^{-1})} = \sum_{i, j = 1}^\dimRep{\tau} \tau_{ij}(u) \conj{{\tau_{ij}(k)}}$,
            we observe that
            \begin{align}
                \DifferenceOperator{q} \eta_l (\lambda) F(u)
                = &\sum_{p, r = 1}^{C_q}
                    \sum_{\JapaneseBracket{\CompactGroup}{\tau} \leq 2^l}
                        \sum_{i, j = 1}^\dimRep{\tau}\notag\\
                            &\tau_{ij}(u) {q'}_{2, p}(u^{-1})
                            f_{l, r}(\tau, \lambda)
                            \int_\CompactGroup
                                q_{r}(k) q_{2, p}(k) F(k) \conj{\tau_{ij}(k)}
                            \dd k
                            \label{eq:theorem:Littlewood-Paley_decomposition:exact_expression_for_operator}\\
                = &\sum_{p, r = 1}^{C_q}
                    \sum_{\JapaneseBracket{\CompactGroup}{\tau} \leq 2^l}
                        \sum_{i, j = 1}^\dimRep{\tau}
                            \tau_{ij}(u) {q'}_{2, p}(u^{-1})
                            f_{l, r}(\tau, \lambda)
                            \Fourier[\CompactGroup]{} {\left\{ q_{r} q_{2, p} F\right\}}_{j i}(\tau).\notag
            \end{align}

            For $p, r = 1, \dots, \dimRep{\tau}$, defining the symbols
            \begin{align}
                \sigma_{l, \lambda, p, r}(u, \tau) =
                \begin{cases}
                    \frac{1}{\dimRep{\tau}} {q'}_{2, p}(u^{-1}) f_{l, r}(\tau, \lambda) \Id{\dimRep{\tau}} & \text{if } \JapaneseBracket{\CompactGroup}{\tau} \leq 2^l\\
                    0 & \text{otherwise}
                \end{cases}
            \end{align}
            and denoting by $T_{l, \lambda, p, r}$ the corresponding operators,
            we see that in fact,
            \begin{align*}
                \DifferenceOperator{q} \eta_l (\lambda) F(u)
                = &\sum_{p, r = 1}^{C_q}
                    \sum_{\tau \in \dualGroup{\CompactGroup}}
                        \dimRep{\tau}
                        \tr \left(
                            \tau(u)
                            \sigma_{l, \lambda, p, r}(u, \tau)
                            \Fourier[\CompactGroup]{} \left\{ q_r q_{2, p} F\right\}(\tau)
                        \right)\\
                = &\sum_{p, r = 1}^{C_q}
                        T_{l, \lambda, p, r} (q_r q_{2, p} F)(u).
            \end{align*}

            By~\eqref{eq:theorem:Littlewood-Paley_decomposition:claim_bound},
            we obtain
            \begin{align*}
                \norm[\Lin{\Lebesgue{2}{\CompactGroup}}]{T_{l, \lambda, p, r}}
                &\leq C \sup_{\tau \in \dualGroup{\CompactGroup}} \sup_{u \in \CompactGroup} \norm[\Lin{\HilbertRep{\tau}}]{\sigma_{l, \lambda, p, r}(u, \tau)}\\
                &\leq C_q < \infty
            \end{align*}
            From there, it follows that
            \begin{align*}
                \norm[\Lebesgue{2}{\CompactGroup}]{\DifferenceOperator{q} \eta_l (\lambda) F}
                &\leq C_q \sum_{p, r = 1}^{C_q} \norm[\Lebesgue{2}{\CompactGroup}]{q_r q_{2, p} F}
            \end{align*}
            where $C_q$ does not depend on $l$.

            Now, using the fact that
            \begin{align*}
                \sup_\CompactGroup \abs{q_r q_{2, p}} \leq C_q < \infty
            \end{align*}
            in the above, this concludes the step.

        \item[Step 4] $\ip[\Lebesgue{2}{\CompactGroup}]{\DifferenceOperator{q} \eta_l(\lambda) \mu_{mn}}{\nu_{kl}}$ is non-zero
            only if $\norm[\dualGroup{\VectorSpace}]{\lambda}, \JapaneseBracket{\CompactGroup}{\mu}, \JapaneseBracket{\CompactGroup}{\nu} \leq C_q 2^l$.

            Choose $C_q \geq 2$ so that
            \begin{align*}
                q_r q_{2, p}, q'_{2, p}(\dummy^{-1})
            \end{align*}
            can be generated by the representations $\{ \tau \in \dualGroup{\CompactGroup} : \JapaneseBracket{\CompactGroup}{\tau} \leq \frac{C_q}{2} \}$.

            Suppose now that $\max\{\JapaneseBracket{\CompactGroup}{\mu}, \JapaneseBracket{\CompactGroup}{\nu}\} > C_q 2^l$.
            It follows from our choice of $C_q$ that if $\JapaneseBracket{\CompactGroup}{\tau} \leq 2^l$,
            either of the following equations hold
            \begin{align*}
                \int_\CompactGroup q_r(k) q_{2, p}(k) \mu_{mn}(k) \conj{\tau_{ij}(k)} \dd k &= 0\\
                \int_\CompactGroup \tau_{ij}(u) q'_{2, p}(u) \conj{\nu_{mn}(k)} \dd k &= 0.
            \end{align*}

            From~\eqref{eq:theorem:Littlewood-Paley_decomposition:exact_expression_for_operator} with $F = \mu_{mn}$,
            we can see that the above implies
            \begin{align*}
                \ip[\Lebesgue{2}{\CompactGroup}]{\DifferenceOperator{q} \eta_l(\lambda) \mu_{mn}}{\nu_{kl}} = 0.
            \end{align*}

            The condition on $\lambda$ is obvious by~\eqref{eq:theorem:Littlewood-Paley_decomposition:rho_condition}
            keeping in mind that $\supp \chi_k \subset \Ball[\dualGroup{\VectorSpace}]{0}{2^{k + 1}}$.

        \item[Step 5] Conclusion.

            Let
            \begin{align*}
                L_l(\lambda) =
                \begin{cases}
                    {\left. \Rep{\lambda} \BesselPotential{1} \right|}_{\oplus_{\JapaneseBracket{\CompactGroup}{\tau} \leq C_q 2^l} V_\tau}
                    & \text{if } \norm[\dualGroup{\VectorSpace}]{\lambda} \leq C_q 2^l\\
                    0 & \text{otherwise},
                \end{cases}
            \end{align*}
            where $C_q$ is given by Step 4.

            Observe that $L_l(\lambda)$ is a bounded operator in $\Lebesgue{2}{\CompactGroup}$,
            and the operator norm is bounded by $C_q 2^l$ uniformly in $\lambda$.

            By Step 4,
            \begin{align*}
                \Rep{\lambda} \BesselPotential{- m - \order(q) + \gamma}&
                \DifferenceOperator{q} \eta_l(\lambda)
                \Rep{\lambda} \BesselPotential{-\gamma}\\
                &= {L_l(\lambda)}^{-m - \order(q) + \gamma}
                \DifferenceOperator{q} \eta_l(\lambda)
                {L_l(\lambda)}^{-\gamma},
            \end{align*}
            from which it follows that
            \begin{align*}
                \norm[\Lin{\Lebesgue{2}{\CompactGroup}}]{\Rep{\lambda} \BesselPotential{- m - \order(q) + \gamma} \DifferenceOperator{q} \eta_l(\lambda) \Rep{\lambda} \BesselPotential{-\gamma}}
                \leq C_q 2^{-l m},
            \end{align*}
            which is what we wanted to show.
    \end{description}
\end{proof}

\section{Kernel estimates}

\section{Adjoint and composition formulas}

\section{$L^2$ boundedness}
