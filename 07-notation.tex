\chapter*{Notation}

The author of this document wishes to stress upon the fact
that he did try very hard to get used to the English notation,
but using \emph{non-negative} and \emph{non-positive}
is the straw that broke the camel's back.

Therefore,
as controversial as it may be,
we shall follow the \emph{French} usage according to which
$0$ is both \emph{positive} and \emph{negative},
while any real number is both \emph{less than} and \emph{greater than} itself.

\begin{aquote}{Laurent Schwartz}
    Notons que nous rompons ici avec l'usage ant\'erieurement acquis
    en appelant inf\'erieur ce qu'on appelait inf\'erieur ou \'egal,
    et strictement inf\'erieur ce qu'on appelait inf\'erieur.
    La raison d'\^etre de ces changements,
    pleinement justifi\'es par la suite,
    est que la notion la plus g\'en\'eralement utilis\'ee est $\leq$,
    et qu'il est bon qu'elle ait l'appelation la plus courte.
    On devra toujours utiliser $\leq$ plut\^ot que $<$,
    toutes les fois que cela sera possible;
    quand on \'ecrira une in\'egalit\'e stricte avec $<$,
    ce sera pour avertir le lecteur qu'il y a un point d\'elicat,
    et que l'in\'egalit\'e large $\leq$ ne conviendrait pas.
\end{aquote}
