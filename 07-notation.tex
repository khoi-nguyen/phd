\chapter*{Notation}

\section*{Sets and symbols}

\begin{itemize}
    \item $\R$, $\R^+$ and $\R^-$ denote the sets of all, positive (including $0$) and negative (including $0$) real numbers respectively.
    \item $\Z$ denotes the sets of all integers,
        while $\N$ contains all positive integers including $0$.
    \item $\Kronecker i j$ is the \emph{Kronecker delta};
        if is equal to $1$ if $i = j$,
        and $0$ otherwise.
    \item $e_\Group$ is the identify element of a group $\Group$.
    \item $\delta_g$ is the desta-distribution at $g \in \Group$.
    \item $\Lebesgue p \Group$ will denote the set of all $\mu$-measurable complex function $f$ on $\Group$ such that $\abs f^p$ is $\mu$-integrable;
        here, $\mu$ is a Haar measure on $\Group$.
\end{itemize}

\section*{Positive and negative}

In this document, we shall use the following conventions.

\begin{itemize}
    \item $0$ is both \emph{``positive''} and \emph{``negative''}.
    \item $\N, \R^+, \R^-$ contain $0$.
    \item Any number is both \emph{``greater than''} and \emph{``less than''} itself.
    \item We shall use strict inequalities only
        when the corresponding non-strict inequalities are not suitable.
\end{itemize}

We shall do this to ensure that
when a strict inequality is necessary (i.e. $< +\infty$ or $> 0$),
it gets the emphasis it deserves.
This is why we shall always insist on writing \emph{``strictly positive''} or \emph{``strictly greater than''} when we require the strict inequality.
To make it less verbose,
we give the non-strict meaning to \emph{``greater or less than''}.

Note that the above conventions originated from Laurent Schwartz's \citetitle{Schwartz1981} (\cite{Schwartz1981})
and have now become standard in Francophone literature.

\begin{aquote}{Laurent Schwartz, \cite[p.~17]{Schwartz1981}}
    Notons que nous rompons ici avec l'usage ant\'erieurement acquis
    en appelant inf\'erieur ce qu'on appelait inf\'erieur ou \'egal,
    et strictement inf\'erieur ce qu'on appelait inf\'erieur.
    La raison d'\^etre de ces changements,
    pleinement justifi\'es par la suite,
    est que la notion la plus g\'en\'eralement utilis\'ee est $\leq$,
    et qu'il est bon qu'elle ait l'appelation la plus courte.
    On devra toujours utiliser $\leq$ plut\^ot que $<$,
    toutes les fois que cela sera possible;
    quand on \'ecrira une in\'egalit\'e stricte avec $<$,
    ce sera pour avertir le lecteur qu'il y a un point d\'elicat,
    et que l'in\'egalit\'e large $\leq$ ne conviendrait pas.
    \footnote{%
        \emph{Translation:}
        Let us observe that we break from traditional usage
        by saying ``less than'' instead of ``less than or equal to'',
        and ``strictly less than'' instead of ``less than''.
        The reason for these changes,
        completely justified hereafter,
        is that $\leq$ is the more widely used notion
        and should therefore have the shortest name.
        We shall always use $\leq$ rather than $<$
        every time we can;
        when we use the strict inequality with $<$,
        it will be to warn the reader that there is a subtlety
        and that the non-strict inequality is not suitable.
    }
\end{aquote}

Naturally, the same terminology applies to operators and matrices.
The operator $0$ is \emph{``positive definite''} but not \emph{``strictly positive definite''}.
