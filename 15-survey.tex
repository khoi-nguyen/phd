\chapter{Survey}

\section{The compact case}

In this section,
$\CompactGroup$ denotes a \emph{connected, compact Lie group}.
We equip $\CompactGroup$ with the unique normalized bi-invariant Riemannian metric,
and denote the corresponding Laplace-Beltrami operator by $\Laplacian [\CompactGroup]$.

The description of the unitary dual is given by the following \emph{Peter-Weyl theorem}.

\begin{theorem}[Peter-Weyl theorem]
\label{theorem:Peter-Weyl_theorem}
    Let $\CompactGroup$ be a compact topological group.
    The unitary dual $\dualGroup \CompactGroup$ of $\Group$ is discrete and
    the set
    \begin{align*}
        \left\{
            \sqrt{\dimRep\tau} \tau_{ij} : \tau \in \dualGroup\CompactGroup,\ i, j = 1, \dots, \dimRep\tau
        \right\}
    \end{align*}
    is an orthonormal basis of $\Lebesgue{2}{\CompactGroup}$ with respect to the \emph{normalised} Haar measure.
\end{theorem}

Since we have
\begin{align*}
    \eval \RightRegularRepresentation {\Span \{\tau_{i j} : j = 1, \dots, \dimRep \tau\}} \simeq \tau,
    \qquad i = 1, \dots, \dimRep \tau,
\end{align*}
the Peter-Weyl theorem gives the following decomposition into irreducible components
\begin{align*}
    \RightRegularRepresentation \simeq \bigoplus_{\tau \in \dualGroup \CompactGroup} \dimRep \tau \tau.
\end{align*}

This allows us to write the entire Fourier analysis with $\RightRegularRepresentation$ as the only representation,
and we shall do so henceforth.
We choose to do so as it will allow us to draw useful comparisons with the representations of the motion group.

It can be shown that the spaces
\begin{align*}
    \Span \{ \tau_{i j} : 1 \leq i, j \leq \dimRep \tau\}, \qquad \tau \in \dualGroup \CompactGroup
\end{align*}
are invariant under both $\RightRegularRepresentation$ and $\LeftRegularRepresentation$,
and are thus eigenspaces of $\Laplacian [\CompactGroup]$.

\subsection{Fourier analysis}

Given $f \in \Lebesgue 1 \CompactGroup$,
we define its \emph{Fourier transform} via
\begin{align*}
    \Fourier [\CompactGroup] f \defeq \int_\CompactGroup f(k) \adj{\RightRegularRepresentation(k)} \dd k.
\end{align*}

If $f \in \SmoothFunctions \CompactGroup$,
then $f$ can be recovered via the following \emph{inversion formula}:
\begin{align*}
    f(k) = \tr(\RightRegularRepresentation(k) \Fourier [\CompactGroup] f),
    \quad k \in \CompactGroup,
\end{align*}
while the \emph{Plancherel formula} takes the form
\begin{align*}
    \ip [\Lebesgue 2 \CompactGroup] {f_1} {f_2}
    = \tr(\Fourier [\CompactGroup] f_1 \adj{\Fourier [\CompactGroup] f_2})
\end{align*}
for $f_1, f_2 \in \Lebesgue 2 \CompactGroup$.

\subsection{Symbolic calculus}

\subsubsection{Difference operators}

The success of our symbolic approach relies on successfully replacing difference operators on $\T^n$
and derivatives in frequency for $\R^n$.

It was observed for the first time in~\cite{RuzhanskyTurunen10}
that the above could be generalised for general groups by considering the action of a well-chosen family of smooth functions
\emph{before} applying the Fourier Transform.

\begin{definition}[Difference operator, \cite{RuzhanskyTurunen10}]
    Let $q \in \SmoothFunctions \CompactGroup$ be such that $q(e_\Group) = 0$.
    The \emph{difference operator associated with $q$} is the operator
    \begin{align*}
        \DifferenceOperator [\CompactGroup] q \Fourier [\CompactGroup] f
        \defeq \Fourier [\CompactGroup] \{q f\}
    \end{align*}

    If $q$ vanishes at order $k$,
    then we shall say that $q$ is a difference operator of \emph{order} $k$.
\end{definition}

\subsubsection{Symbols}

\begin{definition}[Symbol classes on $\CompactGroup$]
    Let $m \in \R$ and fix $\rho, \delta \in \R$ such that $1 \geq \rho \geq \delta \geq 0$.
    We shall say that
    \begin{align*}
        \sigma : \CompactGroup \to \End(\SmoothFunctions \CompactGroup)
    \end{align*}
    is a \emph{symbol of order $m$} and of \emph{type $(\rho, \delta)$}
    if the following properties hold.
    \begin{enumerate}
        \item
            For each $\tau \in \dualGroup \CompactGroup$ and each $k \in \CompactGroup$,
            the map $\sigma(k)$ leaves the space
            \begin{align*}
                V_{i, \tau} \defeq \{\tau_{ij} : 1 \leq j \leq \dimRep \tau\}
            \end{align*}
            invariant for each $i = 1, \dots, \dimRep \tau$.
            Moreover, for each $i, i' \in \{1, \dots, \dimRep \tau\}$,
            we have
            \begin{align*}
                \eval \sigma {V_{i, \tau}} = R_{i i'} \circ \eval \sigma {V_{i', \tau}} \circ R_{i' i},
            \end{align*}
            where $R^\tau_{i i'} : V_{i', \tau} \to V_{i, \tau}$ is a linear transformation determined by
            \begin{align*}
                R^\tau_{i i'} \tau_{i' j} = \tau_{i j},
            \end{align*}
            and naturally $R^\tau_{i' i} = (R^\tau_{i i'})^{-1}$.
        \item Our map $\sigma$ satisfies
            \begin{align*}
                \sup_{k \in \CompactGroup}
                \norm [\Lin {\Lebesgue 2 \CompactGroup}] {%
                    \RightRegularRepresentation \BesselPotential [\CompactGroup] {-m + \rho \abs \alpha - \delta \abs \beta}
                    \LeftDifferentialOperator \beta \DifferenceOperatorOrder [\CompactGroup] \alpha \sigma(k)
                } < \infty
            \end{align*}
            for every $\beta \in \N^{\dim \CompactGroup}$ and every $\dimDifferenceOperators$.
            % TODO: specify set for alpha
    \end{enumerate}
\end{definition}

\section{The graded case}
