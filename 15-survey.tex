\chapter{Survey}

\section{The compact case}

In this section,
$\CompactGroup$ denotes a \emph{connected, compact Lie group}.
We equip $\CompactGroup$ with the unique normalized bi-invariant Riemannian metric,
and denote the corresponding Laplace-Beltrami operator by $\Laplacian [\CompactGroup]$.

The description of the unitary dual is given by the following \emph{Peter-Weyl theorem}.

\begin{theorem}[Peter-Weyl theorem]
\label{theorem:Peter-Weyl_theorem}
    Let $\CompactGroup$ be a compact topological group.
    The unitary dual $\dualGroup \CompactGroup$ of $\Group$ is discrete and
    the set
    \begin{align*}
        \left\{
            \sqrt{\dimRep\tau} \tau_{ij} : \tau \in \dualGroup\CompactGroup,\ i, j = 1, \dots, \dimRep\tau
        \right\}
    \end{align*}
    is an orthonormal basis of $\Lebesgue{2}{\CompactGroup}$ with respect to the \emph{normalised} Haar measure.
\end{theorem}

In particuar,
the Peter-Weyl theorems implies that $\Lebesgue 2 \CompactGroup$ has the following orthogonal decomposition
\begin{align}
    \Lebesgue 2 \CompactGroup
    = \bigoplus_{\tau \in \dualGroup \CompactGroup} \dimRep \tau H_{\tau, j},
    \label{eq:decomposition_of_L2_for_compact_groups}
\end{align}
where $H_{\tau, j} \defeq \Span\{\tau_{j l} \in \Lebesgue 2 \CompactGroup : l = 1, \dots, \dim \CompactGroup\}$,
while
\begin{align*}
    e^{\tau, j}_l \defeq \sqrt {\dimRep \tau} \tau_{j l}, \quad l = 1, \dots, \dim \CompactGroup
\end{align*}
forms an orthonormal basis of $H_{\tau, j}$.

Observe that
\begin{align*}
    \ip [\Lebesgue 2 \CompactGroup] {\RightRegularRepresentation(k) e^{\tau, j}_n} {e^{\tau, j}_m}
    &= \dimRep \tau \int_\CompactGroup \tau_{j n}(h k) \conj{\tau_{j m}}(h) \dd h\\
    &= \sum_{p = 1}^{\dimRep \tau} \left( \dimRep \tau \int_\CompactGroup \tau_{j p}(h) \conj{\tau_{j m}}(h) \dd h \right) \tau_{p n}(k)\\
    &= \sum_{p = 1}^{\dimRep \tau} \Kronecker p m \tau_{p n}(k) = \tau_{m n}(k),
\end{align*}
which means that
\begin{align*}
    (\RightRegularRepresentation(k), H_{\tau, j}) \simeq \tau.
\end{align*}

Combining the above with~\eqref{eq:decomposition_of_L2_for_compact_groups},
we obtain
\begin{align*}
    \RightRegularRepresentation \simeq \bigoplus_{\tau \in \dualGroup \CompactGroup} \dimRep \tau \tau,
\end{align*}
i.e.\ the right-regular representation of $\CompactGroup$ generates the unitary dual $\dualGroup \CompactGroup$.

\subsection{Fourier analysis}

Given $f \in \Lebesgue 1 \CompactGroup$,
we define its \emph{Fourier transform} via
\begin{align*}
    \Fourier [\CompactGroup] f \defeq \int_\CompactGroup f(k) \adj{\RightRegularRepresentation(k)} \dd k.
\end{align*}

If $f \in \SmoothFunctions \CompactGroup$,
then $f$ can be recovered via the following \emph{inversion formula}:
\begin{align*}
    f(k) = \tr(\RightRegularRepresentation(k) \Fourier [\CompactGroup] f),
    \quad k \in \CompactGroup,
\end{align*}
while the \emph{Plancherel formula} takes the form
\begin{align*}
    \ip [\Lebesgue 2 \CompactGroup] {f_1} {f_2}
    = \tr(\Fourier [\CompactGroup] f_1 \adj{\Fourier [\CompactGroup] f_2})
\end{align*}
for $f_1, f_2 \in \Lebesgue 2 \CompactGroup$.

\section{The graded case}
