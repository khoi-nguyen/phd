\chapter{Survey}

\section{The compact case}

In this section,
$\CompactGroup$ denotes a \emph{connected, compact Lie group}.
We equip $\CompactGroup$ with the unique normalized bi-invariant Riemannian metric,
and denote the corresponding Laplace-Beltrami operator by $\Laplacian [\CompactGroup]$.

The description of the unitary dual is given by the following \emph{Peter-Weyl theorem}.

\begin{definition}
    Let $\tau \in \dualGroup\CompactGroup$.
    We define the sets
    \begin{align*}
        \HilbertCompactGroupColumn{\tau}{i} \defeq \span \{ \tau_{i, j} : j = 1, \dots, \dimRep\tau \}
        \subset \Lebesgue{2}\CompactGroup, \quad
        \HilbertCompactGroup{\tau} = \bigoplus_{i = 1}^\dimRep\tau \HilbertCompactGroupColumn{\tau}{i}.
    \end{align*}
\end{definition}

\begin{theorem}[Peter-Weyl theorem]
\label{theorem:Peter-Weyl_theorem}
    Let $\CompactGroup$ be a compact topological group.
    The unitary dual $\dualGroup \CompactGroup$ of $\Groups$ is discrete and
    the set
    \begin{align*}
        \left\{
            \sqrt{\dimRep\tau} \tau_{ij} : \tau \in \dualGroup\CompactGroup,\ i, j = 1, \dots, \dimRep\tau
        \right\}
    \end{align*}
    is an orthonormal basis of $\Lebesgue{2}{\CompactGroup}$ with respect to the \emph{normalised} Haar measure.

    Therefore, we obtain the following decomposition:
    \begin{align*}
        \Lebesgue{2}{\CompactGroup} \defeq
        \bigoplus_{\tau \in \dualGroup\CompactGroup} \bigoplus_{j = 1}^{\dimRep \tau} \Hilbert{\tau}{j}.
    \end{align*}

    Since $\Hilbert{\tau}{j}$ is an invariant subspace of $\RightRegularRepresentation$
    and since $\eval{\RightRegularRepresentation}{\Hilbert{\tau}{j}} \sim \tau$,
    we have
    \begin{align*}
        \RightRegularRepresentation \sim
        \bigoplus_{\tau \in \dualGroup\CompactGroup} \dimRep\tau \tau.
    \end{align*}
    In particular, the unitary dual can be generated by the right-regular representation.
\end{theorem}

\subsection{Fourier analysis on Lie groups}

\section{The graded case}
