\chapter{Applications of pseudo-differential calculus}

\section{Construction of parametrices}

Let $\chi \in \SmoothFunctions {\R^+, [0, 1]}$ be such that
\begin{align*}
    \chi(r) =
    \begin{cases}
        1 & \text{if } r \geq 1\\
        0 & \text{if } r \leq \frac 1 2.
    \end{cases}
\end{align*}

Now, given $\Lambda \geq 0$,
we define the symbol
\begin{align*}
    \psi_\Lambda(\lambda)
    \defeq \sum_{\tau \in \dualGroup \CompactGroup}
    \chi(\Lambda(\JapaneseBracket \VectorSpace \lambda + \JapaneseBracket \CompactGroup \lambda)) \Id {\Span \{\tau_{i j} : 1 \leq i, j \leq \dimRep \tau\}}.
\end{align*}

\begin{definition}[Ellipticity]
\label{definition:ellipticity}
    Let $\sigma \in \SymbolClass m {\rho, \delta}$
    We shall say that that $\sigma$ is \emph{elliptic} and of \emph{elliptic order} $m$
    if there exists $\Lambda \geq 0$ such that the following property holds:
    for each $\gamma \in \R$,
    there exists $C \geq 0$ such that
    \begin{align}
        \norm [\Lebesgue 2 \CompactGroup] {\Rep \lambda \BesselPotential \gamma \sigma(g, \lambda) \psi_\Lambda(\lambda) F}
        \geq C
        \norm [\Lebesgue 2 \CompactGroup] {\Rep \lambda \BesselPotential {\gamma + m} \psi_\Lambda(\lambda) F}
        \label{eq:definition_of_ellipticity}
    \end{align}
    holds for every $g \in \Group$, $\lambda \in \Group$
    and every $F \in \Lebesgue 2 \CompactGroup$.
\end{definition}

\begin{lemma}[Inverse of an elliptic symbol]
    Let $\sigma \in \SymbolClass m {\rho, \delta}$ be an elliptic symbol of elliptic order $m$,
    with $\Lambda \geq 0$ given by Definition~\ref{definition:ellipticity}.

    For any $F \in \SmoothFunctions \CompactGroup$,
    let
    \begin{align*}
        \sigma_\Lambda(g, \lambda)^{-1} \left(\sigma(g, \lambda) \psi_\Lambda(\lambda) F\right) \defeq \psi_\Lambda F.
    \end{align*}
    and consider the unique extension of $\sigma_\Lambda(g, \lambda)$ onto $\SmoothFunctions \CompactGroup$ which satisfies
    \begin{align*}
        \eval {%
            \sigma_\Lambda(g, \lambda)^{-1}
            } {(\Image \{\sigma(g, \lambda) \psi_\Lambda\})^\perp} = 0.
    \end{align*}

    The map $\sigma_\Lambda$ is well-defined,
    and is a symbol in $\SymbolClass {-m} {\rho, \delta}$.
\end{lemma}
\begin{proof}
    Suppose that $F_1, F_2 \in \SmoothFunctions \CompactGroup$ are such that
    \begin{align*}
        \sigma(g, \lambda) \psi_\Lambda F_1
        = \sigma(g, \lambda) \psi_\Lambda F_2.
    \end{align*}
    By definition of ellipticity,
    it follows that
    \begin{align*}
        \norm [\Lebesgue 2 \CompactGroup] {\psi_\Lambda (F_1 - F_2)}
        &\leq C
        \norm [\Lebesgue 2 \CompactGroup] {\Rep \lambda \BesselPotential {-m} \sigma(g, \lambda) \psi_\Lambda(\lambda) (F_1 - F_2)}\\
        &= 0,
    \end{align*}
    so that the map is indeed well-defined.

    Let us now check it is of order $-m$.
    In order to do so,
    let us first observe that
    \begin{align}
        \psi_\Lambda = \sigma_\Lambda(g, \lambda)^{-1} \sigma(g, \lambda) \psi_\Lambda.
        \label{eq:identity_satisfied_by_inverse_of_symbol}
    \end{align}
    Letting $G \defeq \sigma(g, \lambda) \psi_\Lambda F$ with $F \in \SmoothFunctions \CompactGroup$,
    it is clear that by~\eqref{eq:identity_satisfied_by_inverse_of_symbol} we have
    \begin{align*}
        \norm [\Lebesgue 2 \CompactGroup] {\Rep \lambda \BesselPotential m \sigma_\Lambda(g, \lambda)^{-1} G}
        &=
        \norm [\Lebesgue 2 \CompactGroup] {\Rep \lambda \BesselPotential m \psi_\Lambda F}\\
        &\leq C
        \norm [\Lebesgue 2 \CompactGroup] {\sigma(g, \lambda) \psi_\Lambda F}\\
    \end{align*}
    where the second line was obtained
    by using~\eqref{eq:definition_of_ellipticity} with $\gamma = 0$.
    Using $G = \sigma(g, \lambda) \psi_\Lambda F$ in the right-hand side of the above,
    we obtain that
    \begin{align*}
        \sup_{g \in \Group}
        \esssup_{\lambda \in \VectorSpace}
        \norm [\Lin {\Lebesgue 2 \CompactGroup}] {\Rep \lambda \BesselPotential {m - \delta \abs {\beta_0}} \LeftDifferentialOperator {\beta_0} \sigma_\Lambda(g, \lambda)^{-1}}
        < \infty,
    \end{align*}
    for $\beta_0 = 0$,
    since $G$ is arbitrary and $\sigma_\Lambda(g, \lambda)^{-1}$ vanishes on the orthogonal of $\Image \{\sigma(g, \lambda) \psi_\Lambda\}$.
    Let us show it also holds for $\beta_0$ of higher order.

    By the Leibniz rule,
    we have
    \begin{align*}
        0 =
        \LeftDifferentialOperator {\beta_0} \psi_\Lambda
        =
        \sum_{\beta_0 = \beta_1 + \beta_2}
        \left(
        \LeftDifferentialOperator {\beta_1} \sigma_\Lambda(g, \lambda)^{-1}
        \LeftDifferentialOperator {\beta_2} (\sigma(g, \lambda) \psi_\Lambda)\right),
    \end{align*}
    which implies that
    \begin{align*}
        \LeftDifferentialOperator {\beta_0} \sigma_{\Lambda}(g, \lambda)^{-1} \sigma(g, \lambda) \psi_\Lambda
        = -
        \sum_{\substack{
            \beta_0 = \beta_1 + \beta_2\\
            \abs {\beta_1} < \abs {\beta_0}}
        }
        \left(
        \LeftDifferentialOperator {\beta_1} \sigma_\Lambda(g, \lambda)^{-1}
        \LeftDifferentialOperator {\beta_2} (\sigma(g, \lambda) \psi_\Lambda)\right).
    \end{align*}

    Letting $G = \sigma(g, \lambda) \psi_\Lambda F$ like above,
    it should be clear that by the induction hypothesis we have
    \begin{align*}
        &\norm [\Lebesgue 2 \CompactGroup] {%
            \Rep \lambda \BesselPotential {m - \delta \abs {\beta_0}}
            \LeftDifferentialOperator {\beta_0} \sigma_\Lambda(g, \lambda)^{-1} G
        }\\
        &\leq
        \sum_{\substack{
            \beta_0 = \beta_1 + \beta_2\\
            \abs {\beta_1} < \abs {\beta_0}}
        }
        \norm [\Lebesgue 2 \CompactGroup] {%
            \Rep \lambda \BesselPotential {m - \delta \abs {\beta_0}}
            \left(
            \LeftDifferentialOperator {\beta_1} \sigma_\Lambda(g, \lambda)^{-1}
            \LeftDifferentialOperator {\beta_2} (\sigma(g, \lambda) \psi_\Lambda)\right) F
        }\\
        &\leq
        \sum_{\substack{
            \beta_0 = \beta_1 + \beta_2\\
            \abs {\beta_1} < \abs {\beta_0}}
        }
        \norm [\Lin {\Lebesgue 2 \CompactGroup}] {%
            \Rep \lambda \BesselPotential {m - \delta \abs {\beta_0}}
            \LeftDifferentialOperator {\beta_1} \sigma_\Lambda(g, \lambda)^{-1}
            \Rep \lambda \BesselPotential {\delta \abs {\beta_2}}
        }\\
        & \qquad\qquad
        \norm [\Lin {\Lebesgue 2 \CompactGroup}] {%
            \Rep \lambda \BesselPotential {-\delta \abs {\beta_2}}
            \LeftDifferentialOperator {\beta_2} \sigma(g, \lambda)
            \Rep \lambda \BesselPotential {-m}
        }\\
        & \qquad\qquad
        \norm [\Lebesgue 2 \CompactGroup] {\Rep \lambda \BesselPotential {m} \psi_\Lambda F}.
    \end{align*}

    Using ellipticity again,
    we know that
    \begin{align*}
        \norm [\Lebesgue 2 \CompactGroup] {\Rep \lambda \BesselPotential {m} \psi_\Lambda F}
        &\leq C \norm [\Lebesgue 2 \CompactGroup] {\sigma(g, \lambda) \psi_\Lambda F} = C \norm [\Lebesgue 2 \CompactGroup] {G},
    \end{align*}
    which if plugged into the calculation above,
    yields
    \begin{align*}
        \norm [\Lin {\Lebesgue 2 \CompactGroup}] {%
            \Rep \lambda \BesselPotential {m - \delta \abs {\beta_0}}
            \LeftDifferentialOperator {\beta_0} \sigma_\Lambda(g, \lambda)^{-1}
        } \leq C,
    \end{align*}
    where $C$ does not depend on $g$ or $\lambda$.
\end{proof}

\section{Functional calculus}
